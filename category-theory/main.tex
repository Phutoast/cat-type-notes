\documentclass{article}
\usepackage{blindtext}
\usepackage{appendix}
\usepackage{geometry}
\geometry{
    a4paper,
    total={170mm,257mm},
    left=20mm,
    top=20mm,
}
\usepackage{amsthm}
\usepackage{amsmath}
\usepackage{amssymb}
\usepackage{hyperref}
\usepackage{xcolor}
\usepackage{tikz}
\usepackage{tikz-cd}
\usepackage{mathtools}
\usepackage{multicol}
\usepackage{quiver}

\setlength{\parindent}{0pt}
\setlength{\parskip}{0.3em} 


\newtheorem{theorem}{Theorem}
\newtheorem{axiom}{Axiom} 
\newtheorem{proposition}{Proposition}
\newtheorem{definition}{Definition}
\newtheorem{lemma}{Lemma}
\newtheorem{corollary}{Corollary}
\newtheorem{remark}{Remark}
\newtheorem{example}{Example}

\newcommand*{\review}{\textcolor{red}{\textbf{(TO REVIEW)}}}
\newcommand*{\todo}{\textcolor{red}{\textbf{(TO DO)}}}
\newcommand*{\checkproof}{\textcolor{red}{\textbf{(TO CHECK)}}}
\newcommand*{\brackc}[1]{\ensuremath{\left\langle{#1}\right\rangle}}
\newcommand*{\profunct}{ \to\!\!\!\!\!\!\rule{.1pt}{1.3ex} \ \ }

\usepackage[utf8]{inputenc}

\title{Category Theory: Primer (WIP)}
\date{2024}
\author{Phu Sakulwongtana}

\begin{document}

\maketitle

This notes is an mashed up/expository of various books but mainly: \textit{Notes on Category Theory} by Paolo Perrone and \textit{The Dao of Functional Programming} by Bartosz Milewski.

\section{Introduction}

We shall start with the definition of category itself together with the common notations that will be used throughout.

\begin{definition}{\textbf{(Category)}}
    A category $\textbf{C}$ consists of a collection of objects and arrow/morphism between them. Furthermore, with object $a$ and object $b$, there can be a morphism $f:a\to b$ and if there is another morphism $g:b\to c$ to object $c$, then we can composed 2 morphisms together to get a new one, that is $g\circ f:a\to c$ (meaning that there is morphism from $a$ to $c$). The morphisms/compositions have the following properties
    \begin{itemize}
        \item There will always be an morphism from any object to itself and it is called identity morphism i.e given object $a$, its identity morphism is denoted as $\operatorname{id}_a:a\to a$, and have the following properties:
        $$f\circ \operatorname{id}_a=\operatorname{id}_b\circ f = f$$
        where $f:a\to b$
        \item The composition is associative, that is, given $f:a\to b$, $g:b\to c$ and $h:c\to d$, then: $$h\circ(g\circ f)=(h\circ g)\circ f$$
    \end{itemize}
    Finally, one can also gather all the morphism between 2 objects into a set, which call it hom-set. It is denoted, for given objects $a$ and $b$, as $\operatorname{Hom}_\textbf{C}(a,b)$.
\end{definition}

Now, we also have important special kind of arrow that denotes the notion of ``equality'' in category theory sense (note that usually, we can't distinguish 2 objects beyond that).

\begin{definition}{\textbf{(Isomorphism)}}
    The morphism $f:a\to b$ is an isomorphism iff there is another morphism $f^{-1}:b\to a$ such that $$f^{-1}\circ f=\operatorname{id}_a\qquad\text{ and }\qquad f\circ f^{-1}=\operatorname{id}_b$$we call $f^{-1}$, inverse of $f$. Furthermore, if there is an isomorphism between objects $a$ and $b$, then we call them isomorphic and denote it as $a\cong b$
\end{definition}

\section{Relational Point of View}

We will, then, move on the first result that illustrates the ``relational'' aspect of category theory, being the main strength of category theory. It is quite clear that category \textit{prohibits} the viewing of the inside of the object, that is we can only compare/inspect objects using an arrow. Let's start with the simplest example.

\begin{remark}
    Given the category of sets $\textbf{Set}$, where its object are all the sets (we will ignore the issue of Russell's paradox for now) and the morphisms between sets are function between sets. To see the elements of a set $A$, one can observe the following:
    $$A\cong\operatorname{Hom}_\textbf{Set}(1, A)$$
    where $1$ is a singleton set i.e set with only one element. With this, one can ``represent '' the elements of the set based on the set of morphism from singleton set.
\end{remark}

One can also view the remark above as the way object in category are characterized based on the relationship between each other, and we have used the singleton set $1$ to ``probe'' into the set. Let's consider this probing action in a more generalized manner. 

\begin{definition}{\textbf{(Probing Actions/Neutrality Condition)}}
	\label{def:prob-action-nat}
    Let's consider the following diagram, where on the LHS, we consider the action of ``changing target'', while the RHS got to do with the ``changing the source'' (from $x$ to $y$ in both cases), highlighted in red
    \begin{equation*}
	% https://q.uiver.app/#q=WzAsNixbMCwxLCJBIl0sWzIsMSwiQiJdLFsxLDAsIlgiXSxbNCwwLCJYIl0sWzYsMCwiWSJdLFs1LDEsIkEiXSxbMCwxLCJmIiwyXSxbMiwwLCJoIiwyXSxbMiwxLCJoXFxjaXJjIGciXSxbNCw1LCJrXFxjaXJjIGciXSxbMyw1LCJrIiwyXSxbNCwzLCJnIiwyXSxbNyw4LCIoZlxcY2lyYyAtKSIsMix7InNob3J0ZW4iOnsic291cmNlIjoyMCwidGFyZ2V0IjoyMH0sImxldmVsIjoxLCJjb2xvdXIiOlswLDYwLDYwXSwic3R5bGUiOnsiYm9keSI6eyJuYW1lIjoiZGFzaGVkIn19fSxbMCw2MCw2MCwxXV0sWzEwLDksIigtXFxjaXJjIGcpIiwwLHsic2hvcnRlbiI6eyJzb3VyY2UiOjIwLCJ0YXJnZXQiOjIwfSwibGV2ZWwiOjEsImNvbG91ciI6WzAsNjAsNjBdLCJzdHlsZSI6eyJib2R5Ijp7Im5hbWUiOiJkYXNoZWQifX19LFswLDYwLDYwLDFdXV0=
	\begin{tikzcd}
		& X &&& X && Y \\
		A && B &&& A
		\arrow[""{name=0, anchor=center, inner sep=0}, "h"', from=1-2, to=2-1]
		\arrow[""{name=1, anchor=center, inner sep=0}, "{h\circ g}", from=1-2, to=2-3]
		\arrow[""{name=2, anchor=center, inner sep=0}, "k"', from=1-5, to=2-6]
		\arrow["g"', from=1-7, to=1-5]
		\arrow[""{name=3, anchor=center, inner sep=0}, "{k\circ g}", from=1-7, to=2-6]
		\arrow["f"', from=2-1, to=2-3]
		\arrow["{(f\circ -)}"', color={rgb,255:red,214;green,92;blue,92}, shorten <=6pt, shorten >=6pt, dashed, from=0, to=1]
		\arrow["{(-\circ g)}", color={rgb,255:red,214;green,92;blue,92}, shorten <=6pt, shorten >=6pt, dashed, from=2, to=3]
	\end{tikzcd}
    \end{equation*}
    Please be aware that these two diagram are independent of each other. We can see, here, that to change the target of viewing, we can perform the post-composition of the morphism between 2 targets to the probe. Similarly, the pre-composition is used to change the source.
    
    \paragraph{Neutrality:} Now, it wouldn't make sense if we are were to change the target and the source (LHS diagram below), and get different result from when we change the source and then target (RHS diagram below). (Reading from left to right):

    \begin{equation*}
	% https://q.uiver.app/#q=WzAsOCxbMCwxLCJBIl0sWzMsMSwiQiJdLFswLDAsIlgiXSxbNSwwLCJYIl0sWzgsMCwiWSJdLFs1LDEsIkEiXSxbMywwLCJZIl0sWzgsMSwiQiJdLFswLDEsImYiLDJdLFsyLDAsImgiLDJdLFsyLDEsImhcXGNpcmMgZiIsMV0sWzMsNSwiayIsMl0sWzQsNSwia1xcY2lyYyBnIiwxXSxbNiwxXSxbNCwzLCJnIiwyXSxbNiwyLCJnIiwyXSxbNSw3LCJmIiwyXSxbNCw3XSxbOSwxMCwiKGZcXGNpcmMgLSkiLDIseyJzaG9ydGVuIjp7InNvdXJjZSI6MjAsInRhcmdldCI6MjB9LCJsZXZlbCI6MSwiY29sb3VyIjpbMCw2MCw2MF0sInN0eWxlIjp7ImJvZHkiOnsibmFtZSI6ImRhc2hlZCJ9fX0sWzAsNjAsNjAsMV1dLFsxMCwxMywiKC1cXGNpcmMgZykiLDAseyJzaG9ydGVuIjp7InNvdXJjZSI6MjAsInRhcmdldCI6MjB9LCJsZXZlbCI6MSwiY29sb3VyIjpbMCw2MCw2MF0sInN0eWxlIjp7ImJvZHkiOnsibmFtZSI6ImRhc2hlZCJ9fX0sWzAsNjAsNjAsMV1dLFsxMiwxMSwiKC1cXGNpcmMgZykiLDIseyJzaG9ydGVuIjp7InNvdXJjZSI6MjAsInRhcmdldCI6MjB9LCJsZXZlbCI6MSwiY29sb3VyIjpbMCw2MCw2MF0sInN0eWxlIjp7InRhaWwiOnsibmFtZSI6ImFycm93aGVhZCJ9LCJib2R5Ijp7Im5hbWUiOiJkYXNoZWQifSwiaGVhZCI6eyJuYW1lIjoibm9uZSJ9fX0sWzAsNjAsNjAsMV1dLFsxMiwxNywiKGZcXGNpcmMtKSIsMix7InNob3J0ZW4iOnsic291cmNlIjoyMCwidGFyZ2V0IjoyMH0sImxldmVsIjoxLCJjb2xvdXIiOlswLDYwLDYwXSwic3R5bGUiOnsiYm9keSI6eyJuYW1lIjoiZGFzaGVkIn19fSxbMCw2MCw2MCwxXV1d
	\begin{tikzcd}
		X &&& Y && X &&& Y \\
		A &&& B && A &&& B
		\arrow[""{name=0, anchor=center, inner sep=0}, "h"', from=1-1, to=2-1]
		\arrow[""{name=1, anchor=center, inner sep=0}, "{h\circ f}"{description}, from=1-1, to=2-4]
		\arrow["g"', from=1-4, to=1-1]
		\arrow[""{name=2, anchor=center, inner sep=0}, from=1-4, to=2-4]
		\arrow[""{name=3, anchor=center, inner sep=0}, "k"', from=1-6, to=2-6]
		\arrow["g"', from=1-9, to=1-6]
		\arrow[""{name=4, anchor=center, inner sep=0}, "{k\circ g}"{description}, from=1-9, to=2-6]
		\arrow[""{name=5, anchor=center, inner sep=0}, from=1-9, to=2-9]
		\arrow["f"', from=2-1, to=2-4]
		\arrow["f"', from=2-6, to=2-9]
		\arrow["{(f\circ -)}"', color={rgb,255:red,214;green,92;blue,92}, shorten <=10pt, shorten >=10pt, dashed, from=0, to=1]
		\arrow["{(-\circ g)}", color={rgb,255:red,214;green,92;blue,92}, shorten <=10pt, shorten >=10pt, dashed, from=1, to=2]
		\arrow["{(-\circ g)}"', color={rgb,255:red,214;green,92;blue,92}, shorten <=10pt, shorten >=10pt, dashed, tail reversed, no head, from=4, to=3]
		\arrow["{(f\circ-)}"', color={rgb,255:red,214;green,92;blue,92}, shorten <=10pt, shorten >=10pt, dashed, from=4, to=5]
	\end{tikzcd}
    \end{equation*}
    Therefore, we enforce the neutrality condition onto the probing action (the naming does make sense):
    $$
    (-\circ g)\circ(f\circ-) = (f\circ-)\circ(-\circ g)
    $$
       
\end{definition}

Now, we got a more general tool to inspect the relationship between objects. And so, we are ready to consider/proof the following result.

\begin{proposition}
	\label{prop:iso-obj-in}
    Within category $\textbf{C}$, given a pair objects $a$ and $b$ and with any object $x$, we have an \textbf{naturally} isomorphic $\operatorname{Hom}_\textbf{C}(x, a)\cong\operatorname{Hom}_\textbf{C}(x, b)$ iff $a\cong b$. That is, if the arrow coming to both $a$ and $b$ are indistinguishable, then $a$ and $b$ are isomorphic.
\end{proposition}

This proposition is actually a special case of Yoneda lemma (which we will consider later), but the gist of it is that the object can be characterized based on the arrow coming in.

\begin{dem}
\begin{proof}

$(\boldsymbol{\implies}):$ Let's call the map between $\operatorname{Hom}_\textbf{C}(x, a)$ and $\operatorname{Hom}_\textbf{C}(x, b)$ to be called $\alpha_x$. Let's explain what it means for $\alpha$ to be natural. We have that both change of source and target should be the same:

\begin{equation*}
% https://q.uiver.app/#q=WzAsOCxbMCwwLCJYIl0sWzMsMCwiWSJdLFswLDEsIkEiXSxbMywxLCJCIl0sWzUsMCwiWCJdLFs4LDAsIlkiXSxbNSwxLCJBIl0sWzgsMSwiQiJdLFswLDIsImgiLDJdLFswLDMsIlxcYWxwaGFfWChoKSIsMV0sWzEsMCwiZyIsMl0sWzQsNiwiaCIsMl0sWzUsNiwiaFxcY2lyYyBnIiwxXSxbNSw0LCJnIiwyXSxbNSw3XSxbMSwzXSxbOCw5LCJcXGFscGhhX1giLDIseyJzaG9ydGVuIjp7InNvdXJjZSI6MjAsInRhcmdldCI6MjB9LCJsZXZlbCI6MSwiY29sb3VyIjpbMCw2MCw2MF0sInN0eWxlIjp7ImJvZHkiOnsibmFtZSI6ImRhc2hlZCJ9fX0sWzAsNjAsNjAsMV1dLFsxMSwxMiwiKC1cXGNpcmMgZykiLDAseyJzaG9ydGVuIjp7InNvdXJjZSI6MjAsInRhcmdldCI6MjB9LCJsZXZlbCI6MSwiY29sb3VyIjpbMCw2MCw2MF0sInN0eWxlIjp7ImJvZHkiOnsibmFtZSI6ImRhc2hlZCJ9fX0sWzAsNjAsNjAsMV1dLFsxMiwxNCwiXFxhbHBoYV9ZIiwyLHsic2hvcnRlbiI6eyJzb3VyY2UiOjIwLCJ0YXJnZXQiOjIwfSwibGV2ZWwiOjEsImNvbG91ciI6WzAsNjAsNjBdLCJzdHlsZSI6eyJib2R5Ijp7Im5hbWUiOiJkYXNoZWQifX19LFswLDYwLDYwLDFdXSxbOSwxNSwiKC1cXGNpcmMgZykiLDAseyJzaG9ydGVuIjp7InNvdXJjZSI6MjAsInRhcmdldCI6MjB9LCJsZXZlbCI6MSwiY29sb3VyIjpbMCw2MCw2MF0sInN0eWxlIjp7ImJvZHkiOnsibmFtZSI6ImRhc2hlZCJ9fX0sWzAsNjAsNjAsMV1dXQ==
\begin{tikzcd}
	X &&& Y && X &&& Y \\
	A &&& B && A &&& B
	\arrow[""{name=0, anchor=center, inner sep=0}, "h"', from=1-1, to=2-1]
	\arrow[""{name=1, anchor=center, inner sep=0}, "{\alpha_X(h)}"{description}, from=1-1, to=2-4]
	\arrow["g"', from=1-4, to=1-1]
	\arrow[""{name=2, anchor=center, inner sep=0}, from=1-4, to=2-4]
	\arrow[""{name=3, anchor=center, inner sep=0}, "h"', from=1-6, to=2-6]
	\arrow["g"', from=1-9, to=1-6]
	\arrow[""{name=4, anchor=center, inner sep=0}, "{h\circ g}"{description}, from=1-9, to=2-6]
	\arrow[""{name=5, anchor=center, inner sep=0}, from=1-9, to=2-9]
	\arrow["{\alpha_X}"', color={rgb,255:red,214;green,92;blue,92}, shorten <=10pt, shorten >=10pt, dashed, from=0, to=1]
	\arrow["{(-\circ g)}", color={rgb,255:red,214;green,92;blue,92}, shorten <=10pt, shorten >=10pt, dashed, from=1, to=2]
	\arrow["{(-\circ g)}", color={rgb,255:red,214;green,92;blue,92}, shorten <=10pt, shorten >=10pt, dashed, from=3, to=4]
	\arrow["{\alpha_Y}"', color={rgb,255:red,214;green,92;blue,92}, shorten <=10pt, shorten >=10pt, dashed, from=4, to=5]
\end{tikzcd}
\end{equation*}

Or $\alpha_x(h)\circ g = \alpha_y(h\circ g)$. To show that $a\cong b$, we will have to find a map between $a$ and $b$ and its reverse. Let's start with a map $a\to b$, we can have:

\begin{equation*}
% https://q.uiver.app/#q=WzAsNixbMCwxLCJBIl0sWzAsMCwiQSJdLFsyLDEsIkIiXSxbNCwxLCJBIl0sWzYsMSwiQiJdLFs2LDAsIkIiXSxbMSwwLCJcXG9wZXJhdG9ybmFtZXtpZH1fQSIsMl0sWzEsMiwiXFxhbHBoYV9BKFxcb3BlcmF0b3JuYW1le2lkfV9BKSJdLFs1LDQsIlxcb3BlcmF0b3JuYW1le2lkfV9CIl0sWzUsMywiXFxhbHBoYV9CXnstMX0oXFxvcGVyYXRvcm5hbWV7aWR9X0IpIiwyXSxbOSw4LCJcXGFscGhhX0IiLDIseyJzaG9ydGVuIjp7InNvdXJjZSI6MjAsInRhcmdldCI6MjB9LCJsZXZlbCI6MSwiY29sb3VyIjpbMCw2MCw2MF0sInN0eWxlIjp7ImJvZHkiOnsibmFtZSI6ImRhc2hlZCJ9fX0sWzAsNjAsNjAsMV1dLFs2LDcsIlxcYWxwaGFfQSIsMix7InNob3J0ZW4iOnsic291cmNlIjoyMCwidGFyZ2V0IjoyMH0sImxldmVsIjoxLCJjb2xvdXIiOlswLDYwLDYwXSwic3R5bGUiOnsiYm9keSI6eyJuYW1lIjoiZGFzaGVkIn19fSxbMCw2MCw2MCwxXV1d
\begin{tikzcd}
	A &&&&&& B \\
	A && B && A && B
	\arrow[""{name=0, anchor=center, inner sep=0}, "{\operatorname{id}_A}"', from=1-1, to=2-1]
	\arrow[""{name=1, anchor=center, inner sep=0}, "{\alpha_A(\operatorname{id}_A)}", from=1-1, to=2-3]
	\arrow[""{name=2, anchor=center, inner sep=0}, "{\alpha_B^{-1}(\operatorname{id}_B)}"', from=1-7, to=2-5]
	\arrow[""{name=3, anchor=center, inner sep=0}, "{\operatorname{id}_B}", from=1-7, to=2-7]
	\arrow["{\alpha_A}"', color={rgb,255:red,214;green,92;blue,92}, shorten <=6pt, shorten >=6pt, dashed, from=0, to=1]
	\arrow["{\alpha_B}"', color={rgb,255:red,214;green,92;blue,92}, shorten <=6pt, shorten >=6pt, dashed, from=2, to=3]
\end{tikzcd}
\end{equation*}

that is $\alpha_a(\operatorname{id}_a):a\to b$, and its (candidate) inverse is $\alpha^{-1}_b(\operatorname{id}_b):b\to a$. To show that it is actually an inverse, we need to show that:

$$
\begin{aligned}
\alpha_a(\operatorname{id}_a)\circ\alpha^{-1}_b(\operatorname{id}_b) &= \alpha_b(\operatorname{id}_a\circ\alpha^{-1}_b(\operatorname{id}_b)) \\
&=  \alpha_b(\alpha^{-1}_b(\operatorname{id}_b)) = \operatorname{id}_b
\end{aligned} \qquad \begin{aligned}
\alpha^{-1}_b(\operatorname{id}_b)\circ\alpha_a(\operatorname{id}_a) &= \alpha_a^{-1}(\operatorname{id}_b\circ\alpha_a(\operatorname{id}_a)) \\
&=  \alpha_a^{-1}(\alpha_a(\operatorname{id}_a)) = \operatorname{id}_a
\end{aligned}
$$

where we have used the neutrality condition for both cases. Thus, we have shown that $a\cong b$

\paragraph{$(\boldsymbol{\impliedby}):$} Let's assume that $a\cong b$ that is there is an isomorphism $f:a\xrightarrow{\cong} b$, we will have to construct the map $\alpha$, in which we aim to show that it is actually $(f\circ-)$. There are 2 things we have to check:

\begin{itemize}
    \item \textit{Neutrality:} It is clear that $(f\circ-)$ will satisfies the neutrality condition (which is by definition).
    \item \textit{Isomorphism:} The inverse of this probing is $(f^{-1}\circ-)$, it is clear that 
    $$
    \begin{aligned}
    (f^{-1}\circ-)\circ(f\circ-)=(f^{-1}\circ f\circ-)=(\operatorname{id}_a\circ-) \\
    (f\circ-)\circ(f^{-1}\circ-)=(f\circ f^{-1}\circ-)=(\operatorname{id}_b\circ-)
    \end{aligned}
    $$
    in which both of the results are clearly an identity on probing.
\end{itemize}

We can see that this is also consistent with the construction above $\alpha_a(\operatorname{id}_a)=f\circ\operatorname{id}_a=f$, and $f^{-1}$

\end{proof}
\end{dem}

With very similar proof, in the next proposition, object can also be characterized based on arrow going out, and the result on isomorphism can be found as:

\begin{proposition}
	\label{prop:iso-obj-out}
    Within category $\textbf{C}$, given a pair objects $a$ and $b$ and with any object $x$, we have an \textbf{naturally} isomorphic $\operatorname{Hom}_\textbf{C}(a, x)\cong\operatorname{Hom}_\textbf{C}(b, x)$ iff $a\cong b$. 
\end{proposition}

\section{Categorical Construction}

Back to the example on category of sets, we have see the way we can generalize the characterization of the object which is done solely on the relationship going in or out of this object. We will consider various special of objects that we can defined. Starting from the most basic. 


\subsection{initial/Terminal Object}

Now, let's try to generalize the other part, which is the notion of singleton set, so that it works in not just the $\textbf{Set}$ (but not all object) but some category in general.

\begin{definition}{\textbf{(Initial and Terminal Objects)}} 
	We will consider 2 kinds of objects that have the similar way of defining:
	\begin{itemize}
		\item Initial object, denoted $0$, is an object that have \textbf{unique} morphism from itself to every object (including itself). We see it as an object that expands toward all the other objects 
		\item On the other hand, terminal object, denoted $1$, is an object that have \textbf{unique} morphism from every object to itself (including itself). We can see it as an object that all arrows converges.
	\end{itemize}

	We have the following illustration for this kind of objects:

	\begin{equation*}
	% https://q.uiver.app/#q=WzAsMTgsWzEsMSwiMCJdLFsxLDBdLFsyLDBdLFsyLDFdLFsyLDJdLFsxLDJdLFswLDJdLFswLDFdLFswLDBdLFs1LDEsIjEiXSxbNCwwXSxbNSwwXSxbNiwwXSxbNCwxXSxbNCwyXSxbNSwyXSxbNiwyXSxbNiwxXSxbMCwxXSxbMCwyXSxbMCwzXSxbMCw0XSxbMCw1XSxbMCw2XSxbMCw3XSxbMCw4XSxbMTAsOV0sWzExLDldLFsxMiw5XSxbMTMsOV0sWzE0LDldLFsxNSw5XSxbMTYsOV0sWzE3LDldXQ==
	\begin{tikzcd}
		{} & {} & {} && {} & {} & {} \\
		{} & 0 & {} && {} & 1 & {} \\
		{} & {} & {} && {} & {} & {}
		\arrow[from=2-2, to=1-2]
		\arrow[from=2-2, to=1-3]
		\arrow[from=2-2, to=2-3]
		\arrow[from=2-2, to=3-3]
		\arrow[from=2-2, to=3-2]
		\arrow[from=2-2, to=3-1]
		\arrow[from=2-2, to=2-1]
		\arrow[from=2-2, to=1-1]
		\arrow[from=1-5, to=2-6]
		\arrow[from=1-6, to=2-6]
		\arrow[from=1-7, to=2-6]
		\arrow[from=2-5, to=2-6]
		\arrow[from=3-5, to=2-6]
		\arrow[from=3-6, to=2-6]
		\arrow[from=3-7, to=2-6]
		\arrow[from=2-7, to=2-6]
	\end{tikzcd}
	\end{equation*}

	Note that we can still have arrow coming into initial object or arrow coming out of terminal object, but it won't enjoy the special properties. 
\end{definition}

Please note that initial object and terminal object only exist in some particular category i.e there are category that doesn't have initial or terminal object (take discrete category\footnote{Category there is only identity morphism and no morphism between objects}, for example). $\textbf{Set}$ is one of them that has. Let's consider both kind of objects in category of $\textbf{Set}$. 

\begin{remark}
	In $\textbf{Set}$, an empty set $\emptyset$ is an initial object, and singleton set (which we denote collectively as $1$) is a terminal object:
	
	\begin{itemize}
		\item Thus, in general, one can treat the selection of elements in the object $a$ using the morphism from a terminal object to $a$. Therefore, with $f:a\to b$, the notion $f(a)$ is actually a composition with arrow $a:1\to a$ i.e $f\circ a$
		\item It may be quite clear from now that there can be ``multiple initial and terminal object, for example, in $\textbf{Set}$, the terminal objects are $\{1\}, \{2\}, \{\blacksquare\},\dots$. But we can show that it is unique \textit{up to isomorphism}. 
	\end{itemize}
\end{remark}

% \begin{axiom}{\textbf{(Function Determined by its Effect\cite[Axiom 4]{leinster2014rethinking})}}
\begin{axiom}{\textbf{(Function Determined by its Effect)}}
\label{axiom:funct-effect}
Given a set $X$ and $Y$ and functions $f,g:X\to Y$ if $f(x)=g(x)$ for all $x\in X$, we have $f=g$
\end{axiom}

\begin{proposition}
	In $\textbf{Set}$, given the axiom \ref{axiom:funct-effect}, we can show that object $a$ is terminal object iff it is a singleton set, we will need additional axioms 
\end{proposition}


\begin{proof}
	To show that a singleton set is a terminal object, we note that if there are 2 function from any set $X$ to a singleton set $\{a\}$ i.e $f,g:X\to\{a\}$, then we see that for any objects $x\in X$ $f(x)=g(x)=a$, and so $f=g$. This means that there is an unique arrow going to a singleton set.
	
	On the other hand, terminal object is a singleton set because, there is only one arrow from a terminal object to terminal object (by definition).
\end{proof}

	
	
It is clear from proposition \ref{prop:iso-obj-in} and \ref{prop:iso-obj-out} that two initial or terminal objects are isomorphic to each others. Nonetheless, we can also proof this fact directly from the definition of it. Let's state the proposition and the proof.

\begin{proposition}
	A terminal (initial) object are unique up to isomorphism. In other words, given 2 objects $a$ and $a'$ both having a property of terminal (initial) object, then $a\cong a'$
\end{proposition}


\begin{proof}
	We will provide the proof for terminal object only, but the proof for initial object is similar. We note that by definition, there is an unique arrow from $f : a\to a'$ and $g: a'\to a$:
	\begin{itemize}
		\item We see that when both of them are composed i.e $g\circ f:a\to a$ and $f\circ g:a'\to a'$, they will give rise to identities functions $\operatorname{id}_a$ and $\operatorname{id}_{a'}$
		\item That is because identity is the \textit{only} morphism that maps both object to itself by the definition of the terminal objects.
	\end{itemize}
	Thus $f$ and $g$ are inverse of each other and $f$ is an isomorphism, as needed.
\end{proof}


Before we move on to other construction that are more complex than initial/terminal object, we want to point out that in category theory, instead of explicitly describe the object (set with only one element), we defines the particular relation of this particular object instead (\textit{unique} arrow \textit{from} every object to this object). The universal construction follows from this way of working. 

\subsection{Product}

Let's start with how we define the \textit{product} between two objects. If we remember well, given set $X$ and $Y$, one can define the cartesian product between them as:

$$
X\times Y = \big\{ (x, y) : x\in X \text{ and } y \in Y \big\}
$$

The intersting part of the cartesian product is that its element, one can recover the element of both sets via the projecton map $\pi_1:X\times Y\to X$ and $\pi_2:X\times Y\to Y$, where both $p_1((x', y'))=x'$ and $p_2((x',y'))=y'$. One can view cartesian product as total mixing of elements of $X$ and $Y$, while the projection perform a recovery.

With this in mind, we can define a product in general as:

\begin{definition}{\textbf{(Categortical Product)}}
	\label{def:cat-prod}
	The product between $a$ and $b$ in category $\textbf{C}$ is denoted as $a\times b$ together with morphism $p_1:a\times b\to a$ and $p_2:a\times b\to b$ i.e $(a\times b, p_1, p_2)$ such that given any object $x$ with arrow $f_1:x\to a$ and $f_2:x\to b$, there is unique arrow $h:x\to a\times b$ (and given unique $h$ there is unique pair of arrow) such that the following commutative diagram holds:

	\begin{equation*}
	% https://q.uiver.app/#q=WzAsNCxbMSwwLCJ4Il0sWzEsMSwiYVxcdGltZXMgYiJdLFswLDEsImEiXSxbMiwxLCJiIl0sWzEsMiwicF8xIl0sWzEsMywicF8yIiwyXSxbMCwyLCJmXzEiLDJdLFswLDMsImZfMiJdLFswLDEsImgiLDEseyJzdHlsZSI6eyJib2R5Ijp7Im5hbWUiOiJkYXNoZWQifX19XV0=
	\begin{tikzcd}
		& x \\
		a & {a\times b} & b
		\arrow["{p_1}", from=2-2, to=2-1]
		\arrow["{p_2}"', from=2-2, to=2-3]
		\arrow["{f_1}"', from=1-2, to=2-1]
		\arrow["{f_2}", from=1-2, to=2-3]
		\arrow["h"{description}, dashed, from=1-2, to=2-2]
	\end{tikzcd}
	\end{equation*}

	This also means that $f_1=p_1\circ h$ and $f_2= p_2\circ h$ i.e we factorize $(x, f_1, f_2)$ through $(a\times b, p_1, p_2)$.
\end{definition}

Now, we can recover the cartesian product from the categorical product on $\textbf{Set}$ category.

\begin{proposition}
	Categortical product between $A$ and $B$ in $\textbf{Set}$ is the Cartesian product $A\times B$, and Cartesian product $A\times B$ is the categorical product.
\end{proposition}


\begin{proof}
	Let's start with the first statement first. We can set the object $x$ with a terminal object $1$ (see the figure below). Recall that map from a terminal object selects an element of the target. 
	
	\begin{equation*}
	% https://q.uiver.app/#q=WzAsNCxbMSwwLCIxIl0sWzEsMSwiQVxcdGltZXMgQiJdLFswLDEsIkEiXSxbMiwxLCJCIl0sWzEsMiwicF8xIl0sWzEsMywicF8yIiwyXSxbMCwyLCJmXzEiLDJdLFswLDMsImZfMiJdLFswLDEsImgiLDEseyJzdHlsZSI6eyJib2R5Ijp7Im5hbWUiOiJkYXNoZWQifX19XV0=
	\begin{tikzcd}
		& 1 \\
		A & {A\times B} & B
		\arrow["{p_1}", from=2-2, to=2-1]
		\arrow["{p_2}"', from=2-2, to=2-3]
		\arrow["{f_1}"', from=1-2, to=2-1]
		\arrow["{f_2}", from=1-2, to=2-3]
		\arrow["h"{description}, dashed, from=1-2, to=2-2]
	\end{tikzcd}
	\end{equation*}
	
	Then, we can have the $f_1$ selects one object $a$ from $A$ and $f_2$ selects one object $b$ from $B$, then $h$, in order to enforce the uniqueness can be set to selecting a pair of $(a, b)$. Finally, by the commutativity, the map $p_1\circ h=p_1((a, b)):1\to A$ should select the same object as $f_1$ which is $a$. Thus $p_1$ is a projection of first element. The same can be shown with $p_2$. Thus, we have proven the first part.

	To show that cartesian product is a categorical product, we will have to construct $h$ given $(f_1,f_2)$ and vice versa with any set $X$. Let's consider such construction:

	\begin{itemize}
		\item Given the function $f_1:X\to A$ and $f_2:X\to B$, then $h:X\to A\times B$ can be set to be $h(x)=(f_1(x), f_2(x))$
		\item On the other hand, with a function $h$ one can use the commutativity to define both $f_1$ and $f_2$ i.e $f_1 = p_1\circ h$ and $f_2 = p_2\circ h$
	\end{itemize}

\end{proof}


We will also show the power of this kind of definition i.e unique arrow from any object to categorical product, by proving that the product over terminal object, we got:

\begin{proposition}
	Given object $a$ in category $\textbf{C}$ that has terminal object $1$, we can show that $a\times 1\cong a$
\end{proposition}


\begin{proof}
	We consider the following diagram:

	\begin{equation*}
	% https://q.uiver.app/#q=WzAsNCxbMSwxLCJhXFx0aW1lcyAxIl0sWzAsMSwiYSJdLFsyLDEsIjEiXSxbMSwwLCJhIl0sWzAsMiwicF8yIiwyXSxbMCwxLCJwXzEiXSxbMywxLCJcXG9wZXJhdG9ybmFtZXtpZH1fYSIsMl0sWzMsMiwiIV9hIl0sWzMsMCwiaCIsMV1d
	\begin{tikzcd}
		& a \\
		a & {a\times 1} & 1
		\arrow["{p_2}"', from=2-2, to=2-3]
		\arrow["{p_1}", from=2-2, to=2-1]
		\arrow["{\operatorname{id}_a}"', from=1-2, to=2-1]
		\arrow["{!_a}", from=1-2, to=2-3]
		\arrow["h"{description}, from=1-2, to=2-2]
	\end{tikzcd}
	\end{equation*}

	One can see clearly that $p_1\circ h=\operatorname{id}_a$, we are left to show that $h\circ p_1=\operatorname{id}_{a\times1}$. To do this, we have the following comparision diagram (every thing commutes):

	\begin{equation*}
	% https://q.uiver.app/#q=WzAsOSxbMSwyLCJhXFx0aW1lcyAxIl0sWzAsMiwiYSJdLFsyLDIsIjEiXSxbMSwxLCJhIl0sWzUsMiwiYVxcdGltZXMgMSJdLFs0LDIsImEiXSxbNiwyLCIxIl0sWzUsMCwiYVxcdGltZXMgMSJdLFsxLDAsImFcXHRpbWVzIDEiXSxbMCwyLCJwXzIiLDJdLFswLDEsInBfMSJdLFszLDEsIlxcb3BlcmF0b3JuYW1le2lkfV9hIiwxLHsiY29sb3VyIjpbMCwwLDc4XX0sWzAsMCw3OCwxXV0sWzMsMiwiIV9hIiwxLHsiY29sb3VyIjpbMCwwLDc4XX0sWzAsMCw3OCwxXV0sWzMsMCwiaCIsMV0sWzQsNSwicF8xIl0sWzQsNiwicF8yIiwyXSxbNyw1LCJwXzEiLDJdLFs3LDQsIlxcb3BlcmF0b3JuYW1le2lkfV97YVxcdGltZXMxfSIsMSx7InN0eWxlIjp7ImJvZHkiOnsibmFtZSI6ImRhc2hlZCJ9fX1dLFs3LDYsIiFfe2FcXHRpbWVzMX0iXSxbOCwzLCJwXzEiLDFdLFs4LDEsInBfMSIsMl0sWzgsMiwiIV97YVxcdGltZXMxfSJdXQ==
	\begin{tikzcd}
		& {a\times 1} &&&& {a\times 1} \\
		& a \\
		a & {a\times 1} & 1 && a & {a\times 1} & 1
		\arrow["{p_2}"', from=3-2, to=3-3]
		\arrow["{p_1}", from=3-2, to=3-1]
		\arrow["{\operatorname{id}_a}"{description}, color={rgb,255:red,199;green,199;blue,199}, from=2-2, to=3-1]
		\arrow["{!_a}"{description}, color={rgb,255:red,199;green,199;blue,199}, from=2-2, to=3-3]
		\arrow["h"{description}, from=2-2, to=3-2]
		\arrow["{p_1}", from=3-6, to=3-5]
		\arrow["{p_2}"', from=3-6, to=3-7]
		\arrow["{p_1}"', from=1-6, to=3-5]
		\arrow["{\operatorname{id}_{a\times1}}"{description}, dashed, from=1-6, to=3-6]
		\arrow["{!_{a\times1}}", from=1-6, to=3-7]
		\arrow["{p_1}"{description}, from=1-2, to=2-2]
		\arrow["{p_1}"', from=1-2, to=3-1]
		\arrow["{!_{a\times1}}", from=1-2, to=3-3]
	\end{tikzcd}
	\end{equation*}

	Please note that by definition of terminal object, $p_2=!_{a\times1}$ and it is unique (hence the $a\times1\to1$ on the right edge of LHS diagram is correct). Since the definition of product holds that there is a unique pair of arrows, we can see that $h\circ p_1=\operatorname{id}_{a\times1}$
\end{proof}


and we can also prove the commutativity of a categorical product. 

\begin{proposition}
	Given object $a$ and $b$ in category $\textbf{C}$, we can show that $a\times b\cong b\times a$
\end{proposition}


\begin{proof}
	We consider the following diagrams:
	
	\begin{equation*}
	% https://q.uiver.app/#q=WzAsOCxbMSwwLCJiXFx0aW1lcyBhIl0sWzEsMSwiYVxcdGltZXMgYiJdLFswLDEsImEiXSxbMiwxLCJiIl0sWzUsMCwiYVxcdGltZXMgYiJdLFs1LDEsImJcXHRpbWVzIGEiXSxbNCwxLCJhIl0sWzYsMSwiYiJdLFsxLDIsInBfMSJdLFsxLDMsInBfMiIsMl0sWzAsMiwicF8yJyIsMl0sWzAsMywicCdfMSJdLFswLDEsImgiLDEseyJzdHlsZSI6eyJib2R5Ijp7Im5hbWUiOiJkYXNoZWQifX19XSxbNSw3LCJwXzEnIiwyXSxbNSw2LCJwXzInIl0sWzQsNSwiaCciLDEseyJzdHlsZSI6eyJib2R5Ijp7Im5hbWUiOiJkYXNoZWQifX19XSxbNCw2LCJwXzEiLDJdLFs0LDcsInBfMiJdXQ==
	\begin{tikzcd}
		& {b\times a} &&&& {a\times b} \\
		a & {a\times b} & b && a & {b\times a} & b
		\arrow["{p_1}", from=2-2, to=2-1]
		\arrow["{p_2}"', from=2-2, to=2-3]
		\arrow["{p_2'}"', from=1-2, to=2-1]
		\arrow["{p'_1}", from=1-2, to=2-3]
		\arrow["h"{description}, dashed, from=1-2, to=2-2]
		\arrow["{p_1'}"', from=2-6, to=2-7]
		\arrow["{p_2'}", from=2-6, to=2-5]
		\arrow["{h'}"{description}, dashed, from=1-6, to=2-6]
		\arrow["{p_1}"', from=1-6, to=2-5]
		\arrow["{p_2}", from=1-6, to=2-7]
	\end{tikzcd}
	\end{equation*}

	We will claim that $h$ and $h'$ are inverse of each other i.e  $h'\circ h=\operatorname{id}_{b\times a}$ and $h\circ h'=\operatorname{id}_{a\times b}$, thus $h$ is isomorphism. To show this, we can stack diagram above up (to get the LHS version and it is clear that it commutes). 

	\begin{equation*}
	% https://q.uiver.app/#q=WzAsOSxbMiwwLCJiXFx0aW1lcyBhIl0sWzIsMSwiYVxcdGltZXMgYiJdLFswLDIsImEiXSxbNCwyLCJiIl0sWzIsMiwiYlxcdGltZXMgYSJdLFs3LDIsImJcXHRpbWVzIGEiXSxbNiwyLCJiIl0sWzgsMiwiYSJdLFs3LDAsImJcXHRpbWVzIGEiXSxbMSwyLCJwXzEiLDEseyJjb2xvdXIiOlswLDAsNzhdfSxbMCwwLDc4LDFdXSxbMSwzLCJwXzIiLDEseyJjb2xvdXIiOlswLDAsNzhdfSxbMCwwLDc4LDFdXSxbMCwyLCJwXzInIiwxXSxbMCwzLCJwJ18xIiwxXSxbMCwxLCJoIiwxLHsic3R5bGUiOnsiYm9keSI6eyJuYW1lIjoiZGFzaGVkIn19fV0sWzQsMiwicF8yJyIsMV0sWzQsMywicF8xJyIsMV0sWzEsNCwiaCciLDEseyJzdHlsZSI6eyJib2R5Ijp7Im5hbWUiOiJkYXNoZWQifX19XSxbOCw1LCJcXG9wZXJhdG9ybmFtZXtpZH1fe2JcXHRpbWVzIGF9IiwxXSxbOCw2LCJwXzInIiwyXSxbOCw3LCJwXzEnIl0sWzUsNywicF8xJyIsMl0sWzUsNiwicF8yJyJdXQ==
	\begin{tikzcd}
		&& {b\times a} &&&&& {b\times a} \\
		&& {a\times b} \\
		a && {b\times a} && b && b & {b\times a} & a
		\arrow["{p_1}"{description}, color={rgb,255:red,199;green,199;blue,199}, from=2-3, to=3-1]
		\arrow["{p_2}"{description}, color={rgb,255:red,199;green,199;blue,199}, from=2-3, to=3-5]
		\arrow["{p_2'}"{description}, from=1-3, to=3-1]
		\arrow["{p'_1}"{description}, from=1-3, to=3-5]
		\arrow["h"{description}, dashed, from=1-3, to=2-3]
		\arrow["{p_2'}"{description}, from=3-3, to=3-1]
		\arrow["{p_1'}"{description}, from=3-3, to=3-5]
		\arrow["{h'}"{description}, dashed, from=2-3, to=3-3]
		\arrow["{\operatorname{id}_{b\times a}}"{description}, from=1-8, to=3-8]
		\arrow["{p_2'}"', from=1-8, to=3-7]
		\arrow["{p_1'}", from=1-8, to=3-9]
		\arrow["{p_1'}"', from=3-8, to=3-9]
		\arrow["{p_2'}", from=3-8, to=3-7]
	\end{tikzcd}
	\end{equation*}

	On the RHS, we have the obvious commutative diagram, but please note that by definition of categorical product (the universal construction), given a pair $(p_1',p_2')$, we have the \textit{unique} correspondance to the map $\operatorname{id}_{b\times a}$. Thus, $h'\circ h=\operatorname{id}_{b\times a}$ by the uniquenes. The proof that $h\circ h'=\operatorname{id}_{a\times b}$ follows in similar manners.

\end{proof}


Furthermore, we can show that categorical product is associative.

\begin{proposition}
	Given object $a,b$ and $c$ in category $\textbf{C}$, we can show that $a\times(b\times c)\cong(a\times b)\times c$
\end{proposition}


\begin{proof}
	Let's start by defining the (canonical and unique) map $\alpha:a\times(b\times c)\to(a\times b)\times c$. This can be done by the following commutative diagram and univeral properties
	\begin{equation*}
	% https://q.uiver.app/#q=WzAsOCxbMSwwLCJhXFx0aW1lcyhiXFx0aW1lcyBjKSJdLFsxLDEsIihhXFx0aW1lcyBiKVxcdGltZXMgYyJdLFswLDIsImFcXHRpbWVzIGIiXSxbMiwyLCJjIl0sWzUsMCwiYVxcdGltZXMoYlxcdGltZXMgYykiXSxbNSwxLCJhXFx0aW1lcyBiIl0sWzQsMiwiYSJdLFs2LDIsImIiXSxbMCwxLCJcXGFscGhhIiwwLHsic3R5bGUiOnsiYm9keSI6eyJuYW1lIjoiZGFzaGVkIn19fV0sWzAsMiwiaCIsMix7ImN1cnZlIjoyfV0sWzAsMywicF8xXntiY31cXGNpcmMgcF57YSxiY31fMSIsMCx7ImN1cnZlIjotMn1dLFsxLDIsInBfMF57YWIsY30iXSxbMSwzLCJwXzFee2FiLGN9IiwyXSxbNCw1LCJoIiwwLHsic3R5bGUiOnsiYm9keSI6eyJuYW1lIjoiZGFzaGVkIn19fV0sWzQsNiwicF57YSxiY31fMCIsMix7ImN1cnZlIjoyfV0sWzQsNywicF57YmN9XzBcXGNpcmMgcF57YSxiY31fMSIsMCx7ImN1cnZlIjotMn1dLFs1LDYsInBee2FifV8wIl0sWzUsNywicF57YWJ9XzEiLDJdXQ==
	\begin{tikzcd}
		& {a\times(b\times c)} &&&& {a\times(b\times c)} \\
		& {(a\times b)\times c} &&&& {a\times b} \\
		{a\times b} && c && a && b
		\arrow["\alpha", dashed, from=1-2, to=2-2]
		\arrow["h"', curve={height=12pt}, from=1-2, to=3-1]
		\arrow["{p_1^{bc}\circ p^{a,bc}_1}", curve={height=-12pt}, from=1-2, to=3-3]
		\arrow["{p_0^{ab,c}}", from=2-2, to=3-1]
		\arrow["{p_1^{ab,c}}"', from=2-2, to=3-3]
		\arrow["h", dashed, from=1-6, to=2-6]
		\arrow["{p^{a,bc}_0}"', curve={height=12pt}, from=1-6, to=3-5]
		\arrow["{p^{bc}_0\circ p^{a,bc}_1}", curve={height=-12pt}, from=1-6, to=3-7]
		\arrow["{p^{ab}_0}", from=2-6, to=3-5]
		\arrow["{p^{ab}_1}"', from=2-6, to=3-7]
	\end{tikzcd}
	\end{equation*}
	where $h:a\times(b\times c)\to a\times b$ is defined on the RHS. Similarly, the map $\beta:(a\times b)\times c\to a\times(b\times c)$, we have, the following commutative diagrams:
	\begin{equation*}
	% https://q.uiver.app/#q=WzAsOCxbMSwwLCIoYVxcdGltZXMgYilcXHRpbWVzIGMiXSxbMSwxLCJhXFx0aW1lcyhiXFx0aW1lcyBjKSJdLFsyLDIsImJcXHRpbWVzIGMiXSxbMCwyLCJhIl0sWzUsMCwiKGFcXHRpbWVzIGIpXFx0aW1lcyBjIl0sWzUsMSwiYlxcdGltZXMgYyJdLFs0LDIsImIiXSxbNiwyLCJjIl0sWzEsMywicF8wXnthLGJjfSJdLFsxLDIsInBfMV57YSxiY30iLDJdLFswLDMsInBee2FifV8wXFxjaXJjIHBee2FiLGN9XzAiLDIseyJjdXJ2ZSI6Mn1dLFswLDEsIlxcYmV0YSIsMCx7InN0eWxlIjp7ImJvZHkiOnsibmFtZSI6ImRhc2hlZCJ9fX1dLFswLDIsImciLDAseyJjdXJ2ZSI6LTJ9XSxbNCw2LCJwXnthYn1fMVxcY2lyYyBwXnthYixjfV8wIiwyLHsiY3VydmUiOjJ9XSxbNCw3LCJwXnthYixjfV8xIiwwLHsiY3VydmUiOi0yfV0sWzUsNiwicF57YmN9XzAiXSxbNSw3LCJwXntiY31fMSIsMl0sWzQsNSwiZyIsMCx7InN0eWxlIjp7ImJvZHkiOnsibmFtZSI6ImRhc2hlZCJ9fX1dXQ==
	\begin{tikzcd}
		& {(a\times b)\times c} &&&& {(a\times b)\times c} \\
		& {a\times(b\times c)} &&&& {b\times c} \\
		a && {b\times c} && b && c
		\arrow["{p_0^{a,bc}}", from=2-2, to=3-1]
		\arrow["{p_1^{a,bc}}"', from=2-2, to=3-3]
		\arrow["{p^{ab}_0\circ p^{ab,c}_0}"', curve={height=12pt}, from=1-2, to=3-1]
		\arrow["\beta", dashed, from=1-2, to=2-2]
		\arrow["g", curve={height=-12pt}, from=1-2, to=3-3]
		\arrow["{p^{ab}_1\circ p^{ab,c}_0}"', curve={height=12pt}, from=1-6, to=3-5]
		\arrow["{p^{ab,c}_1}", curve={height=-12pt}, from=1-6, to=3-7]
		\arrow["{p^{bc}_0}", from=2-6, to=3-5]
		\arrow["{p^{bc}_1}"', from=2-6, to=3-7]
		\arrow["g", dashed, from=1-6, to=2-6]
	\end{tikzcd}
	\end{equation*}
	We note that $p_0^{bc}\circ g=p^{ab}_1\circ p^{ab,c}_0$ (used on LHS) and $p^{ab,c}_1 = p_1^{bc}\circ g$ (used on RHS):
	$$
	\begin{aligned}
		p_0^{bc}\circ g \circ \alpha &= p^{ab}_1\circ p^{ab,c}_0 \circ \alpha \\
		&= p^{ab}_1\circ h = p^{bc}_0\circ p^{a,bc}_1
	\end{aligned}
	\qquad \quad
	\begin{aligned}
		p_1^{bc}\circ g \circ\alpha  &= p^{ab,c}_1 \circ \alpha  \\
		  &= p_1^{bc}\circ p^{a,bc}_1\\
	\end{aligned}
	$$
	together with folowing commutative diagram (we have proved by commutative in above equation):
	\begin{equation*}
	% https://q.uiver.app/#q=WzAsNCxbMSwwLCJhXFx0aW1lcyhiXFx0aW1lcyBjKSJdLFsxLDEsImJcXHRpbWVzIGMiXSxbMCwyLCJiIl0sWzIsMiwiYyJdLFsxLDIsInBee2JjfV8wIl0sWzEsMywicF57YmN9XzEiLDJdLFswLDMsIiBwXzFee2JjfVxcY2lyYyBwXnthLGJjfV8xIiwwLHsiY3VydmUiOi0yfV0sWzAsMiwicF57YmN9XzBcXGNpcmMgcF57YSxiY31fMSIsMix7ImN1cnZlIjoyfV0sWzAsMSwiZ1xcY2lyY1xcYWxwaGEiLDAseyJjdXJ2ZSI6LTEsInN0eWxlIjp7ImJvZHkiOnsibmFtZSI6ImRhc2hlZCJ9fX1dLFswLDEsInBee2EsYmN9XzEiLDIseyJjdXJ2ZSI6MSwic3R5bGUiOnsiYm9keSI6eyJuYW1lIjoiZGFzaGVkIn19fV1d
	\begin{tikzcd}
		& {a\times(b\times c)} \\
		& {b\times c} \\
		b && c
		\arrow["{p^{bc}_0}", from=2-2, to=3-1]
		\arrow["{p^{bc}_1}"', from=2-2, to=3-3]
		\arrow["{ p_1^{bc}\circ p^{a,bc}_1}", curve={height=-12pt}, from=1-2, to=3-3]
		\arrow["{p^{bc}_0\circ p^{a,bc}_1}"', curve={height=12pt}, from=1-2, to=3-1]
		\arrow["g\circ\alpha", curve={height=-6pt}, dashed, from=1-2, to=2-2]
		\arrow["{p^{a,bc}_1}"', curve={height=6pt}, dashed, from=1-2, to=2-2]
	\end{tikzcd}\qquad \quad
	% https://q.uiver.app/#q=WzAsNSxbMSwwLCJhXFx0aW1lcyhiXFx0aW1lcyBjKSJdLFsxLDEsIihhXFx0aW1lcyBiKVxcdGltZXMgYyJdLFsxLDIsImFcXHRpbWVzKGJcXHRpbWVzIGMpIl0sWzIsMiwiYlxcdGltZXMgYyJdLFswLDIsImEiXSxbMCwxLCJcXGFscGhhIiwwLHsic3R5bGUiOnsiYm9keSI6eyJuYW1lIjoiZGFzaGVkIn19fV0sWzEsMiwiXFxiZXRhIiwwLHsic3R5bGUiOnsiYm9keSI6eyJuYW1lIjoiZGFzaGVkIn19fV0sWzIsMywicF8xXnthLGJjfSIsMl0sWzIsNCwicF57YSxiY31fMCJdLFswLDQsInBee2EsYmN9XzAiLDIseyJjdXJ2ZSI6Mn1dLFswLDMsInBee2EsYmN9XzEiLDAseyJjdXJ2ZSI6LTJ9XSxbMSw0LCJwXnthYn1fMFxcY2lyYyBwXnthYixjfV8wIiwxXSxbMSwzLCJnIiwxLHsic3R5bGUiOnsiYm9keSI6eyJuYW1lIjoiZGFzaGVkIn19fV1d
	\begin{tikzcd}
		& {a\times(b\times c)} \\
		& {(a\times b)\times c} \\
		a & {a\times(b\times c)} & {b\times c}
		\arrow["\alpha", dashed, from=1-2, to=2-2]
		\arrow["\beta", dashed, from=2-2, to=3-2]
		\arrow["{p_1^{a,bc}}"', from=3-2, to=3-3]
		\arrow["{p^{a,bc}_0}", from=3-2, to=3-1]
		\arrow["{p^{a,bc}_0}"', curve={height=12pt}, from=1-2, to=3-1]
		\arrow["{p^{a,bc}_1}", curve={height=-12pt}, from=1-2, to=3-3]
		\arrow["{p^{ab}_0\circ p^{ab,c}_0}"{description}, from=2-2, to=3-1]
		\arrow["g"{description}, dashed, from=2-2, to=3-3]
	\end{tikzcd}
	\end{equation*}
	By the universal property of $b\times c$, we have that $g \circ \alpha=p^{a,bc}_1$. This mean that the right side of the diagram on RHS commutes. On the left side of RHS triangle, the triangle commutes because $p^{ab}_0\circ p^{ab,c}_0\circ\alpha = p^{ab}_0\circ h = p^{a,bc}_0$ and recall that $p^{ab}_0\circ p^{ab,c}_0 = p^{a,bc}_0\circ\beta$. Thus, by univeral properties $\beta\circ\alpha=\operatorname{id}_{a\times(b\times c)}$. Finally, The other cases of $\alpha\circ\beta=\operatorname{id}_{(a\times b)\times c}$ can be proved in simialr manners.
\end{proof}


We can see here that the categorical product actually \textit{looks like} a product on the natural number. Finally, we can consider the \textit{functoriality} of the product, in which the ``lifting'' of the morphism can be performed. (we will discuss the actual notion of functor in later section)

\begin{definition}{\textbf{(Parallel Application of Product)}}
	Given two morphisms $f:a\to a'$ and $g:b\to b'$, then we can construct the map $f\times g :a\times b\to a'\times b'$ based on these 2 function as follows:
	\begin{equation*}
	% https://q.uiver.app/#q=WzAsNixbMSwwLCJhXFx0aW1lcyBiIl0sWzEsMSwiYSdcXHRpbWVzIGInIl0sWzAsMiwiYSciXSxbMiwyLCJiJyJdLFsyLDEsImIiXSxbMCwxLCJhIl0sWzAsMSwiZlxcdGltZXMgZyIsMSx7InN0eWxlIjp7ImJvZHkiOnsibmFtZSI6ImRhc2hlZCJ9fX1dLFsxLDIsInBfMSciLDJdLFsxLDMsInBfMiciXSxbNSwyLCJmIiwyXSxbNCwzLCJnIl0sWzAsNSwicF8xIiwyXSxbMCw0LCJwXzIiXV0=
	\begin{tikzcd}
		& {a\times b} \\
		a & {a'\times b'} & b \\
		{a'} && {b'}
		\arrow["{f\times g}"{description}, dashed, from=1-2, to=2-2]
		\arrow["{p_1'}"', from=2-2, to=3-1]
		\arrow["{p_2'}", from=2-2, to=3-3]
		\arrow["f"', from=2-1, to=3-1]
		\arrow["g", from=2-3, to=3-3]
		\arrow["{p_1}"', from=1-2, to=2-1]
		\arrow["{p_2}", from=1-2, to=2-3]
	\end{tikzcd}
	\end{equation*}
\end{definition}

Given the notion of product one can see the similarity between it and the way we define initial/terminal object above, which involves finding ``the'' object.

\begin{remark}

We would like to point out the common patterns that is starting to arise, whereby we define the categorical object by assigning a particular patterns ($a\leftarrow x\to b$ in product case) and find the best object that represents it:
\begin{itemize}
	\item The \textit{best} one is selected based on (if it exists) the fact that all candiidate objects (of a particular pattern) have a unique map to it.
	\item In the terminal object case, there isn't a pattern, just a simple object.
\end{itemize}

This is called univeral construction. Please note that this can be formalized, but we will have to stay until very last part of this notes. Finally, due to the way it is constructed, every constructed objects will have the same arrow coming in and out, thus all objects that satisfies the construction will be isomorphic with each other.
\end{remark}

Now, we have been more familar with the categorical product and a rought way to construct a new objects. Let's explore 2 more.

\subsection{Currying/Internal-Hom}

This would be quite natural for programmer, where we will introduce the notion of currying, which is quite ubiquitous in functional programming languages. 

\begin{definition}{\textbf{(Currying)}}
Given the 2 argument function $f:X\times Y\to Z$, one can perform the currying, which will givs us $f':X\to (Y\to Z)$ and uncurrying is to reverse it. 
\end{definition}

Let's see how we can use it.

\begin{remark}
	Suppose we have $x\in X$ and $y\in Y$, then we can see that $f$ will map both of them to $z\in Z$ i.e $f(x,y)\mapsto z$, on the other hand, $f'$ can also do the same i.e $f'(x)(y)\mapsto z$ but it can also return a function, upon a partial application i.e $f'(x):y\to z$ is a function such that $y\mapsto f(x, y)$.

	One can write the return of function $f'$ to be $X\to Z^Y$ in which $Z^Y$ is a special kind of object that represent the function from $Y\to Z$ i.e $\operatorname{Hom}_\textbf{Set}(X, Z^Y)\cong\operatorname{Hom}_\textbf{Set}((X,Y),Z)$ (we will see this later in the adjunction), and this is looking like we are taking a logarithm, hence the name exponential object. 
	
	Furthermore, given the object $B^A$, one might define an evaluation function $\varepsilon_{AB}:B^A\times A\to B$, such that:


	\begin{equation*}
	% https://q.uiver.app/#q=WzAsMyxbMCwwLCIxXFx0aW1lczFcXGNvbmcxIl0sWzAsMSwiQl5BXFx0aW1lcyBBIl0sWzEsMSwiQiJdLFswLDEsIihmLCBhKSIsMl0sWzAsMiwiYiJdLFsxLDIsIlxcdmFyZXBzaWxvbl97QUJ9IiwyXV0=
	\begin{tikzcd}
		1\times1\cong1 \\
		{B^A\times A} & B
		\arrow["{(f, a)}"', from=1-1, to=2-1]
		\arrow["b", from=1-1, to=2-2]
		\arrow["{\varepsilon_{AB}}"', from=2-1, to=2-2]
	\end{tikzcd}
	\end{equation*}

	That is we select a function with an element and we evaluation the function to get what we need i.e $f(a)=b$. This is what is called elimination rule.
\end{remark}

With the elimination rule ready, we can now define the universal construction. Again, we have a pattern of 

\begin{definition}{\textbf{(Exponential Object)}} 
	Given the object $a$,$b$, we have the pattern of $\_\times a\to b$. The exponential object is an object $b^a$ such that for any object $c$, with function $f:c\times a\to b$ the following diagram commutes:

	\begin{equation*}
	% https://q.uiver.app/#q=WzAsMyxbMCwxLCJiXmFcXHRpbWVzIGEiXSxbMSwxLCJiIl0sWzAsMCwiY1xcdGltZXMgYSJdLFsyLDAsImhcXHRpbWVzIFxcb3BlcmF0b3JuYW1le2lkfV9hIiwyLHsic3R5bGUiOnsiYm9keSI6eyJuYW1lIjoiZGFzaGVkIn19fV0sWzAsMSwiXFx2YXJlcHNpbG9uIiwyXSxbMiwxLCJmIl1d
	\begin{tikzcd}
		{c\times a} \\
		{b^a\times a} & b
		\arrow["{h\times \operatorname{id}_a}"', dashed, from=1-1, to=2-1]
		\arrow["\varepsilon"', from=2-1, to=2-2]
		\arrow["f", from=1-1, to=2-2]
	\end{tikzcd}
	\end{equation*}

	and, finally, there is an unique $h$ for every $f$ and vice versa (note that the function that represent the pattern for exponential object is the evaluation function $\varepsilon$).
\end{definition}

Now, the currying can be seen in the light of exponential object as:

\begin{remark}
	Consider the following diagram:
	\begin{equation*}
	% https://q.uiver.app/#q=WzAsMyxbMCwxLCJiXmFcXHRpbWVzIGEiXSxbMSwxLCJiIl0sWzAsMCwiMVxcdGltZXMgYVxcY29uZyBhIl0sWzAsMSwiXFx2YXJlcHNpbG9uX3thYn0iLDJdLFsyLDEsImYiXSxbMiwwLCJoXFx0aW1lcyBcXG9wZXJhdG9ybmFtZXtpZH1fYSIsMix7InN0eWxlIjp7ImJvZHkiOnsibmFtZSI6ImRhc2hlZCJ9fX1dXQ==
	\begin{tikzcd}
		{1\times a\cong a} \\
		{b^a\times a} & b
		\arrow["{\varepsilon_{ab}}"', from=2-1, to=2-2]
		\arrow["f", from=1-1, to=2-2]
		\arrow["{h\times \operatorname{id}_a}"', dashed, from=1-1, to=2-1]
	\end{tikzcd}
	\end{equation*}
	We can clearly see that $h:1\to B^A$ can be seen as selecting the element in $B^A$ that is uniquely identified with a function $f:a\cong 1\times a\to b$, that is $h$ is the \textbf{curried} version of $f$.
\end{remark}

\subsection{Pullback}

We will move into very important universal construction, which is the \textit{pullback}. The motivation of this is, again, coming from $\textbf{Set}$, where by, we will consider adding the condition/filtering to the set i.e given functions $f:A\to C$ and $g:B\to C$, we want to find the following set:

$$
\big\{ (a, b) \in A\times B : f(a) = g(b) \big\}
$$

The pattern that we are using are quite clear. With the usual construction of objects, we have the following definition.

\begin{definition}{\textbf{(Pullback)}}
	Given a category $\textbf{C}$, a pullback object is an object $p$ together with map $p_1:p\to a$ and $p_2:p\to b$, such that given an object $c$ and morphisms $f:a\to c$ and $g:b\to c$, and some object $q$ and morphism $q_1:q\to a$ and $q_2:q\to b$. The following diagram commutes:

	\begin{equation*}
	% https://q.uiver.app/#q=WzAsNSxbMCwwLCJxIl0sWzEsMSwicCJdLFsxLDIsImIiXSxbMiwyLCJjIl0sWzIsMSwiYSJdLFswLDEsImgiLDAseyJzdHlsZSI6eyJib2R5Ijp7Im5hbWUiOiJkYXNoZWQifX19XSxbMSw0LCJwXzEiXSxbMiwzLCJnIiwyXSxbNCwzLCJmIl0sWzAsMiwicV8yIiwyLHsiY3VydmUiOjJ9XSxbMCw0LCJxXzEiLDAseyJjdXJ2ZSI6LTJ9XSxbMSwyLCJwXzIiLDJdLFsxLDMsIiIsMCx7InN0eWxlIjp7Im5hbWUiOiJjb3JuZXIifX1dXQ==
	\begin{tikzcd}
		q \\
		& p & a \\
		& b & c
		\arrow["h", dashed, from=1-1, to=2-2]
		\arrow["{p_1}", from=2-2, to=2-3]
		\arrow["g"', from=3-2, to=3-3]
		\arrow["f", from=2-3, to=3-3]
		\arrow["{q_2}"', curve={height=12pt}, from=1-1, to=3-2]
		\arrow["{q_1}", curve={height=-12pt}, from=1-1, to=2-3]
		\arrow["{p_2}"', from=2-2, to=3-2]
		\arrow["\lrcorner"{anchor=center, pos=0.125}, draw=none, from=2-2, to=3-3]
	\end{tikzcd}
	\end{equation*}

	With the function $h:x\to p$ is unique to the the morphisms $(q_1,q_2)$, and vice versa. Furthermore, we note that we also requires a morphism from $p\to c$ but due to the commutativity condition, we can write it implicitly. The $\lrcorner$ denotes that the object $p$ is the pullback. 
\end{definition}

Again, as we have dealt in the product case, we can perform similar proof to show that the pullback is actually a conditioned product:

\begin{itemize}
	\item We can replace $q$ above with singleton set/terminal object $1$, and the conditioning of the elements follows from the commutativity, since it is required that $f\circ a=g\circ b$ where $a:1\to A$ and $b\to B$, which are the object selector. And, we can set $p_1$ and $p_2$ to be  the projectors defined in the same way as the product (thus allowing for factorization of $q_1=p_1\circ h$ and $q_2=p_2\circ h$)
	\item On the other hand, the other direction is simply proved by the observation that the condition is the set is the same as commutativity i.e $f(a)=f\circ a$ and $g(b)=g\circ b$
\end{itemize}

The intersting fact about pullback is that it can stands in for multiple kinds of constructions that we are familar in $\textbf{Set}$, notably: intersection, product, inverse, and subset classifier. 

\begin{definition}{\textbf{(Intersection)}}
	Let's start with a simple case, given a set $A$ and $B$, we can define its intersection to be $A\cap B\cong\{(a, b)\in A\times B: a = b\}$, therefore, one can define the intersection as:

	\begin{equation*}
	% https://q.uiver.app/#q=WzAsNCxbMCwwLCJBXFxjYXAgQiJdLFsxLDEsIkFcXGN1cCBCIl0sWzAsMSwiQSJdLFsxLDAsIkIiXSxbMCwyXSxbMCwzXSxbMywxLCJpX0IiXSxbMiwxLCJpX0EiLDJdLFswLDEsIiIsMSx7InN0eWxlIjp7Im5hbWUiOiJjb3JuZXIifX1dXQ==
	\begin{tikzcd}
		{A\cap B} & B \\
		A & {A\cup B}
		\arrow[from=1-1, to=2-1]
		\arrow[from=1-1, to=1-2]
		\arrow["{i_B}", from=1-2, to=2-2]
		\arrow["{i_A}"', from=2-1, to=2-2]
		\arrow["\lrcorner"{anchor=center, pos=0.125}, draw=none, from=1-1, to=2-2]
	\end{tikzcd}
	\end{equation*}
	
	where the morphism $i_A:A\to A\cup B$ and $i_B:B\to A\cup B$ are the inclusion map to the union.
\end{definition}

\begin{remark}
	We can recover the product from pullback by consider the following pullback diagram, in which we replace object $c$ with terminal object $1$:

	\begin{equation*}
	% https://q.uiver.app/#q=WzAsNSxbMSwxLCJhXFx0aW1lcyBiIl0sWzEsMiwiYiJdLFsyLDIsIjEiXSxbMiwxLCJhIl0sWzAsMCwicSJdLFswLDMsInBfMSJdLFswLDEsInBfMiIsMl0sWzEsMiwiIV9iIiwyXSxbMywyLCIhX2EiXSxbNCwwLCJoIiwwLHsic3R5bGUiOnsiYm9keSI6eyJuYW1lIjoiZGFzaGVkIn19fV0sWzQsMSwicV8yIiwyLHsiY3VydmUiOjJ9XSxbNCwzLCJxXzEiLDAseyJjdXJ2ZSI6LTJ9XSxbMCwyLCIiLDAseyJzdHlsZSI6eyJuYW1lIjoiY29ybmVyIn19XV0=
	\begin{tikzcd}
		q \\
		& {a\times b} & a \\
		& b & 1
		\arrow["{p_1}", from=2-2, to=2-3]
		\arrow["{p_2}"', from=2-2, to=3-2]
		\arrow["{!_b}"', from=3-2, to=3-3]
		\arrow["{!_a}", from=2-3, to=3-3]
		\arrow["h", dashed, from=1-1, to=2-2]
		\arrow["{q_2}"', curve={height=12pt}, from=1-1, to=3-2]
		\arrow["{q_1}", curve={height=-12pt}, from=1-1, to=2-3]
		\arrow["\lrcorner"{anchor=center, pos=0.125}, draw=none, from=2-2, to=3-3]
	\end{tikzcd}
	\end{equation*}
	
	Note that by definition, there is a unique morphism to a terminal object, therefore, $!_b\circ q_2=!_{a\times b} = !_a \circ q_1$ for any morphism $q$ and map $q_1:q\to a$ and $q_2:q\to b$ i.e any pair $(q,q_1,q_2)$ can be used to define $h$ without a restriction of commutativity.
\end{remark}

We can see that if we pick the right morphism or objects $c$, we will be able to derive many others objects. Let's do the inverse, first:

\begin{definition}{\textbf{(Inverse)}}
	Suppose we are given the function $f:B\to C$, we can usually, define the inverse as of $b\in B$ as $f^{-1}(c)=\{b\in B: f(b) = c\}$. With this definition and formulation of pullback, we can see that (as we select object $c$ by picking the map $1\to C$, and note that $B\times 1\cong B$):

	\begin{equation*}
	% https://q.uiver.app/#q=WzAsNCxbMCwwLCJmXnstMX0oYykiXSxbMCwxLCJCIl0sWzEsMSwiQyJdLFsxLDAsIjEiXSxbMCwzXSxbMCwxLCJqIiwyXSxbMSwyLCJmIiwyXSxbMywyLCJjIl0sWzAsMiwiIiwxLHsic3R5bGUiOnsibmFtZSI6ImNvcm5lciJ9fV1d
	\begin{tikzcd}
		{f^{-1}(c)} & 1 \\
		B & C
		\arrow[from=1-1, to=1-2]
		\arrow["j"', from=1-1, to=2-1]
		\arrow["f"', from=2-1, to=2-2]
		\arrow["c", from=1-2, to=2-2]
		\arrow["\lrcorner"{anchor=center, pos=0.125}, draw=none, from=1-1, to=2-2]
	\end{tikzcd}
	\end{equation*}

	Note that this doesn't have to set, in order to define the notion of inverse (but not all categories will have this construction).

\end{definition}

We can see here that the notion of ``subset'' of $B$ arises when we set object $a$ to be the terminal object, and the function $j:f^{-1}(c)\hookrightarrow B$ can be seen as injective function as $f^{-1}(c)\subseteq B$. Finally, the condition itself ($f(a)=g(b)$), can be generalized to any kind of ``boolean'' condition by the following construction.


\begin{definition}{\textbf{(Subset Classifier)}}
	Suppose we are given an set of 2 elements $\Omega=\{0,1\}$, and a set $X$, one can define a characteristic function $\chi:X\to\Omega$ that assigns either $0$ or $1$ to the element of $X$ (i.e true or false value). Then we can have $A=\{ x\in X: \chi(x) \}$, by the following pullback
	
	\begin{equation*}
	% https://q.uiver.app/#q=WzAsNCxbMCwwLCJBIl0sWzAsMSwiWCJdLFsxLDEsIlxcT21lZ2EiXSxbMSwwLCIxIl0sWzAsM10sWzAsMSwiaiIsMl0sWzEsMiwiXFxjaGkiLDJdLFszLDIsInQiXSxbMCwyLCIiLDEseyJzdHlsZSI6eyJuYW1lIjoiY29ybmVyIn19XV0=
	\begin{tikzcd}
		A & 1 \\
		X & \Omega
		\arrow[from=1-1, to=1-2]
		\arrow["j"', from=1-1, to=2-1]
		\arrow["\chi"', from=2-1, to=2-2]
		\arrow["t", from=1-2, to=2-2]
		\arrow["\lrcorner"{anchor=center, pos=0.125}, draw=none, from=1-1, to=2-2]
	\end{tikzcd}
	\end{equation*}
	
	Note that set the condition on the right part is a short hand for $\chi(x)=1$, so we can have $t:X\to \Omega$ being a function that selects an element $1$ ($t$ stands for true). 
\end{definition}

With this construction above, the subject can be defined in a slightly round about manners, in which we have $\chi(x)=1$ if $x\in A$ and $\chi(x)=0$ if $x\not\in A$. 

\subsection{Natural Number}

Finally, we will consider a construction of natural number (and various recursive objects). This part is a bit leaning toward programming. Normally, a natural number $\mathbb{N}$ can be easily defined recursively (inductively) as $(0, s:\mathbb{N}\to \mathbb{N})$, where the successor $s(n)=n+1$. Let's make it in categorical language

\begin{remark}
	We can define the number/element $0$ to be the map from a terminal object $0:1\to N$, while the successor map $s$ is $N\to N$. We can clearly see that number $1$ can be defined as $s\circ 0$ and $2$ can be defined a $s\circ s\circ 0$ and so on.
\end{remark}

With this, we have establishing the pattern that we are going to work on, whereby we have initialization i.e $\operatorname{init}$ in our case and $\operatorname{step}$ to move the object beyond. Now, we can have the natural number to be the one being universally constructed, as:

% \begin{definition}{\textbf{(Natural Number System \cite{leinster2014rethinking})}}
\begin{definition}{\textbf{(Natural Number System)}}
	Given a category $\textbf{C}$, a natural number system consists of object $N$ and morphisms $z:1\to N$ and $s:N\to N$, such that for any object $a$ with morphisms $\operatorname{init}:0\to a$ and $\operatorname{step}:a\to a$, there is a unique arrow $h$, such that the following diagram commutes:

	\begin{equation*}
	% https://q.uiver.app/#q=WzAsNSxbMCwwLCIxIl0sWzEsMCwiTiJdLFsyLDAsIk4iXSxbMSwxLCJhIl0sWzIsMSwiYSJdLFswLDEsInoiXSxbMSwyLCJzIl0sWzAsMywiXFxvcGVyYXRvcm5hbWV7aW5pdH0iLDJdLFszLDQsIlxcb3BlcmF0b3JuYW1le3N0ZXB9IiwyXSxbMSwzLCJoIiwwLHsic3R5bGUiOnsiYm9keSI6eyJuYW1lIjoiZGFzaGVkIn19fV0sWzIsNCwiaCIsMCx7InN0eWxlIjp7ImJvZHkiOnsibmFtZSI6ImRhc2hlZCJ9fX1dXQ==
	\begin{tikzcd}
		1 & N & N \\
		& a & a
		\arrow["z", from=1-1, to=1-2]
		\arrow["s", from=1-2, to=1-3]
		\arrow["{\operatorname{init}}"', from=1-1, to=2-2]
		\arrow["{\operatorname{step}}"', from=2-2, to=2-3]
		\arrow["h", dashed, from=1-2, to=2-2]
		\arrow["h", dashed, from=1-3, to=2-3]
	\end{tikzcd}
	\end{equation*}
	We note that by the commutativity, we got the condition: $\operatorname{step}\circ h=h\circ s$ and $\operatorname{init}=h\circ z$, this is akin to the induction principle, esepscially the first condition.
\end{definition}


\section{More Category Theory/Functors}

Now after we get a hang of the basic concepts of category theory via concrete examples, let's now consider a more abstract notions. Let's start with 2 ways on how we create the new kind of category from the old one:

\subsection{New from Old Category}

\begin{definition}{\textbf{(Opposite Category)}}
    Given a category $\textbf{C}$, we can define the opposite category $\textbf{C}^\text{op}$ with the following components:
    \begin{itemize}
        \item \textit{Object:} We keep all the objects of $\textbf{C}^\text{op}$ to be the same as $\textbf{C}$
        \item \textit{Morphism:} Given the morphism $f:X\rightarrow Y$ of the category $\textbf{C}$, there is an arrow $f^\text{op}:Y\rightarrow X$ for the category $\textbf{C}^\text{op}$
        \item \textit{Identity Morphism:} This stays the same. 
        \item \textit{Composition of Morphism:} Given the morphism $f:X\rightarrow Y$ and $g:Y\rightarrow Z$ of the category $\textbf{C}$, we thus have the morphism $f^\text{op}:Y\rightarrow X$ and $g^\text{op}:Z\rightarrow Y$ in $\textbf{C}^\text{op}$ with the following composition: $$\big(g\circ f\big)^\text{op} = f^\text{op}\circ g^\text{op}$$
    \end{itemize}
\end{definition}

\begin{definition}{\textbf{(Product Category)}}
    Given a category $\textbf{C}$ and $\textbf{D}$, we can define the product category $\textbf{C}\times\textbf{D}$ with the following components:
    \begin{itemize}
        \item \textit{Object:}The product category has objects being a pair $(X, X')$ where $X$ is an object of $\textbf{C}$ and $X'$ is an object of $\textbf{D}$
        \item \textit{Morphism:} Given morphism $f:A\rightarrow B$ of category $\textbf{C}$ and $f':X'\rightarrow Y'$ of category $\textbf{D}$, we can define the morphism to be a pair: $(f,f'):(X, X')\rightarrow (Y, Y')$, which applied element wise.
        \item \textit{Identity Morphism:} We have for object $(X,X')$, the identity morphism is defined as $\operatorname{id}_{(X,X')} = (\operatorname{id}_X,\operatorname{id}_{X'})$
        \item \textit{Composition of Morphism:} Given the morphism $(f,f'):(X,X')\rightarrow(Y,Y')$ and $(g,g'):(Y,Y')\rightarrow(Z,Z')$, their composition is defined as: $$(f,f')\circ(g,g') = (f\circ f', g\circ g')$$
    \end{itemize}
\end{definition}

Let's consider some properties of morphism within the opposite category:

\begin{proposition}
    Given a category $\textbf{C}$ the morphism $f,g:X\rightarrow Y$ are equal iff its opposite in category $\textbf{C}^\text{op}$ are equal too i.e $f=g$ iff $f^\text{op}=g^\text{op}$
\end{proposition}


\begin{proof}
    $(\implies):$ This is obvious from the definition of the opposite morphism, as we can swap the signature of both functions $f$ and $g$. $(\impliedby):$ We can consider the following equality: $f = (f^\text{op})^\text{op} = (g^\text{op})^\text{op} = g$.
\end{proof}


\begin{corollary}
    A diagram in $\textbf{C}$ commutes iff the dual diagram in $\textbf{C}^\text{op}$ commutes. Given a category $\textbf{C}$, the morphism $f:X\rightarrow Y$ is invertible iff $f^{\text{op}}:Y\rightarrow X$ of the category $\textbf{C}^\text{op}$ is invertible i.e if $X$ and $Y$ are isomorphic in $\textbf{C}$, then they are isomorphic to $\textbf{C}^\text{op}$
\end{corollary}


\begin{proof}
    \textbf{(Part 1):} Let's denote the composed function from the first path to be $f_1:X\rightarrow Y$ and the composed function from second path to be $f_2:X\rightarrow Y$. Then since the diagram commute, we have $f_1=f_2$. But by the proposition above, we have that $f_1^\text{op}=f_2^\text{op}$ that is the dual diagram also commutes. 
    
    The other direction follows from the fact that $f_1^\text{op}=f_2^\text{op}$ implies $f_1=f_2$

    \textbf{(Part 2):} If $f:X\rightarrow Y$ is invertible, there is $f^{-1}:Y\rightarrow X$ such that: $f\circ f^{-1} = \operatorname{id}_Y$ and $f^{-1}\circ f = \operatorname{id}_X$. We can find the composition of the opposite morphism, as follows:

    \begin{equation*}
    \begin{aligned}
        &(f\circ f^{-1})^\text{op} = (f^{-1})^\text{op}\circ f^\text{op} = \operatorname{id}_Y^\text{op} \\
        &(f^{-1}\circ f)^\text{op} = f^\text{op}\circ (f^{-1})^\text{op} = \operatorname{id}_X^\text{op}
    \end{aligned}
    \end{equation*}
    thus, we have proven that $f^\text{op}$ is invertible. The other direction follows from the similar proof. 
\end{proof}


\subsection{Types of Functions}

Since we already have talked about isomorphism, we have the following properties of it:

\begin{proposition}
    Given the category $\textbf{C}$ with some objects $X$ and $Y$. The inverse of isomorphism $f:X\rightarrow Y$ is unique i.e show that if $g:Y\rightarrow X$ and $g':Y\rightarrow X$ are the inverses, then $g=g'$. The composite of isomorphism is isomorphism.
\end{proposition}


\begin{proof}
    Let's provide the proof for each points. We have that:
    \begin{equation*}
        g = g\circ\operatorname{id}_Y=g\circ(f\circ g') = (g\circ f)\circ g' = \operatorname{id}_X\circ g'=g'
    \end{equation*}
    For the second statement, given the function $f_1:X\rightarrow Y$ and $f_2:Y\rightarrow Z$ with their inverses $f_1^{-1}:Y\rightarrow X$ and $f_2^{-1}:Z\rightarrow Y$, respectively. We consider the following equations:
    \begin{equation*}
        \begin{aligned}
        (f_2\circ f_1)\circ(f_1^{-1}\circ f^{-1}_2) &= f_2\circ(f_1\circ f_1^{-1})\circ f_2^{-1} \\
        &= f_2\circ\operatorname{id}_Y\circ f_2^{-1} = f_2\circ f_2^{-1}=\operatorname{id}_Z
        \end{aligned}
    \end{equation*}
    The other direction is similar, thus $(f^{-1}_1\circ f^{-1}_2)=(f_2\circ f_1)^{-1}$.
\end{proof}


\begin{definition}{\textbf{(Monomorphism)}}
    Given a category $\textbf{C}$, the morphism $m:X\rightarrow Y$ is called monomorphism if for any object $A$, with a pair of morphism $f,g:A\rightarrow X$, the following diagram commutes:
    \begin{equation*}
    % https://q.uiver.app/#q=WzAsMyxbMCwwLCJBIl0sWzEsMCwiWCJdLFsyLDAsIlkiXSxbMSwyLCJtIl0sWzAsMSwiZiIsMCx7Im9mZnNldCI6LTF9XSxbMCwxLCJnIiwyLHsib2Zmc2V0IjoxfV1d
    \begin{tikzcd}
        A & X & Y
        \arrow["m", from=1-2, to=1-3]
        \arrow["f", shift left, from=1-1, to=1-2]
        \arrow["g"', shift right, from=1-1, to=1-2]
    \end{tikzcd}
    \end{equation*}
    and that $m\circ f = m\circ g$ implies $f=g$.
\end{definition}

\begin{proposition}
    Every isomorphism of $\textbf{C}$ is monomorphism of $\textbf{C}$
\end{proposition}

\begin{proof}
    Given the isomorphism $m:X\rightarrow Y$, and given any morphism $f,g:A\rightarrow X$, then we have:
    $$
    \begin{aligned}
    &m\circ f = m\circ g \\
    \implies&m^{-1}\circ m\circ f = m^{-1}\circ m\circ g \\
    \implies&\operatorname{id}_X\circ f = \operatorname{id}_X\circ g \implies f = g \\
    \end{aligned}
    $$
\end{proof}


\begin{definition}{\textbf{(Left-Inverse/Retraction/Split-Monomorphism)}}
    Given a category $\textbf{C}$, the morphism $m:X\rightarrow Y$, the left-inverse or retraction of $m$ is a morphism $r:Y\rightarrow X$ such that $r\circ m=\operatorname{id}_X$ i.e the following diagram commutes:
    \begin{equation*}
    % https://q.uiver.app/#q=WzAsMyxbMCwwLCJYIl0sWzEsMCwiWSJdLFsxLDEsIlgiXSxbMSwyLCJyIl0sWzAsMSwibSJdLFswLDIsIlxcdGV4dHtpZH1fWCIsMl1d
    \begin{tikzcd}
        X & Y \\
        & X
        \arrow["r", from=1-2, to=2-2]
        \arrow["m", from=1-1, to=1-2]
        \arrow["{\text{id}_X}"', from=1-1, to=2-2]
    \end{tikzcd}
    \end{equation*}
    If $m$ has a left-inverse, then we call it split monomorphism and $r$ is called its splitting.
\end{definition}

\begin{proposition}
    Every split monomorphism of $\textbf{C}$ is a monomorphism of $\textbf{C}$
\end{proposition}

\begin{proposition}
    Every isomorphism of $\textbf{C}$ is a split monomorphism of $\textbf{C}$
\end{proposition}

\begin{proposition}
    \label{prop:mono-iff-injective}
    Given function $f:A\to B$, it is mono iff it is injective in $\textbf{Set}$
\end{proposition}

\begin{proof}
    $(\implies):$ We can have the function $a:1\to A$ and $b:1\to A$ picking the element $a,b\in A$, then we have that $f(a)=f(b)$ implies that $a=b$ by it being mono, so $f$ is also injective.

    $(\impliedby):$ We note that every injective function has a left inverse i.e $f^{-1}\circ f=\operatorname{id}_A$, which can be constructed as follows:
    \begin{equation*}
        f^{-1}(b) = \begin{cases}
            a \text{ where } f(a) = b & \text{ for } b \in \operatorname{im}(f) \\
            1 & \text{ otherwise }
        \end{cases}
    \end{equation*}
    where $1$ is an arbitrary element of $A$. Note that the first condition is well-defined as if there is another $a'$ such that $f(a')=b$ then $f(a)=f(a')$ or $a=a'$. Thus, $f$ is in fact \textit{split} monomorphism, thus a monomorphism.
\end{proof}


As we have seen, if the monomorphism is a generalization of injectivity, then the epimorphism is the generalization of the surjectivity. We have the following notion.

\begin{definition}{\textbf{(Epimorphism)}}
    Given a category $\textbf{C}$, the morphism $e:X\rightarrow Y$ is called epimorphism if for any object $A$, with a pair of morphism $f,g:Y\rightarrow A$, the following diagram commutes:
    \begin{equation*}
    % https://q.uiver.app/#q=WzAsMyxbMCwwLCJYIl0sWzEsMCwiWSJdLFsyLDAsIkEiXSxbMCwxLCJlIl0sWzEsMiwiZiIsMCx7Im9mZnNldCI6LTF9XSxbMSwyLCJnIiwyLHsib2Zmc2V0IjoxfV1d
    \begin{tikzcd}
        X & Y & A
        \arrow["e", from=1-1, to=1-2]
        \arrow["f", shift left, from=1-2, to=1-3]
        \arrow["g"', shift right, from=1-2, to=1-3]
    \end{tikzcd}
    \end{equation*}
    and that $f\circ e = g\circ e$ implies $f=g$.
\end{definition}

\begin{proposition}
    We can show that $f:X\rightarrow Y$ is epi in $\textbf{C}$ iff $f^\text{op}:Y\rightarrow X$ is mono in $\textbf{C}^\text{op}$
\end{proposition}

\begin{proof}
    $(\implies):$ Give any morphism $m,n:Y\to A$ since $f$ is epi, we have that if $m\circ f = n\circ f$ then $m=n$. By proposition above, we also have that: $(m\circ f)^\text{op}=(n\circ f)^\text{op}$ or $f^\text{op}\circ m^\text{op}=f^{\text{op}}\circ n^\text{op}$ then $m^\text{op}=n^{op}$ meaning that $f^\text{op}$ is mono. The other direction is proven in similar manners.
\end{proof}


\begin{proposition}
    Every isomorphism is an epimorphism.
\end{proposition}

\begin{definition}{\textbf{(Right-Inverse/Section/Split-Epimorphism)}}
    Given a category $\textbf{C}$, the morphism $e:X\rightarrow Y$, the right-inverse or section of $m$ is a morphism $s:Y\rightarrow X$ such that $e\circ s=\operatorname{id}_Y$ i.e the following diagram commutes:
    \begin{equation*}
    % https://q.uiver.app/#q=WzAsMyxbMCwwLCJZIl0sWzEsMCwiWCJdLFsxLDEsIlkiXSxbMCwxLCJzIl0sWzEsMiwiZSJdLFswLDIsIlxcdGV4dHtpZH1fWSIsMl1d
    \begin{tikzcd}
        Y & X \\
        & Y
        \arrow["s", from=1-1, to=1-2]
        \arrow["e", from=1-2, to=2-2]
        \arrow["{\text{id}_Y}"', from=1-1, to=2-2]
    \end{tikzcd}
    \end{equation*}
    If $e$ has a right-inverse, then we call it split epimorphism and $s$ is called its splitting.
\end{definition}

And we have the following dual statements (can be proved using the statement in the opposite category) to monomorphism:

\begin{proposition}
    Every split epimorphism of $\textbf{C}$ is a epimorphism of $\textbf{C}$
\end{proposition}

\begin{proposition}
    Every isomorphism of $\textbf{C}$ is a split epimorphism of $\textbf{C}$
\end{proposition}

\begin{proposition}
    \label{prop:epi-iff-surjective}
    Given a function $f:A\to B$, it is epi iff it is surjective in $\textbf{Set}$
\end{proposition}

\begin{proof}
    $(\impliedby)$ Given any $a\in A$, by definition of surjective, we have that if $f(e(a))=g(e(a))$, for any $a\in A$, would implies that $f(b)=g(b)$ for all $b\in B$. That is because $e(A)=B$. 

    $(\implies)$ We will proof that if $e$ isn't surjective then $e$ isn't epi. That is there exists $b'\in B$ such that for all $a\in A$, $e(a)\ne b'$. It is easy to find the function $f$ and $g$ such that $f(b')\ne g(b')$ but $f(b)=g(b)$ for $b=e(a)$ for any $a\in A$. Then, $f(e(a))=g(e(a))$ for all $A$ but it doesn't implies that $f(b)=g(b)$ for all $b\in B$
\end{proof}


Finally, we have the following results of compositions of various special morphism.

\begin{proposition}
    We have the following list of results on the composition of various morphism:
    \begin{itemize}
        \item If $g\circ f$ is mono, then $f$ is mono
        \item If $f\circ g$ is epi, then $f$ is epi
        \item If $f$ and $g$ are mono, then $g\circ f$ is mono
        \item If $f$ and $g$ are epi, then $g\circ f$ is epi
        \item If $f$ is isomorphism, then $f$ is epi and mono
    \end{itemize}
\end{proposition}

\begin{proof}
    Let's provide a proof for each of them: 
    and that $f\circ e = g\circ e$ implies $f=g$.
    \begin{itemize}
        \item We will prove via negation, assume that $f$ isn't mono, that is there exists a morphism $a\ne b$ but $f\circ a=f\circ b$. Then we have that: $g\circ f\circ a=g\circ f\circ b$ and $a\ne b$ i.e $g\circ f$ isn't mono.
        \item Similarly, assume that $f$ isn't epi, that is there is a morphism $a\circ f=b\circ f$ but $a\ne b$. Then we have that $a\circ f\circ g=b\circ f\circ g$ and $a\ne b$ i.e $f\circ g$ isn't epi.
        \item Given a morphism $a, b$ such that $g\circ f\circ a = g\circ f\circ a$, since $g$ is mono, we have that $f\circ a = f\circ b$ and since $f$ is epi, we have that $a=b$. 
        \item Given a morphism $a, b$ such that $a\circ g \circ f = b\circ g \circ f$, since $f$ is epi, we have that $a\circ g= b\circ g$ and since $g$ is epi, we have that $a=b$. 
    \end{itemize}
    The last part follows from the fact that isomorphism is split mono and split epi, which implies that it is both mono and epi as needed.
\end{proof}


% \todo Add Free category.

\subsection{Functor + Presheaf}

Now, we will consider the morphism between category, which we called functor:

\begin{definition}{\textbf{(Functor)}}
    Given a category $\textbf{C}$ and $\textbf{D}$, the functor $F:\textbf{C}\rightarrow\textbf{D}$ transforms the objects and morphism of category $\textbf{C}$ as:
    \begin{itemize}
        \item The functors maps the object $X$ of $\textbf{C}$ to an object $FX$ of $\textbf{D}$
        \item The functors maps The morphism $f:X\rightarrow Y$ of $\textbf{C}$ to a morphism $Ff:FX\rightarrow FY$ of $\textbf{D}$
    \end{itemize}
    With the following conditions.
    \begin{itemize}
        \item \textit{Unitality} For every object $X$ of $\textbf{C}$ the functor have the following mapping $F(\operatorname{id}_X)=\operatorname{id}_{FX}$
        \item \textit{Compositionality} For the morphism $f:X\rightarrow Y$ and $g:Y\rightarrow Z$ in $\textbf{C}$, we have that $F(g\circ f) = Fg\circ Ff$ i.e the following diagram commutes: 
        \begin{equation*}
        % https://q.uiver.app/#q=WzAsMyxbMSwwLCJGWSJdLFswLDEsIkZYIl0sWzIsMSwiRloiXSxbMSwyLCJGKGdcXGNpcmMgZikiLDJdLFsxLDAsIkZmIl0sWzAsMiwiRmciXV0=
        \begin{tikzcd}
            & FY \\
            FX && FZ
            \arrow["{F(g\circ f)}"', from=2-1, to=2-3]
            \arrow["Ff", from=2-1, to=1-2]
            \arrow["Fg", from=1-2, to=2-3]
        \end{tikzcd}
        \end{equation*}
    \end{itemize}
\end{definition}

Since functor preserves composition, we have that:

\begin{proposition}
    Functor preserves commutative diagram. For example, if a diagram commutes in $\textbf{C}$, its image under a functor $F:\textbf{C}\rightarrow\textbf{D}$ commutes in $\textbf{D}$
\end{proposition}

in the specific case of isomorphism, as functor also preserves identity morphism:

\begin{proposition}
    Given a functor $F:\textbf{C}\rightarrow\textbf{D}$, within the category $\textbf{C}$, and we have the following isomorphism (shown in LHS):
    \begin{equation*}
    % https://q.uiver.app/#q=WzAsNCxbMCwwLCJYIl0sWzEsMCwiWSJdLFszLDAsIkZYIl0sWzQsMCwiRlkiXSxbMCwxLCJmIiwwLHsib2Zmc2V0IjotMX1dLFsxLDAsImZeey0xfSIsMCx7Im9mZnNldCI6LTF9XSxbMiwzLCJGZiIsMCx7Im9mZnNldCI6LTF9XSxbMywyLCJGZl57LTF9IiwwLHsib2Zmc2V0IjotMX1dXQ==
    \begin{tikzcd}
        X & Y && FX & FY
        \arrow["f", shift left, from=1-1, to=1-2]
        \arrow["{f^{-1}}", shift left, from=1-2, to=1-1]
        \arrow["Ff", shift left, from=1-4, to=1-5]
        \arrow["{Ff^{-1}}", shift left, from=1-5, to=1-4]
    \end{tikzcd}
    \end{equation*}
    Then, after applying the functors, we have that, in the category $\textbf{D}$ (shown in RHS), which is an isomorphism in $\textbf{D}$. Thus, The functors maps isomorphic objects (and morphism) of the category $\textbf{C}$ to isomorphic (and morphism) of the category $\textbf{D}$.
\end{proposition}

Similar observation can be used for the case of split 

\begin{proposition}
    The functor preserves split monomorphism and split epimorphism. Given functor $F:\textbf{C}\rightarrow\textbf{D}$:
    \begin{itemize}
        \item Given a split monomorphism $m:X\rightarrow Y$ with a retraction $r:Y\rightarrow X$, then $Fm$ is a split monomorphism with retraction of $Fr$.
        \item Dually, given a split epimorphism $e:X\rightarrow Y$ with a section $s:Y\rightarrow X$, then $Fe$ is a split epimorphism with section of $Fs$.
    \end{itemize}
\end{proposition}

However, this doesn't apply to the case of non-split epi and mono. Now, we are going to consider a special kind of functor called presheaves, in which it can be motivated from the probing action, defined in definition \ref{def:prob-action-nat}.

\begin{remark}{\textbf{(Motivation for Presheaves)}}
    \label{remark:presheaf-opposite}
    Given the category $\textbf{C}$, in which we get the hom-set of the morphism $X\rightarrow Y$ as $\operatorname{Hom}_\textbf{C}(X,A)$, where the object $A$ is fixed. We will make use the opposite category more explicit here.
    
    Given $g^\text{op}:X\to Y$, we can have the map $(-\circ g):\operatorname{Hom}_\textbf{C}(X,A)\to \operatorname{Hom}_\textbf{C}(Y, A)$, see the left triangle:
    \begin{equation*}
    % https://q.uiver.app/#q=WzAsNCxbMCwwLCJYIl0sWzIsMCwiWSJdLFs0LDAsIloiXSxbMiwxLCJBIl0sWzAsMywiayIsMl0sWzEsMCwiZyIsMl0sWzIsMSwiZiIsMl0sWzIsMywia1xcY2lyYyBnXFxjaXJjIGYiXSxbMSwzXSxbNCw4LCIoLVxcY2lyYyBnKSIsMCx7InNob3J0ZW4iOnsic291cmNlIjoyMCwidGFyZ2V0IjoyMH0sImxldmVsIjoxLCJjb2xvdXIiOlswLDYwLDYwXSwic3R5bGUiOnsiYm9keSI6eyJuYW1lIjoiZGFzaGVkIn19fSxbMCw2MCw2MCwxXV0sWzgsNywiKC1cXGNpcmMgZikiLDAseyJzaG9ydGVuIjp7InNvdXJjZSI6MjAsInRhcmdldCI6MjB9LCJsZXZlbCI6MSwiY29sb3VyIjpbMCw2MCw2MF0sInN0eWxlIjp7ImJvZHkiOnsibmFtZSI6ImRhc2hlZCJ9fX0sWzAsNjAsNjAsMV1dXQ==
    \begin{tikzcd}
        X && Y && Z \\
        && A
        \arrow[""{name=0, anchor=center, inner sep=0}, "k"', from=1-1, to=2-3]
        \arrow["g"', from=1-3, to=1-1]
        \arrow["f"', from=1-5, to=1-3]
        \arrow[""{name=1, anchor=center, inner sep=0}, "{k\circ g\circ f}", from=1-5, to=2-3]
        \arrow[""{name=2, anchor=center, inner sep=0}, from=1-3, to=2-3]
        \arrow["{(-\circ g)}", color={rgb,255:red,214;green,92;blue,92}, shorten <=6pt, shorten >=6pt, dashed, from=0, to=2]
        \arrow["{(-\circ f)}", color={rgb,255:red,214;green,92;blue,92}, shorten <=6pt, shorten >=6pt, dashed, from=2, to=1]
    \end{tikzcd}
    \end{equation*}
    Actually, this is a functor of $\textbf{C}^\text{op}\to\operatorname{Set}$ (opposite because after apply the functor we have the ``map of $X\to Y$'' (although it is in $\textbf{Set}$)) On can also consider the composition as:
    \begin{equation*}
        (-\circ f)\circ (-\circ g) = (-\circ g\circ f)
    \end{equation*}
    This is the characteristic of opposite category. Finally, the signature of the functor is: $\operatorname{Hom}_\textbf{C}(-,A):\textbf{C}^\text{op}\rightarrow\textbf{Set}$, where when acts on the morphism, we have that $\operatorname{Hom}_\textbf{C}(-,A)[g^\text{op}] = (-\circ g):\operatorname{Hom}_\textbf{C}(X,A)\to\operatorname{Hom}_\textbf{C}(Y,A)$. The way to look at this is that the functor $\textbf{C}^\text{op}\to\textbf{Set}$ accepts $g^\text{op}$ as an ``input'', but the action on the hom-set has to be done in $\textbf{C}$ because that hom-set is in $\textbf{C}$.
\end{remark}

\begin{definition}{\textbf{(Presheaf)}}
    Given a category $\textbf{C}$, the presheaf on $\textbf{C}$ is a functor $\textbf{C}^\text{op}\rightarrow\textbf{Set}$
\end{definition}

\begin{definition}{\textbf{(Contravariant Functors)}}
    Given the category $\textbf{C}$ and $\textbf{D}$, contravariant functor is the functor with reversed arrow as $\textbf{C}^\text{op}\rightarrow\textbf{D}$ or $\textbf{C}\rightarrow\textbf{D}^\text{op}$ (noted that they are the same)
\end{definition}


\subsection{Natural Transformation}

\begin{definition}{\textbf{(Natural Transformation)}}
    Given the categories $\textbf{C}$ and $\textbf{D}$ with a functors $F,G:\textbf{C}\rightarrow\textbf{D}$. We define the natural transformation $\alpha:F\Rightarrow G$ as follows:
    \begin{itemize}
        \item For each object, $C$ of $\textbf{C}$, we have $\alpha_C:FC\rightarrow GC$, which is a morphism in $\textbf{D}$, and it is called a component at $C$, or we can denoted the natural transformation as a collection of morphism $$(\alpha_C:FC\rightarrow GC)_{C\in\textbf{C}_0}$$
        \item Given a morphism $f:C\rightarrow C'$ of category $\textbf{C}$, we have the following commutative diagram, which represents the naturality condition (as seen in LHS).
        \begin{equation*}
        % https://q.uiver.app/#q=WzAsNCxbMCwwLCJGQyJdLFsxLDAsIkZDJyJdLFswLDEsIkdDIl0sWzEsMSwiR0MnIl0sWzAsMSwiRmYiXSxbMiwzLCJHZiIsMl0sWzEsMywiXFxhbHBoYV97Qyd9Il0sWzAsMiwiXFxhbHBoYV97Q30iLDJdXQ==
        \begin{tikzcd}
            FC & {FC'} \\
            GC & {GC'}
            \arrow["Ff", from=1-1, to=1-2]
            \arrow["Gf"', from=2-1, to=2-2]
            \arrow["{\alpha_{C'}}", from=1-2, to=2-2]
            \arrow["{\alpha_{C}}"', from=1-1, to=2-1]
        \end{tikzcd}
        \qquad \quad
        % https://q.uiver.app/#q=WzAsMixbMCwwLCJcXHRleHRiZntDfSJdLFsxLDAsIlxcdGV4dGJme0R9Il0sWzAsMSwiRiIsMCx7ImN1cnZlIjotMn1dLFswLDEsIkciLDIseyJjdXJ2ZSI6Mn1dLFsyLDMsIiIsMCx7InNob3J0ZW4iOnsic291cmNlIjoyMCwidGFyZ2V0IjoyMH19XV0=
        \begin{tikzcd}
            {\textbf{C}} & {\textbf{D}}
            \arrow[""{name=0, anchor=center, inner sep=0}, "F", curve={height=-12pt}, from=1-1, to=1-2]
            \arrow[""{name=1, anchor=center, inner sep=0}, "G"', curve={height=12pt}, from=1-1, to=1-2]
            \arrow[shorten <=3pt, shorten >=3pt, Rightarrow, from=0, to=1]
        \end{tikzcd}
        \end{equation*}
        Note that in the diagram, we also denote the natural condition on the RHS.
    \end{itemize}
\end{definition}

\begin{remark}{\textbf{(Revising Naturality Condition)}}
    Please observe that the naturality condition does go along with what we have defined in definition \ref{def:prob-action-nat} leads naturally to the following natural transformation (LHS) together with the commutative diagram (RHS):
    \begin{equation*}
        \gamma:\operatorname{Hom}_\textbf{C}(X,-)\Rightarrow\operatorname{Hom}_\textbf{C}(Y,-) \qquad \quad % https://q.uiver.app/#q=WzAsNCxbMCwwLCJcXG9wZXJhdG9ybmFtZXtIb219X1xcdGV4dGJme0N9KFgsIEEpIl0sWzMsMCwiXFxvcGVyYXRvcm5hbWV7SG9tfV9cXHRleHRiZntDfShYLCBCKSJdLFswLDEsIlxcb3BlcmF0b3JuYW1le0hvbX1fXFx0ZXh0YmZ7Q30oWSwgQSkiXSxbMywxLCJcXG9wZXJhdG9ybmFtZXtIb219X1xcdGV4dGJme0N9KFksIEIpIl0sWzAsMSwiXFxvcGVyYXRvcm5hbWV7SG9tfV9cXHRleHRiZntDfShYLCAtKVtmXT0oZlxcY2lyYyAtKSJdLFsyLDMsIlxcb3BlcmF0b3JuYW1le0hvbX1fXFx0ZXh0YmZ7Q30oWSwgLSlbZl09KGZcXGNpcmMgLSkiLDJdLFsxLDMsIigtXFxjaXJjIGcpIl0sWzAsMiwiKC1cXGNpcmMgZykiLDJdXQ==
        \begin{tikzcd}
            {\operatorname{Hom}_\textbf{C}(X, A)} &&& {\operatorname{Hom}_\textbf{C}(X, B)} \\
            {\operatorname{Hom}_\textbf{C}(Y, A)} &&& {\operatorname{Hom}_\textbf{C}(Y, B)}
            \arrow["{\operatorname{Hom}_\textbf{C}(X, -)[f]=(f\circ -)}", from=1-1, to=1-4]
            \arrow["{\operatorname{Hom}_\textbf{C}(Y, -)[f]=(f\circ -)}"', from=2-1, to=2-4]
            \arrow["{(-\circ g)}", from=1-4, to=2-4]
            \arrow["{(-\circ g)}"', from=1-1, to=2-1]
        \end{tikzcd}
    \end{equation*}
    where the pre-composition of $g$ is the component of the natural transformation (hence the use of $\gamma$), for any object. One can also have the natural transformation of $\phi:\operatorname{Hom}_\textbf{C}(-,A)\Rightarrow\operatorname{Hom}_\textbf{C}(-,B)$, whose component is $(f\circ-)$ for any object.
\end{remark}


\begin{definition}{\textbf{(Natural Isomorphism)}}
    Given category $\textbf{C}$ and $\textbf{D}$, and the functors $F,G:\textbf{C}\rightarrow\textbf{D}$ with natural transformation of $\alpha:F\Rightarrow G$. This $\alpha$ is called *natural isomorphism* if for each object $C$ of $\textbf{C}$ the components are all $\alpha_C:FC\rightarrow GC$ is isomorphism.
\end{definition}

We can see that natural transformation can be seen as morphism between functor, leading to

\begin{definition}{\textbf{(Functor Category/Vertical Composition)}}
    Given $\textbf{C}$ and $\textbf{D}$ be category, we can define the functor category $[\textbf{C},\textbf{D}]$ can be constructed as:
    \begin{itemize}
        \item Objects: Functors $F:\textbf{C}\rightarrow\textbf{D}$
        \item Morphism: Natural Transformation, given the functors $F,G:\textbf{C}\rightarrow\textbf{D}$, we have $\alpha:\textbf{F}\Rightarrow\textbf{G}$
        \begin{itemize}
            \item The identity of the $F$ is the natural transformation with components as $\operatorname{id}_{FC}:FC\rightarrow FC$
            \item The \textbf{vertical} composition between 2 natural transformation $\alpha:F\Rightarrow G$ and $\beta:G\Rightarrow H$ is a natural transformation $\beta\circ\alpha:F\Rightarrow H$ as: $FC\xrightarrow{\alpha_C}GC\xrightarrow{\beta_C}HC$This is clearly associative.
        \end{itemize}
    \end{itemize}
\end{definition}

\begin{proposition}
    Given 2 objects in the functor category $F$ and $G$ (between category $\textbf{C}$ and $\textbf{D}$), the natural isomorphism between them is isomorphic wrt the functor category.
\end{proposition}


\begin{proof}
    To show that natural isomorphism, let's call it $\alpha:F\Rightarrow G$ is isomorphism, we will define its inverse to be $\alpha^{-1}:G\Rightarrow F$, where its component is an inverse of $\alpha_C$ (which exists because of $\alpha$ being isomorphism) i.e $\alpha^{-1}_C:GC\to FC$ where $\alpha_C\circ \alpha_C^{-1} = \operatorname{id}_{GC}$ and $\alpha_C^{-1}\circ \alpha_C = \operatorname{id}_{FC}$. Thus, by definition, we have that $\alpha^{-1}\circ\alpha=\operatorname{id}\alpha\circ\alpha^{-1}$, as needed.
\end{proof}


\begin{definition}{\textbf{(Diagram/Category of Diagrams)}}
    Given a category $\textbf{C}$ with small category $\textbf{I}$, the diagram in $\textbf{C}$ of shape $\textbf{I}$ is a functor $\textbf{I}\rightarrow\textbf{C}$. The category of $I$-shaped diagram in $\textbf{C}$ is a functor category $[\textbf{I},\textbf{C}]$.
\end{definition}

Now, we will consider how the natural transformation are composed.

\begin{definition}{\textbf{(Vertical Composition)}}
    The way we compose natural transformation within the definition of functor category is called vertical composition: given category $\textbf{C}$ and $\textbf{D}$ together with functors $F,G,H:\textbf{C}\rightarrow\textbf{D}$ and natural transformation $\alpha:F\Rightarrow G$ and $\beta:G\Rightarrow H$, we have:
    \begin{equation*}
    % https://q.uiver.app/#q=WzAsNSxbMCwwLCJcXHRleHRiZntDfSJdLFsyLDAsIlxcdGV4dGJme0R9Il0sWzMsMCwiIFxccmlnaHRzcXVpZ2Fycm93Il0sWzQsMCwiXFx0ZXh0YmZ7Q30iXSxbNiwwLCJcXHRleHRiZntEfSJdLFswLDEsIkYiLDAseyJjdXJ2ZSI6LTN9XSxbMCwxLCJIIiwyLHsiY3VydmUiOjN9XSxbMCwxLCJHIiwxXSxbMyw0LCJGIiwwLHsiY3VydmUiOi0zfV0sWzMsNCwiSCIsMix7ImN1cnZlIjozfV0sWzcsNiwiXFxiZXRhIiwwLHsic2hvcnRlbiI6eyJzb3VyY2UiOjIwLCJ0YXJnZXQiOjIwfX1dLFs1LDcsIlxcYWxwaGEiLDAseyJzaG9ydGVuIjp7InNvdXJjZSI6MjAsInRhcmdldCI6MjB9fV0sWzgsOSwiXFxiZXRhXFxjaXJjXFxhbHBoYSIsMCx7InNob3J0ZW4iOnsic291cmNlIjoyMCwidGFyZ2V0IjoyMH19XV0=
    \begin{tikzcd}
        {\textbf{C}} && {\textbf{D}} & { \rightsquigarrow} & {\textbf{C}} && {\textbf{D}}
        \arrow[""{name=0, anchor=center, inner sep=0}, "F", curve={height=-18pt}, from=1-1, to=1-3]
        \arrow[""{name=1, anchor=center, inner sep=0}, "H"', curve={height=18pt}, from=1-1, to=1-3]
        \arrow[""{name=2, anchor=center, inner sep=0}, "G"{description}, from=1-1, to=1-3]
        \arrow[""{name=3, anchor=center, inner sep=0}, "F", curve={height=-18pt}, from=1-5, to=1-7]
        \arrow[""{name=4, anchor=center, inner sep=0}, "H"', curve={height=18pt}, from=1-5, to=1-7]
        \arrow["\beta", shorten <=2pt, shorten >=2pt, Rightarrow, from=2, to=1]
        \arrow["\alpha", shorten <=2pt, shorten >=2pt, Rightarrow, from=0, to=2]
        \arrow["\beta\circ\alpha", shorten <=5pt, shorten >=5pt, Rightarrow, from=3, to=4]
    \end{tikzcd}
    \end{equation*}
\end{definition}

\begin{definition}{\textbf{(Left/Right Whiskering and Horizontal Composition)}}
    Given the category $\textbf{C}, \textbf{D}$ and $\textbf{E}$ with the functor $F,G:\textbf{C}\rightarrow\textbf{D}$ and $H,I:\textbf{D}\rightarrow\textbf{E}$ together with a natural transformation $\alpha:F\Rightarrow G$ or $\beta:H\Rightarrow I$, giving rise to $\beta F: HF\Rightarrow IF$ (LHS), called left whiskering:
    \begin{equation*}
        % https://q.uiver.app/#q=WzAsOSxbMCwwLCJcXHRleHRiZntDfSJdLFsxLDAsIlxcdGV4dGJme0R9Il0sWzIsMCwiXFx0ZXh0YmZ7RX0iXSxbNCwwLCJcXHRleHRiZntDfSJdLFs1LDAsIlxcdGV4dGJme0R9Il0sWzYsMCwiXFx0ZXh0YmZ7RX0iXSxbOCwwLCJcXHRleHRiZntDfSJdLFs5LDAsIlxcdGV4dGJme0R9Il0sWzEwLDAsIlxcdGV4dGJme0V9Il0sWzAsMSwiRiJdLFsxLDIsIkgiLDAseyJjdXJ2ZSI6LTN9XSxbMSwyLCJJIiwyLHsiY3VydmUiOjN9XSxbMyw0LCJGIiwwLHsiY3VydmUiOi0zfV0sWzMsNCwiRyIsMix7ImN1cnZlIjozfV0sWzQsNSwiSCJdLFs2LDcsIkYiLDAseyJjdXJ2ZSI6LTN9XSxbNiw3LCJHIiwyLHsiY3VydmUiOjN9XSxbNyw4LCJIIiwwLHsiY3VydmUiOi0zfV0sWzcsOCwiSSIsMix7ImN1cnZlIjozfV0sWzEwLDExLCJcXGJldGEiLDAseyJzaG9ydGVuIjp7InNvdXJjZSI6MjAsInRhcmdldCI6MjB9fV0sWzEyLDEzLCJcXGFscGhhIiwwLHsic2hvcnRlbiI6eyJzb3VyY2UiOjIwLCJ0YXJnZXQiOjIwfX1dLFsxNSwxNiwiXFxhbHBoYSIsMCx7InNob3J0ZW4iOnsic291cmNlIjoyMCwidGFyZ2V0IjoyMH19XSxbMTcsMTgsIlxcYmV0YSIsMCx7InNob3J0ZW4iOnsic291cmNlIjoyMCwidGFyZ2V0IjoyMH19XV0=
        \begin{tikzcd}
            {\textbf{C}} & {\textbf{D}} & {\textbf{E}} && {\textbf{C}} & {\textbf{D}} & {\textbf{E}} && {\textbf{C}} & {\textbf{D}} & {\textbf{E}}
            \arrow["F", from=1-1, to=1-2]
            \arrow[""{name=0, anchor=center, inner sep=0}, "H", curve={height=-18pt}, from=1-2, to=1-3]
            \arrow[""{name=1, anchor=center, inner sep=0}, "I"', curve={height=18pt}, from=1-2, to=1-3]
            \arrow[""{name=2, anchor=center, inner sep=0}, "F", curve={height=-18pt}, from=1-5, to=1-6]
            \arrow[""{name=3, anchor=center, inner sep=0}, "G"', curve={height=18pt}, from=1-5, to=1-6]
            \arrow["H", from=1-6, to=1-7]
            \arrow[""{name=4, anchor=center, inner sep=0}, "F", curve={height=-18pt}, from=1-9, to=1-10]
            \arrow[""{name=5, anchor=center, inner sep=0}, "G"', curve={height=18pt}, from=1-9, to=1-10]
            \arrow[""{name=6, anchor=center, inner sep=0}, "H", curve={height=-18pt}, from=1-10, to=1-11]
            \arrow[""{name=7, anchor=center, inner sep=0}, "I"', curve={height=18pt}, from=1-10, to=1-11]
            \arrow["\beta", shorten <=5pt, shorten >=5pt, Rightarrow, from=0, to=1]
            \arrow["\alpha", shorten <=5pt, shorten >=5pt, Rightarrow, from=2, to=3]
            \arrow["\alpha", shorten <=5pt, shorten >=5pt, Rightarrow, from=4, to=5]
            \arrow["\beta", shorten <=5pt, shorten >=5pt, Rightarrow, from=6, to=7]
        \end{tikzcd}
        \end{equation*}
    On the middle diagram, we have similar composition, which is called right whiskering $H\alpha: HF\Rightarrow HG$. One can combine 2 different kinds of whiskering to get the horizontal composition $\beta\alpha:HF\Rightarrow IG$. Note that we didn't use $\circ$ to denote the composition as we reserve for vertical composition.
\end{definition}

\begin{remark}{\textbf{(Example of Whiskering)}}
    Let's use the left whiskering as given in the definition above. To see how the natural transformation acts, we have to given it either an object $C$ or function $f:C\to C'$. We have its component to be $\beta_{FC}:HFC\to IFC$. And, the  commutative diagram is given to be:
    \begin{equation*}
    % https://q.uiver.app/#q=WzAsNCxbMCwwLCJIRkMiXSxbMCwxLCJJRkMiXSxbMSwwLCJIRkMnIl0sWzEsMSwiSUZDJyJdLFsxLDMsIklGZiIsMl0sWzAsMiwiSEZmIl0sWzIsMywiXFxiZXRhX3tGQyd9Il0sWzAsMSwiXFxiZXRhX3tGQ30iLDJdXQ==
    \begin{tikzcd}
        HFC & {HFC'} \\
        IFC & {IFC'}
        \arrow["IFf"', from=2-1, to=2-2]
        \arrow["HFf", from=1-1, to=1-2]
        \arrow["{\beta_{FC'}}", from=1-2, to=2-2]
        \arrow["{\beta_{FC}}"', from=1-1, to=2-1]
    \end{tikzcd}
    \end{equation*}
    and the diagram commute because $\beta$ is a natural transformation (which acts on $Ff:FC\to FC'$)
\end{remark}

Let's see that the horizontal composition makes sense.

\begin{proposition}
    The horizontal composition can be described in two ways as the vertical composition of left whiskering and composition of right whiskering (but different ways at the start), but are equal to each other:
    \begin{equation*}
        \beta\alpha = (\beta G)\circ(H\alpha) = (I\alpha)\circ(\beta F)
    \end{equation*}
\end{proposition}

\begin{proof}
    Let's consider how both LHS and RHS acts on object $C$, which we have the following signatures: $\beta_{GC}\circ H\alpha_C : HFC\to HGC \to IGC$ and $I\alpha_C\circ \beta_{FC} : HFC\to IFC \to IGC$. When apply the naturality condition of $\beta$ on the morphism $\alpha_C:FC\to GC$, we have the following commutative diagram.
    \begin{equation*}
    % https://q.uiver.app/#q=WzAsNCxbMCwwLCJIRkMiXSxbMSwwLCJIR0MiXSxbMSwxLCJJR0MiXSxbMCwxLCJJRkMiXSxbMCwzLCJcXGJldGFfe0ZDfSIsMl0sWzEsMiwiXFxiZXRhX3tHQ30iXSxbMCwxLCJIXFxhbHBoYV9DIl0sWzMsMiwiSVxcYWxwaGFfQyIsMl1d
    \begin{tikzcd}
        HFC & HGC \\
        IFC & IGC
        \arrow["{\beta_{FC}}"', from=1-1, to=2-1]
        \arrow["{\beta_{GC}}", from=1-2, to=2-2]
        \arrow["{H\alpha_C}", from=1-1, to=1-2]
        \arrow["{I\alpha_C}"', from=2-1, to=2-2]
    \end{tikzcd}
    \qquad \quad
    % https://q.uiver.app/#q=WzAsNixbMCwwLCJIRkMiXSxbMSwwLCJIR0MiXSxbMiwwLCJJR0MiXSxbMCwxLCJIRkMnIl0sWzEsMSwiSEdDJyJdLFsyLDEsIklHQyciXSxbMCwxLCJIXFxhbHBoYV9DIl0sWzEsMiwiXFxiZXRhX3tHQ30iXSxbMCwzLCJIRmYiLDJdLFsxLDQsIkhHZiJdLFsyLDUsIklHZiJdLFszLDQsIkhcXGFscGhhX3tDJ30iLDJdLFs0LDUsIlxcYmV0YV97R0MnfSIsMl1d
    \begin{tikzcd}
        HFC & HGC & IGC \\
        {HFC'} & {HGC'} & {IGC'}
        \arrow["{H\alpha_C}", from=1-1, to=1-2]
        \arrow["{\beta_{GC}}", from=1-2, to=1-3]
        \arrow["HFf"', from=1-1, to=2-1]
        \arrow["HGf", from=1-2, to=2-2]
        \arrow["IGf", from=1-3, to=2-3]
        \arrow["{H\alpha_{C'}}"', from=2-1, to=2-2]
        \arrow["{\beta_{GC'}}"', from=2-2, to=2-3]
    \end{tikzcd}
    \end{equation*}
    which is equivalent to see that $\beta_{GC}\circ H\alpha_C = I\alpha_C\circ \beta_{FC}$, as needed. On the RHS diagram, we have shown how one of the acts on $f:C\to C'$, and implies that is indeed satisfies the naturality condition over (following from naturality of each sub-square), see RHS diagram.
\end{proof}


\begin{proposition}{\textbf{(Interchange Law)}}
    Given the following diagram:
    \begin{equation*}
    % https://q.uiver.app/#q=WzAsMyxbMCwwLCJcXHRleHRiZntDfSJdLFsxLDAsIlxcdGV4dGJme0R9Il0sWzIsMCwiXFx0ZXh0YmZ7RX0iXSxbMCwxLCJGIiwwLHsiY3VydmUiOi0zfV0sWzAsMSwiRyIsMV0sWzEsMiwiSCIsMCx7ImN1cnZlIjotM31dLFsxLDIsIkkiLDFdLFswLDEsIlAiLDIseyJjdXJ2ZSI6M31dLFsxLDIsIlEiLDIseyJjdXJ2ZSI6M31dLFszLDQsIlxcYWxwaGEiLDAseyJzaG9ydGVuIjp7InNvdXJjZSI6MjAsInRhcmdldCI6MjB9fV0sWzUsNiwiXFxiZXRhIiwwLHsic2hvcnRlbiI6eyJzb3VyY2UiOjIwLCJ0YXJnZXQiOjIwfX1dLFs2LDgsIlxcZGVsdGEiLDAseyJzaG9ydGVuIjp7InNvdXJjZSI6MjAsInRhcmdldCI6MjB9fV0sWzQsNywiXFxnYW1tYSIsMCx7InNob3J0ZW4iOnsic291cmNlIjoyMCwidGFyZ2V0IjoyMH19XV0=
    \begin{tikzcd}
        {\textbf{C}} & {\textbf{D}} & {\textbf{E}}
        \arrow[""{name=0, anchor=center, inner sep=0}, "F", curve={height=-18pt}, from=1-1, to=1-2]
        \arrow[""{name=1, anchor=center, inner sep=0}, "G"{description}, from=1-1, to=1-2]
        \arrow[""{name=2, anchor=center, inner sep=0}, "H", curve={height=-18pt}, from=1-2, to=1-3]
        \arrow[""{name=3, anchor=center, inner sep=0}, "I"{description}, from=1-2, to=1-3]
        \arrow[""{name=4, anchor=center, inner sep=0}, "P"', curve={height=18pt}, from=1-1, to=1-2]
        \arrow[""{name=5, anchor=center, inner sep=0}, "Q"', curve={height=18pt}, from=1-2, to=1-3]
        \arrow["\alpha", shorten <=2pt, shorten >=2pt, Rightarrow, from=0, to=1]
        \arrow["\beta", shorten <=2pt, shorten >=2pt, Rightarrow, from=2, to=3]
        \arrow["\delta", shorten <=2pt, shorten >=2pt, Rightarrow, from=3, to=5]
        \arrow["\gamma", shorten <=2pt, shorten >=2pt, Rightarrow, from=1, to=4]
    \end{tikzcd}
    \end{equation*}
    This diagram isn't ambiguous in the sense that: $(\delta\gamma)\circ(\beta\alpha) = (\delta\circ\beta)(\gamma\circ\alpha)$
\end{proposition}


\begin{proof}
    Let's consider its component on $C$ on the LHS we have that both of them have signature $HFC\to IGC$. Following from proposition above, on how one define the horizontal composition and their equality:
    \begin{equation*}
    \begin{aligned}
        \beta\alpha &= (\beta G)\circ(H\alpha) = (I\alpha)\circ(\beta F) \\
        \delta\gamma &= (\delta P)\circ(I\gamma) = (Q\gamma) \circ (\delta G) \\
        \beta\gamma &= (I\gamma)\circ(\beta G) = (\beta P)\circ (H\gamma) \\
        \delta\alpha &= (\delta G) \circ (I\alpha) = (Q\alpha)\circ(\delta F) \\
    \end{aligned}
    \qquad \quad
    \begin{aligned}[t]
        (\delta\circ\beta)(\gamma\circ\alpha) &=(Q(\gamma\circ\alpha))\circ((\delta\circ\beta)F)  \\
        &=((\delta\circ\beta)P)\circ(H(\gamma\circ\alpha)) \\
    \end{aligned}
    \end{equation*}
    The last equality comes from horizontal composition of $\beta\gamma$. The one highlighted in red are equality that are needed. Thus, we have:
    \begin{equation*}
    \begin{aligned}
        (\delta\gamma)\circ(\beta\alpha) &= (\delta P)\circ \big[(I\gamma) \circ (\beta G)\big]\circ(H\alpha) \\
        &= (\delta P)\circ \big[(\beta P)\circ (H\gamma)\big]\circ(H\alpha) \\ 
        &= (\delta\circ\beta)(\gamma\circ\alpha)
    \end{aligned} \qquad \quad
    \begin{aligned}
        (\delta\gamma)\circ(\beta\alpha) 
        &= (Q\gamma) \circ \big[(\delta G) \circ (I\alpha)\big]\circ(\beta F) \\
        &= (Q\gamma) \circ \big[(Q\alpha)\circ(\delta F)\big]\circ(\beta F) \\
        &= (\delta\circ\beta)(\gamma\circ\alpha)
    \end{aligned}
    \end{equation*}
    Similarly, we can have the RHS proof of similar technique. Everything can be displayed in the following diagram (where the sub-square represents the naturality condition tht gives rises to the equality related to $\beta\gamma$ and $\delta\alpha$)
    \begin{equation*}
    % https://q.uiver.app/#q=WzAsMTAsWzAsMCwiSEZDIl0sWzIsMCwiSEdDIl0sWzMsMCwiSUdDIl0sWzMsMSwiSVBDIl0sWzMsMiwiUVBDIl0sWzAsMSwiSUZDIl0sWzAsMiwiSUdDIl0sWzEsMiwiUUdDIl0sWzEsMSwiUUZDIl0sWzIsMSwiSFBDIl0sWzAsMSwiSFxcYWxwaGFfQyJdLFsxLDIsIlxcYmV0YV97R0N9Il0sWzIsMywiSVxcZ2FtbWFfQyJdLFszLDQsIlxcZGVsdGFfe1BDfSJdLFswLDUsIlxcYmV0YV97RkN9IiwyXSxbNSw2LCJJXFxhbHBoYV9DIiwyXSxbNiw3LCJcXGRlbHRhX3tHQ30iLDJdLFs3LDQsIlFcXGdhbW1hX0MiLDJdLFs1LDgsIlxcZGVsdGFfe0ZDfSJdLFs4LDcsIlFcXGFscGhhX0MiXSxbOSwzLCJcXGJldGFfe1BDfSIsMl0sWzEsOSwiSFxcZ2FtbWFfQyIsMl1d
    \begin{tikzcd}
        HFC && HGC & IGC \\
        IFC & QFC & HPC & IPC \\
        IGC & QGC && QPC
        \arrow["{H\alpha_C}", from=1-1, to=1-3]
        \arrow["{\beta_{GC}}", from=1-3, to=1-4]
        \arrow["{I\gamma_C}", from=1-4, to=2-4]
        \arrow["{\delta_{PC}}", from=2-4, to=3-4]
        \arrow["{\beta_{FC}}"', from=1-1, to=2-1]
        \arrow["{I\alpha_C}"', from=2-1, to=3-1]
        \arrow["{\delta_{GC}}"', from=3-1, to=3-2]
        \arrow["{Q\gamma_C}"', from=3-2, to=3-4]
        \arrow["{\delta_{FC}}", from=2-1, to=2-2]
        \arrow["{Q\alpha_C}", from=2-2, to=3-2]
        \arrow["{\beta_{PC}}"', from=2-3, to=2-4]
        \arrow["{H\gamma_C}"', from=1-3, to=2-3]
    \end{tikzcd}
    \end{equation*}
\end{proof}


\subsection{Types of Category}

We are now going to give the various types of categories. Let's start with the more general one:

\begin{definition}{\textbf{(Category of Categories):}}
    The category of categories called $\textbf{Cat}$ is an category with the following elements:
    \begin{itemize}
        \item \textit{Objects:} Small Categories
        \item \textit{Morphism:} Functor between 2 Categories.
    \end{itemize}
    The identity morphism is an identity functor, and the compositions are done by the composition of functors.
\end{definition}

Note that $\textbf{Cat}$ isn't much used, but a more refine notion of $\textbf{Cat}$ in which hom-space $\operatorname{Hom}_\textbf{Cat}(C,D)$ are themselves category and not just set, which is called \textit{2-category} is more used. 

We also have the notion of sub-category as:

\begin{definition}{\textbf{(Category of Categories):}}
    Given a category $\mathbf{C}$, the sub-category $\mathcal{S}$ of $\mathbf{C}$ consists of sub-collection of objects and morphism of $\mathbf{C}$ such that:
    \begin{itemize}
        \item \textit{Identity Morphism:} There is identity morphism of object $S$ in $\mathbf{S}$
        \item \textit{Composition:} Given a composable arrrow $f:X\rightarrow Y$ and $g:Y\rightarrow Z$ in $\mathbf{S}$, then $g\circ f$ also in $\mathbf{S}$
    \end{itemize}
\end{definition}

The sub-category can have further different types depend on the what kind of morphism contains or the objects within.

\begin{definition}{\textbf{(Wide/Full Sub-Category)}}
    We have following different types of sub-categories:
    \begin{itemize}
        \item \textit{Wide Sub-Category:} A sub-category $\mathbf{S}$ of $\mathbf{C}$ is called wide if all objects of $\mathbf{C}$ are also object of $\mathbf{S}$ but should have less morphism. 
        \item \textit{Full Sub-Category:} A sub-category $\mathbf{S}$ of $\mathbf{C}$ is called full if given 2 objects $X$ of $Y$ of $\mathbf{S}$, all their morphism in $\mathbf{C}$ is also morphism in $\mathbf{S}$ as: $\operatorname{Hom}_\mathbf{S}(X,Y)\cong\operatorname{Hom}_\mathbf{C}(X,Y)$. But not all objects of $\textbf{C}$ have to be in $\textbf{S}$
    \end{itemize}
\end{definition}

\subsection{Type of Functors and Equivalent of Category}

\begin{definition}{\textbf{(Faithful/Full/Essentially Surjective Functor)}}
    Given a functor $F:\textbf{C}\rightarrow\textbf{D}$, which can have the following properties:
    \begin{itemize}
        \item \textit{Faithful:} Given object $C,C'$ of $\textbf{C}$ with any arrow $f,f':C\rightarrow C'$ of $\textbf{C}$, the functor is faithful if $Ff=Ff'$ for $\textbf{D}$ then $f=f'$ for $\textbf{C}$.
        \item \textit{Full:} Given object $C,C'$ of $\textbf{C}$ and for arrow $g:FC\rightarrow FC'$ of $\textbf{D}$, the functor is full if there is an arrow $f:C\rightarrow C'$ such that $Ff=g$
        \item \textit{Faithful:} The functor $F$ is fully faithful,if it is full and faithful.
        \item \textit{Essentially Surjective:} For any object $D$ of $\textbf{D}$, the functor is essentially surjective if there exists an object $C$ of $\mathbf{C}$ such that there is isomorphism $FC\rightarrow D$. 
    \end{itemize}
\end{definition}

We can link the notion to the injectivity and surjectivity of the hom-set:

\begin{remark}
    With object $C,C'$ of $\textbf{C}$ and functor $F:\textbf{C}\rightarrow\textbf{D}$ gives the following morphism (on $\textbf{Set}$) as: $\operatorname{Hom}_\textbf{C}(C,C') \rightarrow \operatorname{Hom}_\textbf{D}(FC,FC')$. We have that:
    \begin{itemize}
        \item If $F$ is faithful, then the function above is injective, for all $C,C'$.
        \item If $F$ is full, then the function above is surjective, for all $C,C'$.
        \item If $F$ is fully faithful, then the function above is bijective, for all $C,C'$.
        \item For the essentially surjective, it is the surjective up to isomorphism. 
    \end{itemize}
    
    Note that for the last point, this is the only version of surjectivity that we need, as we look at isomorphic object rather than equal.
\end{remark}

\begin{proposition}
    Given fully faithful functor $F:\textbf{C}\rightarrow\textbf{D}$ with objects $C,C'$ of $\textbf{C}$ and isomorphism $\phi:FC\rightarrow FC'$ of $\textbf{D}$. There is an unique isomorphism $\widetilde{\phi}:C\rightarrow C'$ such that: $F\widetilde{\phi}=\phi$
\end{proposition}


\begin{proof}
    \textbf{(Isomorphism):} By the fact that the functor is full, thus, there is an arrow $\widetilde{\phi}:C\to C'$ such that $F\widetilde{\phi}=\phi$. Since $\phi$ is isomorphism, there is an (unique) inverse $\phi^{-1}:FC'\to FC$, and by the fact that functor is full, there is $g:C'\to C$ such that $Fg=\phi^{-1}$. We note that:
    \begin{equation*}
        F(g\circ \widetilde{\phi}) = Fg\circ F\widetilde{\phi} = \phi^{-1}\circ \phi = \operatorname{id}_{FC} = F(\operatorname{id}_{C})
    \end{equation*}
    Since $F$ is faithful, we have that $g\circ \widetilde{\phi} = \operatorname{id}_{C}$. The other direction $\widetilde{\phi}\circ g = \operatorname{id}_{C'}$ is proven similarly. 

    \textbf{(Unique):} Assume that there is another isomorphism $f:C\to C'$ such that $Ff=\phi$, then we have $Ff=F\widetilde{\phi}$. By the fact that $F$ is faithful $f=\widetilde{\phi}$ i.e it is unique.
\end{proof}


\begin{definition}{\textbf{(Equivalence of Categories)}}
    Given a category $\textbf{C}$ and $\textbf{D}$, the \textit{equivalence of category} between both category consists of a pair $F:\textbf{C}\rightarrow\textbf{D}$ and $G:\textbf{D}\rightarrow\textbf{C}$ and natural isomorphism as: $\eta:G\circ F\Rightarrow\operatorname{id}_C$ and $\varepsilon:F\circ G\Rightarrow\operatorname{id}_D$. We call $G$ the pseudo-inverse of $F$ and vice versa.
\end{definition}

To proof important theorem below, we need the following useful lemma:

\begin{lemma}
    \label{lemma:morph-from-isos}
    Given a category $\textbf{C}$ with morphism $f:X\rightarrow Y$ with isomorphisms $\phi:X\rightarrow X'$ and $\psi:Y\rightarrow Y'$, then there is unique morphism $X'\rightarrow Y'$ such that:
    \begin{equation*}
    % https://q.uiver.app/#q=WzAsNCxbMCwwLCJYIl0sWzEsMCwiWSJdLFswLDEsIlgnIl0sWzEsMSwiWSciXSxbMCwxLCJmIl0sWzAsMiwiXFxwaGkiLDJdLFsxLDMsIlxccHNpIiwyXSxbMSwzLCJcXGNvbmciXSxbMCwyLCJcXGNvbmciXSxbMiwzLCIiLDAseyJzdHlsZSI6eyJib2R5Ijp7Im5hbWUiOiJkYXNoZWQifX19XV0=
    \begin{tikzcd}
        X & Y \\
        {X'} & {Y'}
        \arrow["f", from=1-1, to=1-2]
        \arrow["\phi"', from=1-1, to=2-1]
        \arrow["\psi"', from=1-2, to=2-2]
        \arrow["\cong", from=1-2, to=2-2]
        \arrow["\cong", from=1-1, to=2-1]
        \arrow[dashed, from=2-1, to=2-2]
    \end{tikzcd}
    \end{equation*}
    commutes, and this morphism is given by: $\psi\circ f\circ \phi^{-1}$.
\end{lemma}


\begin{proof}
    We note that such morphism exists, by reversing the arrow of $\phi$, giving us $g:X'\to Y'$ equal to $\psi\circ f\circ \phi^{-1}$. Now, we are left to show that it is unique, assume that there is another morphism $g':X'\to Y'$, then by commutativity: $g'\circ\phi = \psi \circ f$ or $g'= \psi\circ f\circ \phi^{-1} = g$.
\end{proof}


\begin{theorem}
    \label{thm:full-faithful-essentially-surj-equiv}
    The functor $F:\textbf{C}\rightarrow\textbf{D}$ is fully faithful and essentially surjective iff it defines an equivalence of category. 
\end{theorem}


\begin{proof}
    $\boldsymbol{(\implies)}:$ We have to find the functor $G:\boldsymbol{D}\to\boldsymbol{C}$. We will have to see how it interacts with object and morphism in $\boldsymbol{D}$. Given $g:D\to D'$, since $F$ is essentially surjective, there is an object $C$ and $C'$ with isomorphisms $\varepsilon_{D}:FC\to D$ and $\varepsilon_{D'}:FC'\to D'$. And from lemma above, there is a unique morphism $h:FC\to FC'$:

    \begin{equation*}
    % https://q.uiver.app/#q=WzAsNCxbMCwwLCJGQyJdLFsxLDAsIkZDJyJdLFswLDEsIkQiXSxbMSwxLCJEJyJdLFsyLDMsImciLDJdLFsxLDMsIlxcY29uZyIsMl0sWzAsMiwiXFxjb25nIl0sWzAsMiwiXFx2YXJlcHNpbG9uX0QiLDJdLFswLDEsImgiLDAseyJzdHlsZSI6eyJib2R5Ijp7Im5hbWUiOiJkYXNoZWQifX19XSxbMSwzLCJcXHZhcmVwc2lsb25fe0QnfSJdXQ==
    \begin{tikzcd}
        FC & {FC'} \\
        D & {D'}
        \arrow["g"', from=2-1, to=2-2]
        \arrow["\cong"', from=1-2, to=2-2]
        \arrow["\cong", from=1-1, to=2-1]
        \arrow["{\varepsilon_D}"', from=1-1, to=2-1]
        \arrow["h", dashed, from=1-1, to=1-2]
        \arrow["{\varepsilon_{D'}}", from=1-2, to=2-2]
    \end{tikzcd}
    \end{equation*}

    where $h=\varepsilon^{-1}_{D'}\circ g\circ\varepsilon_D$. Since $F$ is fully faithful, there is unique $f_h:C\to C'$ such that $Ff_h=h$ and define such morphism as the result of applying $G$ i.e $f_h=Gg$. But we also have that $FGg=h$. Therefore, given the map $\varepsilon_D,\varepsilon_{D'}$, there is a unique pair between $(f_h, g)$ with $GD=C$ and $GD'=C'$.  We will also have to show that $G$ is a functor.
    \begin{itemize}
        \item \textit{(Acting on Identity Morphism):} Given $\operatorname{id}_D$, we have that $h=\operatorname{id}_D$ and the morphism $f_h$ is $\operatorname{id}_C$ due to it being unique. Thus, $G(\operatorname{id}_D)=\operatorname{id}_C$
        \item \textit{(Acting on Composition):} Given 2 morphism $g_1:D\to D'$ and $g_2:D'\to D''$, then we have that $h_1=\varepsilon_{D'}^{-1}\circ g_1\circ\varepsilon_D$ and $h_2=\varepsilon_{D''}^{-1}\circ g_2\circ\varepsilon_{D'}$, $Gg_1=f_{h_1}$ and $Gg_2=f_{h_2}$ where $Ff_{h_1}=h_1$ and $Ff_{h_2}=h_2$. On the other hand, consider $g_3=g_2\circ g_1$, in which the corresponding morphism is $h_3 = \varepsilon_{D''}^{-1}\circ g_2 \circ g_1 \circ\varepsilon_{D}=h_2\circ h_1$. Note that $Ff_{h_3} = h_3=h_2\circ h_1 = F(f_{h_2})\circ F(f_{h_1}) =F(f_{h_2}\circ f_{h_1})$, and since $F$ is full $f_{h_3}=f_{h_2}\circ F_{h_1}$. Thus: $$G(g_2\circ g_1)=Gg_3 = f_{h_3} = f_{h_2}\circ f_{h_1}=G(g_2)\circ G(g_1)$$
    \end{itemize}
    Leading to a natural isomorphism of $\varepsilon:FG\Rightarrow\operatorname{id}_D$ (The naturality condition follows from the commutative diagram above).

    On the other hand, given the morphism $Fg:FC\to FC'$, for $g:C\to C'$, then since $F$ is essentially surjective, there is an object $C_n$ and $C_n'$ such that there is isomorphism $\phi:FC_n\to FC$ and $\psi:FC_n'\to FC'$ (note that $C_n$ and $C_n'$ might as well be $C$ and $C'$, and $\phi$ and $\psi$ would be identity morphism but there can be other choices). We can use our construction of $G$ above, which gives us $GFC=C_n$ and $GFC'=C_n'$, and the following commutative diagram (LHS):

    \begin{equation*}
    % https://q.uiver.app/#q=WzAsNCxbMCwwLCJGR0ZDIl0sWzIsMCwiRkdGQyciXSxbMCwxLCJGQyJdLFsyLDEsIkZDJyJdLFsyLDMsIkZnIiwyXSxbMCwxLCJrPUZHRmciLDAseyJzdHlsZSI6eyJib2R5Ijp7Im5hbWUiOiJkYXNoZWQifX19XSxbMCwyLCJcXHBoaSIsMl0sWzEsMywiXFxwc2kiXV0=
    \begin{tikzcd}
        FGFC && {FGFC'} \\
        FC && {FC'}
        \arrow["Fg"', from=2-1, to=2-3]
        \arrow["{k=FGFg}", dashed, from=1-1, to=1-3]
        \arrow["\phi"', from=1-1, to=2-1]
        \arrow["\psi", from=1-3, to=2-3]
    \end{tikzcd} 
    \qquad\quad \rightsquigarrow\qquad \quad 
    % https://q.uiver.app/#q=WzAsNCxbMCwwLCJHRkMiXSxbMSwwLCJHRkMnIl0sWzAsMSwiQyJdLFsxLDEsIkMnIl0sWzIsMywiZyIsMl0sWzAsMSwiR0ZnIiwwLHsic3R5bGUiOnsiYm9keSI6eyJuYW1lIjoiZGFzaGVkIn19fV0sWzAsMiwiXFxldGFfe0N9IiwyXSxbMSwzLCJcXGV0YV97Qyd9Il1d
    \begin{tikzcd}
        GFC & {GFC'} \\
        C & {C'}
        \arrow["g"', from=2-1, to=2-2]
        \arrow["GFg", dashed, from=1-1, to=1-2]
        \arrow["{\eta_{C}}"', from=1-1, to=2-1]
        \arrow["{\eta_{C'}}", from=1-2, to=2-2]
    \end{tikzcd}
    \end{equation*}

    For morphism $k$, following from above, we have that since $F$ is fully faithful there is $a$ where $Fa=k$ and we define $G$ acting on the morphism $Fg$ to be $a=GFg$, thus $FGFg=k$. By the fact that $F$ is fully faithful, we can have the unique commutative diagram without $F$ in front, where now, we define isomorphism $\eta_C$ and $\eta_{C'}$ to be $F\eta_C=\phi$ and $F\eta_{C'}=\psi$. This gives us the RHS part. Thus, we have $\eta:G\circ F\to \operatorname{id}_C$ (the naturality condition follows from the diagram).

    $\boldsymbol{(\impliedby)}:$ Given equivalent of category, we have $F:\textbf{C}\rightarrow\textbf{D}$ and $G:\textbf{D}\rightarrow\textbf{C}$ and natural isomorphism as: $\eta:G\circ F\Rightarrow\operatorname{id}_C$ and $\varepsilon:F\circ G\Rightarrow\operatorname{id}_D$.

    \textit{(Faithful):} Given $f,f':C\to C'$ and assume that $Ff=Ff'$, then $GFf=GFf'$, but then by the naturality condition, we get two diagrams:

    \begin{equation*}
    % https://q.uiver.app/#q=WzAsOCxbMCwxLCJHRkMiXSxbMSwxLCJHRkMnIl0sWzAsMCwiQyJdLFsxLDAsIkMnIl0sWzMsMSwiR0ZDIl0sWzQsMSwiR0ZDJyJdLFszLDAsIkMiXSxbNCwwLCJDJyJdLFsyLDMsImYiLDAseyJzdHlsZSI6eyJib2R5Ijp7Im5hbWUiOiJkYXNoZWQifX19XSxbMSwzLCJcXGV0YV97Qyd9IiwyXSxbMCwyLCJcXGV0YV97Q30iXSxbMCwxLCJHRmYiLDJdLFs0LDYsIlxcZXRhX3tDfSJdLFs1LDcsIlxcZXRhX3tDJ30iLDJdLFs2LDcsImYnIiwwLHsic3R5bGUiOnsiYm9keSI6eyJuYW1lIjoiZGFzaGVkIn19fV0sWzQsNSwiR0ZmJyIsMl0sWzAsMiwiXFxjb25nIiwyXSxbMSwzLCJcXGNvbmciXSxbNCw2LCJcXGNvbmciLDJdLFs1LDcsIlxcY29uZyJdXQ==
    \begin{tikzcd}
        C & {C'} && C & {C'} \\
        GFC & {GFC'} && GFC & {GFC'}
        \arrow["f", dashed, from=1-1, to=1-2]
        \arrow["{\eta_{C'}}"', from=2-2, to=1-2]
        \arrow["{\eta_{C}}", from=2-1, to=1-1]
        \arrow["GFf"', from=2-1, to=2-2]
        \arrow["{\eta_{C}}", from=2-4, to=1-4]
        \arrow["{\eta_{C'}}"', from=2-5, to=1-5]
        \arrow["{f'}", dashed, from=1-4, to=1-5]
        \arrow["{GFf'}"', from=2-4, to=2-5]
        \arrow["\cong"', from=2-1, to=1-1]
        \arrow["\cong", from=2-2, to=1-2]
        \arrow["\cong"', from=2-4, to=1-4]
        \arrow["\cong", from=2-5, to=1-5]
    \end{tikzcd}
    \end{equation*}
    by the fact that $\eta$ is natural isomorphism and lemma above, we are required to have a unique map i.e $f=f'$.

    \textit{(Fully):} Given $g:FC\to FC'$, let' consider $Gg:GFC\to GFC'$, in which using the component of $\eta_C$ and $\eta_{C'}$, we can find the morphism $f'=\eta_{C'}\circ Gg\circ \eta_C^{-1}$ (LHS):

    \begin{equation*}
    % https://q.uiver.app/#q=WzAsOCxbMCwwLCJHRkMiXSxbMSwwLCJHRkMnIl0sWzAsMSwiQyJdLFsxLDEsIkMnIl0sWzMsMCwiRkdGQyJdLFs0LDEsIkZDJyJdLFszLDEsIkZDIl0sWzQsMCwiRkdGQyciXSxbMCwxLCJHZyJdLFswLDIsIlxcZXRhX0MiLDJdLFsxLDMsIlxcZXRhX3tDJ30iXSxbMiwzLCJmJyIsMix7InN0eWxlIjp7ImJvZHkiOnsibmFtZSI6ImRhc2hlZCJ9fX1dLFs0LDcsIkZHZyIsMCx7InN0eWxlIjp7ImJvZHkiOnsibmFtZSI6ImRhc2hlZCJ9fX1dLFs0LDYsIlxcdmFyZXBzaWxvbl97RkN9IiwyXSxbNiw1LCJnIiwyXSxbNyw1LCJcXHZhcmVwc2lsb25fe0ZDJ30iXV0=
    \begin{tikzcd}
        GFC & {GFC'} && FGFC & {FGFC'} \\
        C & {C'} && FC & {FC'}
        \arrow["Gg", from=1-1, to=1-2]
        \arrow["{\eta_C}"', from=1-1, to=2-1]
        \arrow["{\eta_{C'}}", from=1-2, to=2-2]
        \arrow["{f'}"', dashed, from=2-1, to=2-2]
        \arrow["FGg", dashed, from=1-4, to=1-5]
        \arrow["{\varepsilon_{FC}}"', from=1-4, to=2-4]
        \arrow["g"', from=2-4, to=2-5]
        \arrow["{\varepsilon_{FC'}}", from=1-5, to=2-5]
    \end{tikzcd}
    \end{equation*}

    On the other hand (RHS), we will also consider the naturality condition of $\varepsilon$ on $g$. We will claim that $Ff'=g$. Before we doing that, for any $f:C\to C'$, using functor $F$ on the naturality condition of $\eta$ and using naturality condition on $Ff$, the following diagram commutes:

    \begin{equation*}
    % https://q.uiver.app/#q=WzAsNixbMiwwLCJGQyJdLFsxLDAsIkZHRkMiXSxbMSwxLCJGR0ZDIl0sWzAsMCwiRkMiXSxbMCwxLCJGQyciXSxbMiwxLCJGQyciXSxbMCw1LCJGZiJdLFswLDEsIkZcXGV0YV9DIiwyLHsic3R5bGUiOnsidGFpbCI6eyJuYW1lIjoiYXJyb3doZWFkIn0sImhlYWQiOnsibmFtZSI6Im5vbmUifX19XSxbMSwyLCJGR0ZmIiwxXSxbMSwzLCJcXHZhcmVwc2lsb25fe0ZDfV57LTF9IiwyLHsic3R5bGUiOnsidGFpbCI6eyJuYW1lIjoiYXJyb3doZWFkIn0sImhlYWQiOnsibmFtZSI6Im5vbmUifX19XSxbNSwyLCJGXFxldGFfe0MnfSIsMCx7InN0eWxlIjp7InRhaWwiOnsibmFtZSI6ImFycm93aGVhZCJ9LCJoZWFkIjp7Im5hbWUiOiJub25lIn19fV0sWzIsNCwiXFx2YXJlcHNpbG9uX3tGQyd9IiwwLHsic3R5bGUiOnsidGFpbCI6eyJuYW1lIjoiYXJyb3doZWFkIn0sImhlYWQiOnsibmFtZSI6Im5vbmUifX19XSxbMyw0LCJGZiIsMl1d
    \begin{tikzcd}
        FC & FGFC & FC \\
        {FC'} & FGFC & {FC'}
        \arrow["Ff", from=1-3, to=2-3]
        \arrow["{F\eta_C}"', tail reversed, no head, from=1-3, to=1-2]
        \arrow["FGFf"{description}, from=1-2, to=2-2]
        \arrow["{\varepsilon_{FC}^{-1}}"', tail reversed, no head, from=1-2, to=1-1]
        \arrow["{F\eta_{C'}}", tail reversed, no head, from=2-3, to=2-2]
        \arrow["{\varepsilon_{FC'}}", tail reversed, no head, from=2-2, to=2-1]
        \arrow["Ff"', from=1-1, to=2-1]
    \end{tikzcd}
    \end{equation*}

    Then we have that $Ff' = F\eta_{C'}\circ FGg\circ F\eta_C^{-1}$, then since $FGg=\varepsilon_{FC'}^{-1}\circ g \circ \varepsilon_{FC}$, we have: $F\eta_{C'}^{-1}\circ Ff'\circ F\eta_C = \varepsilon_{FC'}^{-1}\circ g \circ \varepsilon_{FC}$ or, which we have used the commutative diagram above.
    \begin{equation*}
    \begin{aligned}
        g &= \varepsilon_{FC'} \circ F\eta_{C'}^{-1}\circ \big(Ff'\circ F\eta_C\circ\varepsilon_{FC}^{-1}\big)\\
        &= \varepsilon_{FC'} \circ F\eta_{C'}^{-1}\circ F\eta_{C'}\circ \varepsilon_{FC'}^{-1}\circ Ff' = Ff'
    \end{aligned}
    \end{equation*}

    \textit{(Essentially Surjective):} Using the component of $\varepsilon$, we can have the object $GD$ such that $FGC\xrightarrow{\cong} D$, for any object $D$ in $\textbf{D}$.
\end{proof}


\section{Representables Functors/Yoneda}

Since we have familiarize ourselves with functor and natural transformation, now we are going to use these formalism to state Yoneda lemma, which can be seen as a generalization of proposition \ref{prop:iso-obj-in}. But before we are doing that, let's consider a special type of functor call representable

\begin{definition}{\textbf{(Representable Functors)}}
    Given category $\textbf{C}$, with the functor $F:\textbf{C}\rightarrow\textbf{Set}$, this functor is called representable if it is naturally isomorphic to the functor $\operatorname{Hom}_\textbf{C}(S,-):\textbf{C}\rightarrow \textbf{Set}$ for some object $S$ (which is called representing object).

    Note that if we have the functor $F:\textbf{C}^\text{op}\rightarrow\textbf{Set}$ with is naturally isomorphic to $\operatorname{Hom}_\textbf{C}(-, S):\textbf{C}\rightarrow \textbf{Set}$, then we have representable presheaf
\end{definition}

The motivation for Yoneda lemma can be stated as follows:

\begin{remark}
    Given our definition of representable functors, we have 2 further question (which would be answered by Yoneda results) as:
    \begin{itemize}
        \item Is the representing object unique ? 
        \item With the definition of representation above, given objects $X$ and $Y$ of $\textbf{C}$, if $\operatorname{Hom}_\textbf{C}(S,X)$ is naturally isomorphic to $\operatorname{Hom}_\textbf{C}(S,Y)$, can we say that $X$ and $Y$ are isomorphic ? 
    \end{itemize}
    Again, the answer for the second question is already given, but only to a functor of $\operatorname{Hom}_\textbf{C}(-,S)$. We can stated a result for more general functors.
\end{remark}

\begin{lemma}{\textbf{(Yoneda Lemma)}}
    With category $\textbf{C}$, an its object $X$ and presheaf $F:\textbf{C}^\text{op}\rightarrow\textbf{Set}$, then the map:
    \begin{equation*}
        \operatorname{yo}:\operatorname{Hom}_{[\textbf{C}^\text{op}, \textbf{Set}]}\Big( \operatorname{Hom}_\textbf{C}(-, X), F \Big) \xrightarrow{\cong} FX
    \end{equation*}
    
    assigning to natural transformation $\alpha:\operatorname{Hom}_\textbf{C}(-,X)\Rightarrow F$ an element $\alpha_X(\operatorname{id}_X)\in FX$. The assignment is bijective and it is natural both in $X$ and $F$.
    \checkproof Check the opposite direction again.
\end{lemma}

\begin{dem}
\begin{proof}

    Observe that, the use of $\operatorname{id}_X$ as the ``identification'' can also be seen, in a more concrete manner, in the proof of proposition \ref{prop:iso-obj-in}.

    Let's consider why only $\alpha_X$ acting on $\operatorname{id}_X$ is enough to determines the whole component of $\alpha_Y$. Given any $\alpha:\operatorname{Hom}_\textbf{C}(-,X)\Rightarrow F$, it is clear that its component is given by $\alpha_X:\operatorname{Hom}_\textbf{C}(X,X)\Rightarrow FX$, where we have $\operatorname{Hom}_\textbf{C}(X, X)\ni \operatorname{id}_X\mapsto a\in FX$. If we are given $f^\text{op}:X\to Y$, then $\alpha_Y(f)$, can be found by consider the naturality condition on $f^\text{op}$ (The use of opposite category and presheaf is given in remark \ref{remark:presheaf-opposite}):
    \begin{equation*}
    % https://q.uiver.app/#q=WzAsNCxbMCwwLCJcXG9wZXJhdG9ybmFtZXtIb219X1xcdGV4dGJme0N9KFgsWCkiXSxbMCwxLCJGWCJdLFsyLDAsIlxcb3BlcmF0b3JuYW1le0hvbX1fXFx0ZXh0YmZ7Q30oWSxYKSJdLFsyLDEsIkZZIl0sWzAsMSwiXFxhbHBoYV9YIiwyXSxbMiwzLCJcXGFscGhhX1kiXSxbMSwzLCJGZl5cXHRleHR7b3B9IiwyXSxbMCwyLCIoLVxcY2lyYyBmKSJdXQ==
    \begin{tikzcd}
        {\operatorname{Hom}_\textbf{C}(X,X)} && {\operatorname{Hom}_\textbf{C}(Y,X)} \\
        FX && FY
        \arrow["{\alpha_X}"', from=1-1, to=2-1]
        \arrow["{\alpha_Y}", from=1-3, to=2-3]
        \arrow["{Ff^\text{op}}"', from=2-1, to=2-3]
        \arrow["{(-\circ f)}", from=1-1, to=1-3]
    \end{tikzcd}
    \end{equation*}
    Note that $\operatorname{Hom}_\textbf{C}(-,X)[f^\text{op}] =(-\circ f)$. Then we can see that $Ff^\text{op}\circ \alpha_X = \alpha_Y\circ(-\circ f)$ or, when plugging $\operatorname{id}_X$ into both sides $Ff^\text{op}(\alpha_X(\operatorname{id}_X))=\alpha_Y(f)$. Therefore, it is enough just to consider how $\alpha_X$ acts on $\operatorname{id}_X$, since the rest of the function can be derived using the steps here.
    

    \textbf{(Bijective):} Let's show that $\operatorname{yo}$ is both injective and surjective.
    
    \textit{Injectivity:} Recall that $\operatorname{yo}:\alpha\mapsto\alpha_X(\operatorname{id}_X)$. Thus if $\operatorname{yo}(\alpha)=\operatorname{yo}(\alpha')$ i.e $\alpha_X'(\operatorname{id}_X)=\alpha_X(\operatorname{id}_X)$, then by the observation we have above, for any $f^\text{op}:X\to Y$, we have $\alpha_Y(f)=\alpha_Y'(f)$ i.e $\alpha=\alpha'$.

    \textit{Surjectivity:} Given $a\in FX$, we will construct the natural transformation $\alpha:\operatorname{Hom}_\textbf{C}(-,X)\Rightarrow F$, where its components are  $\alpha_Y:\operatorname{Hom}_\textbf{C}(Y, X)\to FY$, where $\alpha_Y(f_1)=Ff^\text{op}_1(\alpha_X(\operatorname{id}_X))$ for $f_1:Y\to X$ and $\alpha_Z:\operatorname{Hom}_\textbf{C}(Z, X)\to FZ$, where $\alpha_Z(f_2)=Ff^\text{op}_2(\alpha_X(\operatorname{id}_X))$ for $f_2:Z\to X$ and $\alpha_X(\operatorname{id}_X)=a$
    
    Then to be a natural transformation, the following diagram has to commute, given $g^\text{op}:Y\to Z$

    \begin{equation*}
    % https://q.uiver.app/#q=WzAsNCxbMCwwLCJcXG9wZXJhdG9ybmFtZXtIb219X1xcdGV4dGJme0N9KFksIFgpIl0sWzAsMSwiRlkiXSxbMSwwLCJcXG9wZXJhdG9ybmFtZXtIb219X1xcdGV4dGJme0N9KFosIFgpIl0sWzEsMSwiRloiXSxbMSwzLCJGZ15cXHRleHR7b3B9IiwyXSxbMiwzLCJcXGFscGhhX1kiXSxbMCwxLCJcXGFscGhhX1giLDJdLFswLDIsIigtXFxjaXJjIGcpIl1d
    \begin{tikzcd}
        {\operatorname{Hom}_\textbf{C}(Y, X)} & {\operatorname{Hom}_\textbf{C}(Z, X)} \\
        FY & FZ
        \arrow["{Fg^\text{op}}"', from=2-1, to=2-2]
        \arrow["{\alpha_Y}", from=1-2, to=2-2]
        \arrow["{\alpha_X}"', from=1-1, to=2-1]
        \arrow["{(-\circ g)}", from=1-1, to=1-2]
    \end{tikzcd}
    \end{equation*}

    That is, we have 
    $$\alpha_Y(f_1\circ g) = F(f_1\circ g)^\text{op}(a) = F(g^\text{op}\circ f^\text{op})(a)=Fg^\text{op} \circ Ff_1^\text{op}(a)$$ and $Fg^\text{op}\circ \alpha_X(f_1) = Fg^\text{op} \circ Ff_1^\text{op}(a)$. Thus, $Fg^\text{op}\circ\alpha_X=\alpha_Y\circ(-\circ g)$ as needed.

    \textbf{(Natural on $X$):} To show this, we are required to have the natural transformation of 
    \begin{equation*}
    \begin{aligned}
        \beta:\operatorname{Hom}_{[\textbf{C}^\text{op}, \textbf{Set}]}&\Big( \operatorname{Hom}_\textbf{C}(-, \bigcirc), F \Big)\Rightarrow F\bigcirc\\
        \text{ where }&\operatorname{Hom}_{[\textbf{C}^\text{op}, \textbf{Set}]}\Big( \operatorname{Hom}_\textbf{C}(-, \bigcirc), F \Big):\textbf{C}^\text{op}\to \textbf{Set}
    \end{aligned}
    \end{equation*}
    the object placeholder for the natural transformation is $\bigcirc$, whose components are given to be $\beta_X=\operatorname{yo}_X$ we added subscript $X$ to $\operatorname{yo}$ to indicate where it belongs. Thus, we will use a similar trick, given $f^\text{op}:X\to Y$ we will set $\operatorname{Hom}_{[\textbf{C}^\text{op}, \textbf{Set}]}\Big( \operatorname{Hom}_\textbf{C}(-, X), F \Big)[f^\text{op}]=(-\circ \phi)$, where $\phi:\operatorname{Hom}_\textbf{C}(-, Y)\Rightarrow\operatorname{Hom}_\textbf{C}(-, X)$, whose component is $\phi_A=(f\circ-)$, and $\circ$ is the vertical composition.  Note that $\phi$ is a natural transformation, as the naturality condition, given $g^\text{op}:A\to B$ is shown to be:
    \begin{equation*}
    % https://q.uiver.app/#q=WzAsNCxbMCwxLCJcXG9wZXJhdG9ybmFtZXtIb219X1xcdGV4dGJme0N9KEEsIFgpIl0sWzAsMCwiXFxvcGVyYXRvcm5hbWV7SG9tfV9cXHRleHRiZntDfShBLCBZKSJdLFsyLDAsIlxcb3BlcmF0b3JuYW1le0hvbX1fXFx0ZXh0YmZ7Q30oQiwgWSkiXSxbMiwxLCJcXG9wZXJhdG9ybmFtZXtIb219X1xcdGV4dGJme0N9KEIsIFgpIl0sWzAsMSwiKGZcXGNpcmMtKSIsMCx7InN0eWxlIjp7InRhaWwiOnsibmFtZSI6ImFycm93aGVhZCJ9LCJoZWFkIjp7Im5hbWUiOiJub25lIn19fV0sWzMsMiwiKGZcXGNpcmMtKSIsMix7InN0eWxlIjp7InRhaWwiOnsibmFtZSI6ImFycm93aGVhZCJ9LCJoZWFkIjp7Im5hbWUiOiJub25lIn19fV0sWzAsMywiKC1cXGNpcmMgZykiLDJdLFsxLDIsIigtXFxjaXJjIGcpIl1d
    \begin{tikzcd}
        {\operatorname{Hom}_\textbf{C}(A, Y)} && {\operatorname{Hom}_\textbf{C}(B, Y)} \\
        {\operatorname{Hom}_\textbf{C}(A, X)} && {\operatorname{Hom}_\textbf{C}(B, X)}
        \arrow["{(f\circ-)}", tail reversed, no head, from=2-1, to=1-1]
        \arrow["{(f\circ-)}"', tail reversed, no head, from=2-3, to=1-3]
        \arrow["{(-\circ g)}"', from=2-1, to=2-3]
        \arrow["{(-\circ g)}", from=1-1, to=1-3]
    \end{tikzcd}
    \end{equation*}
    This is exactly the natural condition within definition \ref{def:prob-action-nat}. Now, let's consider the pre-composition of $\phi$ in action. Given $\alpha: \operatorname{Hom}_\textbf{C}(-, X)\Rightarrow F$, we have $(\alpha\circ \phi):\operatorname{Hom}_\textbf{C}(-, Y)\Rightarrow\operatorname{Hom}_\textbf{C}(-, X)\Rightarrow F$ and so its component is $(\alpha\circ \phi)_X(h)=\alpha_X(f\circ h)$ where $h:X\to Y$. 
    
    To show that the map is natural on $\bigcirc$, we have to show that the following diagram commutes:

    \begin{equation*}
    % https://q.uiver.app/#q=WzAsNCxbMCwwLCJcXG9wZXJhdG9ybmFtZXtIb219X3tbXFx0ZXh0YmZ7Q31eXFx0ZXh0e29wfSwgXFx0ZXh0YmZ7U2V0fV19XFxCaWcoIFxcb3BlcmF0b3JuYW1le0hvbX1fXFx0ZXh0YmZ7Q30oLSwgWCksIEYgXFxCaWcpIl0sWzAsMSwiRlgiXSxbMiwwLCJcXG9wZXJhdG9ybmFtZXtIb219X3tbXFx0ZXh0YmZ7Q31eXFx0ZXh0e29wfSwgXFx0ZXh0YmZ7U2V0fV19XFxCaWcoIFxcb3BlcmF0b3JuYW1le0hvbX1fXFx0ZXh0YmZ7Q30oLSwgWSksIEYgXFxCaWcpIl0sWzIsMSwiRlkiXSxbMCwxLCJcXGNvbmciXSxbMiwzLCJcXGNvbmciLDJdLFsxLDMsIkZmXlxcdGV4dHtvcH0iLDJdLFswLDIsIigtXFxjaXJjIFxccGhpKSJdLFswLDEsIlxcb3BlcmF0b3JuYW1le3lvfV9YIiwyXSxbMiwzLCJcXG9wZXJhdG9ybmFtZXt5b31fWSJdXQ==
    \begin{tikzcd}
        {\operatorname{Hom}_{[\textbf{C}^\text{op}, \textbf{Set}]}\Big( \operatorname{Hom}_\textbf{C}(-, X), F \Big)} && {\operatorname{Hom}_{[\textbf{C}^\text{op}, \textbf{Set}]}\Big( \operatorname{Hom}_\textbf{C}(-, Y), F \Big)} \\
        FX && FY
        \arrow["\cong", from=1-1, to=2-1]
        \arrow["\cong"', from=1-3, to=2-3]
        \arrow["{Ff^\text{op}}"', from=2-1, to=2-3]
        \arrow["{(-\circ \phi)}", from=1-1, to=1-3]
        \arrow["{\operatorname{yo}_X}"', from=1-1, to=2-1]
        \arrow["{\operatorname{yo}_Y}", from=1-3, to=2-3]
    \end{tikzcd}
    \end{equation*}

    That is, for any $\alpha:\operatorname{Hom}_\textbf{C}(-, X)\Rightarrow F$ on both 2 paths on the diagram, we have $$Ff^{\operatorname{op}}(\operatorname{yo}_X(\alpha)) = Ff^{\operatorname{op}}(\alpha_X(\operatorname{id}_X)) = \alpha_Y(f) \quad\text{and} \quad \operatorname{yo}_Y(\alpha\circ\phi) = (\alpha\circ\phi)_Y(\operatorname{id}_Y) = \alpha_Y(f)$$Thus, $Ff^{\operatorname{op}}\circ \operatorname{yo}_X=\operatorname{yo}_Y\circ(-\circ\phi)$, as needed

    \textbf{(Natural on $F$):} On the other hand, we will have to consider the following natural transformation:

    \begin{equation*}
    \begin{aligned}
        \gamma:\operatorname{Hom}_{[\textbf{C}^\text{op}, \textbf{Set}]}&\Big( \operatorname{Hom}_\textbf{C}(-, X), \square \Big)\Rightarrow \square X \\   
        \text{ where }&\operatorname{Hom}_{[\textbf{C}^\text{op}, \textbf{Set}]}\Big( \operatorname{Hom}_\textbf{C}(-, X), \square \Big): [\textbf{C}^\text{op}, \textbf{Set}]\to \textbf{Set}
    \end{aligned}
    \end{equation*}

    again, the object placeholder for the natural transformation is $\square$, whose components are given to be $\gamma_F=\operatorname{yo}_F$ where $F$ is the presheaf (note that we have abuse the notation of $\operatorname{yo}$ abit). Now, to show that it is natural on $\square$, given $\psi:F\Rightarrow G$, where $F,G:\textbf{C}^\text{op}\to\textbf{Set}$

    \begin{equation*}
    % https://q.uiver.app/#q=WzAsNCxbMCwwLCJcXG9wZXJhdG9ybmFtZXtIb219X3tbXFx0ZXh0YmZ7Q31eXFx0ZXh0e29wfSwgXFx0ZXh0YmZ7U2V0fV19XFxCaWcoIFxcb3BlcmF0b3JuYW1le0hvbX1fXFx0ZXh0YmZ7Q30oLSwgWCksIEYgXFxCaWcpIl0sWzAsMSwiRlgiXSxbMiwwLCJcXG9wZXJhdG9ybmFtZXtIb219X3tbXFx0ZXh0YmZ7Q31eXFx0ZXh0e29wfSwgXFx0ZXh0YmZ7U2V0fV19XFxCaWcoIFxcb3BlcmF0b3JuYW1le0hvbX1fXFx0ZXh0YmZ7Q30oLSwgWCksIEcgXFxCaWcpIl0sWzIsMSwiR1giXSxbMCwxLCJcXGNvbmciXSxbMiwzLCJcXGNvbmciLDJdLFsxLDMsIlxccHNpX1giLDJdLFswLDIsIihcXHBzaVxcY2lyYy0pIl0sWzIsMywiXFxvcGVyYXRvcm5hbWV7eW99X0ciXSxbMCwxLCJcXG9wZXJhdG9ybmFtZXt5b31fRiIsMl1d
    \begin{tikzcd}
        {\operatorname{Hom}_{[\textbf{C}^\text{op}, \textbf{Set}]}\Big( \operatorname{Hom}_\textbf{C}(-, X), F \Big)} && {\operatorname{Hom}_{[\textbf{C}^\text{op}, \textbf{Set}]}\Big( \operatorname{Hom}_\textbf{C}(-, X), G \Big)} \\
        FX && GX
        \arrow["\cong", from=1-1, to=2-1]
        \arrow["\cong"', from=1-3, to=2-3]
        \arrow["{\psi_X}"', from=2-1, to=2-3]
        \arrow["{(\psi\circ-)}", from=1-1, to=1-3]
        \arrow["{\operatorname{yo}_G}", from=1-3, to=2-3]
        \arrow["{\operatorname{yo}_F}"', from=1-1, to=2-1]
    \end{tikzcd}
    \end{equation*}

    where we use the vertical composition, which doesn't require any opposite category, and of course, like remark \ref{remark:presheaf-opposite}, we have $\operatorname{Hom}_{[\textbf{C}^\text{op}, \textbf{Set}]}\Big( \operatorname{Hom}_\textbf{C}(-, X), F \Big)[\psi]=(\psi\circ-)$, being vertical composition. That is, for any $\alpha:\operatorname{Hom}_\textbf{C}(-, X)\Rightarrow F$ on both 2 paths on the diagram, we have: $$(\psi_X \circ \operatorname{yo}_F)(\alpha) = \psi_X(\alpha_X(\operatorname{id}_X))\quad\text{ and }\quad \operatorname{yo}_G(\psi\circ \alpha) = (\psi\circ \alpha)_X(\operatorname{id}_X) = \psi_X(\alpha_X(\operatorname{id}_X))$$Thus: $\psi_X \circ \operatorname{yo}_F=\operatorname{yo}_G\circ(\psi\circ -)$, as needed.
\end{proof}
\end{dem}


\begin{theorem}{\textbf{(Yoneda Embedding)}}
    Given a category $\textbf{C}$ with objects $X$ and $Y$, then there is a bijection between the following sets:
    $$
    \operatorname{Hom}_\textbf{C}(X, Y) \cong \operatorname{Hom}_{[\textbf{C}^\text{op}, \textbf{Set}]}\Big( \operatorname{Hom}_{\textbf{C}}(-, X), \operatorname{Hom}_{\textbf{C}}(-, Y) \Big)
    $$
\end{theorem}

\begin{dem}
\begin{proof}
    With the Yoneda lemma above, we can set $F:\operatorname{Hom}_{\textbf{C}^\text{op}}(-, Y)$, and the proof is complete.
\end{proof}
\end{dem}

We also have the following dual result

\begin{corollary}{\textbf{(Dual Yoneda Embedding)}}
    Given a category $\textbf{C}$ with objects $X$ and $Y$, then there is a bijection between the following sets:
    \begin{equation*}
        \operatorname{Hom}_\textbf{C}(X, Y) \cong \operatorname{Hom}_{[\textbf{C}, \textbf{Set}]}\Big( \operatorname{Hom}_{\textbf{C}}(Y, -), \operatorname{Hom}_{\textbf{C}}(X, -) \Big)
    \end{equation*}
\end{corollary}

We have the following corollary on how Yoneda lemma perform, which will give us the proper interpretation of universal properties, and a reworking of proposition \ref{prop:iso-obj-in}

\begin{corollary}
    \label{coro:iso-hom-functor-iso-represent}
    Given objects $X$ and $Y$ in the category $\textbf{C}$, we have that $X\cong Y$ iff:
    \begin{itemize}
        \item The functor (or, presheaf) that they represent are naturally isomorphic.
        \item For any object $S$ of $\textbf{C}$, the set $\operatorname{Hom}_\textbf{C}(S, X)\cong\operatorname{Hom}_\textbf{C}(S, Y)$ (or, $\operatorname{Hom}_\textbf{C}(X, S)\cong\operatorname{Hom}_\textbf{C}(Y, S)$). i.e if $X$ and $Y$ are indistinguishable by $S$, for every $S$ in $\textbf{C}$
    \end{itemize}
\end{corollary}

\begin{dem}
\begin{proof}
    $\boldsymbol{(\implies)}:$ Let $f:X\to Y$ be the isomorphism between $X$ and $Y$, then by Yoneda embedding there is $\alpha:\operatorname{Hom}_{\textbf{C}}(-, X)\Rightarrow \operatorname{Hom}_{\textbf{C}}(-, Y)$, so that $\alpha_X(\operatorname{id}_X)=f$. Let's show that $\alpha$ is naturally isomorphic. 

    Note that the component of $\alpha_A$ for any $A$ in $\textbf{C}$ is $(f\circ -)$, which is isomorphism because it has an inverse of $(f^{-1}\circ-)$, where $f^{-1}$ is an inverse of $f$ that exists because $f$ is isomorphism, and this also works in opposite direction And so, all the components of $\alpha$ are isomorphism i.e $\alpha$ is natural isomorphism.


    $\boldsymbol{(\impliedby)}:$ Given $\alpha:\operatorname{Hom}_{\textbf{C}}(-, X)\Rightarrow \operatorname{Hom}_{\textbf{C}}(-, Y)$ to be natural isomorphism, by Yoneda embedding there is $f\in\operatorname{Hom}(X, Y)$ in which $f=\alpha_X(\operatorname{id}_X)$. We will show that this $f$ is isomorphism. We note that its inverse $g$ would be $\alpha_X^{-1}(\operatorname{id}_X)$ where $\alpha_X^{-1}$ is an inverse of $\alpha_X$ as it is an isomorphism. That is because:

    \begin{equation*}
        g\circ f = \alpha_X^{-1}(\operatorname{id}_X)\circ\alpha_X(\operatorname{id}_X) = \operatorname{Id}(\operatorname{id}_X) = \operatorname{id}_X
    \end{equation*}

    and this also works in opposite direction, so $f$ is an isomorphism. Similar proof also works on opposite direction.

\end{proof}
\end{dem}

Finally, we can then us this corollary to formally define the universal properties (which will be used to define all the construction that we have seen so far i.e product, internal-hom, pull-back, etc.) as:

\begin{definition}\textbf{(Universal Properties)}
    Given an object $X$ of category $\textbf{C}$, then a universal properties of $X$ consists of a functor $F:\textbf{C}\rightarrow\textbf{Set}$ and natural isomorphism $\operatorname{Hom}_\textbf{C}(X, -)\Rightarrow F$. Or, a presheaf $P:\textbf{C}^\text{op}\rightarrow\textbf{Set}$ and natural isomorphism $\operatorname{Hom}_\textbf{C}(-, X)\Rightarrow P$ 
\end{definition}

\begin{remark}{\textbf{(Example of Universal Properties)}}
    \label{remark:universal-prop}
    Let's consider the definition of universal properties when applied onto the concept of product. Given two objects $Y$ and $X$ of $\textbf{C}$, we can define their product via presheaf $P:\operatorname{Hom}_\textbf{C}(-,X)\times\operatorname{Hom}_\textbf{C}(-,Y):\textbf{C}^\text{op}\to\textbf{Set}$. Let's consider its action:
    \begin{itemize}
        \item \textit{Object:} Given object $S$, we have the set $\operatorname{Hom}_\textbf{C}(S,X)\times\operatorname{Hom}_\textbf{C}(S,Y)$ i.e a set of arrows of  $X\leftarrow S\rightarrow Y$
        \item \textit{Arrow:} Given arrow $f^\text{op}:S\to T$ we have the (lifted) map, given to be: \begin{equation*}
        % https://q.uiver.app/#q=WzAsMixbMCwwLCJcXG9wZXJhdG9ybmFtZXtIb219X1xcdGV4dGJme0N9KFMsWClcXHRpbWVzXFxvcGVyYXRvcm5hbWV7SG9tfV9cXHRleHRiZntDfShTLFkpIl0sWzAsMSwiXFxvcGVyYXRvcm5hbWV7SG9tfV9cXHRleHRiZntDfShULFgpXFx0aW1lc1xcb3BlcmF0b3JuYW1le0hvbX1fXFx0ZXh0YmZ7Q30oVCxZKSJdLFswLDEsIigtXFxjaXJjIGYpXFx0aW1lcygtXFxjaXJjIGYpIiwxXV0=
        \begin{tikzcd}
            {\operatorname{Hom}_\textbf{C}(S,X)\times\operatorname{Hom}_\textbf{C}(S,Y)} \\
            {\operatorname{Hom}_\textbf{C}(T,X)\times\operatorname{Hom}_\textbf{C}(T,Y)}
            \arrow["{(-\circ f)\times(-\circ f)}"{description}, from=1-1, to=2-1]
        \end{tikzcd}
        \end{equation*}
        as we have seen in the earlier example.
    \end{itemize}
    This gives us the following question: does the presheaf $P$ representable ? i.e we follows the definition of universal property above. That is we are intersted in an object $Z$, whose related functor is in natural isomorphism to $P$ that is $\operatorname{Hom}_\textbf{C}(-,Z)\Rightarrow\operatorname{Hom}_\textbf{C}(-,X)\times\operatorname{Hom}_\textbf{C}(-,Y)$. Let's considder its component and the naturality condition:
    \begin{itemize}
        \item \textit{Component:} Given object $A$, the component of the natural isomorphism is $\operatorname{Hom}_\textbf{C}(A,Z)\xrightarrow{\cong}\operatorname{Hom}_\textbf{C}(A,X)\times\operatorname{Hom}_\textbf{C}(A,Y)$. That is given an arrow $f:A\to Z$ there is a unique corresponding pair of arrows $q_1:A\to X$ and $q_2:A\to Y$. From the Yoneda lemma, one might also consider component at $Z$, then the isomorphism $\operatorname{Hom}_\textbf{C}(Z,Z)\xrightarrow{\cong}\operatorname{Hom}_\textbf{C}(Z,X)\times\operatorname{Hom}_\textbf{C}(Z,Y)$ maps $\operatorname{id}_Z\mapsto(p_1, p_2)$
        \item \textit{Naturality Condition:} Given the arrow $f^\text{op}:Z\to A$ (coincide with the $f$ above), then the LHS diagram commutes:
        \begin{equation*}
        % https://q.uiver.app/#q=WzAsOCxbMCwwLCJcXG9wZXJhdG9ybmFtZXtIb219X1xcdGV4dGJme0N9KFosWikiXSxbMCwxLCJcXG9wZXJhdG9ybmFtZXtIb219X1xcdGV4dGJme0N9KFosWClcXHRpbWVzXFxvcGVyYXRvcm5hbWV7SG9tfV9cXHRleHRiZntDfShaLFkpIl0sWzIsMSwiXFxvcGVyYXRvcm5hbWV7SG9tfV9cXHRleHRiZntDfShBLFgpXFx0aW1lc1xcb3BlcmF0b3JuYW1le0hvbX1fXFx0ZXh0YmZ7Q30oQSxZKSJdLFsyLDAsIlxcb3BlcmF0b3JuYW1le0hvbX1fXFx0ZXh0YmZ7Q30oQSxaKSJdLFs1LDAsIkEiXSxbNSwxLCJaIl0sWzQsMSwiWCJdLFs2LDEsIlkiXSxbMywyLCJcXGNvbmciXSxbMCwxLCJcXGNvbmciLDJdLFsxLDIsIigtXFxjaXJjIGYpXFx0aW1lcygtXFxjaXJjIGYpIiwyXSxbMCwzLCIoLVxcY2lyYyBmKSJdLFs1LDYsInBfMSJdLFs1LDcsInBfMiIsMl0sWzQsNSwiZiIsMCx7InN0eWxlIjp7ImJvZHkiOnsibmFtZSI6ImRhc2hlZCJ9fX1dLFs0LDYsInFfMSIsMl0sWzQsNywicV8yIl1d
        \begin{tikzcd}
            {\operatorname{Hom}_\textbf{C}(Z,Z)} && {\operatorname{Hom}_\textbf{C}(A,Z)} &&& A \\
            {\operatorname{Hom}_\textbf{C}(Z,X)\times\operatorname{Hom}_\textbf{C}(Z,Y)} && {\operatorname{Hom}_\textbf{C}(A,X)\times\operatorname{Hom}_\textbf{C}(A,Y)} && X & Z & Y
            \arrow["\cong", from=1-3, to=2-3]
            \arrow["\cong"', from=1-1, to=2-1]
            \arrow["{(-\circ f)\times(-\circ f)}"', from=2-1, to=2-3]
            \arrow["{(-\circ f)}", from=1-1, to=1-3]
            \arrow["{p_1}", from=2-6, to=2-5]
            \arrow["{p_2}"', from=2-6, to=2-7]
            \arrow["f", dashed, from=1-6, to=2-6]
            \arrow["{q_1}"', from=1-6, to=2-5]
            \arrow["{q_2}", from=1-6, to=2-7]
        \end{tikzcd}
        \end{equation*}
        Especially when we start with $\operatorname{id}_Z$ on the upper LHS. We have the following conditions: $p_1\circ f=q_1$ and $p_2\circ f = q_2$, or on the RHS diagram.
    \end{itemize}
    Thus, the natural isomorphism will gives us the universal properties of the product i.e unique $f$ that makes the projection maps commutes. This idea will be generalized further with the notion of limit and colimits.
\end{remark}



\section{Limits}

As noted in the example of universal property, we can unify all the special ``construction'' under the term of limit. To do this we have to define what the diagram is, formally. After we have done with formulating the limit, we are going to look into its properties, and then end this section with some more facts on specific constructions.

\subsection{Defining Limit}

\begin{definition}{\textbf{(Constant Diagram)}}
    Given a category $\textbf{C}$ with object $X$, and small category $\textbf{J}$, we define \textit{constant diagram} at $X$ to be a diagram indexed by $\textbf{J}$ being the functor $X:\textbf{J}\rightarrow\textbf{C}$, where:
    \begin{itemize}
        \item All objects of $\textbf{J}$ are assigned to object $X$
        \item All morphism of $\textbf{J}$, they are assigned to identity morphism $\operatorname{id}_X$
    \end{itemize}
\end{definition}

\begin{definition}{\textbf{(Cone/Co-Cone)}}
    Given a small category $\textbf{J}$ with a category $\textbf{C}$ with object $X$, let $F:\textbf{J}\rightarrow\textbf{C}$ be the diagram, then:
    \begin{itemize}
        \item A \textit{cone} over $F$ with tips $X$ is a natural transformation from the constant diagram at $X$ to functor $F$.
        \item A \textit{co-cone} with bottom $X$ is a natural transformation from the functor $F$ to constant diagram at $X$.
    \end{itemize}
    Explicitly, for each object $J$ of $\textbf{J}$, we have the component $\alpha_J:X\rightarrow FJ$ such that for any morphism $m:J\rightarrow J'$, we have, following commutative diagram:
    \begin{equation*}
    % https://q.uiver.app/#q=WzAsMyxbMCwxLCJGSiJdLFsyLDEsIkZKJyJdLFsxLDAsIlgiXSxbMCwxLCJGbSIsMl0sWzIsMCwiXFxhbHBoYV9KIiwyXSxbMiwxLCJcXGFscGhhX3tKJ30iXV0=
    \begin{tikzcd}
        & X \\
        FJ && {FJ'}
        \arrow["Fm"', from=2-1, to=2-3]
        \arrow["{\alpha_J}"', from=1-2, to=2-1]
        \arrow["{\alpha_{J'}}", from=1-2, to=2-3]
    \end{tikzcd}
    \end{equation*}
\end{definition}

\begin{definition}{\textbf{(Functor of a Cone/Co-Cone)}}
    Given a diagram $F:\textbf{J}\rightarrow\textbf{C}$ be the diagram, in which we construct the \textit{pre-sheaf of cone} $\operatorname{Cone}(-,F):\textbf{C}^\text{op}\rightarrow\textbf{Set}$ as:
    \begin{itemize}
        \item Given object $X$ of $\textbf{C}$, it maps the set of cones $\operatorname{Cone}(X,F)$ over $F$ with tip $X$
        \item Given the map $f:X\rightarrow Y$, we have the function $\operatorname{Cone}(Y,F)\rightarrow\operatorname{Cone}(X,F)$ such that:
        \begin{itemize}
            \item Given a cone $\alpha:Y\Rightarrow F$ with components $\alpha_J:Y\rightarrow FJ$ for each object $J$ of $\textbf{J}$
            \item We have $\alpha\circ f:X\Rightarrow F$ in which its components are defined as $(\alpha\circ f)_J:=\alpha_J\circ f$
        \end{itemize}
    \end{itemize}
    The functor $\operatorname{Cone}(F,-):\textbf{C}\rightarrow\textbf{Set}$ is defined in similar manners, and is called \textit{co-cone}
\end{definition}

With the cone defined, we can used the pattern from remark \ref{remark:universal-prop} i.e finding the object that \textit{represents} the cone/co-cone itself i.e universal property. This leads to the notion of limit.

\begin{definition}{\textbf{(Limit/Co-Limit)}}
    Given the diagram $F:\textbf{J}\rightarrow\textbf{C}$ be the diagram:
    \begin{itemize}
        \item A \textit{limit} of $F$, if it exists, is an object $\lim F$ of $\textbf{C}$ representing a pre-sheaf $\operatorname{Cone}(-,F):\textbf{C}^\text{op}\rightarrow\textbf{Set}$ with its universal property.
        \item A \textit{co-limit} of $F$, if it exists, is an object $\operatorname{colim} F$ of $\textbf{C}$ representing a functor $\operatorname{Cone}(F, -):\textbf{C}^\text{op}\rightarrow\textbf{Set}$ with its universal property.
    \end{itemize}
\end{definition}

With this notion of cone/co-cone, we can see that one can specify any ``shape'' of it that we want, for example, spanning like a product ($\operatorname{Hom}_\textbf{C}(-,Z)\Rightarrow\operatorname{Hom}_\textbf{C}(-,X)\times\operatorname{Hom}_\textbf{C}(-,Y)$) or more.

\begin{remark}
    To recap on the interpretation of limit, we have the following natural isomorphism:
    \begin{equation*}
        \gamma:\operatorname{Hom}_\textbf{C}(-, \lim F) \Rightarrow \operatorname{Cone}(-, F)
    \end{equation*}
    With the following diagram, given a unique map $g:X\to\lim F$, the naturality condition will gives us (LHS diagram) together with the commutative diagram of the limit in $\textbf{C}$

    \begin{equation*}
    % https://q.uiver.app/#q=WzAsOCxbMCwwLCJcXG9wZXJhdG9ybmFtZXtIb219X1xcdGV4dGJme0N9KFxcbGltIEYsXFxsaW0gRikiXSxbMCwyLCJcXG9wZXJhdG9ybmFtZXtDb25lfShcXGxpbSBGLCBGKSJdLFsyLDIsIlxcb3BlcmF0b3JuYW1le0NvbmV9KFgsIEYpIl0sWzIsMCwiXFxvcGVyYXRvcm5hbWV7SG9tfV9cXHRleHRiZntDfShYLCBcXGxpbSBGKSJdLFs1LDAsIlgiXSxbNSwxLCJcXGxpbSBGIl0sWzQsMiwiRkoiXSxbNiwyLCJGSiciXSxbMywyLCJcXGNvbmciXSxbMCwxLCJcXGNvbmciLDJdLFsxLDIsIlxcb3BlcmF0b3JuYW1le0NvbmV9KC0sXFxsaW0gRilbZ10iLDJdLFswLDMsIigtXFxjaXJjIGcpIl0sWzAsMSwiXFxnYW1tYV97XFxsaW0gRn0iXSxbMywyLCJcXGdhbW1hX1giLDJdLFs1LDYsIlxcYWxwaGFfSiJdLFs1LDcsIlxcYWxwaGFfe0onfSIsMl0sWzYsNywiRm0iLDJdLFs0LDYsIlxcYmV0YV9qIiwyXSxbNCw3LCJcXGJldGFfe2onfSJdLFs0LDUsImciLDIseyJzdHlsZSI6eyJib2R5Ijp7Im5hbWUiOiJkYXNoZWQifX19XV0=
    \begin{tikzcd}
        {\operatorname{Hom}_\textbf{C}(\lim F,\lim F)} && {\operatorname{Hom}_\textbf{C}(X, \lim F)} &&& X \\
        &&&&& {\lim F} \\
        {\operatorname{Cone}(\lim F, F)} && {\operatorname{Cone}(X, F)} && FJ && {FJ'}
        \arrow["\cong", from=1-3, to=3-3]
        \arrow["\cong"', from=1-1, to=3-1]
        \arrow["{\operatorname{Cone}(-,\lim F)[g]}"', from=3-1, to=3-3]
        \arrow["{(-\circ g)}", from=1-1, to=1-3]
        \arrow["{\gamma_{\lim F}}", from=1-1, to=3-1]
        \arrow["{\gamma_X}"', from=1-3, to=3-3]
        \arrow["{\alpha_J}", from=2-6, to=3-5]
        \arrow["{\alpha_{J'}}"', from=2-6, to=3-7]
        \arrow["Fm"', from=3-5, to=3-7]
        \arrow["{\beta_j}"', from=1-6, to=3-5]
        \arrow["{\beta_{j'}}", from=1-6, to=3-7]
        \arrow["g"', dashed, from=1-6, to=2-6]
    \end{tikzcd}
    \end{equation*}

    Note that the morphism lifted by the cone functor is defined (and just like $(-\circ f)\times(-\circ f)$ above), and required to be such that $\alpha_J\circ g=\beta_J$. 
\end{remark}

By the immediate application of Yoneda lemma, we have that:

\begin{proposition}
    Limit and co-limit are unqiue up to isomorphism (if they exists).
\end{proposition}

\begin{dem}
\begin{proof}
    Assume that there are 2 limit of diagram $F$ i.e $\lim F_1$ and $\lim F_2$ i.e:
    \begin{equation*}
        % https://q.uiver.app/#q=WzAsMyxbMCwwLCJcXG9wZXJhdG9ybmFtZXtIb219X1xcdGV4dGJme0N9KC0sXFxsaW0gRl8xKSJdLFsyLDAsIlxcb3BlcmF0b3JuYW1le0hvbX1fXFx0ZXh0YmZ7Q30oLSxcXGxpbSBGXzIpIl0sWzEsMSwiXFxvcGVyYXRvcm5hbWV7Q29uZX0oLSwgRikiXSxbMCwyLCJcXGNvbmciLDIseyJsZXZlbCI6Mn1dLFsxLDIsIlxcY29uZyIsMCx7ImxldmVsIjoyfV1d
        \begin{tikzcd}
            {\operatorname{Hom}_\textbf{C}(-,\lim F_1)} && {\operatorname{Hom}_\textbf{C}(-,\lim F_2)} \\
            & {\operatorname{Cone}(-, F)}
            \arrow["\cong"', Rightarrow, from=1-1, to=2-2]
            \arrow["\cong", Rightarrow, from=1-3, to=2-2]
        \end{tikzcd}
    \end{equation*}
    Then we see that there is a natural isomorphism between $\operatorname{Hom}_\textbf{C}(-,\lim F_1)\Rightarrow\operatorname{Hom}_\textbf{C}(-,\lim F_2)$ by the corollary \ref{coro:iso-hom-functor-iso-represent} we have that $\lim F_1\cong\lim F_2$
\end{proof}
\end{dem}

\begin{definition}{\textbf{(Complete Category)}}
    A catagory $\textbf{C}$ is called \textit{complete} if every diagram in $\textbf{C}$ has limit. Similary, it is \textit{co-complete} if every diagram in $\textbf{C}$ has a colimit.
\end{definition}

Given the complete category, one can explicitly construct the limits. Let's consider the case of $\textbf{Set}$.

\begin{remark}{\textbf{(Construction of Limits in $\textbf{Set}$)}}
    Given the diagram $D:\textbf{J}\rightarrow\textbf{Set}$, and assume $\textbf{J}$ to be small (the objects form a set). We will start with the cartesian product of all sets in diagram $\textbf{D}$: $P:= \prod_{I\in\operatorname{Obj}(\textbf{J})}DI$. We have the projection map of $\pi_I:P\to DI$ for each object $I$ of $\textbf{J}$. However, with morphism $m:I\rightarrow I'$, we add need constrain to the product set so that the diagram commutes:
    \begin{equation*}
    % https://q.uiver.app/#q=WzAsMyxbMSwwLCJQIl0sWzAsMSwiREkiXSxbMiwxLCJESSciXSxbMSwyLCJEbSIsMl0sWzAsMSwiXFxwaV97SX0iLDJdLFswLDIsIlxccGlfe0knfSJdXQ==
    \begin{tikzcd}
        & P \\
        DI && {DI'}
        \arrow["Dm"', from=2-1, to=2-3]
        \arrow["{\pi_{I}}"', from=1-2, to=2-1]
        \arrow["{\pi_{I'}}", from=1-2, to=2-3]
    \end{tikzcd}
    \end{equation*}
    Given $p\in P$, we required to have $Dm(\pi_I(p))=\pi_{I'}(p)$. If $p$ satisfies the constrain for morphism $m$, we say $p\in S_m$. This should be applied to all morphism i.e we have:
    \begin{equation*}
        S := \bigcap_{m\in\operatorname{Morp}(\mathbf{J})}S_m
    \end{equation*}
    This $S$ is indeed a limit over the diagram $D$, following the lemma below.
\end{remark}

\begin{lemma}
    The object $S$ defined above with map $S\to DI$ as:
    \begin{equation*}
    % https://q.uiver.app/#q=WzAsMyxbMCwwLCJTIl0sWzEsMCwiUCJdLFsyLDAsIkRJIl0sWzAsMSwiaSIsMCx7InN0eWxlIjp7InRhaWwiOnsibmFtZSI6Imhvb2siLCJzaWRlIjoidG9wIn19fV0sWzEsMiwiXFxwaV9JIl1d
    \begin{tikzcd}
        S & P & DI
        \arrow["i", hook, from=1-1, to=1-2]
        \arrow["{\pi_I}", from=1-2, to=1-3]
    \end{tikzcd}
    \end{equation*}
    where $i:S\rightarrow P$ is the inclusion, is a limit over the diagram $D$.
\end{lemma}

\begin{dem}
\begin{proof}
    To show that $S$ is a limit, we need to show that there is a natural isomorphism between the functor: $\operatorname{Hom}_\textbf{Set}(-,S)\Rightarrow \operatorname{Cone}(-,D)$. Considering its component, given a set $A$, we have $\operatorname{Hom}_\textbf{Set}(A,S)\to \operatorname{Cone}(A,D)$. Let's show that it is a bijection.

    \textbf{(Surjectivity):} Let a cone be $\alpha:A\Rightarrow D$, we note that by definition there is an arrow $\alpha_I:A\to DI$ for all $I\in\operatorname{Obj}(\textbf{J})$. By the universal property of the product there is a function $f:A\to P$ such that $\alpha_I=\pi_I\circ f$.
    
    However, for $\alpha$ to be a natural transformation, we required for any $a\in A$, we have $Dm(\pi_I(f(a)))=\pi_{I'}(f(a))$ for any $m\in\operatorname{Morp}(\textbf{J})$, or in other words $f(a)\in S$. Thus $f$ can only restricted to $f:A\to S$. Thus for any $\alpha\in\operatorname{Cone}(A,D)$, its component $\alpha_I$ has the form of $\pi_I\circ f$.


    \textbf{(Injectivity):} Given the fact that $\alpha=\alpha'$ with associated $f,f'\in\operatorname{Hom}_\textbf{Set}(A, S)$, that is for every object $I\in \operatorname{Obj}(\textbf{J})$, we have $\pi_J\circ f=\pi_J\circ f'$. That is the output of $f$ and $f'$ for every entry of the output are the same. Thus, $f=f'$, as needed.
    
    Finally, the naturality condition are authomatically satisfied since the component of $\alpha$ at $I$ is equal to $\pi_I\circ f$. 
\end{proof}
\end{dem}

Therefore, this leads to the following theorem:

\begin{theorem}
    The category $\textbf{Set}$ is complete.
\end{theorem}

\begin{proposition}{\textbf{(Functoriality of Limit)}}
    \label{prop:limit-functoriality}
    Given a diagram $F:\textbf{J}\to\textbf{C}$ and a natural transformation $\alpha:F\Rightarrow F'$, then we can define the morphism $\alpha_{\lim}:\lim F\to\lim\alpha(F)$ in $\textbf{C}$ via the universal property as:
    \begin{equation*}
    % https://q.uiver.app/#q=WzAsNixbMSwxLCJcXGxpbSBGIl0sWzAsMiwiRkoiXSxbMiwyLCJGSiciXSxbMywwLCJcXGxpbVxcYWxwaGEoRikiXSxbMiwxLCJGJ0oiXSxbNCwxLCJGJ0onIl0sWzAsMV0sWzAsMl0sWzEsMiwiRm0iLDFdLFswLDMsIlxcYWxwaGFfe1xcbGltfSIsMCx7InN0eWxlIjp7ImJvZHkiOnsibmFtZSI6ImRhc2hlZCJ9fX1dLFsxLDQsIlxcYWxwaGFfSiIsMV0sWzIsNSwiXFxhbHBoYV97Sid9IiwxXSxbNCw1LCJGJ20iLDFdLFszLDRdLFszLDVdXQ==
    \begin{tikzcd}
        &&& {\lim\alpha(F)} \\
        & {\lim F} & {F'J} && {F'J'} \\
        FJ && {FJ'}
        \arrow[from=1-4, to=2-3]
        \arrow[from=1-4, to=2-5]
        \arrow["{\alpha_{\lim}}", dashed, from=2-2, to=1-4]
        \arrow[from=2-2, to=3-1]
        \arrow[from=2-2, to=3-3]
        \arrow["{F'm}"{description}, from=2-3, to=2-5]
        \arrow["{\alpha_J}"{description}, from=3-1, to=2-3]
        \arrow["Fm"{description}, from=3-1, to=3-3]
        \arrow["{\alpha_{J'}}"{description}, from=3-3, to=2-5]
    \end{tikzcd}
    \end{equation*}
    Then we can see that: $\lim (\beta\circ\alpha)(F)=\beta_{\lim}\circ\alpha_{\lim}(\lim F)$ and $\lim(\operatorname{id} F)=\lim F$, thus the limit can be seen as functorial. Recall that we can see limit as functor with type signature of $\lim:[\textbf{J},\textbf{C}]\to\textbf{C}$.
\end{proposition}
\begin{proof}
    Following functoriality of (Co)Ends shown in proposition \ref{prop:functoriality-of-ends}, as limits can be described by ends. In a nutshell, the equality between composition is due to the uniqueness of universal properties.
\end{proof}

\subsection{Continuous Functors}

\begin{definition}{\textbf{(Continuity)}}
    Given a functor $F:\textbf{C}\rightarrow\textbf{C}^\prime$, it is called continuous if it preserves all limits that exists in $\textbf{C}$ i.e every diagram $D$ in $\textbf{C}$, where its limit $\lim D$ exists, then $\lim FD$ exists in $\textbf{C}'$ and: 

    \begin{equation*}
        \lim FD\cong F(\lim D)
    \end{equation*}
\end{definition}

Note also that the functor in general doesn't preserve the property of mono or epi or generally any other universal property, hence the need of continuity.

\begin{remark}
    Suppose $F$ is continuous, then by Yoneda lemma, we also have that 
    \begin{equation*}
        \operatorname{Hom}_\textbf{C}(-,F\lim D)\cong\operatorname{Hom}_\textbf{C}(-,\lim FD)\cong\operatorname{Cone}(-, FD)
    \end{equation*}
    Thus $F\lim D$ is also a limit of the diagram $FD$. One can also proof the propositions below by showing that $,F\lim D$ is actually a limit (or construct the isomorphism directly).
\end{remark}

\begin{proposition}
    Given a functor $F:\textbf{C}\rightarrow\textbf{C}{^\prime}$ that preserves the limit of diagram $D:\textbf{J}\rightarrow\textbf{C}$, and functor $G$ be functor that is naturally isomorphic to $F$, then $G$ also preserve limit of diagram $D$.
\end{proposition}

\begin{dem}
\begin{proof}
    We let $\alpha:F\Rightarrow G$ be the natural isomorphism between $F$ and $G$. We want to show that $\lim GD\cong G(\lim D)$. We will also denote the isomorphism $f:\lim F(\lim D)\xrightarrow{\cong}FD$. 
    
    We will claim that there is an isomorphism between $\lim FD$ and $\lim GD$. To do this, we consider similar technique to show the isomorphism between products and using the fact that $\alpha$ is natural isomorphism, as shown in the LHS diagram.

    \begin{equation*}
    % https://q.uiver.app/#q=WzAsMTMsWzQsMSwiRlxcbGltIEQiXSxbNCwyLCJcXGxpbSBGRCJdLFs2LDIsIlxcbGltIEdEIl0sWzYsMSwiR1xcbGltIEQiXSxbMCwxLCJGREkiXSxbMSwwLCJcXGxpbSBGRCJdLFsxLDEsIlxcbGltIEdEIl0sWzEsMiwiXFxsaW0gRkQiXSxbMCwzLCJGREkiXSxbMCwyLCJHREkiXSxbMiwxLCJGREknIl0sWzIsMiwiR0RJJyJdLFsyLDMsIkZESSciXSxbMCwzLCJcXGFscGhhX3tcXGxpbSBEfSJdLFswLDMsIlxcY29uZyIsMl0sWzEsMiwiayIsMix7InN0eWxlIjp7ImJvZHkiOnsibmFtZSI6ImRhc2hlZCJ9fX1dLFsxLDIsIlxcY29uZyIsMCx7InN0eWxlIjp7ImJvZHkiOnsibmFtZSI6ImRhc2hlZCJ9fX1dLFszLDIsImgiLDAseyJzdHlsZSI6eyJib2R5Ijp7Im5hbWUiOiJkYXNoZWQifX19XSxbMCwxLCJmIiwyXSxbMCwxLCJcXGNvbmciXSxbNSw2LCJrIiwwLHsic3R5bGUiOnsiYm9keSI6eyJuYW1lIjoiZGFzaGVkIn19fV0sWzUsNCwiXFxiZXRhX0kiLDJdLFs1LDEwLCJcXGJldGFfe0knfSJdLFs2LDksIlxccGhpX0kiXSxbNiwxMSwiXFxwaGlfe0knfSIsMl0sWzYsNywiayciLDAseyJzdHlsZSI6eyJib2R5Ijp7Im5hbWUiOiJkYXNoZWQifX19XSxbNyw4LCJcXGJldGFfe0l9Il0sWzcsMTIsIlxcYmV0YV97SSd9IiwyXSxbOSw4LCJcXGFscGhhX3tESX1eey0xfSIsMl0sWzEwLDExLCJcXGFscGhhX3tESX1eey0xfSJdLFsxMSwxMiwiXFxhbHBoYV97REknfV57LTF9Il0sWzQsOSwiXFxhbHBoYV97REl9IiwyXV0=
    \begin{tikzcd}
        & {\lim FD} \\
        FDI & {\lim GD} & {FDI'} && {F\lim D} && {G\lim D} \\
        GDI & {\lim FD} & {GDI'} && {\lim FD} && {\lim GD} \\
        FDI && {FDI'}
        \arrow["{\alpha_{\lim D}}", from=2-5, to=2-7]
        \arrow["\cong"', from=2-5, to=2-7]
        \arrow["k"', dashed, from=3-5, to=3-7]
        \arrow["\cong", dashed, from=3-5, to=3-7]
        \arrow["h", dashed, from=2-7, to=3-7]
        \arrow["f"', from=2-5, to=3-5]
        \arrow["\cong", from=2-5, to=3-5]
        \arrow["k", dashed, from=1-2, to=2-2]
        \arrow["{\beta_I}"', from=1-2, to=2-1]
        \arrow["{\beta_{I'}}", from=1-2, to=2-3]
        \arrow["{\phi_I}", from=2-2, to=3-1]
        \arrow["{\phi_{I'}}"', from=2-2, to=3-3]
        \arrow["{k'}", dashed, from=2-2, to=3-2]
        \arrow["{\beta_{I}}", from=3-2, to=4-1]
        \arrow["{\beta_{I'}}"', from=3-2, to=4-3]
        \arrow["{\alpha_{DI}^{-1}}"', from=3-1, to=4-1]
        \arrow["{\alpha_{DI}^{-1}}", from=2-3, to=3-3]
        \arrow["{\alpha_{DI'}^{-1}}", from=3-3, to=4-3]
        \arrow["{\alpha_{DI}}"', from=2-1, to=3-1]
    \end{tikzcd}
    \end{equation*}
    We can see that $k$ and $k'$ are invere of each other, thus $k$ is an isomorphism. Finally, since there is an isomorphism from $\alpha_{\lim D}:F\lim D\to G\lim D$, from lemma \ref{lemma:morph-from-isos}, we can find a isomorphism $h:G\lim D\to \lim GD$, as shown in RHS diagram.
\end{proof}
\end{dem}


\begin{proposition}
    Given a functor $F:\textbf{C}\rightarrow\textbf{D}$ that indicates the equvialent of categories, we can also show that $F$ is continuous.
    \checkproof Check the proof again, or simplify them.
\end{proposition}

\begin{dem}
\begin{proof}
    We want to show that $F\lim D\cong \lim FD$. Let's start with the obvious map between $F\lim D\to \lim FD$, which follows from the universal property of $\lim FD$ (as shown in the LHS diagram)
    \begin{equation*}
    % https://q.uiver.app/#q=WzAsNCxbMSwwLCJGXFxsaW0gRCJdLFsxLDEsIlxcbGltIEZEIl0sWzAsMiwiRkRJIl0sWzIsMiwiRkRJJyJdLFsxLDIsIlxcYmV0YV9JIl0sWzEsMywiXFxiZXRhX3tJJ30iLDJdLFswLDEsImYiLDAseyJzdHlsZSI6eyJib2R5Ijp7Im5hbWUiOiJkYXNoZWQifX19XSxbMCwyLCJGXFxhbHBoYV9JIiwyLHsiY3VydmUiOjJ9XSxbMCwzLCJGXFxhbHBoYV97SSd9IiwwLHsiY3VydmUiOi0yfV0sWzIsMywiRkRtIiwyXV0=
    \begin{tikzcd}
        & {F\lim D} \\
        & {\lim FD} \\
        FDI && {FDI'}
        \arrow["{\beta_I}", from=2-2, to=3-1]
        \arrow["{\beta_{I'}}"', from=2-2, to=3-3]
        \arrow["f", dashed, from=1-2, to=2-2]
        \arrow["{F\alpha_I}"', curve={height=12pt}, from=1-2, to=3-1]
        \arrow["{F\alpha_{I'}}", curve={height=-12pt}, from=1-2, to=3-3]
        \arrow["FDm"', from=3-1, to=3-3]
    \end{tikzcd}
    \qquad \quad
    % https://q.uiver.app/#q=WzAsNCxbMCwyLCJESSJdLFsxLDEsIlxcbGltIEQiXSxbMiwyLCJESSciXSxbMSwwLCJBIl0sWzEsMCwiXFxhbHBoYV9JIiwyXSxbMCwyLCJEbSIsMl0sWzMsMCwiXFxwaGlfSSIsMix7ImN1cnZlIjoyfV0sWzMsMSwiaCIsMCx7InN0eWxlIjp7ImJvZHkiOnsibmFtZSI6ImRhc2hlZCJ9fX1dLFsxLDIsIlxcYWxwaGFfe0knfSJdLFszLDIsIlxccGhpX3tJJ30iLDAseyJjdXJ2ZSI6LTJ9XV0=
    \begin{tikzcd}
        & A \\
        & {\lim D} \\
        DI && {DI'}
        \arrow["{\alpha_I}"', from=2-2, to=3-1]
        \arrow["Dm"', from=3-1, to=3-3]
        \arrow["{\phi_I}"', curve={height=12pt}, from=1-2, to=3-1]
        \arrow["h", dashed, from=1-2, to=2-2]
        \arrow["{\alpha_{I'}}", from=2-2, to=3-3]
        \arrow["{\phi_{I'}}", curve={height=-12pt}, from=1-2, to=3-3]
    \end{tikzcd}
    \end{equation*}
    where $\alpha_I$ is part of universal property of $\lim D$. Since $F$ is essentially surjective, there is object $A$ of $\textbf{C}$ such that $g:FA\xrightarrow{\cong}\lim FD$. Then we can consider the component of natural transformation to be $\gamma_I = \beta_I\circ g$ and so on, for each $I$ of $\textbf{J}$. Since $F$ is fully faithful, there is $\phi_I$ such that $F\phi_I=\gamma_I$. We will use $\phi_I$ to consturct the map $h:A\to \lim D$ via universal property (see RHS).
    \begin{equation*}
    % https://q.uiver.app/#q=WzAsNSxbMCwzLCJGREkiXSxbMSwxLCJGXFxsaW0gRCJdLFsyLDMsIkZESSciXSxbMSwyLCJcXGxpbSBGRCJdLFsxLDAsIlxcbGltIEZEIl0sWzEsMCwiRlxcYWxwaGFfSSIsMSx7ImN1cnZlIjoyfV0sWzAsMiwiRkRtIiwyXSxbMSwyLCJGXFxhbHBoYV97SSd9IiwxLHsiY3VydmUiOi0yfV0sWzEsMywiZiIsMCx7InN0eWxlIjp7ImJvZHkiOnsibmFtZSI6ImRhc2hlZCJ9fX1dLFszLDAsIlxcYmV0YV9JIl0sWzMsMiwiXFxiZXRhX3tJJ30iLDJdLFs0LDAsIlxcYmV0YV97SX0iLDIseyJjdXJ2ZSI6M31dLFs0LDIsIlxcYmV0YV97SSd9IiwwLHsiY3VydmUiOi0zfV0sWzQsMSwiRmhcXGNpcmMgZ157LTF9IiwxXV0=
    \begin{tikzcd}
        & {\lim FD} \\
        & {F\lim D} \\
        & {\lim FD} \\
        FDI && {FDI'}
        \arrow["{F\alpha_I}"{description}, curve={height=12pt}, from=2-2, to=4-1]
        \arrow["FDm"', from=4-1, to=4-3]
        \arrow["{F\alpha_{I'}}"{description}, curve={height=-12pt}, from=2-2, to=4-3]
        \arrow["f", dashed, from=2-2, to=3-2]
        \arrow["{\beta_I}", from=3-2, to=4-1]
        \arrow["{\beta_{I'}}"', from=3-2, to=4-3]
        \arrow["{\beta_{I}}"', curve={height=18pt}, from=1-2, to=4-1]
        \arrow["{\beta_{I'}}", curve={height=-18pt}, from=1-2, to=4-3]
        \arrow["{Fh\circ g^{-1}}"{description}, from=1-2, to=2-2]
    \end{tikzcd}\qquad \quad
    % https://q.uiver.app/#q=WzAsNSxbMCwzLCJGREkiXSxbMSwyLCJGXFxsaW0gRCJdLFsyLDMsIkZESSciXSxbMSwwLCJcXGxpbSBGRCJdLFsxLDEsIkZBIl0sWzEsMCwiRlxcYWxwaGFfSSIsMSx7ImN1cnZlIjoxfV0sWzAsMiwiRkRtIiwyXSxbMSwyLCJGXFxhbHBoYV97SSd9IiwxLHsiY3VydmUiOi0xfV0sWzMsMCwiXFxiZXRhX3tJfSIsMix7ImN1cnZlIjozfV0sWzMsMiwiXFxiZXRhX3tJJ30iLDAseyJjdXJ2ZSI6LTN9XSxbNCwwLCJcXGJldGFfSVxcY2lyYyBnIiwxLHsiY3VydmUiOjJ9XSxbNCwxLCJGaCIsMCx7InN0eWxlIjp7ImJvZHkiOnsibmFtZSI6ImRhc2hlZCJ9fX1dLFs0LDIsIlxcYmV0YV97SSd9XFxjaXJjIGciLDEseyJjdXJ2ZSI6LTJ9XSxbMyw0LCJnXnstMX0iLDFdXQ==
    \begin{tikzcd}
        & {\lim FD} \\
        & FA \\
        & {F\lim D} \\
        FDI && {FDI'}
        \arrow["{F\alpha_I}"{description}, curve={height=6pt}, from=3-2, to=4-1]
        \arrow["FDm"', from=4-1, to=4-3]
        \arrow["{F\alpha_{I'}}"{description}, curve={height=-6pt}, from=3-2, to=4-3]
        \arrow["{\beta_{I}}"', curve={height=18pt}, from=1-2, to=4-1]
        \arrow["{\beta_{I'}}", curve={height=-18pt}, from=1-2, to=4-3]
        \arrow["{\beta_I\circ g}"{description}, curve={height=12pt}, from=2-2, to=4-1]
        \arrow["Fh", dashed, from=2-2, to=3-2]
        \arrow["{\beta_{I'}\circ g}"{description}, curve={height=-12pt}, from=2-2, to=4-3]
        \arrow["{g^{-1}}"{description}, from=1-2, to=2-2]
    \end{tikzcd}
    \end{equation*}
    Note that the diagram above all arrows commutes. Thus we have that $f\circ(Fh\circ g^{-1})=\operatorname{id}_{\lim FD}$. And the other direction can be proven in similar manners. Thus, we have shown that $f$ is an isomorphism.
\end{proof}
\end{dem}

\begin{theorem}
    \label{thm:rep-functor-continuous}
    Representable functors are continuous
\end{theorem}
\begin{dem}
\begin{proof}
    We note that the action are given to be $(Dm\circ-):\operatorname{Hom}_\textbf{C}(X,DI)\to\operatorname{Hom}_\textbf{C}(X,DI')$. Then, we can consider the following diagram (and we have used the notation from the construction of limit in a $\textbf{Set}$):
    \begin{equation*}
    % https://q.uiver.app/#q=WzAsMyxbMCwxLCJcXG9wZXJhdG9ybmFtZXtIb219X1xcdGV4dGJme0N9KFgsREkpIl0sWzIsMSwiXFxvcGVyYXRvcm5hbWV7SG9tfV9cXHRleHRiZntDfShYLERJJykiXSxbMSwwLCJcXGxpbVxcYmlnKFxcb3BlcmF0b3JuYW1le0hvbX1fXFx0ZXh0YmZ7Q30oWCxELSlcXGJpZykiXSxbMCwxLCIoRG1cXGNpcmMtKSIsMl0sWzIsMCwiXFxwaV9JIiwyXSxbMiwxLCJcXHBpX3tJJ30iXV0=
    \begin{tikzcd}
        & {\lim\big(\operatorname{Hom}_\textbf{C}(X,D-)\big)} \\
        {\operatorname{Hom}_\textbf{C}(X,DI)} && {\operatorname{Hom}_\textbf{C}(X,DI')}
        \arrow["{\pi_I}"', from=1-2, to=2-1]
        \arrow["{\pi_{I'}}", from=1-2, to=2-3]
        \arrow["{(Dm\circ-)}"', from=2-1, to=2-3]
    \end{tikzcd}
    \end{equation*}
    where we note that $\lim\big(\operatorname{Hom}_\textbf{C}(X,D-)\big)\subseteq \prod_{I\in\operatorname{Obj}(\textbf{J})}\operatorname{Hom}_\textbf{C}(X,DI)$ such that for an element $\alpha$ of the limit (displayed below) 
    \begin{equation*}
        \alpha=(f_I:X\to DI)_{I\in\operatorname{Obj}(\textbf{J})}\in \lim\big(\operatorname{Hom}_\textbf{C}(X,D-)\big)
    \end{equation*}
    Being the tuple of functions $X\to DI$ such that $Dm\circ\pi_I(\alpha)=\pi_{I'}(\alpha)$, or $Dm\circ f_I=f_{I'}$, where $\pi_I(\alpha)=f_I$ and $\pi_{I'}(\alpha)=f_I'$ as $\pi_I,\pi_{I'}$ are the projection of the product. Observe that this $\alpha$ acts like the natural transformation between the constant functor $X$ and diagram $D$ i.e it is a cone. 
    
    On the other hand, we can change an element of cone to be the element in the $\lim\big(\operatorname{Hom}_\textbf{C}(X,D-)\big)$ by ``bundle'' all of its compoent into a tuple. And the commutative condition is still satisifed (as it is simply the naturality condition). Thus we have:
    \begin{equation*}
        \lim\big(\operatorname{Hom}_\textbf{C}(X,D-)\big) \cong \operatorname{Cone}(X,D)\cong \operatorname{Hom}_\textbf{C}(X, \lim D)
    \end{equation*}
\end{proof}
\end{dem}

\begin{corollary}
    Given representable pre-sheaf $P:\textbf{C}^\text{op}\rightarrow\textbf{Set}$, then $P$ turns co-limit to limit i.e with diagram $D:\textbf{J}\rightarrow\textbf{C}$ that has colimit, then limit of $P\circ D$ exists and: 
    \begin{equation*}
        \lim P\circ D = P(\operatorname{colim}D)
    \end{equation*}
\end{corollary}

\subsection{Instances of Limit}

Let's consider various instances of limit that we have consider in the sections above, starting with the simpliest one:

\begin{definition}{\textbf{(Initial/Terminal Object)}}
    We start with the empty category $\textbf{O}$, which is small. Consider the empty diagram $E:\textbf{O}\rightarrow\textbf{C}$ i.e:
    \begin{equation*}
    \end{equation*}
    The cone (and co-cone) over this diagram is just an object $X$ with an identity morphism of $X$. Now, we have that the limit and co-limit are given to be:
    \begin{equation*}
    % https://q.uiver.app/#q=WzAsNCxbMCwwLCJYIl0sWzEsMCwiXFxsaW0gRiJdLFszLDAsIlxcb3BlcmF0b3JuYW1le2NvbGltfUYiXSxbNCwwLCJYIl0sWzAsMSwiIiwwLHsic3R5bGUiOnsiYm9keSI6eyJuYW1lIjoiZGFzaGVkIn19fV0sWzIsMywiIiwwLHsic3R5bGUiOnsiYm9keSI6eyJuYW1lIjoiZGFzaGVkIn19fV1d
    \begin{tikzcd}
        X & {\lim F} && {\operatorname{colim}F} & X
        \arrow[dashed, from=1-1, to=1-2]
        \arrow[dashed, from=1-4, to=1-5]
    \end{tikzcd}
    \end{equation*}
\end{definition}

Then we have:

\begin{definition}{\textbf{(Categorical Product)}}
    From definition \ref{def:cat-prod}, we can define the \textit{product} (or \textit{co-product}) to be the limit (or co-limit) of the discrete diagram $F$: 
    \begin{equation*}
        A \qquad \quad B
    \end{equation*}
\end{definition}

\begin{definition}{\textbf{(Equalizer/Co-Equalizer)}}
    The notion of \textit{equalizer} (or \textit{coequalizer}) is the limit (or co-limit) of the digram:
    \begin{equation*}
    % https://q.uiver.app/#q=WzAsMixbMCwwLCJBIl0sWzEsMCwiQiJdLFswLDEsImYiLDAseyJvZmZzZXQiOi0xfV0sWzAsMSwiZyIsMix7Im9mZnNldCI6MX1dXQ==
    \begin{tikzcd}
        A & B
        \arrow["f", shift left, from=1-1, to=1-2]
        \arrow["g"', shift right, from=1-1, to=1-2]
    \end{tikzcd}
    \end{equation*}
    In which, we have the following commutative diagram:
    \begin{equation*}
    % https://q.uiver.app/#q=WzAsOCxbMSwxLCJBIl0sWzIsMSwiQiJdLFsxLDAsIlxcbGltIEYiXSxbMCwxLCJYIl0sWzQsMSwiQSJdLFs1LDEsIkIiXSxbNiwwLCJcXG9wZXJhdG9ybmFtZXtjb2xpbX1GIl0sWzYsMSwiWSJdLFswLDEsImciLDIseyJvZmZzZXQiOjF9XSxbMCwxLCJmIiwwLHsib2Zmc2V0IjotMX1dLFsyLDAsIm0iXSxbMywwLCJwIiwyXSxbMywyLCIiLDEseyJzdHlsZSI6eyJib2R5Ijp7Im5hbWUiOiJkYXNoZWQifX19XSxbNCw1LCJnIiwyLHsib2Zmc2V0IjoxfV0sWzQsNSwiZiIsMCx7Im9mZnNldCI6LTF9XSxbNSw3LCJxIiwyXSxbNiw3LCIiLDAseyJzdHlsZSI6eyJib2R5Ijp7Im5hbWUiOiJkYXNoZWQifX19XSxbNSw2LCJuIl1d
    \begin{tikzcd}
        & {\lim F} &&&&& {\operatorname{colim}F} \\
        X & A & B && A & B & Y
        \arrow["g"', shift right, from=2-2, to=2-3]
        \arrow["f", shift left, from=2-2, to=2-3]
        \arrow["m", from=1-2, to=2-2]
        \arrow["p"', from=2-1, to=2-2]
        \arrow[dashed, from=2-1, to=1-2]
        \arrow["g"', shift right, from=2-5, to=2-6]
        \arrow["f", shift left, from=2-5, to=2-6]
        \arrow["q"', from=2-6, to=2-7]
        \arrow[dashed, from=1-7, to=2-7]
        \arrow["n", from=2-6, to=1-7]
    \end{tikzcd}
    \end{equation*}
    Note that we have ignore the map $p':X\rightarrow B$ on the LHS diagram because, by definition of the cone: $f\circ p=p'=g\circ p$ (on the limit's diagram on the LHS), $p'$ is just a composition.
\end{definition}

\begin{proposition}
    Given the definiton of equalizer above, we can show that the universal map $m:\lim F\rightarrow A$ is necessary mono.
\end{proposition}
\begin{dem}
\begin{proof}
    Given any function $a:Y\to\lim F$ and $b:Y\to\lim F$ for any object $Y$, by the universal property, for every $a$, there is a \textit{unique} $a\circ m:Y\to A$. Therefore, if $a\circ m=b\circ m$ then $a=b$. Thus $m$ is mono.
\end{proof}
\end{dem}

This also gives us the dual statement:

\begin{proposition}
    Given the definition of coequalizer above, we can show that the universal map $e:B\rightarrow\operatorname{colim}F$ is necessary epi.
\end{proposition}

\begin{remark}{\textbf{(Equalizer in $\textbf{Set}$)}}
    With our common tradition, we can consider the role of equalizer in the context of $\textbf{Set}$. Before we doing that, from the commutative diagram, we can see that given $p:X\to A$ we have $f\circ p=g\circ p$. Note that we can't claim that $f=g$ as $p$ doesn't have to be an epi.

    Let $X$ being the intial object $1$, and note that since $m$ can be shown to be monomorphism or injective in $\textbf{Set}$ by proposition \ref{prop:mono-iff-injective}. Thus, one can treat $\lim F\subseteq A$. Thus we can define it to be:

    \begin{equation*}
        \lim F = \big\{ a\in A : f(a) = g(a) \big\}
    \end{equation*}
    
    As the intial object $1$ requires to to every element in $\lim F$, and $A$. In other words, the equalizer $\lim F$ is the largetest subset of $A$ that $f$ and $g$ agree.
\end{remark}

\begin{remark}{\textbf{(Co-Equalizer in $\textbf{Set}$)}}
    By definition, given $a\in A$, we have that $n(f(a))=n(g(a))$, where $n$ is the universal map. Observe that $\operatorname{colim} F$ acts akin to set of equivalence classes over the set $B$, where $n$ being the surjective function (proposition \ref{prop:epi-iff-surjective}) that maps each element into an equivalent class:

    \begin{itemize}
        \item Let's try to define an relation within $B$ first, we denote $b\sim b'$ iff $b=f(a)$ and $b'=g(a)$ for some $a\in A$. Furthermore, we can extends this relation into an smallest equivalence relation that contains the relation (intersect of all equivalence classes that contaisn this relation). Thus, $\operatorname{colim} F=B/\sim$.
        \item That is because, for any $Y$ with $q:B\to Y$, we are required that for any $a\in A$, we have $q(f(a))=q(g(a))$ thus $q$ shouldn't distinguish between $b=f(a)$ and $b'=g(a)$ i.e $q$ should based on the equivalence classes defined above. And the bear minimum of being able to (non-)distinguish follows from definition of $\operatorname{colim} F$, hence the universal property. 
    \end{itemize}
\end{remark}

\begin{definition}{\textbf{(Pullback/Pushout)}}
    The notion of pullback (or pushout) is the limit (or co-limit) of the diagram:
    \begin{equation*}
    % https://q.uiver.app/#q=WzAsNixbMSwwLCJBIl0sWzEsMSwiQyJdLFswLDEsIkIiXSxbMywwLCJBIl0sWzQsMCwiQiJdLFszLDEsIkMiXSxbMiwxLCJnIiwyXSxbMCwxLCJmIl0sWzMsNSwiZyIsMl0sWzMsNCwiZiJdXQ==
    \begin{tikzcd}
        & A && A & B \\
        B & C && C
        \arrow["g"', from=2-1, to=2-2]
        \arrow["f", from=1-2, to=2-2]
        \arrow["g"', from=1-4, to=2-4]
        \arrow["f", from=1-4, to=1-5]
    \end{tikzcd}
    \end{equation*}
    We have the following cone (or co-cone) with the following commutative diagram as:
    \begin{equation*}
    % https://q.uiver.app/#q=WzAsMTAsWzIsMSwiQSJdLFsyLDIsIkMiXSxbMSwyLCJCIl0sWzQsMCwiQSJdLFs1LDAsIkIiXSxbNCwxLCJDIl0sWzEsMSwiXFxsaW0gRiJdLFswLDAsIlgiXSxbNSwxLCJcXG9wZXJhdG9ybmFtZXtjb2xpbX0gRiJdLFs2LDIsIlkiXSxbMiwxLCJnIiwyXSxbMCwxLCJmIl0sWzMsNSwiZyIsMl0sWzMsNCwiZiJdLFs2LDAsImZeKmciXSxbNiwyLCJnXipmIiwyXSxbNiwxLCIiLDEseyJzdHlsZSI6eyJuYW1lIjoiY29ybmVyIn19XSxbNyw2LCIiLDEseyJzdHlsZSI6eyJib2R5Ijp7Im5hbWUiOiJkYXNoZWQifX19XSxbNywyLCJxIiwxLHsiY3VydmUiOjJ9XSxbNywwLCJwIiwwLHsiY3VydmUiOi0yfV0sWzUsOCwiZ14qZiIsMl0sWzQsOCwiZl4qZyJdLFs4LDksIiIsMCx7InN0eWxlIjp7ImJvZHkiOnsibmFtZSI6ImRhc2hlZCJ9fX1dLFs1LDksIiIsMCx7ImN1cnZlIjoyfV0sWzQsOSwiIiwwLHsiY3VydmUiOi0yfV0sWzgsMywiIiwwLHsic3R5bGUiOnsibmFtZSI6ImNvcm5lciJ9fV1d
    \begin{tikzcd}
        X &&&& A & B \\
        & {\lim F} & A && C & {\operatorname{colim} F} \\
        & B & C &&&& Y
        \arrow["g"', from=3-2, to=3-3]
        \arrow["f", from=2-3, to=3-3]
        \arrow["g"', from=1-5, to=2-5]
        \arrow["f", from=1-5, to=1-6]
        \arrow["{f^*g}", from=2-2, to=2-3]
        \arrow["{g^*f}"', from=2-2, to=3-2]
        \arrow["\lrcorner"{anchor=center, pos=0.125}, draw=none, from=2-2, to=3-3]
        \arrow[dashed, from=1-1, to=2-2]
        \arrow["q"{description}, curve={height=12pt}, from=1-1, to=3-2]
        \arrow["p", curve={height=-12pt}, from=1-1, to=2-3]
        \arrow["{g^*f}"', from=2-5, to=2-6]
        \arrow["{f^*g}", from=1-6, to=2-6]
        \arrow[dashed, from=2-6, to=3-7]
        \arrow[curve={height=12pt}, from=2-5, to=3-7]
        \arrow[curve={height=-12pt}, from=1-6, to=3-7]
        \arrow["\lrcorner"{anchor=center, pos=0.125, rotate=180}, draw=none, from=2-6, to=1-5]
    \end{tikzcd}
    \end{equation*}
    Note that the pullback can be denoted as $A\times_CB$. On the other hand, the pushout $B\sqcup_AC$. The symbol of the corner also denote it is the pullback and pushout. 
\end{definition}

\subsection{All about Pullback}

\todo Add references. We will consider the fact that pullback diagram was quite ubiquitous. Let's start with a re-definition of monomorphism and epimorphism:

\begin{remark}{\textbf{(Mono/Epi in Pullback/Pushout)}}
    \label{remark:mono-pullback}
    Given a morphism $f:A\to B$, $f$ is monomorphism (or epimorphism) if the left (or right) diagram is a pullback (or pushout)
    \begin{equation*}
    % https://q.uiver.app/#q=WzAsOCxbMSwwLCJBIl0sWzEsMSwiQyJdLFswLDEsIkEiXSxbMywwLCJBIl0sWzQsMCwiQiJdLFszLDEsIkIiXSxbMCwwLCJBIl0sWzQsMSwiQiJdLFsyLDEsImYiLDJdLFswLDEsImYiXSxbMyw1LCJmIiwyXSxbMyw0LCJmIl0sWzYsMCwiXFxvcGVyYXRvcm5hbWV7aWR9X0EiXSxbNiwyLCJcXG9wZXJhdG9ybmFtZXtpZH1fQSIsMl0sWzYsMSwiIiwxLHsic3R5bGUiOnsibmFtZSI6ImNvcm5lciJ9fV0sWzUsNywiXFxvcGVyYXRvcm5hbWV7aWR9X0IiLDJdLFs0LDcsIlxcb3BlcmF0b3JuYW1le2lkfV9CIl0sWzcsMywiIiwxLHsic3R5bGUiOnsibmFtZSI6ImNvcm5lci1pbnZlcnNlIn19XV0=
    \begin{tikzcd}
        A & A && A & B \\
        A & C && B & B
        \arrow["f"', from=2-1, to=2-2]
        \arrow["f", from=1-2, to=2-2]
        \arrow["f"', from=1-4, to=2-4]
        \arrow["f", from=1-4, to=1-5]
        \arrow["{\operatorname{id}_A}", from=1-1, to=1-2]
        \arrow["{\operatorname{id}_A}"', from=1-1, to=2-1]
        \arrow["\lrcorner"{anchor=center, pos=0.125}, draw=none, from=1-1, to=2-2]
        \arrow["{\operatorname{id}_B}"', from=2-4, to=2-5]
        \arrow["{\operatorname{id}_B}", from=1-5, to=2-5]
        \arrow["\ulcorner"{anchor=center, pos=0.125, rotate=180}, draw=none, from=2-5, to=1-4]
    \end{tikzcd}
    \end{equation*}
    For the pullback case, we can see that the commutativity forces $p:X\to A$ and $q:X\to A$ to be same function, if we were to have a diagram being commute. Thus satisfies the original definition of monomorphism. Please recall that the limit here requires \textit{both} the map $\operatorname{id}_A:A\to A$ and the object $A$ itself. 
\end{remark}

\begin{remark}{\textbf{(Product as Pullback)}}
    Given objects $X$ and $Y$ their product can be defined via a pullback as their product as $X\times Y\cong X\times_1Y$.
\end{remark}

We will now consider how the pullback preserves the isomorphism and identity, in the following sense.

\begin{proposition}
    The pullback of an isomorphism along any morphis is an isomorphism. That is given the diagram $D$ with an isomorphism $f_1:A\to C$ and arbitrary morphism $g_1:B\to C$, then their pullback:
    \begin{equation*}
    % https://q.uiver.app/#q=WzAsNCxbMSwwLCJBIl0sWzEsMSwiQyJdLFswLDEsIkIiXSxbMCwwLCJcXGxpbSBEIl0sWzIsMSwiZ18xIiwyXSxbMywwLCJnXzIiXSxbMywyLCJmXzIiLDJdLFszLDEsIiIsMSx7InN0eWxlIjp7Im5hbWUiOiJjb3JuZXIifX1dLFswLDEsImZfMSJdLFszLDIsIlxcY29uZyJdLFswLDEsIlxcY29uZyIsMl1d
    \begin{tikzcd}
        {\lim D} & A \\
        B & C
        \arrow["{g_1}"', from=2-1, to=2-2]
        \arrow["{g_2}", from=1-1, to=1-2]
        \arrow["{f_2}"', from=1-1, to=2-1]
        \arrow["\lrcorner"{anchor=center, pos=0.125}, draw=none, from=1-1, to=2-2]
        \arrow["{f_1}", from=1-2, to=2-2]
        \arrow["\cong", from=1-1, to=2-1]
        \arrow["\cong"', from=1-2, to=2-2]
    \end{tikzcd}
    \end{equation*}
    gives us morphism $f_2:\lim D\to B$, which is isomorphism.
\end{proposition}

\begin{dem}
\begin{proof}
    We have to find the inverse of $f_2$ to show that it is isomorphic. Using the universal property, we can define a cone with a tip of $B$ and the map: $\operatorname{id}_B:B\to B$ and $f^{-1}_1\circ g_1:B\to A$ (as $f$ is invertible). Note that it is clear that the cone commutes i.e $f_1\circ f^{-1}_1\circ g_1=g_1\circ\operatorname{id}_B$. And so, there would be a unique map $f':B\to\lim D$, see red diagram:
    \begin{equation*}
    % https://q.uiver.app/#q=WzAsNSxbMiwxLCJBIl0sWzIsMiwiQyJdLFsxLDEsIlxcbGltIEQiXSxbMCwwLCJCIl0sWzEsMiwiQiJdLFsyLDAsImdfMiJdLFsyLDEsIiIsMSx7InN0eWxlIjp7Im5hbWUiOiJjb3JuZXIifX1dLFswLDEsImZfMSJdLFswLDEsIlxcY29uZyIsMl0sWzMsMCwiZl8xXnstMX1cXGNpcmMgZ18xIiwxLHsiY3VydmUiOi0yfV0sWzQsMSwiZ18xIiwyXSxbMiw0LCJmXzIiLDJdLFszLDQsIlxcb3BlcmF0b3JuYW1le2lkfV9CIiwxLHsiY3VydmUiOjJ9XSxbMywyLCJmJyIsMSx7InN0eWxlIjp7ImJvZHkiOnsibmFtZSI6ImRhc2hlZCJ9fX1dXQ==
    \begin{tikzcd}
        B \\
        & {\lim D} & A \\
        & B & C
        \arrow["{g_2}", from=2-2, to=2-3]
        \arrow["\lrcorner"{anchor=center, pos=0.125}, draw=none, from=2-2, to=3-3]
        \arrow["{f_1}", from=2-3, to=3-3]
        \arrow["\cong"', from=2-3, to=3-3]
        \arrow["{f_1^{-1}\circ g_1}"{description}, curve={height=-12pt}, from=1-1, to=2-3]
        \arrow["{g_1}"', from=3-2, to=3-3]
        \arrow["{f_2}"', from=2-2, to=3-2]
        \arrow["{\operatorname{id}_B}"{description}, curve={height=12pt}, from=1-1, to=3-2]
        \arrow["{f'}"{description}, dashed, from=1-1, to=2-2]
    \end{tikzcd}
    \qquad \quad 
    % https://q.uiver.app/#q=WzAsNSxbMiwxLCJBIl0sWzIsMiwiQyJdLFsxLDEsIlxcbGltIEQiXSxbMSwyLCJCIl0sWzAsMCwiXFxsaW0gRCJdLFsyLDAsImdfMiJdLFsyLDEsIiIsMSx7InN0eWxlIjp7Im5hbWUiOiJjb3JuZXIifX1dLFswLDEsImZfMSJdLFswLDEsIlxcY29uZyIsMl0sWzMsMSwiZ18xIiwyXSxbMiwzLCJmXzIiLDJdLFs0LDAsImZfMV57LTF9XFxjaXJjIGdfMVxcY2lyYyBmXzIiLDAseyJjdXJ2ZSI6LTJ9XSxbNCwzLCJmXzIiLDIseyJjdXJ2ZSI6Mn1dLFs0LDIsImYnXFxjaXJjIGZfMiIsMSx7InN0eWxlIjp7ImJvZHkiOnsibmFtZSI6ImRhc2hlZCJ9fX1dXQ==
    \begin{tikzcd}
        {\lim D} \\
        & {\lim D} & A \\
        & B & C
        \arrow["{g_2}", from=2-2, to=2-3]
        \arrow["\lrcorner"{anchor=center, pos=0.125}, draw=none, from=2-2, to=3-3]
        \arrow["{f_1}", from=2-3, to=3-3]
        \arrow["\cong"', from=2-3, to=3-3]
        \arrow["{g_1}"', from=3-2, to=3-3]
        \arrow["{f_2}"', from=2-2, to=3-2]
        \arrow["{f_1^{-1}\circ g_1\circ f_2}", curve={height=-12pt}, from=1-1, to=2-3]
        \arrow["{f_2}"', curve={height=12pt}, from=1-1, to=3-2]
        \arrow["{f'\circ f_2}"{description}, dashed, from=1-1, to=2-2]
    \end{tikzcd}
    \qquad \quad 
    % https://q.uiver.app/#q=WzAsNixbMywyLCJBIl0sWzMsMywiQyJdLFsyLDIsIlxcbGltIEQiXSxbMSwxLCJCIl0sWzIsMywiQiJdLFswLDAsIlxcbGltIEQiXSxbMiwwLCJnXzIiXSxbMiwxLCIiLDEseyJzdHlsZSI6eyJuYW1lIjoiY29ybmVyIn19XSxbMCwxLCJmXzEiXSxbMCwxLCJcXGNvbmciLDJdLFszLDAsImZfMV57LTF9XFxjaXJjIGdfMSIsMSx7ImN1cnZlIjotMiwiY29sb3VyIjpbMCw2MCw2MF19LFswLDYwLDYwLDFdXSxbNCwxLCJnXzEiLDJdLFsyLDQsImZfMiIsMl0sWzMsNCwiXFxvcGVyYXRvcm5hbWV7aWR9X0IiLDEseyJjdXJ2ZSI6MiwiY29sb3VyIjpbMCw2MCw2MF19LFswLDYwLDYwLDFdXSxbMywyLCJmJyIsMSx7ImNvbG91ciI6WzAsNjAsNjBdLCJzdHlsZSI6eyJib2R5Ijp7Im5hbWUiOiJkYXNoZWQifX19LFswLDYwLDYwLDFdXSxbNSwzLCJmXzIiLDEseyJjb2xvdXIiOlsyNDAsNjAsNjBdfSxbMjQwLDYwLDYwLDFdXSxbNSwwLCJnXzIiLDAseyJjdXJ2ZSI6LTMsImNvbG91ciI6WzI0MCw2MCw2MF19LFsyNDAsNjAsNjAsMV1dLFs1LDQsImZfMiIsMix7ImN1cnZlIjozLCJjb2xvdXIiOlsyNDAsNjAsNjBdfSxbMjQwLDYwLDYwLDFdXV0=
    \end{equation*}
    By the way we define, we have $f_2\circ f'=\operatorname{id}_B$. We then use the same trick to get RHS diagram, where we note that $f_1^{-1}\circ g_1\circ f_2=g_2$ (thus the diagram all commute) and so $f'\circ f_2=\operatorname{id}_{\lim D}$ as constructed from a cone whose edge are $g_2$ and $f_2$.
\end{proof}
\end{dem}

\begin{proposition}
    \label{prop:identity-pullback}
    The pullback of an identity along any morphism $g_1:B\to C$ is an identity. Then we have the following pullback (on the LHS).
    \begin{equation*}
    % https://q.uiver.app/#q=WzAsNCxbMSwwLCJBIl0sWzEsMSwiQSJdLFswLDAsIkIiXSxbMCwxLCJCIl0sWzIsMCwiZyJdLFsyLDEsIiIsMSx7InN0eWxlIjp7Im5hbWUiOiJjb3JuZXIifX1dLFszLDEsImciLDJdLFsyLDMsIiIsMix7ImxldmVsIjoyLCJzdHlsZSI6eyJoZWFkIjp7Im5hbWUiOiJub25lIn19fV0sWzAsMSwiIiwwLHsibGV2ZWwiOjIsInN0eWxlIjp7ImhlYWQiOnsibmFtZSI6Im5vbmUifX19XV0=
    \begin{tikzcd}
        B & A \\
        B & A
        \arrow["g", from=1-1, to=1-2]
        \arrow["\lrcorner"{anchor=center, pos=0.125}, draw=none, from=1-1, to=2-2]
        \arrow["g"', from=2-1, to=2-2]
        \arrow[Rightarrow, no head, from=1-1, to=2-1]
        \arrow[Rightarrow, no head, from=1-2, to=2-2]
    \end{tikzcd}
    \qquad \quad 
    % https://q.uiver.app/#q=WzAsNCxbMSwwLCJBIl0sWzEsMSwiQSJdLFswLDAsIlxcbGltIEQiXSxbMCwxLCJCIl0sWzIsMCwiZyciXSxbMiwxLCIiLDEseyJzdHlsZSI6eyJuYW1lIjoiY29ybmVyIn19XSxbMywxLCJnIiwyXSxbMiwzLCJmXzIiLDJdLFswLDEsIiIsMCx7ImxldmVsIjoyLCJzdHlsZSI6eyJoZWFkIjp7Im5hbWUiOiJub25lIn19fV1d
    \begin{tikzcd}
        {\lim D} & A \\
        B & A
        \arrow["{g'}", from=1-1, to=1-2]
        \arrow["\lrcorner"{anchor=center, pos=0.125}, draw=none, from=1-1, to=2-2]
        \arrow["g"', from=2-1, to=2-2]
        \arrow["{f_2}"', from=1-1, to=2-1]
        \arrow[Rightarrow, no head, from=1-2, to=2-2]
    \end{tikzcd}
    \end{equation*}
    \textbf{Warning}, we are only required to show that the LHS diagram is a pullback, but we note that there can be another pullback given by the RHS diagram where $\lim D\ne B$ only $\lim D\cong B$ since $\operatorname{id}_A$ can be consider to be an isomorphism too.
\end{proposition}
\begin{dem}
\begin{proof}
    Given any object $C$, we have the following diagram:
    \begin{equation*}
    % https://q.uiver.app/#q=WzAsNSxbMiwxLCJBIl0sWzIsMiwiQSJdLFsxLDEsIkIiXSxbMSwyLCJCIl0sWzAsMCwiQyJdLFsyLDAsImciXSxbMiwxLCIiLDEseyJzdHlsZSI6eyJuYW1lIjoiY29ybmVyIn19XSxbMywxLCJnIiwyXSxbMiwzLCIiLDIseyJsZXZlbCI6Miwic3R5bGUiOnsiaGVhZCI6eyJuYW1lIjoibm9uZSJ9fX1dLFswLDEsIiIsMCx7ImxldmVsIjoyLCJzdHlsZSI6eyJoZWFkIjp7Im5hbWUiOiJub25lIn19fV0sWzQsMiwiaCIsMSx7InN0eWxlIjp7ImJvZHkiOnsibmFtZSI6ImRhc2hlZCJ9fX1dLFs0LDAsImdcXGNpcmMgaCIsMCx7ImN1cnZlIjotMn1dLFs0LDMsImgiLDIseyJjdXJ2ZSI6Mn1dXQ==
    \begin{tikzcd}
        C \\
        & B & A \\
        & B & A
        \arrow["g", from=2-2, to=2-3]
        \arrow["\lrcorner"{anchor=center, pos=0.125}, draw=none, from=2-2, to=3-3]
        \arrow["g"', from=3-2, to=3-3]
        \arrow[Rightarrow, no head, from=2-2, to=3-2]
        \arrow[Rightarrow, no head, from=2-3, to=3-3]
        \arrow["h"{description}, dashed, from=1-1, to=2-2]
        \arrow["{g\circ h}", curve={height=-12pt}, from=1-1, to=2-3]
        \arrow["h"', curve={height=12pt}, from=1-1, to=3-2]
    \end{tikzcd}
    \end{equation*}
    The $C\to A$ can be derived from the commutativity of the cone, thus everyhing is specified by the map $C\to B$ uniquely, as needed.
\end{proof}
\end{dem}

Now, we consider the 2 pullback diagram together, and show that there is no ambiguity in doing so. 

\begin{proposition}
    Given the following commutative diagram, suppose that $(B, C, B', C')$ square are pullback:
    \begin{equation*}
    % https://q.uiver.app/#q=WzAsNixbMCwwLCJBIl0sWzEsMCwiQiJdLFsyLDAsIkMiXSxbMCwxLCJBJyJdLFsxLDEsIkInIl0sWzIsMSwiQyciXSxbMiw1LCJjIl0sWzEsMiwiZyJdLFs0LDUsImcnIiwyXSxbMSw0LCJiIiwyXSxbMCwzLCJhIiwyXSxbMyw0LCJmJyIsMl0sWzAsMSwiZiJdLFsxLDUsIiIsMCx7InN0eWxlIjp7Im5hbWUiOiJjb3JuZXIifX1dXQ==
    \begin{tikzcd}
        A & B & C \\
        {A'} & {B'} & {C'}
        \arrow["c", from=1-3, to=2-3]
        \arrow["g", from=1-2, to=1-3]
        \arrow["{g'}"', from=2-2, to=2-3]
        \arrow["b"', from=1-2, to=2-2]
        \arrow["a"', from=1-1, to=2-1]
        \arrow["{f'}"', from=2-1, to=2-2]
        \arrow["f", from=1-1, to=1-2]
        \arrow["\lrcorner"{anchor=center, pos=0.125}, draw=none, from=1-2, to=2-3]
    \end{tikzcd}
    \end{equation*}
    Then $(A, B, A', B')$ square is a pullback iff $(A, C, A', C')$ rectangle is a pullback.
\end{proposition}
\begin{dem}
\begin{proof}
    $\boldsymbol{(\implies)}$ If $(A, B, A', B')$ is a pullback, then is if we consider arbitrary object $X$ in $\textbf{C}$ with the morphism $h:X\to A$, then we got 2 commutes diagrams:
    \begin{equation*}
    % https://q.uiver.app/#q=WzAsMTAsWzEsMSwiQSJdLFsyLDEsIkIiXSxbNSwxLCJDIl0sWzEsMiwiQSciXSxbMiwyLCJCJyJdLFs1LDIsIkMnIl0sWzAsMCwiWCJdLFs0LDEsIkIiXSxbNCwyLCJCJyJdLFszLDAsIlgiXSxbMiw1LCJjIl0sWzEsNCwiYiJdLFswLDMsImEiLDJdLFszLDQsImYnIiwyXSxbMCwxLCJmIl0sWzYsMCwiaCIsMSx7InN0eWxlIjp7ImJvZHkiOnsibmFtZSI6ImRhc2hlZCJ9fX1dLFswLDQsIiIsMCx7InN0eWxlIjp7Im5hbWUiOiJjb3JuZXIifX1dLFs2LDMsInAiLDIseyJjdXJ2ZSI6Mn1dLFs2LDEsInEiLDAseyJjdXJ2ZSI6LTJ9XSxbNyw1LCIiLDAseyJzdHlsZSI6eyJuYW1lIjoiY29ybmVyIn19XSxbNywyLCJnIl0sWzgsNSwiZyciLDJdLFs3LDgsImIiLDJdLFs5LDcsInEiLDAseyJzdHlsZSI6eyJib2R5Ijp7Im5hbWUiOiJkYXNoZWQifX19XSxbOSw4LCJxXzEiLDIseyJjdXJ2ZSI6Mn1dLFs5LDIsInFfMiIsMCx7ImN1cnZlIjotMn1dXQ==
    \begin{tikzcd}
        X &&& X \\
        & A & B && B & C \\
        & {A'} & {B'} && {B'} & {C'}
        \arrow["c", from=2-6, to=3-6]
        \arrow["b", from=2-3, to=3-3]
        \arrow["a"', from=2-2, to=3-2]
        \arrow["{f'}"', from=3-2, to=3-3]
        \arrow["f", from=2-2, to=2-3]
        \arrow["h"{description}, dashed, from=1-1, to=2-2]
        \arrow["\lrcorner"{anchor=center, pos=0.125}, draw=none, from=2-2, to=3-3]
        \arrow["p"', curve={height=12pt}, from=1-1, to=3-2]
        \arrow["q", curve={height=-12pt}, from=1-1, to=2-3]
        \arrow["\lrcorner"{anchor=center, pos=0.125}, draw=none, from=2-5, to=3-6]
        \arrow["g", from=2-5, to=2-6]
        \arrow["{g'}"', from=3-5, to=3-6]
        \arrow["b"', from=2-5, to=3-5]
        \arrow["q", dashed, from=1-4, to=2-5]
        \arrow["{q_1}"', curve={height=12pt}, from=1-4, to=3-5]
        \arrow["{q_2}", curve={height=-12pt}, from=1-4, to=2-6]
    \end{tikzcd}
    \end{equation*}
    That is $h$ induces the 2 unique arrows $q:X\to B$ and $p:X\to A'$ from pullback square $(A, B, A', B')$, especially the former, where $q$ will also induces 2 more unique arrow $q_2:X\to C$ and $q_1:X\to B$ , since the square $(B, C, B', C')$ is a pullback. It is clear that given a unique $h$, we will get a unique $q_2$. Furthermore, we have the following equalities:
    \begin{equation*}
    \begin{aligned}
        f'\circ p = f'\circ a \circ h = b\circ f\circ h = b\circ q = q_1
    \end{aligned}
    \end{equation*}
    Thus $q_1$ and $p$ are actually ``coherence'', so we can define an unique pair of morphism $(p, q_2)$ to be assigned to $h$ and vice versa i.e universal property. Thus $(A, C, A', C')$ is a pullback.

    $\boldsymbol{(\impliedby)}$ If $(A, C, A', C')$ is a pullback, then if we consider arbitrary object $X$ in $\textbf{C}$ with morphism $h:X\to X$, then we got 2 commutative diagrams:
    \begin{equation*}
    % https://q.uiver.app/#q=WzAsMTAsWzEsMSwiQSJdLFsyLDEsIkMiXSxbMSwyLCJBJyJdLFsyLDIsIkMnIl0sWzAsMCwiWCJdLFs1LDEsIkIiXSxbNSwyLCJCJyJdLFs2LDEsIkMiXSxbNiwyLCJDJyJdLFs0LDAsIlgiXSxbMSwzLCJjIl0sWzAsMiwiYSIsMl0sWzIsMywiZydcXGNpcmMgZiciLDJdLFswLDEsImdcXGNpcmMgZiJdLFs0LDAsImgiLDEseyJzdHlsZSI6eyJib2R5Ijp7Im5hbWUiOiJkYXNoZWQifX19XSxbMCwzLCIiLDAseyJzdHlsZSI6eyJuYW1lIjoiY29ybmVyIn19XSxbNCwyLCJxXzEiLDIseyJjdXJ2ZSI6Mn1dLFs0LDEsInFfMiIsMCx7ImN1cnZlIjotMn1dLFs2LDgsImcnIiwyXSxbNSw2LCJiIiwyXSxbNyw4LCJjIl0sWzUsNywiZyJdLFs5LDcsInFfMiIsMCx7ImN1cnZlIjotMn1dLFs5LDYsImYnXFxjaXJjIHFfMSIsMix7ImN1cnZlIjoyfV0sWzksNSwicSIsMCx7InN0eWxlIjp7ImJvZHkiOnsibmFtZSI6ImRhc2hlZCJ9fX1dXQ==
    \begin{tikzcd}
        X &&&& X \\
        & A & C &&& B & C \\
        & {A'} & {C'} &&& {B'} & {C'}
        \arrow["c", from=2-3, to=3-3]
        \arrow["a"', from=2-2, to=3-2]
        \arrow["{g'\circ f'}"', from=3-2, to=3-3]
        \arrow["{g\circ f}", from=2-2, to=2-3]
        \arrow["h"{description}, dashed, from=1-1, to=2-2]
        \arrow["\lrcorner"{anchor=center, pos=0.125}, draw=none, from=2-2, to=3-3]
        \arrow["{q_1}"', curve={height=12pt}, from=1-1, to=3-2]
        \arrow["{q_2}", curve={height=-12pt}, from=1-1, to=2-3]
        \arrow["{g'}"', from=3-6, to=3-7]
        \arrow["b"', from=2-6, to=3-6]
        \arrow["c", from=2-7, to=3-7]
        \arrow["g", from=2-6, to=2-7]
        \arrow["{q_2}", curve={height=-12pt}, from=1-5, to=2-7]
        \arrow["{f'\circ q_1}"', curve={height=12pt}, from=1-5, to=3-6]
        \arrow["q", dashed, from=1-5, to=2-6]
    \end{tikzcd}
    \end{equation*}
    That is $h$ induces 2 arrows $q_1:X\to A'$ and $q_2:X\to C$ from pullback square $(A, C, A', C')$. Then, we can create 1 more arrow $q_1\circ f':X\to B'$ which, together with $q_2$, we are able to induce a unqiue arrow $q:X\to B$ via pullback square $(B, C, B', C')$. First, we note that: $g\circ q = q_2 = g\circ f\circ h$. Let's consider, the alternative pullback diagram of $(B, C, B', C')$, on LHS 
    \begin{equation*}
    % https://q.uiver.app/#q=WzAsMTAsWzEsMSwiQiJdLFsxLDIsIkInIl0sWzIsMSwiQyJdLFsyLDIsIkMnIl0sWzAsMCwiWCJdLFs0LDAsIlgiXSxbNSwxLCJBIl0sWzUsMiwiQSciXSxbNiwxLCJCIl0sWzYsMiwiQiciXSxbMSwzLCJnJyIsMl0sWzAsMSwiYiIsMix7ImNvbG91ciI6WzAsNjAsNjBdfSxbMCw2MCw2MCwxXV0sWzIsMywiYyJdLFswLDIsImciXSxbNCwyLCJxXzIiLDAseyJjdXJ2ZSI6LTJ9XSxbNCwxLCJmJ1xcY2lyYyBxXzEiLDIseyJjdXJ2ZSI6MiwiY29sb3VyIjpbMCw2MCw2MF19LFswLDYwLDYwLDFdXSxbNCwwLCJmXFxjaXJjIGgiLDEseyJjb2xvdXIiOlswLDYwLDYwXSwic3R5bGUiOnsiYm9keSI6eyJuYW1lIjoiZGFzaGVkIn19fSxbMCw2MCw2MCwxXV0sWzcsOSwiZiciLDJdLFs2LDcsImEiLDJdLFs4LDksImIiXSxbNiw4LCJmIl0sWzUsOCwicSIsMCx7ImN1cnZlIjotMiwic3R5bGUiOnsiYm9keSI6eyJuYW1lIjoiZGFzaGVkIn19fV0sWzUsNywicV8xIiwyLHsiY3VydmUiOjJ9XSxbNSw2LCJoIiwxLHsic3R5bGUiOnsiYm9keSI6eyJuYW1lIjoiZGFzaGVkIn19fV0sWzAsMywiIiwwLHsic3R5bGUiOnsibmFtZSI6ImNvcm5lciJ9fV1d
    \begin{tikzcd}
        X &&&& X \\
        & B & C &&& A & B \\
        & {B'} & {C'} &&& {A'} & {B'}
        \arrow["{g'}"', from=3-2, to=3-3]
        \arrow["b"', color={rgb,255:red,214;green,92;blue,92}, from=2-2, to=3-2]
        \arrow["c", from=2-3, to=3-3]
        \arrow["g", from=2-2, to=2-3]
        \arrow["{q_2}", curve={height=-12pt}, from=1-1, to=2-3]
        \arrow["{f'\circ q_1}"', color={rgb,255:red,214;green,92;blue,92}, curve={height=12pt}, from=1-1, to=3-2]
        \arrow["{f\circ h}"{description}, color={rgb,255:red,214;green,92;blue,92}, dashed, from=1-1, to=2-2]
        \arrow["{f'}"', from=3-6, to=3-7]
        \arrow["a"', from=2-6, to=3-6]
        \arrow["b", from=2-7, to=3-7]
        \arrow["f", from=2-6, to=2-7]
        \arrow["q", curve={height=-12pt}, dashed, from=1-5, to=2-7]
        \arrow["{q_1}"', curve={height=12pt}, from=1-5, to=3-6]
        \arrow["h"{description}, dashed, from=1-5, to=2-6]
        \arrow["\lrcorner"{anchor=center, pos=0.125}, draw=none, from=2-2, to=3-3]
    \end{tikzcd}
    \end{equation*}
    We can show that the area highlighted in red commtues: $b\circ f\circ h = f'\circ a \circ h = f'\circ q_1$. Thus, $f\circ h$ also corresponds to a pair of arrows $(q_2, f'\circ q_1)$ just like $q$. Therefore, $f\circ h=q$ or the RHS diagram commutes. Thus, we have a unique pair of arrows $(q, q_1)$ given any $h:X\to A$ i.e a pullback of $(A, B, A', B')$, as needed.
\end{proof}
\end{dem}

\begin{proposition}
    Consider the following pullback diagram 
    \begin{equation*}
    % https://q.uiver.app/#q=WzAsNCxbMCwwLCJBJyJdLFsxLDAsIkEiXSxbMCwxLCJCJyJdLFsxLDEsIkIiXSxbMSwzLCJmIiwwLHsic3R5bGUiOnsidGFpbCI6eyJuYW1lIjoibW9ubyJ9fX1dLFswLDIsImYnIiwyXSxbMCwxLCJnIl0sWzIsMywiaCIsMl0sWzAsMywiIiwxLHsic3R5bGUiOnsibmFtZSI6ImNvcm5lciJ9fV1d
    \begin{tikzcd}
        {A'} & A \\
        {B'} & B
        \arrow["f", tail, from=1-2, to=2-2]
        \arrow["{f'}"', from=1-1, to=2-1]
        \arrow["g", from=1-1, to=1-2]
        \arrow["h"', from=2-1, to=2-2]
        \arrow["\lrcorner"{anchor=center, pos=0.125}, draw=none, from=1-1, to=2-2]
    \end{tikzcd}
    \end{equation*}
    If $f$ is a monomorphism, then $f'$ is also a monomorphism.
\end{proposition}

\begin{dem}
\begin{proof}
    From remark \ref{remark:mono-pullback}, we want to proof that the left square of the LHS diagram is a pullback:
    \begin{equation*}
    % https://q.uiver.app/#q=WzAsMTYsWzcsMCwiQSciLFsxMzksMjMsNDEsMV1dLFs4LDAsIkEiXSxbNywxLCJBJyIsWzEzOSwyMyw0MSwxXV0sWzksMSwiQiIsWzI0MCw2MCw2MCwxXV0sWzgsMSwiQSJdLFs5LDAsIkEiLFsyNDAsNjAsNjAsMV1dLFswLDAsIkEnIl0sWzEsMCwiQSciXSxbMSwxLCJCJyJdLFsyLDEsIkIiXSxbMiwwLCJBIl0sWzAsMSwiQSciXSxbNCwwLCJBJyJdLFs0LDEsIkEnIl0sWzUsMCwiQSJdLFs1LDEsIkIiXSxbMCwxLCJnIiwwLHsiY29sb3VyIjpbMTM5LDIzLDQxXX0sWzEzOSwyMyw0MSwxXV0sWzIsNCwiZyIsMix7ImNvbG91ciI6WzEzOSwyMyw0MV19LFsxMzksMjMsNDEsMV1dLFs0LDMsImYiLDIseyJjb2xvdXIiOlsyNDAsNjAsNjBdfSxbMjQwLDYwLDYwLDFdXSxbNSwzLCJmIiwwLHsiY29sb3VyIjpbMjQwLDYwLDYwXX0sWzI0MCw2MCw2MCwxXV0sWzEsNSwiXFxvcGVyYXRvcm5hbWV7aWR9X0EiLDAseyJjb2xvdXIiOlsyNDAsNjAsNjBdfSxbMjQwLDYwLDYwLDFdXSxbMSw0LCIiLDEseyJjb2xvdXIiOlswLDYwLDYwXX1dLFs2LDExLCJcXG9wZXJhdG9ybmFtZXtpZH1fe0EnfSIsMl0sWzcsOCwiZiciLDJdLFs3LDEwLCJnIl0sWzcsOSwiIiwxLHsic3R5bGUiOnsibmFtZSI6ImNvcm5lciJ9fV0sWzgsOSwiaCIsMl0sWzExLDgsImYnIiwyXSxbMTAsOSwiZiIsMCx7InN0eWxlIjp7InRhaWwiOnsibmFtZSI6Im1vbm8ifX19XSxbNiw3LCJcXG9wZXJhdG9ybmFtZXtpZH1fe0EnfSJdLFsxNCwxNSwiZiJdLFsxMiwxNCwiZyJdLFsxMywxNSwiaFxcY2lyYyBmJyIsMl0sWzEyLDEzLCIiLDIseyJsZXZlbCI6Miwic3R5bGUiOnsiaGVhZCI6eyJuYW1lIjoibm9uZSJ9fX1dLFswLDIsIiIsMix7ImxldmVsIjoyLCJjb2xvdXIiOlsxMzksMjMsNDFdLCJzdHlsZSI6eyJoZWFkIjp7Im5hbWUiOiJub25lIn19fV0sWzAsNCwiIiwwLHsic3R5bGUiOnsibmFtZSI6ImNvcm5lciJ9fV0sWzEsMywiIiwwLHsic3R5bGUiOnsibmFtZSI6ImNvcm5lciJ9fV1d
    \begin{tikzcd}
        {A'} & {A'} & A && {A'} & A && \textcolor{rgb,255:red,81;green,129;blue,96}{A'} & A & \textcolor{rgb,255:red,92;green,92;blue,214}{A} \\
        {A'} & {B'} & B && {A'} & B && \textcolor{rgb,255:red,81;green,129;blue,96}{A'} & A & \textcolor{rgb,255:red,92;green,92;blue,214}{B}
        \arrow["g", color={rgb,255:red,81;green,129;blue,96}, from=1-8, to=1-9]
        \arrow["g"', color={rgb,255:red,81;green,129;blue,96}, from=2-8, to=2-9]
        \arrow["f"', color={rgb,255:red,92;green,92;blue,214}, from=2-9, to=2-10]
        \arrow["f", color={rgb,255:red,92;green,92;blue,214}, from=1-10, to=2-10]
        \arrow["{\operatorname{id}_A}", color={rgb,255:red,92;green,92;blue,214}, from=1-9, to=1-10]
        \arrow[color={rgb,255:red,214;green,92;blue,92}, from=1-9, to=2-9]
        \arrow["{\operatorname{id}_{A'}}"', from=1-1, to=2-1]
        \arrow["{f'}"', from=1-2, to=2-2]
        \arrow["g", from=1-2, to=1-3]
        \arrow["\lrcorner"{anchor=center, pos=0.125}, draw=none, from=1-2, to=2-3]
        \arrow["h"', from=2-2, to=2-3]
        \arrow["{f'}"', from=2-1, to=2-2]
        \arrow["f", tail, from=1-3, to=2-3]
        \arrow["{\operatorname{id}_{A'}}", from=1-1, to=1-2]
        \arrow["f", from=1-6, to=2-6]
        \arrow["g", from=1-5, to=1-6]
        \arrow["{h\circ f'}"', from=2-5, to=2-6]
        \arrow[Rightarrow, no head, from=1-5, to=2-5]
        \arrow[color={rgb,255:red,81;green,129;blue,96}, Rightarrow, no head, from=1-8, to=2-8]
        \arrow["\lrcorner"{anchor=center, pos=0.125}, draw=none, from=1-8, to=2-9]
        \arrow["\lrcorner"{anchor=center, pos=0.125}, draw=none, from=1-9, to=2-10]
    \end{tikzcd}
    \end{equation*}
    Using the proposition above, it would be suffices to proof that the middle diagram is a pullback. We first notice that $h\circ f' = f\circ g$, and so we can ``extends'' the diagram to be, the diagram on the right. 
    
    If we have the {\color{rgb,255:red,214;green,92;blue,92} red arrow} to be an identity arrow, then the {\color{rgb,255:red,92;green,92;blue,214} blue square} is a pullback as $f$ is an monomorphism, while the {\color{rgb,255:red,81;green,129;blue,96} green square} is a pullback as shown in proposition \ref{prop:identity-pullback}. Thus by proposition above, the square $(A', A, A', B)$ is a pullback as needed.
\end{proof}
\end{dem}

\begin{definition}{\textbf{(Kernel Pair)}}
    Given the morphism $f:X\rightarrow Y$, we have the pullback of:
    \begin{equation*}
    % https://q.uiver.app/#q=WzAsNCxbMCwwLCJYXFx0aW1lc19ZIFgiXSxbMSwwLCJYIl0sWzAsMSwiWCJdLFsxLDEsIlkiXSxbMSwzLCJmIl0sWzIsMywiZiIsMl0sWzAsMl0sWzAsMV0sWzAsMywiIiwxLHsic3R5bGUiOnsibmFtZSI6ImNvcm5lciJ9fV1d
    \begin{tikzcd}
        {X\times_Y X} & X \\
        X & Y
        \arrow["f", from=1-2, to=2-2]
        \arrow["f"', from=2-1, to=2-2]
        \arrow[from=1-1, to=2-1]
        \arrow[from=1-1, to=1-2]
        \arrow["\lrcorner"{anchor=center, pos=0.125}, draw=none, from=1-1, to=2-2]
    \end{tikzcd}
    \end{equation*}
    This gives us the universal map $X\times_YX\rightarrow X$, which are parallel maps and called \textit{kernel pair} of $f$. 
\end{definition}

\begin{proposition}
    The morphism $f$ (as defined above) is mono iff those maps coincides and $X\times_YX\cong X$.
\end{proposition}
\begin{dem}
\begin{proof}
    $\boldsymbol{(\implies):}$ If $f$ is mono, then the map $a,b:X\times_Y X\to X$ are the same because $f\circ a=f\circ b$ by the commutativity and by the definition of monomorphism. Let's consider the isomorphism, we have the following 2 commutative diagrams:
    \begin{equation*}
    % https://q.uiver.app/#q=WzAsNSxbMSwxLCJYXFx0aW1lc19ZIFgiXSxbMiwxLCJYIl0sWzEsMiwiWCJdLFsyLDIsIlkiXSxbMCwwLCJYIl0sWzEsMywiZiJdLFsyLDMsImYiLDJdLFswLDIsImEiLDJdLFswLDEsImEiXSxbMCwzLCIiLDEseyJzdHlsZSI6eyJuYW1lIjoiY29ybmVyIn19XSxbNCwwLCJoIiwxLHsic3R5bGUiOnsiYm9keSI6eyJuYW1lIjoiZGFzaGVkIn19fV0sWzQsMSwiXFxvcGVyYXRvcm5hbWV7aWR9X1giLDEseyJjdXJ2ZSI6LTJ9XSxbNCwyLCJcXG9wZXJhdG9ybmFtZXtpZH1fWCIsMSx7ImN1cnZlIjoyfV1d
    \begin{tikzcd}
        X \\
        & {X\times_Y X} & X \\
        & X & Y
        \arrow["f", from=2-3, to=3-3]
        \arrow["f"', from=3-2, to=3-3]
        \arrow["a"', from=2-2, to=3-2]
        \arrow["a", from=2-2, to=2-3]
        \arrow["\lrcorner"{anchor=center, pos=0.125}, draw=none, from=2-2, to=3-3]
        \arrow["h"{description}, dashed, from=1-1, to=2-2]
        \arrow["{\operatorname{id}_X}"{description}, curve={height=-12pt}, from=1-1, to=2-3]
        \arrow["{\operatorname{id}_X}"{description}, curve={height=12pt}, from=1-1, to=3-2]
    \end{tikzcd}
    \qquad \quad
    % https://q.uiver.app/#q=WzAsNSxbMSwxLCJYXFx0aW1lc19ZIFgiXSxbMiwxLCJYIl0sWzEsMiwiWCJdLFsyLDIsIlkiXSxbMCwwLCJYXFx0aW1lc19ZIFgiXSxbMSwzLCJmIl0sWzIsMywiZiIsMl0sWzAsMiwiYSIsMl0sWzAsMSwiYSJdLFswLDMsIiIsMSx7InN0eWxlIjp7Im5hbWUiOiJjb3JuZXIifX1dLFs0LDAsImhcXGNpcmMgYSIsMSx7InN0eWxlIjp7ImJvZHkiOnsibmFtZSI6ImRhc2hlZCJ9fX1dLFs0LDEsImEiLDEseyJjdXJ2ZSI6LTJ9XSxbNCwyLCJhIiwxLHsiY3VydmUiOjJ9XV0=
    \begin{tikzcd}
        {X\times_Y X} \\
        & {X\times_Y X} & X \\
        & X & Y
        \arrow["f", from=2-3, to=3-3]
        \arrow["f"', from=3-2, to=3-3]
        \arrow["a"', from=2-2, to=3-2]
        \arrow["a", from=2-2, to=2-3]
        \arrow["\lrcorner"{anchor=center, pos=0.125}, draw=none, from=2-2, to=3-3]
        \arrow["{h\circ a}"{description}, dashed, from=1-1, to=2-2]
        \arrow["a"{description}, curve={height=-12pt}, from=1-1, to=2-3]
        \arrow["a"{description}, curve={height=12pt}, from=1-1, to=3-2]
    \end{tikzcd}
    \end{equation*}
    where in LHS diagram, the unique morphism $h$ derives from the cone of edge $(\operatorname{id}_X, \operatorname{id}_X)$ and $a\circ h=\operatorname{id}_X$. On the other hand, we can see that $h\circ a=\operatorname{id}_{X\times_Y X}$ due to universal property.
\end{proof}
\end{dem}

This is where the name "kernel" comes from, where $f$ is trivial iff $f$ is mono. To see this in more details, with the example of category of $\textbf{Set}$, we have:

\begin{proposition}
    Given the function $m:X\rightarrow Y$ and consider the kernel pair $f,g:X\times_YX\rightarrow X$, we define the relation $\sim$ (similar to that above), for $x,x'\in X$, where $x\sim x'$ if given $p\in X\times_YX$ such that $f(p)=x$ and $g(p)=x'$. We can also show that:
    \begin{multicols}{2}
    \begin{itemize}
        \item $\sim$ is an equivalence relation
        \item $x\sim x'$ iff $m(x)=m(x')$ that is $\sim$ is the partition induced by $m$ whenever $m$ is surjective.
        \item If there exists $p\in X\times_YX$ such that $f(p)=x$ and $g(p)=x'$, then $p$ is unique.
    \end{itemize}
    \columnbreak
    \begin{equation*}
    % https://q.uiver.app/#q=WzAsNCxbMSwwLCJYIl0sWzEsMSwiWSJdLFswLDEsIlgiXSxbMCwwLCJYXFx0aW1lc19ZIFgiXSxbMiwxLCJtIiwyXSxbMCwxLCJtIl0sWzMsMiwiZiIsMl0sWzMsMCwiZyJdLFszLDEsIiIsMSx7InN0eWxlIjp7Im5hbWUiOiJjb3JuZXIifX1dXQ==
    \begin{tikzcd}
        {X\times_Y X} & X \\
        X & Y
        \arrow["m"', from=2-1, to=2-2]
        \arrow["m", from=1-2, to=2-2]
        \arrow["f"', from=1-1, to=2-1]
        \arrow["g", from=1-1, to=1-2]
        \arrow["\lrcorner"{anchor=center, pos=0.125}, draw=none, from=1-1, to=2-2]
    \end{tikzcd}
    \end{equation*}
    \end{multicols}
    Note that the right diagram shows the relationship between the kernel pairs.
\end{proposition}
\begin{dem}
\begin{proof}
    \textbf{(Part 1):}
    \textit{(Reflexive):} $x\sim x$ because suppose we are given $1$ together with the cone edge of $x:1\to X$ picking the same element.  Note that the diagram as a whole has to commute i.e $m(x)=m(x)$ too (see the LHS diagram), then this induces the unique element $p:1\to X\times_YX$ such that $f(p)=x=g(p)$
    \begin{equation*}
    % https://q.uiver.app/#q=WzAsMTUsWzYsMSwiWCJdLFs2LDIsIlkiXSxbNSwyLCJYIl0sWzUsMSwiWFxcdGltZXNfWSBYIl0sWzQsMCwiMSJdLFs5LDEsIlhcXHRpbWVzX1kgWCJdLFsxMCwxLCJYIl0sWzEwLDIsIlkiXSxbOSwyLCJYIl0sWzgsMCwiMSJdLFswLDAsIjEiXSxbMSwxLCJYXFx0aW1lc19ZIFgiXSxbMSwyLCJYIl0sWzIsMiwiWSJdLFsyLDEsIlgiXSxbMiwxLCJtIiwyXSxbMCwxLCJtIl0sWzMsMiwiZiIsMl0sWzMsMCwiZyJdLFszLDEsIiIsMSx7InN0eWxlIjp7Im5hbWUiOiJjb3JuZXIifX1dLFs0LDAsIngnIiwwLHsiY3VydmUiOi0yfV0sWzQsMywiaCIsMSx7InN0eWxlIjp7ImJvZHkiOnsibmFtZSI6ImRhc2hlZCJ9fX1dLFs0LDIsIngiLDIseyJjdXJ2ZSI6Mn1dLFs1LDgsImYiLDJdLFs1LDYsImciXSxbNiw3LCJtIl0sWzksNSwiaCciLDEseyJzdHlsZSI6eyJib2R5Ijp7Im5hbWUiOiJkYXNoZWQifX19XSxbOSw2LCJ4IiwwLHsiY3VydmUiOi0yfV0sWzksOCwieCciLDIseyJjdXJ2ZSI6Mn1dLFsxMCwxMSwicCIsMSx7InN0eWxlIjp7ImJvZHkiOnsibmFtZSI6ImRhc2hlZCJ9fX1dLFsxMCwxNCwieCIsMCx7ImN1cnZlIjotMn1dLFsxMCwxMiwieCIsMix7ImN1cnZlIjoyfV0sWzExLDEyLCJmIiwyXSxbMTEsMTQsImciXSxbMTIsMTMsIm0iLDJdLFsxNCwxMywibSJdLFs4LDcsIm0iLDJdXQ==
    \begin{tikzcd}
        1 &&&& 1 &&&& 1 \\
        & {X\times_Y X} & X &&& {X\times_Y X} & X &&& {X\times_Y X} & X \\
        & X & Y &&& X & Y &&& X & Y
        \arrow["m"', from=3-6, to=3-7]
        \arrow["m", from=2-7, to=3-7]
        \arrow["f"', from=2-6, to=3-6]
        \arrow["g", from=2-6, to=2-7]
        \arrow["\lrcorner"{anchor=center, pos=0.125}, draw=none, from=2-6, to=3-7]
        \arrow["{x'}", curve={height=-12pt}, from=1-5, to=2-7]
        \arrow["h"{description}, dashed, from=1-5, to=2-6]
        \arrow["x"', curve={height=12pt}, from=1-5, to=3-6]
        \arrow["f"', from=2-10, to=3-10]
        \arrow["g", from=2-10, to=2-11]
        \arrow["m", from=2-11, to=3-11]
        \arrow["{h'}"{description}, dashed, from=1-9, to=2-10]
        \arrow["x", curve={height=-12pt}, from=1-9, to=2-11]
        \arrow["{x'}"', curve={height=12pt}, from=1-9, to=3-10]
        \arrow["p"{description}, dashed, from=1-1, to=2-2]
        \arrow["x", curve={height=-12pt}, from=1-1, to=2-3]
        \arrow["x"', curve={height=12pt}, from=1-1, to=3-2]
        \arrow["f"', from=2-2, to=3-2]
        \arrow["g", from=2-2, to=2-3]
        \arrow["m"', from=3-2, to=3-3]
        \arrow["m", from=2-3, to=3-3]
        \arrow["m"', from=3-10, to=3-11]
    \end{tikzcd}
    \end{equation*}
    \textit{(Symmetric):} If $x\sim x'$, then let $h$ be element such that $f(h)=x$ and $g(h)=x'$ i.e the middle diagram above commutes. By the fact that $m(x')=m(x)$ (not all pair of $x$ and $x'$ will work), we can swap the arrow $x$ and $x'$ place, which will leads to a unique arrow $h':1\to X\times_YX$, in which $f(h')=x'$ and $g(h')=x$, as shown in the RHS diagram above. Thus $x'\sim x$ as needed.

    \textit{(Transiviity):} Assume that $a\sim b$ and $b\sim c$, then, by the observation above, we have $m(a)=m(b)=m(c)$ so we can have arrows $a:1\to X$ and $c:1\to X$ as edge of the cone (and its commutes!), which induces $k:1\to X\times_YX$ such that $a=f(k)$ and $c=g(k)$.

    \textbf{(Part 2):} With the observation that we had i.e the requirement that $m(x)=m(x')$ to be the cone commutes, this part is already shown. 

    \textbf{(Part 3):} This follows directly from the universal propery of a pullback where we consider the \textit{unique} arrow $p:1\to X\times_YX$ inducing the arrow $x,x':1\to X$. Note that the condition $m(x)=m(x')$ is guaranteed by the universal proeprty.
\end{proof}
\end{dem}

\begin{proposition}
    Given a morphism $f:X\to Y$, which as a kernel pair, furthermore, assume that $f$ is a coequalizer of something (see RHS diagram). We can show that $f$ is also a coequalizer of its kernel pair. This is given in the following diagram:
    \begin{equation*}
    % https://q.uiver.app/#q=WzAsNCxbMSwwLCJYIl0sWzEsMSwiWSJdLFswLDEsIlgiXSxbMCwwLCJYXFx0aW1lc19ZIFgiXSxbMiwxLCJmIiwyXSxbMCwxLCJmIl0sWzMsMiwiYiIsMl0sWzMsMCwiYSJdLFszLDEsIiIsMSx7InN0eWxlIjp7Im5hbWUiOiJjb3JuZXIifX1dXQ==
    \begin{tikzcd}
        {X\times_Y X} & X \\
        X & Y
        \arrow["f"', from=2-1, to=2-2]
        \arrow["f", from=1-2, to=2-2]
        \arrow["b"', from=1-1, to=2-1]
        \arrow["a", from=1-1, to=1-2]
        \arrow["\lrcorner"{anchor=center, pos=0.125}, draw=none, from=1-1, to=2-2]
    \end{tikzcd}\qquad \quad
    % https://q.uiver.app/#q=WzAsMyxbMCwwLCJBIl0sWzEsMCwiWCJdLFsyLDAsIlkiXSxbMCwxLCJjIiwwLHsib2Zmc2V0IjotMX1dLFswLDEsImMnIiwyLHsib2Zmc2V0IjoxfV0sWzEsMiwiZiJdXQ==
    \begin{tikzcd}
        A & X & Y
        \arrow["c", shift left, from=1-1, to=1-2]
        \arrow["{c'}"', shift right, from=1-1, to=1-2]
        \arrow["f", from=1-2, to=1-3]
    \end{tikzcd}
    \end{equation*}
\end{proposition}

\begin{dem}
\begin{proof}
    First we note that $f\circ a= f\circ b$ by the commutativity of the pullback diagram, so $f$ \textit{could be} the candidate for the coequalizer of $a$ and $b$. We are interested in finding the unique arrow $a:Y\to A$ where we are given morphism $g:X\to A$. 
    \begin{equation*}
    % https://q.uiver.app/#q=WzAsNCxbMCwxLCJYXFx0aW1lc19ZIFgiXSxbMSwxLCJYIl0sWzIsMCwiWSJdLFsyLDEsIkEiXSxbMCwxLCJiIiwyLHsib2Zmc2V0IjoxfV0sWzAsMSwiYSIsMCx7Im9mZnNldCI6LTF9XSxbMSwyLCJmIl0sWzEsMywiZyIsMl0sWzIsMywiYSIsMCx7InN0eWxlIjp7ImJvZHkiOnsibmFtZSI6ImRhc2hlZCJ9fX1dXQ==
    \begin{tikzcd}
        && Y \\
        {X\times_Y X} & X & A
        \arrow["b"', shift right, from=2-1, to=2-2]
        \arrow["a", shift left, from=2-1, to=2-2]
        \arrow["f", from=2-2, to=1-3]
        \arrow["g"', from=2-2, to=2-3]
        \arrow["a", dashed, from=1-3, to=2-3]
    \end{tikzcd}
    \end{equation*}
    To do this, we notice that we can find the unique arrow $h:A\to X\times_YX$ from the pullback $X\times_YX$, which is valid because $f\circ c=f\circ c'$. This si given in the LHS diagram. Furthermore, we notice further that $c$ is factorized by unique $h$, and so we are given the RHS diagram:
    \begin{equation*}
    % https://q.uiver.app/#q=WzAsNSxbMiwyLCJZIl0sWzIsMSwiWCJdLFsxLDEsIlhcXHRpbWVzX1kgWCJdLFsxLDIsIlgiXSxbMCwwLCJBIl0sWzMsMCwiZiIsMl0sWzEsMCwiZiJdLFsyLDAsIiIsMSx7InN0eWxlIjp7Im5hbWUiOiJjb3JuZXIifX1dLFsyLDEsImEiXSxbNCwxLCJjIiwwLHsiY3VydmUiOi0yfV0sWzIsMywiYiIsMl0sWzQsMiwiaCIsMSx7InN0eWxlIjp7ImJvZHkiOnsibmFtZSI6ImRhc2hlZCJ9fX1dLFs0LDMsImMnIiwyLHsiY3VydmUiOjJ9XV0=
    \begin{tikzcd}
        A \\
        & {X\times_Y X} & X \\
        & X & Y
        \arrow["f"', from=3-2, to=3-3]
        \arrow["f", from=2-3, to=3-3]
        \arrow["\lrcorner"{anchor=center, pos=0.125}, draw=none, from=2-2, to=3-3]
        \arrow["a", from=2-2, to=2-3]
        \arrow["c", curve={height=-12pt}, from=1-1, to=2-3]
        \arrow["b"', from=2-2, to=3-2]
        \arrow["h"{description}, dashed, from=1-1, to=2-2]
        \arrow["{c'}"', curve={height=12pt}, from=1-1, to=3-2]
    \end{tikzcd}
    \qquad \quad
    % https://q.uiver.app/#q=WzAsNSxbMSwxLCJYXFx0aW1lc19ZIFgiXSxbMCwxLCJBIl0sWzMsMSwiQSJdLFszLDAsIlkiXSxbMiwxLCJYIl0sWzAsNCwiYSIsMCx7Im9mZnNldCI6LTF9XSxbMSw0LCJjIiwwLHsiY3VydmUiOi00fV0sWzEsNCwiYyciLDIseyJjdXJ2ZSI6NH1dLFs0LDIsImciLDJdLFszLDIsImEiLDAseyJzdHlsZSI6eyJib2R5Ijp7Im5hbWUiOiJkYXNoZWQifX19XSxbNCwzLCJmIl0sWzAsNCwiYiIsMix7Im9mZnNldCI6MX1dLFsxLDAsImgiLDEseyJzdHlsZSI6eyJib2R5Ijp7Im5hbWUiOiJkYXNoZWQifX19XV0=
    \begin{tikzcd}
        &&& Y \\
        A & {X\times_Y X} & X & A
        \arrow["a", shift left, from=2-2, to=2-3]
        \arrow["c", curve={height=-24pt}, from=2-1, to=2-3]
        \arrow["{c'}"', curve={height=24pt}, from=2-1, to=2-3]
        \arrow["g"', from=2-3, to=2-4]
        \arrow["a", dashed, from=1-4, to=2-4]
        \arrow["f", from=2-3, to=1-4]
        \arrow["b"', shift right, from=2-2, to=2-3]
        \arrow["h"{description}, dashed, from=2-1, to=2-2]
    \end{tikzcd}
    \end{equation*}
    with the same $g:X\to A$ and the unique arrow $a:Y\to A$, but created from the definition of coequalizer. Since $h$ is unique, given $c$ and $c'$, we thus shown that $f$ is the coequalizer of the kernel pair.
\end{proof}
\end{dem}


\section{Adjunction}

\begin{definition}{\textbf{(Adjunction)}}
    Given the category $\textbf{C}$ and $\textbf{D}$, with the functor $F:\textbf{C}\rightarrow\textbf{D}$ and $G:\textbf{D}\rightarrow\textbf{C}$, then the adjunction between $F$ and $G$ is natural bijection (on both object):
    \begin{equation*}
        \operatorname{Hom}_\textbf{D}(FC, D) \xrightarrow{\cong}\operatorname{Hom}_\textbf{C}(C, GD)       
    \end{equation*}
    for each objects $C$ of $\textbf{C}$ and $D$ of $\textbf{D}$. Furthermore, we have the following additional notations: 
    \begin{itemize}
        \item We call $F$ left-adjoint and $G$ right-adjoint and the adjunction between them is denoted by $F\dashv G$.
        \item With 2 maps $f^\sharp:FC\rightarrow D$ and $f^\flat:C\rightarrow GD$, that are related by bijection are called transpose or adjunct to each other.
    \end{itemize}
\end{definition}

\begin{remark}{\textit{(Exploration on Adjunction)}}
    Given the categories $\textbf{C},\textbf{D}$ with functors $F:\textbf{C}\rightarrow\textbf{D}$ and $G:\textbf{D}\rightarrow\textbf{C}$, in which $F\dashv G$. Then we can fixed the object $C$, leading to the following functors:
    \begin{equation*}
        \operatorname{Hom}_\textbf{D}(FC, -), \operatorname{Hom}_\textbf{C}(C, G-):\textbf{D}\rightarrow\textbf{Set}
    \end{equation*}
    This means that the functor $\operatorname{Hom}_\textbf{C}(C,G-)$ can be represented by $FC$. We can interpret this as. Given functor $\operatorname{Hom}_\textbf{C}(C,G-)$, we can probe into object $D$ by the mapping between $C\to GD$, and using $C$ to judge its properties. But since they are natural isomorphism, then the judging procedure can be done using $FC$ directly.

\end{remark}

\begin{remark}{\textit{(Adjunction and Yoneda Lemma)}}
    By Yoneda embedding, the set of natural \textbf{isomorphism} given on the LHS is isomorphic to:
    \begin{equation*}
        \operatorname{Hom}_{[\textbf{D}, \textbf{Set}]}\Big( \operatorname{Hom}_\textbf{D}(FC, -), \operatorname{Hom}_\textbf{C}(C, G-) \Big)\cong\operatorname{Hom}_\textbf{C}(C, GFC)
    \end{equation*}
    where each of the natural isomorphism is governed by its action on identity morphism of $FC$. Furthermore, each of the natural isomorphism gives us the component of $\operatorname{Hom}_\textbf{D}(FC,FC)\xrightarrow{\cong}\operatorname{Hom}_\textbf{C}(C,GFC)$
\end{remark}

\subsection{Unit \& Co-Unit}

The following two remarks leads to the following notion of unit:

\begin{definition}{\textbf{(Unit of Adjunction)}}
    The image of $\operatorname{id}_{FC}$ on the function $\operatorname{Hom}_\textbf{D}(FC, FC)\xrightarrow{\cong}\operatorname{Hom}_\textbf{C}(C, GFC)$ defined above is called unit of the adjunction (at $C$) is denoted by $(\operatorname{id}_{FC})^\flat=\eta_C:C\rightarrow GFC$
\end{definition}

\begin{remark}{(Notes on Unit)}
    \label{remark:note-unit}
    One can also see the Unit as the representation of the adjunction as a whole, due to the fact that the action on identity of a natural transformation is enough to determine the whole natural transformation. Suppose we are given $f^\sharp:FC\to D$ we can get $f^\flat$ by the following LHS diagram.
    \begin{equation*}
    % https://q.uiver.app/#q=WzAsNyxbMywwLCJcXG9wZXJhdG9ybmFtZXtIb219X1xcdGV4dGJme0R9KEZDLCBGQykiXSxbMywxLCJcXG9wZXJhdG9ybmFtZXtIb219X1xcdGV4dGJme0N9KEMsIEdGQykiXSxbNiwxLCJcXG9wZXJhdG9ybmFtZXtIb219X1xcdGV4dGJme0N9KEMsIEdEKSJdLFs2LDAsIlxcb3BlcmF0b3JuYW1le0hvbX1fXFx0ZXh0YmZ7RH0oRkMsIEQpIl0sWzAsMCwiQyJdLFsxLDAsIkdGQyJdLFsxLDEsIkdEIl0sWzMsMiwiXFxjb25nIl0sWzAsMSwiXFxjb25nIiwyLHsiY29sb3VyIjpbMCw2MCw2MF19LFswLDYwLDYwLDFdXSxbMCwzLCJmXlxcc2hhcnBcXGNpcmMtIl0sWzEsMiwiXFxvcGVyYXRvcm5hbWV7SG9tfV9cXHRleHRiZntDfShDLEctKVtmXlxcc2hhcnBdPUdmXlxcc2hhcnBcXGNpcmMtIiwyLHsiY29sb3VyIjpbMCw2MCw2MF19LFswLDYwLDYwLDFdXSxbNSw2LCJHZl5cXHNoYXJwIl0sWzQsNSwiXFxldGFfQyJdLFs0LDYsImZeXFxmbGF0IiwyXV0=
    \begin{tikzcd}
        C & GFC && {\operatorname{Hom}_\textbf{D}(FC, FC)} &&& {\operatorname{Hom}_\textbf{D}(FC, D)} \\
        & GD && {\operatorname{Hom}_\textbf{C}(C, GFC)} &&& {\operatorname{Hom}_\textbf{C}(C, GD)}
        \arrow["\cong", from=1-7, to=2-7]
        \arrow["\cong"', color={rgb,255:red,214;green,92;blue,92}, from=1-4, to=2-4]
        \arrow["{f^\sharp\circ-}", from=1-4, to=1-7]
        \arrow["{\operatorname{Hom}_\textbf{C}(C,G-)[f^\sharp]=Gf^\sharp\circ-}"', color={rgb,255:red,214;green,92;blue,92}, from=2-4, to=2-7]
        \arrow["{Gf^\sharp}", from=1-2, to=2-2]
        \arrow["{\eta_C}", from=1-1, to=1-2]
        \arrow["{f^\flat}"', from=1-1, to=2-2]
    \end{tikzcd}
    \end{equation*}
    For the RHS diagram, we make use of the natural transformation and Yoneda's trick. The arrow moving down represents the ``transposition'', and the red represents the path of the LHS diagram when we start with $\operatorname{id}_{FC}$.
\end{remark}

Now we have only consider the unit at $C$, so let's try varying $C$. This is given in the following lemma:

\begin{lemma}
    \label{lemma:adj-commutative-diagram}
    {\color{purple} Given the pair of adjunction $f^\sharp:FC\rightarrow D$ and $g^\sharp:FC'\rightarrow D'$ with morphism $h:C\rightarrow C'$ and $k:D\rightarrow D'$, then by the naturality condition of adjunction shows that the following diagrams are commutes iff they are of each other:}
    \begin{equation*}
    % https://q.uiver.app/#q=WzAsOCxbMCwwLCJGQyJdLFsxLDAsIkQiXSxbMCwxLCJGQyciXSxbMSwxLCJEJyJdLFszLDAsIkMiXSxbNCwwLCJHRCJdLFs0LDEsIkdEJyJdLFszLDEsIkMnIl0sWzIsMywiZ15cXHNoYXJwIiwyXSxbMCwxLCJmXlxcc2hhcnAiXSxbMCwyLCJGaCIsMl0sWzEsMywiayJdLFs1LDYsIkdrIl0sWzQsNSwiZl5cXGZsYXQiXSxbNyw2LCJnXlxcZmxhdCIsMl0sWzQsNywiaCIsMl1d
    \begin{tikzcd}
        FC & D && C & GD \\
        {FC'} & {D'} && {C'} & {GD'}
        \arrow["{g^\sharp}"', from=2-1, to=2-2]
        \arrow["{f^\sharp}", from=1-1, to=1-2]
        \arrow["Fh"', from=1-1, to=2-1]
        \arrow["k", from=1-2, to=2-2]
        \arrow["Gk", from=1-5, to=2-5]
        \arrow["{f^\flat}", from=1-4, to=1-5]
        \arrow["{g^\flat}"', from=2-4, to=2-5]
        \arrow["h"', from=1-4, to=2-4]
    \end{tikzcd}
    \end{equation*}
\end{lemma}

\begin{dem}
\begin{proof}
    $\boldsymbol{(\implies)}:$ We start by consider the following commutative diagram that are created from the naturality condition of the adjunction:
    \begin{equation*}
    % https://q.uiver.app/#q=WzAsNixbMCwwLCJcXG9wZXJhdG9ybmFtZXtIb219X1xcdGV4dGJme0R9KEZDLCBEKSJdLFsyLDAsIlxcb3BlcmF0b3JuYW1le0hvbX1fXFx0ZXh0YmZ7RH0oRkMsIEQnKSJdLFsyLDEsIlxcb3BlcmF0b3JuYW1le0hvbX1fXFx0ZXh0YmZ7RH0oQywgR0QnKSJdLFswLDEsIlxcb3BlcmF0b3JuYW1le0hvbX1fXFx0ZXh0YmZ7RH0oQywgR0QpIl0sWzQsMCwiXFxvcGVyYXRvcm5hbWV7SG9tfV9cXHRleHRiZntEfShGQycsIEQnKSJdLFs0LDEsIlxcb3BlcmF0b3JuYW1le0hvbX1fXFx0ZXh0YmZ7RH0oQycsIEdEJykiXSxbMCwxLCJrXFxjaXJjLSJdLFsxLDIsIlxcY29uZyIsMV0sWzAsMywiXFxjb25nIiwyXSxbMywyLCJHa1xcY2lyYy0iLDJdLFs0LDEsIi1cXGNpcmMgRmgiLDJdLFs1LDIsIi1cXGNpcmMgaCJdLFs0LDUsIlxcY29uZyJdXQ==
    \begin{tikzcd}
        {\operatorname{Hom}_\textbf{D}(FC, D)} && {\operatorname{Hom}_\textbf{D}(FC, D')} && {\operatorname{Hom}_\textbf{D}(FC', D')} \\
        {\operatorname{Hom}_\textbf{D}(C, GD)} && {\operatorname{Hom}_\textbf{D}(C, GD')} && {\operatorname{Hom}_\textbf{D}(C', GD')}
        \arrow["{k\circ-}", from=1-1, to=1-3]
        \arrow["\cong"{description}, from=1-3, to=2-3]
        \arrow["\cong"', from=1-1, to=2-1]
        \arrow["{Gk\circ-}"', from=2-1, to=2-3]
        \arrow["{-\circ Fh}"', from=1-5, to=1-3]
        \arrow["{-\circ h}", from=2-5, to=2-3]
        \arrow["\cong", from=1-5, to=2-5]
    \end{tikzcd}
    \end{equation*}
    We see that $f^\sharp\in\operatorname{Hom}_\textbf{D}(FC, D)$ and $g^\sharp\in\operatorname{Hom}_\textbf{D}(FC', D')$, by the hypothesis, we have that $k\circ f^\sharp = g^\sharp\circ Fh$, and the fact that it is commuting we can take a different path, in which we have that $Gk\circ f^\flat=f^\flat\circ h$.

    The other direction is the same.
\end{proof}
\end{dem}

\begin{lemma}
    With the maps $\eta_C:C\rightarrow GFC$ can be put together to define the natural transformation $\eta:\operatorname{id}_\textbf{C}\Rightarrow G\circ F$ of endofunctors $\textbf{C}\rightarrow\textbf{C}$
\end{lemma}
\begin{dem}
\begin{proof}
    We note that, since the LHS is obviously commutes, using the lemma above.
    \begin{equation*}
    % https://q.uiver.app/#q=WzAsOCxbMCwwLCJGQyJdLFsxLDAsIkZDIl0sWzAsMSwiRkMnIl0sWzEsMSwiRkMnIl0sWzMsMCwiQyJdLFszLDEsIkMnIl0sWzQsMCwiR0ZDIl0sWzQsMSwiR0ZDJyJdLFswLDEsIlxcb3BlcmF0b3JuYW1le2lkfV97RkN9Il0sWzIsMywiXFxvcGVyYXRvcm5hbWV7aWR9X3tGQyd9IiwyXSxbMCwyLCJGZiIsMl0sWzEsMywiRmYiXSxbNSw3LCJcXGV0YV97RkMnfSIsMl0sWzQsNiwiXFxldGFfe0ZDfSJdLFs0LDUsImYiLDJdLFs2LDcsIkdGZiJdXQ==
    \begin{tikzcd}
        FC & FC && C & GFC \\
        {FC'} & {FC'} && {C'} & {GFC'}
        \arrow["{\operatorname{id}_{FC}}", from=1-1, to=1-2]
        \arrow["{\operatorname{id}_{FC'}}"', from=2-1, to=2-2]
        \arrow["Ff"', from=1-1, to=2-1]
        \arrow["Ff", from=1-2, to=2-2]
        \arrow["{\eta_{FC'}}"', from=2-4, to=2-5]
        \arrow["{\eta_{FC}}", from=1-4, to=1-5]
        \arrow["f"', from=1-4, to=2-4]
        \arrow["GFf", from=1-5, to=2-5]
    \end{tikzcd}
    \end{equation*}
\end{proof}
\end{dem}

\begin{remark}{(Co-Units Formulation)}
    Let's variate $D$ in the natural bijection of $\operatorname{Hom}_\textbf{D}(FC,D) \xrightarrow{\cong}\operatorname{Hom}_\textbf{C}(C,GD)$ consider the natural isomorphism between presheaves $\operatorname{Hom}_\textbf{D}(F-,D), \operatorname{Hom}_\textbf{C}(C,G-):\textbf{C}^\text{op}\rightarrow\textbf{Set}$. Repeating the analysis above, we have the universal morphism $\varepsilon_D:FGD\rightarrow D$, and natural transformation $\varepsilon:F\circ G\Rightarrow\operatorname{id}_\textbf{D}$
\end{remark}

\begin{definition}{\textbf{(Counit)}}
    The adjunction $F\dashv G$ induced the natural isomorphism $\varepsilon:F\circ G\Rightarrow\operatorname{id}_\textbf{D}$  of endofunctors $\textbf{D}\rightarrow\textbf{D}$ is called counit of the adjunction.
\end{definition}

\begin{remark}
    \label{remark:note-counit}
    Similar to the case of unit, one can see that the following diagram commutes:
    \begin{equation*}
    % https://q.uiver.app/#q=WzAsMixbMCwxLCJHRCJdLFswLDAsIkMiXSxbMSwwLCJnXlxcZmxhdCIsMix7InN0eWxlIjp7ImJvZHkiOnsibmFtZSI6ImRhc2hlZCJ9fX1dXQ==
    \begin{tikzcd}
        C \\
        GD
        \arrow["{g^\flat}"', dashed, from=1-1, to=2-1]
    \end{tikzcd}
    \longmapsto
    % https://q.uiver.app/#q=WzAsMyxbMCwwLCJGQyJdLFswLDEsIkZHRCJdLFsxLDEsIkQiXSxbMCwyLCJnXlxcc2hhcnAiXSxbMCwxLCJGZ15cXGZsYXQiLDJdLFsxLDIsIlxcdmFyZXBzaWxvbl9EIiwyXV0=
    \begin{tikzcd}
        FC \\
        FGD & D
        \arrow["{Fg^\flat}"', from=1-1, to=2-1]
        \arrow["{g^\sharp}", from=1-1, to=2-2]
        \arrow["{\varepsilon_D}"', from=2-1, to=2-2]
    \end{tikzcd}
    \end{equation*}
\end{remark}

\subsection{Alternative Definition of Adjunction}

We have the following question: Given the pair of functors $F:\textbf{C}\rightarrow\textbf{D}$ and $G:\textbf{D}\rightarrow\textbf{C}$, the pair of natural transformations $\eta:\operatorname{id}_\textbf{C}\Rightarrow G\circ F$ and $\varepsilon:G\circ F\Rightarrow\operatorname{id}_\textbf{D}$ induces an adjunction $F\dashv G$ ? The answer is almost yes, but there is a problem. However, the Yoneda embedding doesn't guranteed the natural isomorphism, so we need additional condition.

\begin{lemma}{\textbf{(Triangle Identities):}}
    \label{lemma:triangle-id-adjoint}
    Given the unit $\eta:\operatorname{id}_\textbf{C}\Rightarrow G\circ F$ and co-unit $\varepsilon:F\circ G\Rightarrow\operatorname{id}_\textbf{D}$ for adjunction $F\dashv G$. Then the following diagram commutes:
    \begin{equation*}
    % https://q.uiver.app/#q=WzAsNixbMCwwLCJGIl0sWzEsMCwiRkdGIl0sWzEsMSwiRiJdLFszLDAsIkciXSxbNCwwLCJHRkciXSxbNCwxLCJHIl0sWzAsMSwiRlxcZXRhIiwwLHsibGV2ZWwiOjJ9XSxbMSwyLCJcXHZhcmVwc2lsb24gRiIsMCx7ImxldmVsIjoyfV0sWzAsMiwiXFxvcGVyYXRvcm5hbWV7aWR9X0YiLDIseyJsZXZlbCI6Mn1dLFszLDUsIlxcb3BlcmF0b3JuYW1le2lkfV9HIiwyLHsibGV2ZWwiOjJ9XSxbMyw0LCJcXGV0YSBHIiwwLHsibGV2ZWwiOjJ9XSxbNCw1LCJHXFx2YXJlcHNpbG9uIiwwLHsibGV2ZWwiOjJ9XV0=
    \begin{tikzcd}
        F & FGF && G & GFG \\
        & F &&& G
        \arrow["F\eta", Rightarrow, from=1-1, to=1-2]
        \arrow["{\operatorname{id}_F}"', Rightarrow, from=1-1, to=2-2]
        \arrow["{\varepsilon F}", Rightarrow, from=1-2, to=2-2]
        \arrow["{\eta G}", Rightarrow, from=1-4, to=1-5]
        \arrow["{\operatorname{id}_G}"', Rightarrow, from=1-4, to=2-5]
        \arrow["G\varepsilon", Rightarrow, from=1-5, to=2-5]
    \end{tikzcd}
    \end{equation*}
\end{lemma}
\begin{proof}
    For the LHS diagram, given an object $C$, we have to show that $\varepsilon_{FC}\circ F\eta_C=\operatorname{id}_{FC}$. We starts by note that, from the remark \ref{remark:note-counit}, when using $\eta_C:FC\to GFC$ and turn it to its transpose, i.e $(\eta_C)^\sharp=\operatorname{id}_FC$, that is:
    \begin{equation*}
    % https://q.uiver.app/#q=WzAsMixbMCwxLCJHRkMiXSxbMCwwLCJDIl0sWzEsMCwiXFxldGFfQyIsMix7InN0eWxlIjp7ImJvZHkiOnsibmFtZSI6ImRhc2hlZCJ9fX1dXQ==
    \begin{tikzcd}
        C \\
        GFC
        \arrow["{\eta_C}"', dashed, from=1-1, to=2-1]
    \end{tikzcd}
    \longmapsto
    % https://q.uiver.app/#q=WzAsMyxbMCwwLCJGQyJdLFswLDEsIkZHRkMiXSxbMSwxLCJGQyJdLFswLDIsIlxcb3BlcmF0b3JuYW1le2lkfV97RkN9Il0sWzAsMSwiRlxcZXRhX0MiLDJdLFsxLDIsIlxcdmFyZXBzaWxvbl97RkN9IiwyXV0=
    \begin{tikzcd}
        FC \\
        FGFC & FC
        \arrow["{F\eta_C}"', from=1-1, to=2-1]
        \arrow["{\operatorname{id}_{FC}}", from=1-1, to=2-2]
        \arrow["{\varepsilon_{FC}}"', from=2-1, to=2-2]
    \end{tikzcd}
    \qquad \qquad 
    % https://q.uiver.app/#q=WzAsMixbMCwxLCJEIl0sWzAsMCwiRkdEIl0sWzEsMCwiXFx2YXJlcHNpbG9uX0QiLDAseyJzdHlsZSI6eyJib2R5Ijp7Im5hbWUiOiJkYXNoZWQifX19XV0=
    \begin{tikzcd}
        FGD \\
        D
        \arrow["{\varepsilon_D}", dashed, from=1-1, to=2-1]
    \end{tikzcd}
    \longmapsto
    % https://q.uiver.app/#q=WzAsMyxbMCwxLCJHRCJdLFswLDAsIkdGR0QiXSxbMSwwLCJHRCJdLFsyLDAsIlxcb3BlcmF0b3JuYW1le2lkfV97R0R9Il0sWzEsMCwiR1xcdmFyZXBzaWxvbl9EIiwyXSxbMiwxLCJcXGV0YV97R0R9IiwyXV0=
    \begin{tikzcd}
        GFGD & GD \\
        GD
        \arrow["{G\varepsilon_D}"', from=1-1, to=2-1]
        \arrow["{\eta_{GD}}"', from=1-2, to=1-1]
        \arrow["{\operatorname{id}_{GD}}", from=1-2, to=2-1]
    \end{tikzcd}
    \end{equation*}
    and we have used the horizontal composition of natural transformation. The second diagram is done in similar manners, where we consider the unit diagram instead as shown in the RHS of the diagram above.
\end{proof}

This is the condition that the natural transformations have to satisfy to get the adjunction i.e:

\begin{lemma}
    Given functors $F:\textbf{C}\rightarrow\textbf{D}$ and $G:\textbf{D}\rightarrow\textbf{C}$, the pair of natural transformation $\eta:\operatorname{id}_\textbf{C}\Rightarrow G\circ F$ and $\varepsilon:F\circ G\Rightarrow\operatorname{id}_\textbf{D}$ that satisfying the triangle identity. then the assignment:
    \begin{equation*}
    \begin{aligned}
        \operatorname{Hom}_\textbf{D}(FC,D)\xrightarrow{\quad\flat\quad}&\operatorname{Hom}_\textbf{C}(C,GD) \\
        f^\sharp\xmapsto{\qquad\quad}&f^\flat = Gf^\sharp\circ\eta_C
    \end{aligned}
    \qquad \qquad 
    \begin{aligned}
    \operatorname{Hom}_\textbf{C}(C,GD)\xrightarrow{\quad\sharp\quad}&\operatorname{Hom}_\textbf{D}(FC,D) \\
    g^\flat\xmapsto{\qquad\quad}&g^\sharp = \varepsilon_D\circ Fg^\flat
    \end{aligned}
    \end{equation*}
    are mutually inverse i.e there is bijection between $\operatorname{Hom}_\textbf{D}(FC,D)\rightarrow\operatorname{Hom}_\textbf{C}(C,GD)$, thus an adjunction $F\dashv G$. Note that the naturality on the argument is guranteed by the natural transformations used to defined such a mapping.
\end{lemma}

\begin{proof}
    We have that, for LHS given $f:FC\to D$ we have: $f^\flat=Gf\circ\eta_C$, and given $g:C\to GD$ for the RHS with $g^\sharp=\varepsilon_D\circ Fg$
    \begin{equation*}
    \begin{aligned}
        (f^\flat)^\sharp&=\varepsilon_D\circ Ff^\flat = \varepsilon_D\circ FGf\circ F\eta_C \\
        &= f\circ\varepsilon_{FC}\circ F\eta_C = f\circ \operatorname{id}_{FC} = f
    \end{aligned}
    \qquad\quad
    \begin{aligned}
        (g^\sharp)^\flat&= Gg^\flat\circ\eta_C = G\varepsilon_D\circ GFg\circ \eta_C \\
        &= G\varepsilon_D\circ\eta_{GD}\circ g = \operatorname{id}_{GD} \circ g = g
    \end{aligned}
    \end{equation*}
    Note that the second equality follows from the naturality condition of $\varepsilon$ on the function $f:FC\to D$ as we have $\varepsilon_D\circ FGf=\varepsilon_{FC}\circ F\eta_C$. This is similar for the case of $(g^\sharp)^\flat$, as show on the RHS, where we use naturality condition of $\eta$ for which $GFg\circ\eta_C=\eta_{GD}\circ g$.
\end{proof}

\subsection{Adjunctions, Limits and Colimits}

We are going to consider the connection betweet left/right adjunction and limits. We are going to show that right adjoint functor are continous. Let's consider a simplier example.

\begin{proposition}
    Given object $A$ and $B$ in $\textbf{D}$ with product $A\times B$, and right adjoint $R:\textbf{D}\rightarrow\textbf{C}$. We can show that: $R(A\times B) \cong RA\times RB$.
\end{proposition}
\begin{proof}
    Consider the map $f_1:C\to RA$ and $f_2:C\to RB$, by the adjunction we also have a unique map of $f_1^\sharp:LC\to A$ and $f_2^\sharp:LC\to B$, then by univeral properties there is a unique map $f_1^\sharp\times f_2^\sharp=f^\sharp:LC\to A\times B$. And by adjunction, we have a unique map $f^\flat:C\to R(A\times B)$. Thus, $R(A\times B)$ is also has the universal property i.e a limit. Therefore, $R(A\times B)\cong RA\times RB$ as limtis are all isomorphic to each other.
\end{proof}

In a more general manners, we can consider the action of adjunction on cones. 

\begin{lemma}
Given the adjuntion $L\dashv R$, in which $L:\textbf{C}\rightarrow\textbf{D}$ and $R:\textbf{D}\rightarrow\textbf{C}$ with diagram $E:\textbf{J}\rightarrow\textbf{D}$. Then adjunction induces a bijection:
\begin{equation*}
    \operatorname{Cone}(LC, E) \cong \operatorname{Cone}(C, RE)
\end{equation*}
for all object $C$ and natural in $C$.
\end{lemma}

\begin{proof}
    Note that $\alpha\in\operatorname{Cone}(LC, E)$ is a natural transformation between $LC\Rightarrow E$, where $LC:\boldsymbol{J}\to\boldsymbol{D}$ is a constant diagram where for all object $J$ of $\boldsymbol{J}$, we have $LC(J)=LC$. Then, for each object $J$, there is a component of $\alpha_J:LC\to EJ$ by the adjunction, we have a unique (per isomorphism of adjunction) $(\alpha_J)^\flat:C\to (RE)J$, which we take as the component of $\alpha$ to component of $\alpha^\flat\in \operatorname{Cone}(C, RE)$. Note that the naturality of $C$ follows from properties of adjunction.

    We are left to show that $\alpha^\flat$ is a natural transformation, that is given a morphism $m:J_1\to J_2$ where $J_1,J_2$ are objects of $J$. Then, we we have that (following from the lemma \ref{lemma:adj-commutative-diagram}, where $f^\sharp$, $g^\sharp$, $Fh$ and $k$ in its case are $\alpha_{J_1}$, $\alpha_{J_2}$, $\operatorname{id}_{LC}$ and $Dm$, respectively):

    \begin{equation*}
    % https://q.uiver.app/#q=WzAsMyxbMSwwLCJMQyJdLFswLDEsIkVKXzEiXSxbMiwxLCJFSl8yIl0sWzAsMSwiXFxhbHBoYV97Sl8xfSIsMl0sWzAsMiwiXFxhbHBoYV97Sl8yfSJdLFsxLDIsIkVtIiwyXV0=
    \begin{tikzcd}
        & LC \\
        {EJ_1} && {EJ_2}
        \arrow["{\alpha_{J_1}}"', from=1-2, to=2-1]
        \arrow["{\alpha_{J_2}}", from=1-2, to=2-3]
        \arrow["Em"', from=2-1, to=2-3]
    \end{tikzcd}
    \qquad\rightsquigarrow\qquad
    % https://q.uiver.app/#q=WzAsMyxbMSwwLCJDIl0sWzAsMSwiUkVKXzEiXSxbMiwxLCJSRUpfMiJdLFswLDEsIlxcYWxwaGFfe0pfMX1eXFxmbGF0IiwyXSxbMCwyLCJcXGFscGhhX3tKXzJ9XlxcZmxhdCJdLFsxLDIsIlJFbSIsMl1d
    \begin{tikzcd}
        & C \\
        {REJ_1} && {REJ_2}
        \arrow["{\alpha_{J_1}^\flat}"', from=1-2, to=2-1]
        \arrow["{\alpha_{J_2}^\flat}", from=1-2, to=2-3]
        \arrow["REm"', from=2-1, to=2-3]
    \end{tikzcd}
    \end{equation*}
    
    this hows that $\alpha^\flat$ is natural transformation as needed. 
\end{proof}

\begin{theorem}
    Right-adjoint functors are continous.
\end{theorem}
\begin{proof}
    We want to show that (note that it isn't just isomorphism but also naturally isomorphic):
    \begin{equation*}
    \begin{aligned}
        \operatorname{Hom}_\textbf{C}(-, \lim RF) &\cong \operatorname{Cone}(-, RF) \cong \operatorname{Cone}(L-,F) \\
        &\cong \operatorname{Hom}_\textbf{D}(L-,\lim F) \cong \operatorname{Hom}_\textbf{D}(-,R\lim F)
    \end{aligned}
    \end{equation*}
    And by Yoneda embedding, $\lim RF\cong R\lim F$ as needed.
\end{proof}


\section{Monad and Comonad}

\subsection{Monad}

\begin{definition}{\textbf{(Monad)}}
    Given a category $\textbf{C}$, the monad on $\textbf{C}$ is $(T, \eta, \mu)$ where $T:\textbf{C}\to\textbf{C}$, an unit $\eta:\operatorname{id}_\textbf{C}\Rightarrow T$ and a composition or multiplication $\mu:TT\Rightarrow T$, making the following diagram commutes (they are called left and right unitality and associativity, respectively):
    \begin{equation*}
    % https://q.uiver.app/#q=WzAsMTAsWzAsMCwiVCJdLFsxLDAsIlRUIl0sWzEsMSwiVCJdLFszLDAsIlQiXSxbNCwwLCJUVCJdLFs0LDEsIlQiXSxbNiwwLCJUVFQiXSxbNywwLCJUVCJdLFs2LDEsIlRUIl0sWzcsMSwiVCJdLFswLDEsIlxcZXRhIFQiLDAseyJsZXZlbCI6Mn1dLFsxLDIsIlxcbXUiLDAseyJsZXZlbCI6Mn1dLFswLDIsIlxcb3BlcmF0b3JuYW1le2lkfSIsMix7ImxldmVsIjoyfV0sWzMsNCwiVFxcZXRhIiwwLHsibGV2ZWwiOjJ9XSxbNCw1LCJcXG11IiwwLHsibGV2ZWwiOjJ9XSxbMyw1LCJcXG9wZXJhdG9ybmFtZXtpZH0iLDIseyJsZXZlbCI6Mn1dLFs2LDcsIlRcXG11IiwwLHsibGV2ZWwiOjJ9XSxbNiw4LCJcXG11IFQiLDIseyJsZXZlbCI6Mn1dLFs4LDksIlxcbXUiLDIseyJsZXZlbCI6Mn1dLFs3LDksIlxcbXUiLDAseyJsZXZlbCI6Mn1dXQ==
    \begin{tikzcd}
        T & TT && T & TT && TTT & TT \\
        & T &&& T && TT & T
        \arrow["{\eta T}", Rightarrow, from=1-1, to=1-2]
        \arrow["{\operatorname{id}}"', Rightarrow, from=1-1, to=2-2]
        \arrow["\mu", Rightarrow, from=1-2, to=2-2]
        \arrow["T\eta", Rightarrow, from=1-4, to=1-5]
        \arrow["{\operatorname{id}}"', Rightarrow, from=1-4, to=2-5]
        \arrow["\mu", Rightarrow, from=1-5, to=2-5]
        \arrow["T\mu", Rightarrow, from=1-7, to=1-8]
        \arrow["{\mu T}"', Rightarrow, from=1-7, to=2-7]
        \arrow["\mu", Rightarrow, from=1-8, to=2-8]
        \arrow["\mu"', Rightarrow, from=2-7, to=2-8]
    \end{tikzcd}
    \end{equation*}
\end{definition}

\begin{definition}{\textbf{(Kleisli Category)}}
    Given a monad $(T, \eta, \mu)$ on category $\textbf{C}$, a Kleisli category $\textbf{C}_T$ is defined to have the following components:
    \begin{itemize}
        \item \textbf{(Object):} The objects are objects of $\textbf{C}$.
        \item \textbf{(Morphism):} The morphism between 2 objects $X$ to $Y$ is $k:X\to TY$, where the composition between this morphism and $h:Y\to TZ$ is given to be:
        \begin{equation*}
        % https://q.uiver.app/#q=WzAsNCxbMCwwLCJYIl0sWzEsMCwiVFkiXSxbMiwwLCJUVFoiXSxbMywwLCJUWiJdLFswLDEsImsiXSxbMSwyLCJUaCJdLFsyLDMsIlxcbXVfWiJdXQ==
        \begin{tikzcd}
            X & TY & TTZ & TZ
            \arrow["k", from=1-1, to=1-2]
            \arrow["Th", from=1-2, to=1-3]
            \arrow["{\mu_Z}", from=1-3, to=1-4]
        \end{tikzcd}
        \end{equation*}
        With $\eta_X:X\to TX$ being the identity morphism.
    \end{itemize}
\end{definition}

\begin{proposition}
    Kleisli Category is indeed a category.
\end{proposition}
\begin{proof}
    There are 2 things we have to proof here: the composition of the identity morphism and the associativity of the composition. That are:

    \textbf{(Identity Composition):} Consider $k:X\to TY$ and $\eta_X:X\to TX$, then $k\circ_{kl} \eta_X$ is given in the blue path of the LHS diagram, while the square represents the naturality of $\eta$.
    \begin{equation*}
    % https://q.uiver.app/#q=WzAsNSxbMCwwLCJYIl0sWzAsMSwiVFgiXSxbMSwxLCJUVFkiXSxbMiwxLCJUWSJdLFsxLDAsIlRZIl0sWzEsMiwiVGsiLDIseyJjb2xvdXIiOlsyNDAsNjAsNjBdfSxbMjQwLDYwLDYwLDFdXSxbMiwzLCJcXG11X1kiLDIseyJjb2xvdXIiOlsyNDAsNjAsNjBdfSxbMjQwLDYwLDYwLDFdXSxbMCwxLCJcXGV0YV9YIiwyLHsiY29sb3VyIjpbMjQwLDYwLDYwXX0sWzI0MCw2MCw2MCwxXV0sWzAsNCwiayJdLFs0LDIsIlxcZXRhX3tUWX0iXV0=
    \begin{tikzcd}
        X & TY \\
        TX & TTY & TY
        \arrow["k", from=1-1, to=1-2]
        \arrow["{\eta_X}"', color={rgb,255:red,92;green,92;blue,214}, from=1-1, to=2-1]
        \arrow["{\eta_{TY}}", from=1-2, to=2-2]
        \arrow["Tk"', color={rgb,255:red,92;green,92;blue,214}, from=2-1, to=2-2]
        \arrow["{\mu_Y}"', color={rgb,255:red,92;green,92;blue,214}, from=2-2, to=2-3]
    \end{tikzcd}
    \qquad\quad
    % https://q.uiver.app/#q=WzAsNCxbMCwwLCJYIl0sWzIsMCwiVFRZIl0sWzMsMCwiVFkiXSxbMSwwLCJUWSJdLFsxLDIsIlxcbXVfWSJdLFswLDMsImsiXSxbMywxLCJUXFxldGFfWSJdXQ==
    \begin{tikzcd}
        X & TY & TTY & TY
        \arrow["k", from=1-1, to=1-2]
        \arrow["{T\eta_Y}", from=1-2, to=1-3]
        \arrow["{\mu_Y}", from=1-3, to=1-4]
    \end{tikzcd}
    \end{equation*}
    The left unitality gives $\mu\circ \eta_{TY}=\operatorname{id}_Y$ and so the result is $k$, as needed. On the other hand, given $\eta_Y:Y\to TY$, then we have $\eta_Y\circ_{kl}k$ in the RHS diagram. With only right unitality, we get the final result as $k$ as needed.
    
    \textbf{(Associativity):} Given $k:X\to TY,h:Y\to TZ, l:Z\to TW$, then we want to show that 
    
    \begin{equation*}
        {\color{rgb,255:red,92;green,92;blue,214} l\circ_{kl} (h\circ_{kl} k)} = {\color{rgb,255:red,214;green,92;blue,92} (l\circ_{kl} h)\circ_{kl} k} 
    \end{equation*}

    We have the commutative diagram, where the LHS is the composition represented by the colors, while the RHS shows the commutativity of 2 paths, where the first square is the naturality condition of $\mu$ and the second square is the associativity condition of monad.
    \begin{equation*}
    % https://q.uiver.app/#q=WzAsOSxbMCwwLCJYIl0sWzEsMCwiVFkiXSxbMiwwLCJUVFoiXSxbMywwLCJUWiJdLFs0LDAsIlRUVyJdLFs1LDAsIlRXIl0sWzIsMSwiVFRaIl0sWzMsMSwiVFRUVyJdLFs0LDEsIlRUVyJdLFswLDEsImsiXSxbMSwyLCJUaCJdLFsyLDMsIlxcbXVfWiIsMCx7ImNvbG91ciI6WzI0MCw2MCw2MF19LFsyNDAsNjAsNjAsMV1dLFszLDQsIlRsIiwwLHsiY29sb3VyIjpbMjQwLDYwLDYwXX0sWzI0MCw2MCw2MCwxXV0sWzQsNSwiXFxtdV9XIiwwLHsiY29sb3VyIjpbMjQwLDYwLDYwXX0sWzI0MCw2MCw2MCwxXV0sWzYsNywiVFRsIiwyLHsiY29sb3VyIjpbMCw2MCw2MF19LFswLDYwLDYwLDFdXSxbNyw4LCJUXFxtdV9XIiwyLHsiY29sb3VyIjpbMCw2MCw2MF19LFswLDYwLDYwLDFdXSxbMSw2LCJUaCIsMl0sWzgsNSwiXFxtdV9XIiwyLHsiY29sb3VyIjpbMCw2MCw2MF19LFswLDYwLDYwLDFdXV0=
    \begin{tikzcd}
        X & TY & TTZ & TZ & TTW & TW \\
        && TTZ & TTTW & TTW
        \arrow["k", from=1-1, to=1-2]
        \arrow["Th", from=1-2, to=1-3]
        \arrow["Th"', from=1-2, to=2-3]
        \arrow["{\mu_Z}", color={rgb,255:red,92;green,92;blue,214}, from=1-3, to=1-4]
        \arrow["Tl", color={rgb,255:red,92;green,92;blue,214}, from=1-4, to=1-5]
        \arrow["{\mu_W}", color={rgb,255:red,92;green,92;blue,214}, from=1-5, to=1-6]
        \arrow["TTl"', color={rgb,255:red,214;green,92;blue,92}, from=2-3, to=2-4]
        \arrow["{T\mu_W}"', color={rgb,255:red,214;green,92;blue,92}, from=2-4, to=2-5]
        \arrow["{\mu_W}"', color={rgb,255:red,214;green,92;blue,92}, from=2-5, to=1-6]
    \end{tikzcd}
    \quad
    % https://q.uiver.app/#q=WzAsNixbMCwwLCJUVFoiXSxbMCwxLCJUWiJdLFsxLDAsIlRUVFciXSxbMSwxLCJUVFciXSxbMiwwLCJUVFciXSxbMiwxLCJUVyJdLFswLDIsIlRUbCIsMCx7ImNvbG91ciI6WzAsNjAsNjBdfSxbMCw2MCw2MCwxXV0sWzAsMSwiXFxtdV96IiwyLHsiY29sb3VyIjpbMjQwLDYwLDYwXX0sWzI0MCw2MCw2MCwxXV0sWzEsMywiVGwiLDIseyJjb2xvdXIiOlsyNDAsNjAsNjBdfSxbMjQwLDYwLDYwLDFdXSxbMiwzLCJcXG11X3tUV30iLDJdLFsyLDQsIlRcXG11X1ciLDAseyJjb2xvdXIiOlswLDYwLDYwXX0sWzAsNjAsNjAsMV1dLFszLDUsIlxcbXVfVyIsMix7ImNvbG91ciI6WzI0MCw2MCw2MF19LFsyNDAsNjAsNjAsMV1dLFs0LDUsIlxcbXVfVyIsMCx7ImNvbG91ciI6WzAsNjAsNjBdfSxbMCw2MCw2MCwxXV1d
    \begin{tikzcd}
        TTZ & TTTW & TTW \\
        TZ & TTW & TW
        \arrow["TTl", color={rgb,255:red,214;green,92;blue,92}, from=1-1, to=1-2]
        \arrow["{\mu_z}"', color={rgb,255:red,92;green,92;blue,214}, from=1-1, to=2-1]
        \arrow["{T\mu_W}", color={rgb,255:red,214;green,92;blue,92}, from=1-2, to=1-3]
        \arrow["{\mu_{TW}}"', from=1-2, to=2-2]
        \arrow["{\mu_W}", color={rgb,255:red,214;green,92;blue,92}, from=1-3, to=2-3]
        \arrow["Tl"', color={rgb,255:red,92;green,92;blue,214}, from=2-1, to=2-2]
        \arrow["{\mu_W}"', color={rgb,255:red,92;green,92;blue,214}, from=2-2, to=2-3]
    \end{tikzcd}
    \end{equation*}

\end{proof}

\begin{example}{\textbf{(Monoid as Monad)}}
    In the example below, we will consider monoid $(M, 0, +)$ to represet side effects. We can turn it into monad associated with $M$ to be $(T_M, \eta,\mu)$ on category $\textbf{Set}$:
    \begin{itemize}
        \item Functor: is defined as $T_MX=X\times M$ and given a function $f:X\rightarrow Y$, we lift it to $T_Mf=f\times\operatorname{id}_M:X\times M\rightarrow Y\times M$ i.e $(x, m)\mapsto(f(x), m)$
        \item Unit: is defined to be, the map: $\eta_X:X\to X\times M$, where $x\mapsto(x, 0)$
        \item Multiplication: is defined to be, the map $\mu_X:X\times M\times M\to X\times M$, where $(x, m, n)\mapsto(x, m+ n)$
    \end{itemize}

    Let's check that it is acually a monad:
    \begin{itemize}
        \item Left/Right Unitality: we have the following diagram for left and right law respectively:
        \begin{equation*}
        % https://q.uiver.app/#q=WzAsNixbMCwwLCJYXFx0aW1lcyBNIl0sWzIsMCwiWFxcdGltZXMgTVxcdGltZXMgTSJdLFsyLDEsIlhcXHRpbWVzIE0iXSxbNCwwLCJYXFx0aW1lcyBNIl0sWzYsMCwiWFxcdGltZXMgTVxcdGltZXMgTSJdLFs2LDEsIlhcXHRpbWVzIE0iXSxbMSwyLCJcXG11X1giXSxbMCwxLCJcXGV0YV97WFxcdGltZXMgTX0iXSxbMCwyXSxbMyw0LCJcXGV0YV9YXFx0aW1lc1xcb3BlcmF0b3JuYW1le2lkfV9NIl0sWzQsNSwiXFxtdV9YIl0sWzMsNV1d
        \begin{tikzcd}
            {X\times M} && {X\times M\times M} && {X\times M} && {X\times M\times M} \\
            && {X\times M} &&&& {X\times M}
            \arrow["{\eta_{X\times M}}", from=1-1, to=1-3]
            \arrow[from=1-1, to=2-3]
            \arrow["{\mu_X}", from=1-3, to=2-3]
            \arrow["{\eta_X\times\operatorname{id}_M}", from=1-5, to=1-7]
            \arrow[from=1-5, to=2-7]
            \arrow["{\mu_X}", from=1-7, to=2-7]
        \end{tikzcd}
        \end{equation*}
        We can see that for the LHS diagram, we have $(x, m)\mapsto(x, m, 0)\mapsto(x,m +0)=(x,m)$. On the other hand, for the RHS diagram, we can have $(x, m)\mapsto(x, 0, m)\mapsto(x,0 +m)=(x,m)$, as needed.
        \item Multiplication: We consider the diagram below:
        \begin{equation*}
        % https://q.uiver.app/#q=WzAsNCxbMCwwLCJYXFx0aW1lcyBNXFx0aW1lcyBNXFx0aW1lcyBNIl0sWzIsMCwiWFxcdGltZXMgTVxcdGltZXMgTSJdLFswLDEsIlhcXHRpbWVzIE1cXHRpbWVzIE0iXSxbMiwxLCJYXFx0aW1lcyBNIl0sWzIsMywiXFxtdV9YIiwyXSxbMSwzLCJcXG11X1giXSxbMCwxLCJcXG11X1hcXHRpbWVzXFxvcGVyYXRvcm5hbWV7aWR9X00iXSxbMCwyLCJcXG11X3tYXFx0aW1lcyBNfSIsMl1d
        \begin{tikzcd}
            {X\times M\times M\times M} && {X\times M\times M} \\
            {X\times M\times M} && {X\times M}
            \arrow["{\mu_X\times\operatorname{id}_M}", from=1-1, to=1-3]
            \arrow["{\mu_{X\times M}}"', from=1-1, to=2-1]
            \arrow["{\mu_X}", from=1-3, to=2-3]
            \arrow["{\mu_X}"', from=2-1, to=2-3]
        \end{tikzcd}
        \end{equation*}
        On the blue path, we have $(x, a, b, c)\mapsto(x, a+b, c)\mapsto(x, a+b+c)$ on the red path, we have $(x, a, b, c)\mapsto(x, a, b+c)\mapsto(x, a+b+C)$, and the associativity follows from associativity of monoid.
    \end{itemize}
\end{example}

\begin{example}{\textbf{(Modeling Side-Effects with Monad)}}
    In this example, we will consider the how monad is used in CS, where it can be used to model the side effect of the functions. With monad $(T_M,\eta,\mu)$, the Kleisli morphism $k:X\to T_MY=Y\times M$ computes the input and output $X\to Y$ and also emits the side effect $M$. Then with Kleisli composition with $h:Y\to Z\times M$, we can see how the side effects are ``accumulated'':

    \begin{equation*}
    \begin{matrix}
    X & \xrightarrow{\quad k\quad} & Y\times M & \xrightarrow{h\times\operatorname{id}_M} & Z\times M\times M & \xrightarrow{\quad\mu\quad} &Z\times M \\
    x & \xmapsto{\quad k\quad} & (y, m)& \xmapsto{h\times\operatorname{id}_M} & (z,n,m) & \xmapsto{\quad\mu\quad} &(z, n+m) \\
    \end{matrix}
    \end{equation*}

    where we note that $T_Mh=h\times\operatorname{id}_M$
\end{example}

\begin{definition}{\textbf{(Left and Right Kleisli Adjunction)}}
    Given a monad $(T, \eta, \mu)$ on category $\textbf{C}$, we have the following construction:
    \begin{itemize}
        \item Left Adjointion $L_T:\textbf{C}\to\textbf{C}_T$: (Object) we have $L_TX=X$. (Morphism) Given $f:X\rightarrow Y$, the functor creates Kleisli morphism $L_Tf:X\to TY$ by $L_Tf=\eta_Y\circ f$ i.e 
        \begin{equation*}
            X\xrightarrow{f}Y\xrightarrow{\eta_Y}TY
        \end{equation*}
        \item Right Adjointion $R_T:\textbf{C}_T\to\textbf{C}$: (Object) we have $R_TX=TX$. (Morphism) Given $f:X\to TY$, the functor gives creates $R_Tf:TX\to TY$ by $R_Tf=\mu_Y\circ Tf$ i.e:
        \begin{equation*}
            TX\xrightarrow{Tf}TTY\xrightarrow{\mu_Y}TY
        \end{equation*}
    \end{itemize}
\end{definition}

\begin{proposition}
    Both $L_T$ and $R_T$ are functors
\end{proposition}

\begin{proof}
    \textbf{(Part 1):} To show that $L_T$ is a functor, we first consider the identity $\operatorname{id}_X:X\to X$, then $L_T\operatorname{id}_X=\eta_X$, which is an identity arrow in $\textbf{C}_T$.

    On the other hand, given $f:X\to Y$ and $g:Y\to Z$, we can see that: $L_Tg\circ_{kl} L_Tf = \big(\eta_Z\circ g\big)\circ_{kl}\big( \eta_Y\circ f \big)$ is given in the colored path of the diagram below
    \begin{equation*}
    % https://q.uiver.app/#q=WzAsNyxbMCwwLCJYIl0sWzEsMCwiWSJdLFsxLDEsIlRZIl0sWzIsMSwiVFoiXSxbMywxLCJUVFoiXSxbNCwxLCJUWiJdLFsyLDAsIloiXSxbMCwxLCJmIiwwLHsiY29sb3VyIjpbMjQwLDYwLDYwXX0sWzI0MCw2MCw2MCwxXV0sWzEsMiwiXFxldGFfWSIsMix7ImNvbG91ciI6WzI0MCw2MCw2MF19LFsyNDAsNjAsNjAsMV1dLFsyLDMsIlRnIiwyLHsiY29sb3VyIjpbMjQwLDYwLDYwXX0sWzI0MCw2MCw2MCwxXV0sWzMsNCwiVFxcZXRhX1oiLDIseyJjb2xvdXIiOlswLDYwLDYwXX0sWzAsNjAsNjAsMV1dLFs0LDUsIlxcbXVfWiIsMix7ImNvbG91ciI6WzAsNjAsNjBdfSxbMCw2MCw2MCwxXV0sWzEsNiwiZyJdLFs2LDMsIlxcZXRhX1oiXSxbMyw1LCJcXG9wZXJhdG9ybmFtZXtpZH1fe1RafSIsMCx7ImN1cnZlIjotMiwic3R5bGUiOnsiYm9keSI6eyJuYW1lIjoiZG90dGVkIn19fV1d
    \begin{tikzcd}
        X & Y & Z \\
        & TY & TZ & TTZ & TZ
        \arrow["f", color={rgb,255:red,92;green,92;blue,214}, from=1-1, to=1-2]
        \arrow["g", from=1-2, to=1-3]
        \arrow["{\eta_Y}"', color={rgb,255:red,92;green,92;blue,214}, from=1-2, to=2-2]
        \arrow["{\eta_Z}", from=1-3, to=2-3]
        \arrow["Tg"', color={rgb,255:red,92;green,92;blue,214}, from=2-2, to=2-3]
        \arrow["{T\eta_Z}"', color={rgb,255:red,214;green,92;blue,92}, from=2-3, to=2-4]
        \arrow["{\operatorname{id}_{TZ}}", curve={height=-12pt}, dotted, from=2-3, to=2-5]
        \arrow["{\mu_Z}"', color={rgb,255:red,214;green,92;blue,92}, from=2-4, to=2-5]
    \end{tikzcd}
    \end{equation*}
    Note that the red path is canceled per right unitality law, and the square is the naturality condition. Thus we have $L_Tg\circ_{kl}L_Tf=\eta_Z\circ g\circ f=L_T(g\circ f)$ as needed.

    \textbf{(Part 2):} Given the identity in $\textbf{C}_T$ at object $X$, which is $\eta_X:X\to TX$, then we see that $R_T\eta_X=\mu_X\circ T\eta_X=\operatorname{id}_{TX}$ by the right unitality law.

    On the other hand, consider the functor over composition between $f:X\to TY$ and $g:Y\to TZ$:

    \begin{equation*}
    \begin{aligned}
        R_T(g\circ_{kl}f) &= R_T(\mu_Z\circ Tg\circ f) = \mu_Y\circ T\mu_Z\circ TTg\circ Tf \\
        R_T(g)\circ R_T(f) &= \mu_Z\circ Tg\circ \mu_Y\circ Tf
    \end{aligned}
    \end{equation*}

    The first line corresponds to the blue path on the diagram below, and the second line corresponds to the red path. Note that every things commutes because the square represents the naturality of $\mu$ and the triangle represented the associativity property of monad.

    \begin{equation*}
    % https://q.uiver.app/#q=WzAsNyxbMCwwLCJUWCJdLFsxLDAsIlRUWSJdLFsyLDAsIlRUVFoiXSxbMywwLCJUVFoiXSxbNCwwLCJUWiJdLFsxLDEsIlRZIl0sWzIsMSwiVFRaIl0sWzAsMSwiVGYiLDAseyJjb2xvdXIiOlsyNzAsNjAsNjBdfSxbMjcwLDYwLDYwLDFdXSxbMSwyLCJUVGciLDAseyJjb2xvdXIiOlsyNDAsNjAsNjBdfSxbMjQwLDYwLDYwLDFdXSxbMiwzLCJUXFxtdV9aIiwwLHsiY29sb3VyIjpbMjQwLDYwLDYwXX0sWzI0MCw2MCw2MCwxXV0sWzMsNCwiXFxtdV9aIiwwLHsiY29sb3VyIjpbMjQwLDYwLDYwXX0sWzI0MCw2MCw2MCwxXV0sWzEsNSwiXFxtdV9ZIiwyLHsiY29sb3VyIjpbMCw2MCw2MF19LFswLDYwLDYwLDFdXSxbNSw2LCJUZyIsMix7ImNvbG91ciI6WzAsNjAsNjBdfSxbMCw2MCw2MCwxXV0sWzYsNCwiXFxtdV9aIiwyLHsiY29sb3VyIjpbMCw2MCw2MF19LFswLDYwLDYwLDFdXSxbMiw2LCJcXG11X3tUWX0iXV0=
    \begin{tikzcd}
        TX & TTY & TTTZ & TTZ & TZ \\
        & TY & TTZ
        \arrow["Tf", color={rgb,255:red,153;green,92;blue,214}, from=1-1, to=1-2]
        \arrow["TTg", color={rgb,255:red,92;green,92;blue,214}, from=1-2, to=1-3]
        \arrow["{\mu_Y}"', color={rgb,255:red,214;green,92;blue,92}, from=1-2, to=2-2]
        \arrow["{T\mu_Z}", color={rgb,255:red,92;green,92;blue,214}, from=1-3, to=1-4]
        \arrow["{\mu_{TY}}", from=1-3, to=2-3]
        \arrow["{\mu_Z}", color={rgb,255:red,92;green,92;blue,214}, from=1-4, to=1-5]
        \arrow["Tg"', color={rgb,255:red,214;green,92;blue,92}, from=2-2, to=2-3]
        \arrow["{\mu_Z}"', color={rgb,255:red,214;green,92;blue,92}, from=2-3, to=1-5]
    \end{tikzcd}
    \end{equation*}

\end{proof}

\begin{proposition}
    \label{prop:adjointion-kleisli}
    Both functors are indeed adjunction. (1) The composition between functors $R_T\circ L_T:\textbf{C}\to\textbf{C}$ are naturality isomorphic to $T$. (2) The functor $L_T$ is left-adjoint to $R_T$ (3) unit of the adjunction is unit of monad given by $\eta$.
\end{proposition}

\begin{proof}
    \textbf{(Part 1):} Let's consider what the composition between functors looks like. Given an object $X$ we have that $R_T\circ L_TX=TW$. On the other hand, given morphism $f:X\to Y$, then we have:
    \begin{equation*}
        R_T\circ L_Tf = \mu_Y\circ T(\eta_Y\circ f) = \mu_Y\circ T\eta_Y\circ Tf= Tf
    \end{equation*}
    We used the right unitality on the last equality. Thus, we can see that $R_T\circ L_T=T$ thus it is naturally isomorphic by the natual isomorphism of $\operatorname{id}_T$. 

    \textbf{(Part 2):} To show that $L_T\dashv R_T$, we starts with showing that:
    \begin{equation*}
        \operatorname{Hom}_{\textbf{C}_T}(L_TC, D) \cong \operatorname{Hom}_{\textbf{C}}(C, R_TD)
    \end{equation*}

    Given $f\in \operatorname{Hom}_{\textbf{C}_T}(L_TC, D)=\operatorname{Hom}_{\textbf{C}_T}(C, D)$, we have that $f:C\to TD$ (morphism in $\textbf{C}$), in which we can define $f^\flat=f$. On the other hand, given $g\in\operatorname{Hom}_{\textbf{C}_T}(C, R_TD)$, we have that $g:C\to TD$, in which we can set $g^\sharp=g$. And, it is clear that $(f^\flat)^\sharp=f$ and $(g^\sharp)^\flat=g$. So we have the isomorphism.

    Now, we will have to show the naturality on objects in $\textbf{C}$, that is, given $f^\text{op}:C\to C'$, we want to show that the following LHS commutative diagram commutes:

    \begin{equation*}
    % https://q.uiver.app/#q=WzAsNCxbMCwxLCJcXG9wZXJhdG9ybmFtZXtIb219X3tcXHRleHRiZntDfV9UfShMX1RDLCBEKSJdLFswLDAsIlxcb3BlcmF0b3JuYW1le0hvbX1fe1xcdGV4dGJme0N9fShDLCBSX1REKSJdLFsyLDEsIlxcb3BlcmF0b3JuYW1le0hvbX1fe1xcdGV4dGJme0N9X1R9KExfVEMnLCBEKSJdLFsyLDAsIlxcb3BlcmF0b3JuYW1le0hvbX1fe1xcdGV4dGJme0N9fShDJywgUl9URCkiXSxbMCwyLCItXFxjaXJjX3trbH1MX1RmIiwyXSxbMCwxLCJcXGNvbmciLDAseyJzdHlsZSI6eyJ0YWlsIjp7Im5hbWUiOiJhcnJvd2hlYWQifSwiaGVhZCI6eyJuYW1lIjoibm9uZSJ9fX1dLFsyLDMsIlxcY29uZyIsMix7InN0eWxlIjp7InRhaWwiOnsibmFtZSI6ImFycm93aGVhZCJ9LCJoZWFkIjp7Im5hbWUiOiJub25lIn19fV0sWzEsMywiLVxcY2lyYyBmIl1d
    \begin{tikzcd}
        {\operatorname{Hom}_{\textbf{C}}(C, R_TD)} && {\operatorname{Hom}_{\textbf{C}}(C', R_TD)} \\
        {\operatorname{Hom}_{\textbf{C}_T}(L_TC, D)} && {\operatorname{Hom}_{\textbf{C}_T}(L_TC', D)}
        \arrow["{-\circ f}", from=1-1, to=1-3]
        \arrow["\cong", tail reversed, no head, from=2-1, to=1-1]
        \arrow["{-\circ_{kl}L_Tf}"', from=2-1, to=2-3]
        \arrow["\cong"', tail reversed, no head, from=2-3, to=1-3]
    \end{tikzcd}
    \qquad \quad
    % https://q.uiver.app/#q=WzAsNixbMiwwLCJUVEQiXSxbMCwxLCJDJyJdLFsxLDEsIkMiXSxbMSwwLCJUQyJdLFszLDAsIlREIl0sWzIsMSwiVEQiXSxbMSwyLCJmIiwwLHsiY29sb3VyIjpbMjQwLDYwLDYwXX0sWzI0MCw2MCw2MCwxXV0sWzIsMywiXFxldGFfQyIsMCx7ImNvbG91ciI6WzI0MCw2MCw2MF19LFsyNDAsNjAsNjAsMV1dLFszLDAsIlRnIiwwLHsiY29sb3VyIjpbMjQwLDYwLDYwXX0sWzI0MCw2MCw2MCwxXV0sWzAsNCwiXFxtdV9EIiwwLHsiY29sb3VyIjpbMjQwLDYwLDYwXX0sWzI0MCw2MCw2MCwxXV0sWzIsNSwiZyIsMl0sWzUsMCwiXFxldGFfe1REfSJdLFs1LDQsIlxcb3BlcmF0b3JuYW1le2lkfV97VER9IiwyLHsic3R5bGUiOnsiYm9keSI6eyJuYW1lIjoiZG90dGVkIn19fV1d
    \begin{tikzcd}
        & TC & TTD & TD \\
        {C'} & C & TD
        \arrow["Tg", color={rgb,255:red,92;green,92;blue,214}, from=1-2, to=1-3]
        \arrow["{\mu_D}", color={rgb,255:red,92;green,92;blue,214}, from=1-3, to=1-4]
        \arrow["f", color={rgb,255:red,92;green,92;blue,214}, from=2-1, to=2-2]
        \arrow["{\eta_C}", color={rgb,255:red,92;green,92;blue,214}, from=2-2, to=1-2]
        \arrow["g"', from=2-2, to=2-3]
        \arrow["{\eta_{TD}}", from=2-3, to=1-3]
        \arrow["{\operatorname{id}_{TD}}"', dotted, from=2-3, to=1-4]
    \end{tikzcd}
    \end{equation*}

    Note that $\operatorname{Hom}_{\textbf{C}_T}(L_T-, D)[f^\text{op}]=-\circ_{kl}L_Tf$, as given in the arrow below. Similarly, for arrow above, we have $\operatorname{Hom}_{\textbf{C}}(-, R_TD)[f^\text{op}] = -\circ f$. Then given $g\in\operatorname{Hom}_\textbf{C}(C, R_TD)$ as $g:C\to TD$, where the blue path on the diagram on the RHS represents: $g\circ_{kl} L_Tf$ and with naturality of $\eta$ and the left unitality law (dotted line) shows that the composition is $g\circ f$.

    On the other hand, consider the naturality on objects in $\textbf{D}$, that is given the arrow $h:D\to TD'$ for $h\in\operatorname{Hom}_{\textbf{C}_T}(D, D')$, then the following LHS diagram commutes
    \begin{equation*}
    % https://q.uiver.app/#q=WzAsNCxbMCwwLCJcXG9wZXJhdG9ybmFtZXtIb219X3tcXHRleHRiZntDfV9UfShMX1RDLCBEKSJdLFswLDEsIlxcb3BlcmF0b3JuYW1le0hvbX1fe1xcdGV4dGJme0N9fShDLCBSX1REKSJdLFsyLDAsIlxcb3BlcmF0b3JuYW1le0hvbX1fe1xcdGV4dGJme0N9X1R9KExfVEMsIEQnKSJdLFsyLDEsIlxcb3BlcmF0b3JuYW1le0hvbX1fe1xcdGV4dGJme0N9fShDLCBSX1REJykiXSxbMCwyLCJoXFxjaXJjX3trbH0tIl0sWzAsMSwiXFxjb25nIiwyXSxbMSwzLCJSX1RoXFxjaXJjLSIsMl0sWzMsMiwiXFxjb25nIiwyLHsic3R5bGUiOnsidGFpbCI6eyJuYW1lIjoiYXJyb3doZWFkIn0sImhlYWQiOnsibmFtZSI6Im5vbmUifX19XV0=
    \begin{tikzcd}
        {\operatorname{Hom}_{\textbf{C}_T}(L_TC, D)} && {\operatorname{Hom}_{\textbf{C}_T}(L_TC, D')} \\
        {\operatorname{Hom}_{\textbf{C}}(C, R_TD)} && {\operatorname{Hom}_{\textbf{C}}(C, R_TD')}
        \arrow["{h\circ_{kl}-}", from=1-1, to=1-3]
        \arrow["\cong"', from=1-1, to=2-1]
        \arrow["{R_Th\circ-}"', from=2-1, to=2-3]
        \arrow["\cong"', tail reversed, no head, from=2-3, to=1-3]
    \end{tikzcd}
    \end{equation*}
    where $\operatorname{Hom}_{\textbf{C}_T}(L_TC, -)[h]=h\circ_{kl}-$ and $\operatorname{Hom}_{\textbf{C}}(C, R_T-)[h] = R_Th\circ-$. Then given $g\in\operatorname{Hom}_{\textbf{C}_T}(L_TC, D)$ we can see that: $h\circ_{kl}g = R_Th\circ g$ directly from the definition. Thus the diagram commutes.

    \textbf{(Part 3):} Note that the identity morphism over object $L_TC$ in $\textbf{C}_T$ is $\eta_C$, and we have that $(\eta_C)^\flat=\eta_C$. Furthermore, it is a natual transformation with signature of $\operatorname{id}_T\Rightarrow T$ as $R_T\circ L_T=T$, as needed.
    
\end{proof}

\begin{definition}{\textbf{(Idempotent)}}
    Given a monad $(T, \eta, \mu)$ on category $\textbf{C}$, it is called idempotent if the multiplication $\mu:TT\Rightarrow T$ is natural isomorphism.
\end{definition}

\begin{remark}{(Interpretation of Monad and Kleisli)}
    With the Kleisli morphism, one can view monad as the extension of spaces or objects with the equipped extended functions of Kleisli morphism (since it outputs $TY$ the extended space).
\end{remark}

We can, also, view the monad as ``consistence choice of formal expression together with way of evaluating them''. Let's consider the following definition:

\begin{definition}{\textbf{(Free commutative Monoid Monad)}}
    It is a monad $(F, \eta, \mu)$ operating on $\textbf{Set}$, where we defined each components as:
    \begin{itemize}
        \item Functor: Given a set $X$, then $FX$ is the set of formal sum of elements, for example, $x_1+\cdots+x_n$. Furthermore, with $f:X\to Y$, we can defined $Ff:FX\to FY$ as:
        \begin{equation*}
            Ff(x_1+\cdots+x_n) = f(x_1) + \cdots + f(x_n)
        \end{equation*}
        If $X$ is empty, then we will have empty expression, and $Ff$ on empty expression is empty expression. Thus, it is clear that it is functor.
        \item Unit: We define $\eta_X:X\to FX$ as seting each element of $x\in X$ to its formal expression, being $x$. Note that it is natural in $X$ i.e 
        \begin{equation*}
        % https://q.uiver.app/#q=WzAsNCxbMCwwLCJYIl0sWzEsMCwiWSJdLFswLDEsIkZYIl0sWzEsMSwiRlkiXSxbMCwyLCJcXGV0YV9YIiwyXSxbMSwzLCJcXGV0YV9ZIl0sWzIsMywiRmYiLDJdLFswLDEsImYiXV0=
        \begin{tikzcd}
            X & Y \\
            FX & FY
            \arrow["f", from=1-1, to=1-2]
            \arrow["{\eta_X}"', from=1-1, to=2-1]
            \arrow["{\eta_Y}", from=1-2, to=2-2]
            \arrow["Ff"', from=2-1, to=2-2]
        \end{tikzcd}
        \end{equation*}
        as we note that $f(x)\in Y$ will get turned to its formal expression of $f(x)\in FY$.
        \item Multiplication: Note that $FFX$ is the formal expression of formal expression, constructed as adding the bracket i.e $(x_1+x_2)+(x_1+x_3)$. Then $\mu_X:FFX\to FX$, is the removal of bracket. It is natural as we consider the following diagram:
        \begin{equation*}
        % https://q.uiver.app/#q=WzAsOCxbMCwwLCJGRlgiXSxbMSwwLCJGRlkiXSxbMCwxLCJGWCJdLFsxLDEsIkZZIl0sWzMsMCwiKHhfMSt4XzIpK3hfMyJdLFs0LDAsIihmKHhfMSkgKyBmKHhfMikpK2YoeF8zKSJdLFszLDEsInhfMSt4XzIreF8zIl0sWzQsMSwiZih4XzEpICsgZih4XzIpK2YoeF8zKSJdLFswLDIsIlxcbXVfWCIsMl0sWzEsMywiXFxtdV9ZIl0sWzAsMSwiRkZmIl0sWzIsMywiRmYiLDJdLFs0LDUsIiIsMix7InN0eWxlIjp7InRhaWwiOnsibmFtZSI6Im1hcHMgdG8ifX19XSxbNCw2LCIiLDAseyJzdHlsZSI6eyJ0YWlsIjp7Im5hbWUiOiJtYXBzIHRvIn19fV0sWzYsNywiIiwwLHsic3R5bGUiOnsidGFpbCI6eyJuYW1lIjoibWFwcyB0byJ9fX1dLFs1LDcsIiIsMix7InN0eWxlIjp7InRhaWwiOnsibmFtZSI6Im1hcHMgdG8ifX19XV0=
        \begin{tikzcd}
            FFX & FFY && {(x_1+x_2)+x_3} & {(f(x_1) + f(x_2))+f(x_3)} \\
            FX & FY && {x_1+x_2+x_3} & {f(x_1) + f(x_2)+f(x_3)}
            \arrow["FFf", from=1-1, to=1-2]
            \arrow["{\mu_X}"', from=1-1, to=2-1]
            \arrow["{\mu_Y}", from=1-2, to=2-2]
            \arrow[maps to, from=1-4, to=1-5]
            \arrow[maps to, from=1-4, to=2-4]
            \arrow[maps to, from=1-5, to=2-5]
            \arrow["Ff"', from=2-1, to=2-2]
            \arrow[maps to, from=2-4, to=2-5]
        \end{tikzcd}
        \end{equation*}
    \end{itemize}
    Now, we will consider the laws of the monad, and confirmed that they are satisfied (by consider the examples). Consider the following diagram for left, right unitality and associativity, respectively
    \begin{equation*}
    % https://q.uiver.app/#q=WzAsMyxbMCwwLCJ4XzEreF8yIl0sWzEsMCwiKHhfMSt4XzIpIl0sWzEsMSwieF8xK3hfMiJdLFswLDEsIlxcZXRhX3tGWH0iXSxbMSwyLCJcXG11X1giXSxbMCwyXV0=
    \begin{tikzcd}
        {x_1+x_2} & {(x_1+x_2)} \\
        & {x_1+x_2}
        \arrow["{\eta_{FX}}", from=1-1, to=1-2]
        \arrow[from=1-1, to=2-2]
        \arrow["{\mu_X}", from=1-2, to=2-2]
    \end{tikzcd}
    \quad
    % https://q.uiver.app/#q=WzAsMyxbMCwwLCJ4XzEreF8yIl0sWzEsMCwiKHhfMSkrKHhfMikiXSxbMSwxLCJ4XzEreF8yIl0sWzAsMl0sWzAsMSwiRlxcZXRhX1giXSxbMSwyLCJcXG11X1giXV0=
    \begin{tikzcd}
        {x_1+x_2} & {(x_1)+(x_2)} \\
        & {x_1+x_2}
        \arrow["{F\eta_X}", from=1-1, to=1-2]
        \arrow[from=1-1, to=2-2]
        \arrow["{\mu_X}", from=1-2, to=2-2]
    \end{tikzcd}
    \quad 
    % https://q.uiver.app/#q=WzAsNCxbMCwwLCIoKHhfMSkrKHhfMikpIl0sWzAsMSwiKHhfMSkgKyAoeF8yKSJdLFsxLDAsIih4XzEreF8yKSJdLFsxLDEsInhfMSt4XzIiXSxbMCwyLCJUXFxtdV9YIl0sWzAsMSwiXFxtdV97VFh9IiwyXSxbMSwzLCJcXG11X1giLDJdLFsyLDMsIlxcbXVfWCJdLFsxLDNdLFsxLDNdXQ==
    \begin{tikzcd}
        {((x_1)+(x_2))} & {(x_1+x_2)} \\
        {(x_1) + (x_2)} & {x_1+x_2}
        \arrow["{T\mu_X}", from=1-1, to=1-2]
        \arrow["{\mu_{TX}}"', from=1-1, to=2-1]
        \arrow["{\mu_X}", from=1-2, to=2-2]
        \arrow["{\mu_X}"', from=2-1, to=2-2]
        \arrow[from=2-1, to=2-2]
        \arrow[from=2-1, to=2-2]
    \end{tikzcd}
    \end{equation*}
\end{definition}

We can see that the formal expression doesn't give out the evaluation. By the ues of algebra over the monad, we can do something like this. 

\begin{definition}{\textbf{(Eilenberg-Moore Algebras of Monad)}}
    Given a monad $(T, \eta, \mu)$ on category $\textbf{C}$, we define algebra of/over $T$ or $T$-algebra or Eilenberg-Moore algebra to be a pair $(A, e)$ where $A$ is an object of $\textbf{C}$ and a morphism $e:TA\to A$ in $\textbf{C}$ that satisfies the unit and composition law, defined by:
    \begin{equation*}
    % https://q.uiver.app/#q=WzAsNyxbMCwwLCJBIl0sWzEsMCwiVEEiXSxbMSwxLCJBIl0sWzMsMCwiVFRBIl0sWzQsMCwiVEEiXSxbNCwxLCJBIl0sWzMsMSwiVEEiXSxbMCwxLCJcXGV0YV9BIl0sWzEsMiwiZV9BIl0sWzAsMiwiXFxvcGVyYXRvcm5hbWV7aWR9X0EiLDJdLFszLDQsIlRlX0EiXSxbMyw2LCJcXG11X0EiLDJdLFs0LDUsImVfQSJdLFs2LDUsImVfQSIsMl1d
    \begin{tikzcd}
        A & TA && TTA & TA \\
        & A && TA & A
        \arrow["{\eta_A}", from=1-1, to=1-2]
        \arrow["{\operatorname{id}_A}"', from=1-1, to=2-2]
        \arrow["{e_A}", from=1-2, to=2-2]
        \arrow["{Te_A}", from=1-4, to=1-5]
        \arrow["{\mu_A}"', from=1-4, to=2-4]
        \arrow["{e_A}", from=1-5, to=2-5]
        \arrow["{e_A}"', from=2-4, to=2-5]
    \end{tikzcd}
    \end{equation*}
\end{definition}

\begin{remark}{(Algebra and Monoid)}
    We can see that the unit law of algebra means, for free commutative monoid monda that the evaluation of a single element's formal expression is that element. On the other hand, the composition can be described in the following example:
    \begin{equation*}
    % https://q.uiver.app/#q=WzAsNCxbMCwwLCIoMiszKSsoMSsyKSJdLFswLDEsIjIrMysxKzIiXSxbMSwwLCI1KzMiXSxbMSwxLCI4Il0sWzAsMSwiXFxtdSIsMix7InN0eWxlIjp7InRhaWwiOnsibmFtZSI6Im1hcHMgdG8ifX19XSxbMCwyLCJGZSIsMCx7InN0eWxlIjp7InRhaWwiOnsibmFtZSI6Im1hcHMgdG8ifX19XSxbMSwzLCJlIiwyLHsic3R5bGUiOnsidGFpbCI6eyJuYW1lIjoibWFwcyB0byJ9fX1dLFsyLDMsImUiLDAseyJzdHlsZSI6eyJ0YWlsIjp7Im5hbWUiOiJtYXBzIHRvIn19fV1d
    \begin{tikzcd}
        {(2+3)+(1+2)} & {5+3} \\
        {2+3+1+2} & 8
        \arrow["Fe", maps to, from=1-1, to=1-2]
        \arrow["\mu"', maps to, from=1-1, to=2-1]
        \arrow["e", maps to, from=1-2, to=2-2]
        \arrow["e"', maps to, from=2-1, to=2-2]
    \end{tikzcd}
    \end{equation*} 
    Thus, we see that every monoid is an $F$-algebra. On the other hand, $F$-algebra is an commutative monoid, as we can set the natural element to be empty expression, and the sum between elements are the formal expression.
\end{remark}

\begin{definition}\textbf{(Eilenberg-Moore Category/Category of Algebras of $T$)}
    Given a monad $(T, \eta,\mu)$, then we can define Category of Algebras (denoted $\textbf{C}^T$) to have the objects being $(A, e_A)$ for each $A$ in $\textbf{C}$ with suitable $e_A:TA\to A$. The $T$-morphism $f:(A, e_A)\to(B, e_B)$ is an morphism $f:A\to B$ such that:
    \begin{equation*}
    % https://q.uiver.app/#q=WzAsNCxbMCwwLCJUQSJdLFsxLDAsIlRCIl0sWzAsMSwiQSJdLFsxLDEsIkIiXSxbMCwxLCJUZiJdLFswLDIsImVfQSIsMl0sWzIsMywiZiIsMl0sWzEsMywiZV9CIl1d
    \begin{tikzcd}
        TA & TB \\
        A & B
        \arrow["Tf", from=1-1, to=1-2]
        \arrow["{e_A}"', from=1-1, to=2-1]
        \arrow["{e_B}", from=1-2, to=2-2]
        \arrow["f"', from=2-1, to=2-2]
    \end{tikzcd}
    \end{equation*}
    Clearly the identity morphism $\operatorname{id}_{(A,e_A)}$ is clearly $\operatorname{id}_A$, and the associativity of composition is clearly inherted from the category $\textbf{C}$ (and the commutativity is still valid). Note that $e_A:(A, e_A)\to(TA,\mu_A)$ is $T$-morphism as the multiplication law guarantee the commutativity.
\end{definition}

\begin{definition}{\textbf{(Free Algebra)}}
    The $T$-algebra of the form $(TX,\mu_X)$ for some $X$ of $\textbf{C}$ is called free $T$-algebra.
\end{definition}

\begin{remark}
    Note that free algebra is an algebra because the first 2 diagram specifices the condition of the algebra (which are left unitality, and associativity):
    \begin{equation*}
    % https://q.uiver.app/#q=WzAsMyxbMCwwLCJUWCJdLFsxLDAsIlRUWCJdLFsxLDEsIlRYIl0sWzAsMSwiXFxldGFfe1RYfSJdLFswLDIsIlxcb3BlcmF0b3JuYW1le2lkfV97VFh9IiwyXSxbMSwyLCJcXG11X1giXV0=
    \begin{tikzcd}
        TX & TTX \\
        & TX
        \arrow["{\eta_{TX}}", from=1-1, to=1-2]
        \arrow["{\operatorname{id}_{TX}}"', from=1-1, to=2-2]
        \arrow["{\mu_X}", from=1-2, to=2-2]
    \end{tikzcd}
    \qquad 
    % https://q.uiver.app/#q=WzAsNCxbMCwwLCJUVFRYIl0sWzAsMSwiVFRYIl0sWzEsMSwiVFgiXSxbMSwwLCJUVFgiXSxbMCwzLCJUXFxtdV97WH0iXSxbMCwxLCJcXG11X3tUWH0iLDJdLFsxLDIsIlxcbXVfWCIsMl0sWzMsMiwiXFxtdV9YIl1d
    \begin{tikzcd}
        TTTX & TTX \\
        TTX & TX
        \arrow["{T\mu_{X}}", from=1-1, to=1-2]
        \arrow["{\mu_{TX}}"', from=1-1, to=2-1]
        \arrow["{\mu_X}", from=1-2, to=2-2]
        \arrow["{\mu_X}"', from=2-1, to=2-2]
    \end{tikzcd}
    \qquad
    % https://q.uiver.app/#q=WzAsNCxbMCwxLCJUWCJdLFsxLDEsIlRZIl0sWzEsMCwiVFRZIl0sWzAsMCwiVFRYIl0sWzAsMSwiVGYiLDJdLFszLDAsIlxcbXVfWCIsMl0sWzIsMSwiXFxtdV9ZIl0sWzMsMiwiVFRmIl1d
    \begin{tikzcd}
        TTX & TTY \\
        TX & TY
        \arrow["TTf", from=1-1, to=1-2]
        \arrow["{\mu_X}"', from=1-1, to=2-1]
        \arrow["{\mu_Y}", from=1-2, to=2-2]
        \arrow["Tf"', from=2-1, to=2-2]
    \end{tikzcd}
    \end{equation*}
    On the other hand, the morphism $f:X\to Y$ of $\textbf{C}$ is always a map between $(TX, \mu_X)\to(TY, \mu_Y)$ because of the naturality of $\mu$, as shown in the right most diagram.
\end{remark}

\begin{remark}
    In the context of free commutative monoid monad $F$, the free algebra can be seen as the simplification of terms, for example:
    \begin{equation*}
    \begin{aligned}
        (x_1&+\cdots+x_n) + (y_1+\cdots+y_n)  \\
        &:= (x_1+\cdots+x_n + y_1+\cdots+y_n)
    \end{aligned}
    \end{equation*}
\end{remark}

Now, with similar move to the Kleisli category, one can define the Eilenberg-Moore adjunction. 

\begin{definition}{\textbf{(Left and Right Eilenberg-Moore Adjunction)}}
    Given the monad $(T,\eta,\mu)$ of category $\textbf{C}$, we define the fully faithful ``forgetful'' functor of $R^T:\textbf{C}^T\rightarrow\textbf{C}$, and the functor $L^T:\textbf{C}\to\textbf{C}^T$, in which: $L^TX=(TX, \mu_X)$ being the free algebra. On the other hand, given morphism $f:X\to Y$, we have the lifted function to be $Tf:(TX,\mu_X)\to(TY,\mu_Y)$ (which is a $T$-morphism due to naturality of $\mu$)
\end{definition}

Note that the functorial properties follows directly from the functoriality of $T$. 

\begin{proposition}
    Both functors are indeed adjunction. (1) The composition between functors $R^T\circ L^T:\textbf{C}\to\textbf{C}$ are naturality isomorphic to $T$. (2) The functor $L^T$ is left-adjoint to $R^T$ (3) unit of the adjunction is unit of comonad given by $\eta$, while its counit is the structure map $e:L^T\circ R^T\Rightarrow\operatorname{id}_{\textbf{C}^T}$
\end{proposition}
\begin{proof}
    \textbf{(Part 1):} Let's consider how the composed function $R^T\circ L^T$ acts on $X$, we have that $R^T(L^TX)=R^T(TX,\mu_X)=TX$. On the other hand, given morphism $f:X\to Y$, then $R^T(L^Tf)=R^T(Tf)=Tf$. Thus $R^T\circ L^T=T$ and so naturally isomorphic between them.

    \textbf{(Part 2):} We want to show that:
    \begin{equation*}
        \operatorname{Hom}_{\textbf{C}^T}((TC, \mu_C),(D, e_D)) = \operatorname{Hom}_{\textbf{C}^T}(L^TC,(D, e_D)) \cong \operatorname{Hom}_{\textbf{C}}(C, R^T(D, e_D)) = \operatorname{Hom}_{\textbf{C}}(C, D)
    \end{equation*}

    That is given $f^\sharp:(TC, \mu_C)\to(D, e_D)$ and send it to $f^\flat:C\to D$, then we will define $R^Tf^\sharp\circ\eta_C=f^\flat$, and we also define its inverse to be $e\circ L^Tf^\flat=f^\sharp$ (the $e$ here is the structure map of the \textbf{co-domain}). To show that they are inverse of each other, we have:
    \begin{itemize}
        \item Given $f:C\to D$, then we have $f^\sharp=e_c\circ L^T = e_D\circ Tf$ (where the structure map is $e_{TD}=\mu_D$, for the middle $T$-algebra), and this is given in the diagram within category $\textbf{C}^T$ below as:
        \begin{equation*}
        % https://q.uiver.app/#q=WzAsMyxbMCwwLCIoVEMsXFxtdV9DKSJdLFsxLDAsIihURCxcXG11X0QpIl0sWzIsMCwiKEQsZV9EKSJdLFswLDEsIlRmIl0sWzEsMiwiZV9EIl1d
        \begin{tikzcd}
            {(TC,\mu_C)} & {(TD,\mu_D)} & {(D,e_D)}
            \arrow["Tf", from=1-1, to=1-2]
            \arrow["{e_D}", from=1-2, to=1-3]
        \end{tikzcd}
        \end{equation*}
        By passing this through $R^T$, we forgot the structure, thus giving us the map of $e_D\circ Tf:TC\to D$ but now we are back to category $\textbf{C}^T$, in which $(f^\sharp)^\flat=e_D\circ Tf\circ\eta_C=e_D\circ\eta_{D}\circ f=f$ following from the the naturality of $\eta$ and unit law of algebra.
        \item Given $g:(TC, \mu_C)\to(D, e_D)$, then we note that $R^Tg:TC\to D$, then we see that $g^\flat=g\circ\eta_C$, then we have $(g^\flat)^\sharp=e\circ L^T(g\circ\eta_C) = e_D\circ Tg\circ T\eta_C$, which is given in the LHS diagram
        \begin{equation*}
        % https://q.uiver.app/#q=WzAsOCxbMCwwLCIoVEMsIFxcbXVfQykiXSxbMSwwLCIoVFRDLC0pIl0sWzIsMCwiKFRELFxcbXVfRCkiXSxbMiwxLCIoRCwgZV9EKSJdLFs0LDAsIlRUQyJdLFs1LDAsIlREIl0sWzUsMSwiRCJdLFs0LDEsIlRDIl0sWzEsMiwiVGciXSxbMCwxLCJUXFxldGFfQyJdLFsyLDMsImVfRCJdLFs0LDUsIlRnIl0sWzUsNiwiZV9EIl0sWzQsNywiXFxtdV9DIiwyXSxbNyw2LCJnIiwyXV0=
        \begin{tikzcd}
            {(TC, \mu_C)} & {(TTC,-)} & {(TD,\mu_D)} && TTC & TD \\
            && {(D, e_D)} && TC & D
            \arrow["{T\eta_C}", from=1-1, to=1-2]
            \arrow["Tg", from=1-2, to=1-3]
            \arrow["{e_D}", from=1-3, to=2-3]
            \arrow["Tg", from=1-5, to=1-6]
            \arrow["{\mu_C}"', from=1-5, to=2-5]
            \arrow["{e_D}", from=1-6, to=2-6]
            \arrow["g"', from=2-5, to=2-6]
        \end{tikzcd}
        \end{equation*}
        Note that $e_D\circ Tg=g\circ\mu_C$ because $e_D$ is an morphism of algebra which makes the RHS diagram commutes. Thus, we have $(g^\flat)^\sharp=g\circ\mu_C\circ T\eta_c=g$ per right commutativity law.
    \end{itemize}
    Then we can see that we have defined the uniit and counit of adjunction to be unit $\eta$ and structure map $e$. Note that this works for any kinds of $e_D$ (although we should keep in fixed for each unique objects).

    \textbf{(Part 3):} With the result of part above, we have determined how the actions works in details. To show that $e$ are unit and co-unit, we starts with showing that $e$ is an natural transformation that is $e:L^T\circ R^T\Rightarrow\operatorname{id}_{\textbf{C}^T}$. This is clear, as given a $T$-morphism $f:(A, e_A)\to(B, e_B)$, we have $f\circ e_A=e_B\circ Tf$. 
    
    Now, we are left to show the triangle identities. We have the following diagrams, where the LHS we consider the component for object $C$, while the RHS we consider the component for object $(C,e_C)$
    \begin{equation*}
    % https://q.uiver.app/#q=WzAsNixbMCwwLCJMXlQiXSxbMSwwLCJMXlRSXlRMXlQiXSxbMiwwLCJMXlQiXSxbMCwxLCIoVEMsIFxcbXVfQykiXSxbMSwxLCIoVFRDLFxcbXVfe1RDfSkiXSxbMiwxLCIoVEMsXFxtdV9DKSJdLFswLDEsIkxeVFxcZXRhIiwwLHsibGV2ZWwiOjJ9XSxbMSwyLCJlTF5UIiwwLHsibGV2ZWwiOjJ9XSxbMyw0LCJMXlRcXGV0YV9DIl0sWzQsNSwiZV97VEN9Il0sWzMsNCwiVFxcZXRhX0MiLDJdLFs0LDUsIlxcbXVfQyIsMl0sWzMsNSwiXFxvcGVyYXRvcm5hbWV7aWR9X3tUQ30iLDIseyJjdXJ2ZSI6NCwic3R5bGUiOnsiYm9keSI6eyJuYW1lIjoiZG90dGVkIn19fV1d
    \begin{tikzcd}
        {L^T} & {L^TR^TL^T} & {L^T} \\
        {(TC, \mu_C)} & {(TTC,\mu_{TC})} & {(TC,\mu_C)}
        \arrow["{L^T\eta}", Rightarrow, from=1-1, to=1-2]
        \arrow["{eL^T}", Rightarrow, from=1-2, to=1-3]
        \arrow["{L^T\eta_C}", from=2-1, to=2-2]
        \arrow["{T\eta_C}"', from=2-1, to=2-2]
        \arrow["{\operatorname{id}_{TC}}"', curve={height=24pt}, dotted, from=2-1, to=2-3]
        \arrow["{e_{TC}}", from=2-2, to=2-3]
        \arrow["{\mu_C}"', from=2-2, to=2-3]
    \end{tikzcd}
    \quad\quad
    % https://q.uiver.app/#q=WzAsNixbMCwwLCJSXlQiXSxbMSwwLCJSXlRMXlRSXlQiXSxbMiwwLCJSXlQiXSxbMCwxLCJDIl0sWzEsMSwiQyJdLFsyLDEsIkMiXSxbMCwxLCJcXGV0YSBSXlQiLDAseyJsZXZlbCI6Mn1dLFsxLDIsIlJeVGUiLDAseyJsZXZlbCI6Mn1dLFszLDQsIlxcZXRhX0MiXSxbNCw1LCJlX0MiXSxbMyw1LCJcXG9wZXJhdG9ybmFtZXtpZH1fe0N9IiwyLHsiY3VydmUiOjMsInN0eWxlIjp7ImJvZHkiOnsibmFtZSI6ImRvdHRlZCJ9fX1dXQ==
    \begin{tikzcd}
        {R^T} & {R^TL^TR^T} & {R^T} \\
        C & C & C
        \arrow["{\eta R^T}", Rightarrow, from=1-1, to=1-2]
        \arrow["{R^Te}", Rightarrow, from=1-2, to=1-3]
        \arrow["{\eta_C}", from=2-1, to=2-2]
        \arrow["{\operatorname{id}_{C}}"', curve={height=18pt}, dotted, from=2-1, to=2-3]
        \arrow["{e_C}", from=2-2, to=2-3]
    \end{tikzcd}
    \end{equation*}
    Both are composed to identity because for LHS, we have the right unitality law, while the left has the unit law for $T$-algebra, as needed.
\end{proof}

\begin{corollary}
    For each object $X$ and $T$-algebra $(A, e_A)$, and the morphism $f:X\rightarrow A$, there is a unique $T$-morphism $TX\rightarrow A$ such that following diagram commutes:
    \begin{equation*}
    % https://q.uiver.app/#q=WzAsMyxbMCwwLCJYIl0sWzAsMSwiVFgiXSxbMSwxLCJBIl0sWzAsMSwiXFxldGFfWCIsMl0sWzEsMiwiIiwyLHsic3R5bGUiOnsiYm9keSI6eyJuYW1lIjoiZGFzaGVkIn19fV0sWzAsMiwiZiJdXQ==
    \begin{tikzcd}
        X \\
        TX & A
        \arrow["{\eta_X}"', from=1-1, to=2-1]
        \arrow["f", from=1-1, to=2-2]
        \arrow[dashed, from=2-1, to=2-2]
    \end{tikzcd}
    \end{equation*}
\end{corollary}
\begin{proof}
    Due to the adjunction (see remark \ref{remark:note-unit}), the pair of morphism $f$ and $f^\sharp$ are unique, where we note that $f^\sharp=R^Tf\circ \eta_X=f\circ eta_X$ that is why $f$ has to be a $T$-morphism.
\end{proof}

\subsection{Comonad}

\begin{definition}{\textbf{(Comonad)}}
    Given a category $\textbf{C}$, the monad on $\textbf{C}$ is $(C, \varepsilon, \nu)$ where $T:\textbf{C}\to\textbf{C}$, an unit $\varepsilon:C\Rightarrow\operatorname{id}_\textbf{C}$ and a composition or multiplication $\nu:C\Rightarrow CC$, making the following diagram commutes (they are called left and right co-unitality and co-associativity, respectively):
    \begin{equation*}
    % https://q.uiver.app/#q=WzAsMTAsWzAsMCwiQyJdLFsxLDAsIkNDIl0sWzEsMSwiQyJdLFszLDAsIkMiXSxbNCwwLCJDQyJdLFs0LDEsIkMiXSxbNiwwLCJDIl0sWzcsMCwiQ0MiXSxbNiwxLCJDQyJdLFs3LDEsIkNDQyJdLFswLDEsIlxcbnUiLDAseyJsZXZlbCI6Mn1dLFsxLDIsIlxcdmFyZXBzaWxvbiBDIiwwLHsibGV2ZWwiOjJ9XSxbMCwyLCJcXG9wZXJhdG9ybmFtZXtpZH0iLDIseyJsZXZlbCI6Mn1dLFszLDQsIlxcbnUiLDAseyJsZXZlbCI6Mn1dLFs0LDUsIkNcXHZhcmVwc2lsb24iLDAseyJsZXZlbCI6Mn1dLFszLDUsIlxcb3BlcmF0b3JuYW1le2lkfSIsMix7ImxldmVsIjoyfV0sWzYsNywiXFxudSIsMCx7ImxldmVsIjoyfV0sWzYsOCwiXFxudSIsMix7ImxldmVsIjoyfV0sWzgsOSwiQ1xcbnUiLDIseyJsZXZlbCI6Mn1dLFs3LDksIlxcbnUgQyIsMCx7ImxldmVsIjoyfV1d
    \begin{tikzcd}
        C & CC && C & CC && C & CC \\
        & C &&& C && CC & CCC
        \arrow["\nu", Rightarrow, from=1-1, to=1-2]
        \arrow["{\operatorname{id}}"', Rightarrow, from=1-1, to=2-2]
        \arrow["{\varepsilon C}", Rightarrow, from=1-2, to=2-2]
        \arrow["\nu", Rightarrow, from=1-4, to=1-5]
        \arrow["{\operatorname{id}}"', Rightarrow, from=1-4, to=2-5]
        \arrow["C\varepsilon", Rightarrow, from=1-5, to=2-5]
        \arrow["\nu", Rightarrow, from=1-7, to=1-8]
        \arrow["\nu"', Rightarrow, from=1-7, to=2-7]
        \arrow["{\nu C}", Rightarrow, from=1-8, to=2-8]
        \arrow["C\nu"', Rightarrow, from=2-7, to=2-8]
    \end{tikzcd}
    \end{equation*}
\end{definition}

\begin{definition}{\textbf{(Co-Kleisli Category)}}
    Given a comonad $(C, \varepsilon, \nu)$ on category $\textbf{C}$, then the co-Kleisli category denoted as $\textbf{C}_C$ is defined to have the following component:
    \begin{itemize}
        \item \textbf{(Object):} The objects are objects in $C$.
        \item \textbf{(Morphism):} The morphism between 2 objects $X$ to $Y$ is $k:CX\to Y$, where the composition between this morphism and $h:CY\to Z$ is i.e $h\circ_{ck}k:CX\to Z$ is:
        \begin{equation*}
            CX\xrightarrow{\ \nu_X \ } CCX \xrightarrow{\ Ck\ } CY\xrightarrow{\ h\ } Z
        \end{equation*}
        With $\varepsilon:CX\to X$ being identity morphism.
    \end{itemize}
\end{definition}

\begin{proposition}
    Co-Kleisli Category is indeed a category.
\end{proposition}
\begin{proof}
    There are 2 things we have to proof here: the composition of the identity morphism and the associativity of the composition. That are:

    \textbf{(Identity Composition):} Given $k:CX\to Y$ and $\varepsilon_Y:CY\to Y$, then we have the following composition on the LHS diagram, where we have used the naturality of $\varepsilon$. We can see here, by left co-unitality law i.e $\varepsilon_{CX}\circ\nu_X=\operatorname{id}_{CX}$, thus we have only $k$ left:
    \begin{equation*}
    % https://q.uiver.app/#q=WzAsNSxbMCwxLCJDWCJdLFsxLDEsIkNDWCJdLFsyLDEsIkNZIl0sWzIsMCwiWSJdLFsxLDAsIkNYIl0sWzIsMywiXFx2YXJlcHNpbG9uX1kiLDIseyJjb2xvdXIiOlswLDYwLDYwXX0sWzAsNjAsNjAsMV1dLFsxLDIsIkNrIiwyLHsiY29sb3VyIjpbMCw2MCw2MF19LFswLDYwLDYwLDFdXSxbMCwxLCJcXG51X1giLDIseyJjb2xvdXIiOlswLDYwLDYwXX0sWzAsNjAsNjAsMV1dLFsxLDQsIlxcdmFyZXBzaWxvbl97Q1h9Il0sWzQsMywiayJdLFswLDQsIlxcb3BlcmF0b3JuYW1le2lkfV97Q1h9IiwwLHsiY3VydmUiOi0yLCJzdHlsZSI6eyJib2R5Ijp7Im5hbWUiOiJkb3R0ZWQifX19XV0=
    \begin{tikzcd}
        & CX & Y \\
        CX & CCX & CY
        \arrow["k", from=1-2, to=1-3]
        \arrow["{\operatorname{id}_{CX}}", curve={height=-12pt}, dotted, from=2-1, to=1-2]
        \arrow["{\nu_X}"', color={rgb,255:red,214;green,92;blue,92}, from=2-1, to=2-2]
        \arrow["{\varepsilon_{CX}}", from=2-2, to=1-2]
        \arrow["Ck"', color={rgb,255:red,214;green,92;blue,92}, from=2-2, to=2-3]
        \arrow["{\varepsilon_Y}"', color={rgb,255:red,214;green,92;blue,92}, from=2-3, to=1-3]
    \end{tikzcd}
    \qquad
    % https://q.uiver.app/#q=WzAsNCxbMCwwLCJDWCJdLFsxLDAsIkNDWCJdLFsyLDAsIkNYIl0sWzMsMCwiWSJdLFswLDEsIlxcbnVfWCIsMl0sWzIsMywiayIsMl0sWzEsMiwiQ1xcdmFyZXBzaWxvbl9YIiwyXSxbMCwyLCJcXG9wZXJhdG9ybmFtZXtpZH1fe0NYfSIsMCx7ImN1cnZlIjotMiwic3R5bGUiOnsiYm9keSI6eyJuYW1lIjoiZG90dGVkIn19fV1d
    \begin{tikzcd}
        CX & CCX & CX & Y
        \arrow["{\nu_X}"', from=1-1, to=1-2]
        \arrow["{\operatorname{id}_{CX}}", curve={height=-12pt}, dotted, from=1-1, to=1-3]
        \arrow["{C\varepsilon_X}"', from=1-2, to=1-3]
        \arrow["k"', from=1-3, to=1-4]
    \end{tikzcd}
    \end{equation*}
    On the other hand, if we compute $k\circ\varepsilon_X$, we can use the co-right unitality law i.e $C\varepsilon_X\circ\nu_X=\operatorname{id}_{CX}$ so that we are left with $k$, as needed.

    \textbf{(Commutativity):} Now, suppose we are given $a:CX\to Y, b:CY\to Z$ and $c:CZ\to W$, then we consider the composition of $c\circ_{ck}(b\circ_{ck}a)$ (represented in red path) and $(c\circ_{ck}b)\circ_{ck}a$ (represented in blue part), as:
    \begin{equation*}
    % https://q.uiver.app/#q=WzAsOCxbMCwwLCJDQ1giXSxbMSwwLCJDQ0NYIl0sWzIsMCwiQ0NZIl0sWzMsMCwiQ1oiXSxbMiwxLCJDWSJdLFswLDEsIkNYIl0sWzQsMCwiVyJdLFsxLDEsIkNDWCJdLFsxLDIsIkNDYSIsMCx7ImNvbG91ciI6WzAsNjAsNjBdfSxbMCw2MCw2MCwxXV0sWzAsMSwiQ1xcbnVfWCIsMCx7ImNvbG91ciI6WzAsNjAsNjBdfSxbMCw2MCw2MCwxXV0sWzIsMywiQ2IiLDAseyJjb2xvdXIiOlsyNzAsNjAsNjBdfSxbMjcwLDYwLDYwLDFdXSxbNSwwLCJcXG51X1giLDAseyJjb2xvdXIiOlswLDYwLDYwXX0sWzAsNjAsNjAsMV1dLFszLDYsImMiLDAseyJjb2xvdXIiOlsyNzAsNjAsNjBdfSxbMjcwLDYwLDYwLDFdXSxbNyw0LCJDYSIsMix7ImNvbG91ciI6WzI0MCw2MCw2MF19LFsyNDAsNjAsNjAsMV1dLFs1LDcsIlxcbnVfWCIsMix7ImNvbG91ciI6WzI0MCw2MCw2MF19LFsyNDAsNjAsNjAsMV1dLFs0LDIsIlxcbnVfWSIsMix7ImNvbG91ciI6WzI0MCw2MCw2MF19LFsyNDAsNjAsNjAsMV1dLFs3LDEsIlxcbnVfe0NYfSJdXQ==
    \begin{tikzcd}
        CCX & CCCX & CCY & CZ & W \\
        CX & CCX & CY
        \arrow["{C\nu_X}", color={rgb,255:red,214;green,92;blue,92}, from=1-1, to=1-2]
        \arrow["CCa", color={rgb,255:red,214;green,92;blue,92}, from=1-2, to=1-3]
        \arrow["Cb", color={rgb,255:red,153;green,92;blue,214}, from=1-3, to=1-4]
        \arrow["c", color={rgb,255:red,153;green,92;blue,214}, from=1-4, to=1-5]
        \arrow["{\nu_X}", color={rgb,255:red,214;green,92;blue,92}, from=2-1, to=1-1]
        \arrow["{\nu_X}"', color={rgb,255:red,92;green,92;blue,214}, from=2-1, to=2-2]
        \arrow["{\nu_{CX}}", from=2-2, to=1-2]
        \arrow["Ca"', color={rgb,255:red,92;green,92;blue,214}, from=2-2, to=2-3]
        \arrow["{\nu_Y}"', color={rgb,255:red,92;green,92;blue,214}, from=2-3, to=1-3]
    \end{tikzcd}
    \end{equation*}
    We note that the LHS square comes from the co-commutativity law, while the RHS square is the naturality condition of $\nu$. Since every thing commutes, we can say that both compositions are equal to each other.
\end{proof}

\begin{definition}{\textbf{(co-Kleisli Adjunction)}}
    Given a comonad $(C,\varepsilon,v)$ on category $\textbf{C}$, we have the following construction:
    \begin{itemize}
        \item Right Adjointion: $R_C:\textbf{C}\rightarrow\textbf{C}_C$: (Object) Given the object $X$, then $R_CX=X$ (Morphism) Given the morphism $f:X\rightarrow Y$, then $R_Cf=f\circ\varepsilon_X$ where $R_Cf:CX\rightarrow Y$
        \item Left Adjointion: $L_C:\textbf{C}_C\to\textbf{C}$: (Object) Given the object $X$, then $L_CX=CX$ (Morphism) Given the co-Kleisli morphism $k:CX\rightarrow Y$, then $L_Ck=Ck\circ \nu_X$ where $L_Cf:CX\rightarrow CY$
    \end{itemize}
\end{definition}

\begin{proposition}
    Both $R_C$ and $L_C$ are functors
\end{proposition}
\begin{proof}
    \textbf{(Right Adjunction):} Its action on identity is clear, as $R_C\operatorname{id}_X=\varepsilon_X$, which is an identities in co-Kleisli adjunction. Now, we are left to show that the functors preseves composition, in which, suppose we have $f:X\to Y$ and $g:Y\to Z$, then:
    \begin{equation*}
    % https://q.uiver.app/#q=WzAsNyxbMCwwLCJDWCJdLFsxLDAsIkNDWCJdLFsyLDAsIkNYIl0sWzIsMSwiQ1kiXSxbMywxLCJZIl0sWzQsMSwiWiJdLFszLDAsIlgiXSxbMCwxLCJcXG51X1giLDAseyJjb2xvdXIiOlswLDYwLDYwXX0sWzAsNjAsNjAsMV1dLFsxLDIsIkNcXHZhcmVwc2lsb25fWCIsMCx7ImNvbG91ciI6WzAsNjAsNjBdfSxbMCw2MCw2MCwxXV0sWzIsMywiQ2YiLDIseyJjb2xvdXIiOlswLDYwLDYwXX0sWzAsNjAsNjAsMV1dLFszLDQsIlxcdmFyZXBzaWxvbl9ZIiwyLHsiY29sb3VyIjpbMCw2MCw2MF19LFswLDYwLDYwLDFdXSxbNCw1LCJnIiwyLHsiY29sb3VyIjpbMCw2MCw2MF19LFswLDYwLDYwLDFdXSxbNiw0LCJmIl0sWzIsNiwiXFx2YXJlcHNpbG9uX1giXSxbMCwyLCJcXG9wZXJhdG9ybmFtZXtpZH1fWCIsMix7ImN1cnZlIjoyLCJzdHlsZSI6eyJib2R5Ijp7Im5hbWUiOiJkb3R0ZWQifX19XV0=
    \begin{tikzcd}
        CX & CCX & CX & X \\
        && CY & Y & Z
        \arrow["{\nu_X}", color={rgb,255:red,214;green,92;blue,92}, from=1-1, to=1-2]
        \arrow["{\operatorname{id}_X}"', curve={height=12pt}, dotted, from=1-1, to=1-3]
        \arrow["{C\varepsilon_X}", color={rgb,255:red,214;green,92;blue,92}, from=1-2, to=1-3]
        \arrow["{\varepsilon_X}", from=1-3, to=1-4]
        \arrow["Cf"', color={rgb,255:red,214;green,92;blue,92}, from=1-3, to=2-3]
        \arrow["f", from=1-4, to=2-4]
        \arrow["{\varepsilon_Y}"', color={rgb,255:red,214;green,92;blue,92}, from=2-3, to=2-4]
        \arrow["g"', color={rgb,255:red,214;green,92;blue,92}, from=2-4, to=2-5]
    \end{tikzcd}
    \end{equation*}
    Note that the red path is given to be $R_Cg\circ_{ck} R_Cf = (g\circ\varepsilon_Y)\circ_{ck}(f\circ\varepsilon_X)$, by the co-right unitality law and the naturality condition of $\varepsilon$.

    \textbf{(Left Adjunction):} Given an identity on $\textbf{C}_C$ that is $\varepsilon$, we can see that $L_C\nu_X=C\varepsilon_X\circ\nu_X=\operatorname{id}_{CX}$ by the co-left unitality law, as needed. On the other hand, let's consider the composition, suppose we are given $f:CX\to Y$ and $g:CY\to Z$, then we have:
    \begin{equation*}
    % https://q.uiver.app/#q=WzAsNixbMiwxLCJDQ1kiXSxbMywxLCJDWiJdLFsyLDAsIkNZIl0sWzEsMCwiQ0NYIl0sWzAsMCwiQ1giXSxbMSwxLCJDQ0NYIl0sWzAsMSwiQ2ciXSxbMiwwLCJcXG51X1kiLDAseyJjb2xvdXIiOlsyNDAsNjAsNjBdfSxbMjQwLDYwLDYwLDFdXSxbMywyLCJDZiIsMCx7ImNvbG91ciI6WzI0MCw2MCw2MF19LFsyNDAsNjAsNjAsMV1dLFs0LDMsIlxcbnVfWCJdLFs1LDAsIkNDZiIsMix7ImNvbG91ciI6WzAsNjAsNjBdfSxbMCw2MCw2MCwxXV0sWzMsNSwiQ1xcbnVfWCIsMix7ImNvbG91ciI6WzAsNjAsNjBdfSxbMCw2MCw2MCwxXV1d
    \begin{tikzcd}
        CX & CCX & CY \\
        & CCCX & CCY & CZ
        \arrow["{\nu_X}", from=1-1, to=1-2]
        \arrow["Cf", color={rgb,255:red,92;green,92;blue,214}, from=1-2, to=1-3]
        \arrow["{C\nu_X}"', color={rgb,255:red,214;green,92;blue,92}, from=1-2, to=2-2]
        \arrow["{\nu_Y}", color={rgb,255:red,92;green,92;blue,214}, from=1-3, to=2-3]
        \arrow["CCf"', color={rgb,255:red,214;green,92;blue,92}, from=2-2, to=2-3]
        \arrow["Cg", from=2-3, to=2-4]
    \end{tikzcd}
    \end{equation*}
    Using the naturality condition of $\nu_X$, we can see that the blue path represented the composition $L_Cg\circ L_Cf = (Cg\circ\nu_Y)\circ(Cf\circ\nu_X)$, while $L_C(g\circ_{ck}f)$ is represented in the red path. Sicne the diagram commutes, they are all equal, as needed
\end{proof}

\begin{proposition}
    This is a dual statement to \ref{prop:adjointion-kleisli}, where both functors above are indeed adjunction such that (1) $L_C\circ R_C=C$ (2) $L_C$ is left-adjoint to $R_C$ (3) The counit $\varepsilon$ of comomad is the counit of the adjunction.
\end{proposition}
\begin{proof}
    \textbf{(Part 1):} Consider the action of $L_C\circ R_C$ on object $X$, we have that $L_C\circ R_CX=L_CX=X$. On the other hand, given morphism $f:X\to Y$, then we have $L_C\circ R_Cf=L_C(f\circ\varepsilon_X)=C(f\circ\varepsilon_X)\circ\nu_X=Cf$ where we have used co-right unitality.

    \textbf{(Part 2):} We want to show that there is natual isomorphism between: $\operatorname{Hom}_{\textbf{C}}(L_CX, Y) \cong \operatorname{Hom}_{\textbf{C}_T}(X, R_CY)$. Please note that function on both sides has the signature of $CX\to Y$ in $\textbf{C}$, so we can consider the relation between them to be identity. However, we are left to show that they are natural under both argument. Starting with $X$, given $f^\text{op}:CX\to X'$ being a morphism in $\textbf{C}_C$, we want to show that:
    \begin{equation*}
    % https://q.uiver.app/#q=WzAsNCxbMCwwLCIgICAgICAgIFxcb3BlcmF0b3JuYW1le0hvbX1fe1xcdGV4dGJme0N9fShMX0NYLCBZKSJdLFsyLDEsIlxcb3BlcmF0b3JuYW1le0hvbX1fe1xcdGV4dGJme0N9X1R9KFgnLCBSX0NZKSJdLFsyLDAsIiAgICAgICAgXFxvcGVyYXRvcm5hbWV7SG9tfV97XFx0ZXh0YmZ7Q319KExfQ1gnLCBZKSJdLFswLDEsIlxcb3BlcmF0b3JuYW1le0hvbX1fe1xcdGV4dGJme0N9X1R9KFgsIFJfQ1kpIl0sWzAsMywiXFxjb25nIiwyXSxbMiwxLCJcXGNvbmciXSxbMCwyLCItXFxjaXJjIExfQ2YiXSxbMywxLCItXFxjaXJjX3tja30gZiIsMl1d
    \begin{tikzcd}
        {        \operatorname{Hom}_{\textbf{C}}(L_CX, Y)} && {        \operatorname{Hom}_{\textbf{C}}(L_CX', Y)} \\
        {\operatorname{Hom}_{\textbf{C}_T}(X, R_CY)} && {\operatorname{Hom}_{\textbf{C}_T}(X', R_CY)}
        \arrow["{-\circ L_Cf}", from=1-1, to=1-3]
        \arrow["\cong"', from=1-1, to=2-1]
        \arrow["\cong", from=1-3, to=2-3]
        \arrow["{-\circ_{ck} f}"', from=2-1, to=2-3]
    \end{tikzcd}
    \end{equation*}
    Note that $\operatorname{Hom}_{\textbf{C}}(L_C-, Y)[f]=-\circ L_Cf$ and similarly, $\operatorname{Hom}_{\textbf{C}_T}(-, R_CY)[f]=-\circ_{ck} f$. Then, we see that, given $g:CX\to Y$, we will have that: $g\circ L_Cf = g\circ Cf\circ\nu_X = g\circ_{ck}f$, thus the diagram commutes. On the other hand, we consider the LHS diagram, given $g:Y\to Z$
    \begin{equation*}
    % https://q.uiver.app/#q=WzAsNCxbMCwwLCIgICAgICAgIFxcb3BlcmF0b3JuYW1le0hvbX1fe1xcdGV4dGJme0N9fShMX0NYLCBZKSJdLFsyLDEsIlxcb3BlcmF0b3JuYW1le0hvbX1fe1xcdGV4dGJme0N9X1R9KFgsIFJfQ1knKSJdLFsyLDAsIiAgICAgICAgXFxvcGVyYXRvcm5hbWV7SG9tfV97XFx0ZXh0YmZ7Q319KExfQ1gsIFknKSJdLFswLDEsIlxcb3BlcmF0b3JuYW1le0hvbX1fe1xcdGV4dGJme0N9X1R9KFgsIFJfQ1kpIl0sWzAsMywiXFxjb25nIiwyXSxbMiwxLCJcXGNvbmciXSxbMCwyLCJnXFxjaXJjIC0iXSxbMywxLCJSX0NnXFxjaXJjX3tja30tIiwyXV0=
    \begin{tikzcd}
        {        \operatorname{Hom}_{\textbf{C}}(L_CX, Y)} && {        \operatorname{Hom}_{\textbf{C}}(L_CX, Y')} \\
        {\operatorname{Hom}_{\textbf{C}_T}(X, R_CY)} && {\operatorname{Hom}_{\textbf{C}_T}(X, R_CY')}
        \arrow["{g\circ -}", from=1-1, to=1-3]
        \arrow["\cong"', from=1-1, to=2-1]
        \arrow["\cong", from=1-3, to=2-3]
        \arrow["{R_Cg\circ_{ck}-}"', from=2-1, to=2-3]
    \end{tikzcd}
    \qquad
    % https://q.uiver.app/#q=WzAsNixbMCwxLCJDWCJdLFsxLDEsIkNDWCJdLFsyLDEsIkNZIl0sWzIsMCwiWSJdLFszLDAsIloiXSxbMSwwLCJDWCJdLFswLDEsIlxcbnVfWCIsMix7ImNvbG91ciI6WzAsNjAsNjBdfSxbMCw2MCw2MCwxXV0sWzEsMiwiQ2siLDIseyJjb2xvdXIiOlswLDYwLDYwXX0sWzAsNjAsNjAsMV1dLFsyLDMsIlxcdmFyZXBzaWxvbl9ZIiwyLHsiY29sb3VyIjpbMCw2MCw2MF19LFswLDYwLDYwLDFdXSxbMyw0LCJnIiwwLHsiY29sb3VyIjpbMCw2MCw2MF19LFswLDYwLDYwLDFdXSxbMSw1LCJcXHZhcmVwc2lsb25fe0NYfSJdLFs1LDMsImsiXSxbMCw1LCJcXG9wZXJhdG9ybmFtZXtpZH1fe0NYfSIsMCx7ImN1cnZlIjotMiwic3R5bGUiOnsiYm9keSI6eyJuYW1lIjoiZG90dGVkIn19fV1d
    \begin{tikzcd}
        & CX & Y & Z \\
        CX & CCX & CY
        \arrow["k", from=1-2, to=1-3]
        \arrow["g", color={rgb,255:red,214;green,92;blue,92}, from=1-3, to=1-4]
        \arrow["{\operatorname{id}_{CX}}", curve={height=-12pt}, dotted, from=2-1, to=1-2]
        \arrow["{\nu_X}"', color={rgb,255:red,214;green,92;blue,92}, from=2-1, to=2-2]
        \arrow["{\varepsilon_{CX}}", from=2-2, to=1-2]
        \arrow["Ck"', color={rgb,255:red,214;green,92;blue,92}, from=2-2, to=2-3]
        \arrow["{\varepsilon_Y}"', color={rgb,255:red,214;green,92;blue,92}, from=2-3, to=1-3]
    \end{tikzcd}
    \end{equation*}
    where we note that: $\operatorname{Hom}_{\textbf{C}_T}(X, R_C-)[g]=R_Cg\circ_{ck}-$. Given $k:CX\to Y$, then we see that, for $R_Cg\circ_{ck}\circ k$ is represented on the red path of the RHS diagram. Which by the co-left unitality law gives the identity. So $R_Cg\circ_{ck}\circ k=g\circ k$, which makes the LHS diagram commutes.

    So we have showed that $L_C$ is left adjoint to $R_C$, then to consider the counit, which is $(\operatorname{id}_{R_TX})^\sharp$ where the identities is on the $\textbf{C}_T$ which is the $\varepsilon_X$ in $\textbf{C}$, and since the isomorphism is identity, then, we have that $(\operatorname{id}_{R_TX})^\sharp=\varepsilon_X$, as needed.
\end{proof}

\begin{definition}{\textbf{(Eilenberg-Moore Coalgebra)}}
    Given a comonad $(C,\varepsilon,v)$ on category $\textbf{C}$, we define algebra of $C$ or $C$-coalgebra to be a pair $(A, i_A)$ where $A$ is an object of $\textbf{C}$ and a morphism $i_A:A\to CA$ in $\textbf{C}$ that satisfies the counit and comultiplication or coalgebra square law:
    \begin{equation*}
    % https://q.uiver.app/#q=WzAsNyxbMCwwLCJBIl0sWzEsMCwiQ0EiXSxbMSwxLCJBIl0sWzMsMCwiQSJdLFs0LDAsIkNBIl0sWzMsMSwiQ0EiXSxbNCwxLCJDQ0EiXSxbMCwxLCJpIl0sWzEsMiwiXFx2YXJlcHNpbG9uX0EiXSxbMCwyLCJcXG9wZXJhdG9ybmFtZXtpZH1fQSIsMl0sWzUsNiwiQ2kiLDJdLFs0LDYsIlxcbnVfQSJdLFszLDUsImkiLDJdLFszLDQsImkiXV0=
    \begin{tikzcd}
        A & CA && A & CA \\
        & A && CA & CCA
        \arrow["i", from=1-1, to=1-2]
        \arrow["{\operatorname{id}_A}"', from=1-1, to=2-2]
        \arrow["{\varepsilon_A}", from=1-2, to=2-2]
        \arrow["i", from=1-4, to=1-5]
        \arrow["i"', from=1-4, to=2-4]
        \arrow["{\nu_A}", from=1-5, to=2-5]
        \arrow["Ci"', from=2-4, to=2-5]
    \end{tikzcd}
    \end{equation*}
\end{definition}

\begin{definition}{\textbf{(Eilenberg-Moore Category of Coalgebra)}}
    \label{def:coalg-cat}
    Given a comonad $(C,\varepsilon,\nu)$ on category $\textbf{C}$ (denoted as $\textbf{C}^C$) to have objects being $(A, i_A)$ for each $A$ in $\textbf{C}$ with suitable $i_A:TA\to A$. The $C$-morphism $f:(A, i_A)\to (B, i_B)$ is an morphism $f:A\to B$ such that:
    \begin{equation*}
    % https://q.uiver.app/#q=WzAsNCxbMCwwLCJBIl0sWzEsMCwiQiJdLFswLDEsIkNBIl0sWzEsMSwiQ0IiXSxbMCwxLCJmIl0sWzAsMiwiaV9BIiwyXSxbMiwzLCJDZiIsMl0sWzEsMywiaV9CIl1d
    \begin{tikzcd}
        A & B \\
        CA & CB
        \arrow["f", from=1-1, to=1-2]
        \arrow["{i_A}"', from=1-1, to=2-1]
        \arrow["{i_B}", from=1-2, to=2-2]
        \arrow["Cf"', from=2-1, to=2-2]
    \end{tikzcd}
    \end{equation*}
    Clearly the identity morphism $\operatorname{id}_{(A, i_A)}$ is clearly $\operatorname{id}_A$, and the associativity of composition is clearly inherted from the category $\textbf{C}$ (and the commutativity is still valid). Observe that $i_A$ is an $C$-morphism of $(A, i_A)\to(CA,\nu_A)$ as the comultiplication law guarantee the commutativity.
\end{definition}

\begin{definition}{\textbf{(Free Coalgebra)}}
    Given object $X$ in category $\textbf{C}$, the $C$-coalgebra of $(CX, \nu_X)$ of $\textbf{C}$ is called free $C$-coalgebra.
\end{definition}

\begin{remark}{(Free Coalgebra is Algebra)}
    This can be checked using the counit and comultiplication, in which the following 2 left diagram commutes (follows from co-left unitality law and co-commutativity law):
    \begin{equation*}
    % https://q.uiver.app/#q=WzAsMTEsWzAsMCwiQ1giXSxbMSwwLCJDQ1giXSxbMSwxLCJDWCJdLFszLDAsIkNYIl0sWzQsMCwiQ0NYIl0sWzMsMSwiQ0NYIl0sWzQsMSwiQ0NDWCJdLFs2LDAsIkNYIl0sWzcsMCwiQ1kiXSxbNiwxLCJDQ1giXSxbNywxLCJDQ1kiXSxbMSwyLCJcXHZhcmVwc2lsb25fe0NYfSJdLFswLDEsIlxcbnVfWCJdLFswLDIsIlxcb3BlcmF0b3JuYW1le2lkfV97Q1h9IiwyXSxbMyw0LCJcXG51X1giXSxbMyw1LCJcXG51X1giLDJdLFs1LDYsIkNcXG51X1giLDJdLFs0LDYsIlxcbnVfe0NYfSJdLFs3LDksIlxcbnVfWCIsMl0sWzgsMTAsIlxcbnVfWSJdLFs5LDEwLCJDQ2YiLDJdLFs3LDgsIkNmIl1d
    \begin{tikzcd}
        CX & CCX && CX & CCX && CX & CY \\
        & CX && CCX & CCCX && CCX & CCY
        \arrow["{\nu_X}", from=1-1, to=1-2]
        \arrow["{\operatorname{id}_{CX}}"', from=1-1, to=2-2]
        \arrow["{\varepsilon_{CX}}", from=1-2, to=2-2]
        \arrow["{\nu_X}", from=1-4, to=1-5]
        \arrow["{\nu_X}"', from=1-4, to=2-4]
        \arrow["{\nu_{CX}}", from=1-5, to=2-5]
        \arrow["Cf", from=1-7, to=1-8]
        \arrow["{\nu_X}"', from=1-7, to=2-7]
        \arrow["{\nu_Y}", from=1-8, to=2-8]
        \arrow["{C\nu_X}"', from=2-4, to=2-5]
        \arrow["CCf"', from=2-7, to=2-8]
    \end{tikzcd}
    \end{equation*}
    On the other hand, any function $f:A\to B$ is a $C$-morphism due to the naturality of $\nu$, as shown in the right most diagram.
\end{remark}

\begin{definition}{\textbf{(Co-Algebra Adjunction)}}
    Given comonad $(C, \varepsilon, v)$ on category $\textbf{C}$, then we can define the fully faithful ``forgetful'' functor $L^C:\textbf{C}^C\rightarrow \textbf{C}$, and the functor $R^C:\textbf{C}\rightarrow\textbf{C}^C$, in which: $R^CX=(CX,\nu_X)$ being free coalgebra. On the other hand, given morphism $f:X\to Y$, we lifted the morphism of $Cf:(CX,\nu_X)\to(CY, \nu_Y)$.
\end{definition}

\begin{proposition}
    Both functors are indeed adjunction. (1) The composition between functors $L^C\circ R^C:\textbf{C}\to\textbf{C}$ are naturality isomorphic to $C$. (2) The functor $L^C$ is left-adjoint to $R^C$ (3) unit of the adjunction is the structure map $i:R^C\circ L^C\Rightarrow\operatorname{id}_{\textbf{C}^C}$ (and it is valid $C$-morphism because of naturality condition of $\nu$), while its counit is counit of comonad given by $\varepsilon$.
\end{proposition}
\begin{proof}
    \textbf{(Part 1):} Let's consider $L^C\circ R^CC=L^C(CX,\nu_X)=CX$.  On the other hand, given $f:X\to Y$, then $L^C\circ R^Cf=L^CCf=Cf$. This mean that $L^C\circ R^C=C$.

    \textbf{(Part 2):} We want to show that:
    \begin{equation*}
        \operatorname{Hom}_{\textbf{C}}(A, B)=\operatorname{Hom}_{\textbf{C}}\big( L^C(A, i_A), B \big) \cong \operatorname{Hom}_{\textbf{C}^C}\big( (A, i_A), R^CB \big) = \operatorname{Hom}_{\textbf{C}^C}\big( (A, i_A), (CB, \nu_B) \big)
    \end{equation*}
    That is given $f^\sharp:A\to B$ and can send it to $f^\flat:(A, i_A)\to(CB, \nu_B)$, then we will define $R^Cf^\sharp\circ i_A=f^\flat$ where we have the signature of $R^Cf^\sharp:(CA, \nu_A)\to(CB, \nu_B)$ and $i_A:(A, i_A)\to (CA, \nu_A)$. On the other hand, we have $\varepsilon_B\circ L^Cf^\flat=f^\sharp$. Let's see that they are inverse of each other:
    \begin{itemize}
        \item Given $f:A\to B$ in $\textbf{C}$, then we have $R^Cf\circ i_A=f^\flat$ and then we have:
        \begin{equation*}
        \begin{aligned}
            (f^\flat)^\sharp=\varepsilon_B\circ L^C(R^Cf\circ i_A) &= \varepsilon_B\circ L^CR^Cf\circ L^Ci_A = \varepsilon_B\circ Cf\circ i_A \\
            &= f\circ\varepsilon_A\circ i_A = f
        \end{aligned}
        \end{equation*}
        where the second to last equation, we have used the naturality condition of $\varepsilon$ and the last equation, we have used the counit law of coalgebra.
        \item Given $g:(A, i_A)\to(CB, \nu_B)$, then we have $\varepsilon_B\circ L^Cg=g^\sharp$ and then we have: $(g^\sharp)^\flat = R^C(\varepsilon_B\circ L^Cg)\circ i_A$, which will be represented by the red path of following diagram:
        \begin{equation*}
        % https://q.uiver.app/#q=WzAsNSxbMCwwLCIoQSwgaV9BKSJdLFswLDEsIihDQSxcXG51X0EpIl0sWzEsMSwiKENDQixcXG51X0IpIl0sWzIsMSwiKENCLFxcbnVfQikiXSxbMSwwLCIoQ0IsIFxcbnVfQikiXSxbMCwxLCJpX0EiLDIseyJjb2xvdXIiOlswLDYwLDYwXX0sWzAsNjAsNjAsMV1dLFsxLDIsIkNnIiwyLHsiY29sb3VyIjpbMCw2MCw2MF19LFswLDYwLDYwLDFdXSxbMiwzLCJDXFx2YXJlcHNpbG9uX0IiLDIseyJjb2xvdXIiOlswLDYwLDYwXX0sWzAsNjAsNjAsMV1dLFswLDQsImciXSxbNCwyLCJcXG51X0IiXSxbNCwzLCJcXG9wZXJhdG9ybmFtZXtpZH1fe0NCfSIsMCx7ImN1cnZlIjotMiwic3R5bGUiOnsiYm9keSI6eyJuYW1lIjoiZG90dGVkIn19fV1d
        \begin{tikzcd}
            {(A, i_A)} & {(CB, \nu_B)} \\
            {(CA,\nu_A)} & {(CCB,\nu_B)} & {(CB,\nu_B)}
            \arrow["g", from=1-1, to=1-2]
            \arrow["{i_A}"', color={rgb,255:red,214;green,92;blue,92}, from=1-1, to=2-1]
            \arrow["{\nu_B}", from=1-2, to=2-2]
            \arrow["{\operatorname{id}_{CB}}", curve={height=-12pt}, dotted, from=1-2, to=2-3]
            \arrow["Cg"', color={rgb,255:red,214;green,92;blue,92}, from=2-1, to=2-2]
            \arrow["{C\varepsilon_B}"', color={rgb,255:red,214;green,92;blue,92}, from=2-2, to=2-3]
\end{tikzcd}
        \end{equation*}
        where we note that $R^CL^Cg=Cg$ (the changes were happening at the structure map) and $R^C\varepsilon_B=C\varepsilon_B$. We use the fact that $g$ is an $C$-morphism to get the square. And by the right co-unitality law, we recovers $g$, as needed.
    \end{itemize}
    Before we move on, we note that the structure map $i$ is an natual transformation because any morphism $f$ in $\textbf{C}^C$ is $C$-morphism so the naturality condition holds. Now, we have confirmed the unit and counit of adjunction, we are left to show that they satisfies triangle identities.  In which we have:
    \begin{equation*}
    % https://q.uiver.app/#q=WzAsNixbMCwwLCJMXkMiXSxbMSwwLCJMXkNSXkNMXkMiXSxbMiwwLCJMXkMiXSxbMCwxLCJBIl0sWzEsMSwiQSJdLFsyLDEsIkEiXSxbMCwxLCJMXkNpIiwwLHsibGV2ZWwiOjJ9XSxbMSwyLCJcXHZhcmVwc2lsb24gTF5DIiwwLHsibGV2ZWwiOjJ9XSxbMyw0LCJpX0EiXSxbNCw1LCJcXHZhcmVwc2lsb25fQSJdLFszLDUsIlxcb3BlcmF0b3JuYW1le2lkfV9BIiwyLHsiY3VydmUiOjMsInN0eWxlIjp7ImJvZHkiOnsibmFtZSI6ImRvdHRlZCJ9fX1dXQ==
    \begin{tikzcd}
        {L^C} & {L^CR^CL^C} & {L^C} \\
        A & A & A
        \arrow["{L^Ci}", Rightarrow, from=1-1, to=1-2]
        \arrow["{\varepsilon L^C}", Rightarrow, from=1-2, to=1-3]
        \arrow["{i_A}", from=2-1, to=2-2]
        \arrow["{\operatorname{id}_A}"', curve={height=18pt}, dotted, from=2-1, to=2-3]
        \arrow["{\varepsilon_A}", from=2-2, to=2-3]
    \end{tikzcd}
    \qquad \qquad
    % https://q.uiver.app/#q=WzAsNixbMCwwLCJSXkMiXSxbMSwwLCJSXkNMXkNSXkMiXSxbMiwwLCJSXkMiXSxbMCwxLCIoQ0EsIFxcbnVfQSkiXSxbMSwxLCIoQ0NBLCBcXG51X3tDQX0pIl0sWzIsMSwiKENBLFxcbnVfQSkiXSxbMCwxLCJpUl5DIiwwLHsibGV2ZWwiOjJ9XSxbMSwyLCJSXkNcXHZhcmVwc2lsb24iLDAseyJsZXZlbCI6Mn1dLFszLDQsIlxcbnVfe0F9IiwyXSxbNCw1LCJDXFx2YXJlcHNpbG9uX3tBfSIsMl0sWzMsNCwiaV97Q0F9Il0sWzMsNSwiXFxvcGVyYXRvcm5hbWV7aWR9X3tDQX0iLDIseyJjdXJ2ZSI6NCwic3R5bGUiOnsiYm9keSI6eyJuYW1lIjoiZG90dGVkIn19fV1d
    \begin{tikzcd}
        {R^C} & {R^CL^CR^C} & {R^C} \\
        {(CA, \nu_A)} & {(CCA, \nu_{CA})} & {(CA,\nu_A)}
        \arrow["{iR^C}", Rightarrow, from=1-1, to=1-2]
        \arrow["{R^C\varepsilon}", Rightarrow, from=1-2, to=1-3]
        \arrow["{\nu_{A}}"', from=2-1, to=2-2]
        \arrow["{i_{CA}}", from=2-1, to=2-2]
        \arrow["{\operatorname{id}_{CA}}"', curve={height=24pt}, dotted, from=2-1, to=2-3]
        \arrow["{C\varepsilon_{A}}"', from=2-2, to=2-3]
    \end{tikzcd}
    \end{equation*}
    where the LHS diagram, we consider the component of $(C, i_A)$ and use the counit law of coalgebra, while the RHS diagram we consider the component of $A$, which we have used the right counitality law. Therefore, the triangle identities is satisfied and so they are all adjunction.
\end{proof}

\begin{corollary}
    Given the comonad $(C, \varepsilon, v)$ of category $\textbf{C}$ with object $X$, and $(A, i_A)$ in $C$-coalgebra. Given any morphism $f:A\to X$, there exists an unique morphism of coalgebras $(A, i)\rightarrow(CX,v)$ such that following diagram commutes:
    \begin{equation*}
    % https://q.uiver.app/#q=WzAsMyxbMCwxLCJBIl0sWzEsMCwiQ1giXSxbMSwxLCJYIl0sWzEsMiwiXFx2YXJlcHNpbG9uX1giXSxbMCwyLCJmIiwyXSxbMCwxLCIiLDAseyJzdHlsZSI6eyJib2R5Ijp7Im5hbWUiOiJkYXNoZWQifX19XV0=
    \begin{tikzcd}
        & CX \\
        A & X
        \arrow["{\varepsilon_X}", from=1-2, to=2-2]
        \arrow[dashed, from=2-1, to=1-2]
        \arrow["f"', from=2-1, to=2-2]
    \end{tikzcd}
    \end{equation*}
\end{corollary}
\begin{proof}
    We note that the $f^\flat:(A,i_A)\to(CX,\nu_X)$, and we note that the diagram in remark \ref{remark:note-counit} gives (note that the pair $f^\flat$ and $f$ is unique by the isomorphism of adjunction), hence the uniqueness:
    \begin{equation*}
    % https://q.uiver.app/#q=WzAsMixbMCwwLCIoQSwgaV9BKSJdLFswLDEsIihDWCxcXG51X1gpIl0sWzAsMSwiZl5cXGZsYXQiLDIseyJzdHlsZSI6eyJib2R5Ijp7Im5hbWUiOiJkYXNoZWQifX19XV0=
    \begin{tikzcd}
        {(A, i_A)} \\
        {(CX,\nu_X)}
        \arrow["{f^\flat}"', dashed, from=1-1, to=2-1]
    \end{tikzcd}
    \longmapsto
    % https://q.uiver.app/#q=WzAsMyxbMCwwLCJMXkNBPUEiXSxbMCwxLCJMXkNSXkNYPUNYIl0sWzEsMSwiWCJdLFswLDEsIkxeQ2ZeXFxmbGF0IiwyLHsic3R5bGUiOnsiYm9keSI6eyJuYW1lIjoiZGFzaGVkIn19fV0sWzAsMiwiZiJdLFsxLDIsIlxcdmFyZXBzaWxvbl9YIiwyXV0=
    \begin{tikzcd}
        {L^CA=A} \\
        {L^CR^CX=CX} & X
        \arrow["{L^Cf^\flat}"', dashed, from=1-1, to=2-1]
        \arrow["f", from=1-1, to=2-2]
        \arrow["{\varepsilon_X}"', from=2-1, to=2-2]
    \end{tikzcd}
    \end{equation*}
\end{proof}

\subsection{Adjunction, Monads and Comonads}

\begin{theorem}
    \label{thm:adj-to-monad}
    Given a category $\textbf{C}$ and $\textbf{D}$ and the adjoint functors $F:\textbf{C}\to\textbf{D}$ and $G:\textbf{D}\to\textbf{C}$ in which $F\dashv G$ where the units are $\eta:\operatorname{id}_\textbf{C}\Rightarrow G\circ F$ and $\varepsilon:F\circ G\Rightarrow \operatorname{id}_\textbf{D}$, then:
    \begin{itemize}
        \item $G\circ F$ is monad on $\textbf{C}$ with unit $\eta$ and multiplication $G\varepsilon F$
        \item $F\circ G$ is comonad on $\textbf{D}$ with counit $\varepsilon$ and comultiplication $F\eta G$
    \end{itemize}
\end{theorem}
\begin{proof}
    \textbf{(Part 1):} It is clear that the unit and multiplication are natural transformation. We will show that they satisfies the law of monad, where we have, the left and right unitality:
    \begin{equation*}
    % https://q.uiver.app/#q=WzAsNixbMCwwLCJHRiJdLFsxLDAsIkdGR0YiXSxbMSwxLCJHRiJdLFszLDAsIkdGIl0sWzQsMCwiR0ZHRiJdLFs0LDEsIkdGIl0sWzAsMSwiXFxldGEgR0YiLDAseyJsZXZlbCI6Mn1dLFsxLDIsIkdcXHZhcmVwc2lsb24gRiIsMCx7ImxldmVsIjoyfV0sWzMsNCwiR0ZcXGV0YSIsMCx7ImxldmVsIjoyfV0sWzQsNSwiR1xcdmFyZXBzaWxvbiBGIiwwLHsibGV2ZWwiOjJ9XSxbMCwyLCJcXG9wZXJhdG9ybmFtZXtpZH1fe0dGfSIsMix7ImxldmVsIjoyfV0sWzMsNSwiXFxvcGVyYXRvcm5hbWV7aWR9X3tHRn0iLDIseyJsZXZlbCI6Mn1dXQ==
    \begin{tikzcd}
        GF & GFGF && GF & GFGF \\
        & GF &&& GF
        \arrow["{\eta GF}", Rightarrow, from=1-1, to=1-2]
        \arrow["{\operatorname{id}_{GF}}"', Rightarrow, from=1-1, to=2-2]
        \arrow["{G\varepsilon F}", Rightarrow, from=1-2, to=2-2]
        \arrow["GF\eta", Rightarrow, from=1-4, to=1-5]
        \arrow["{\operatorname{id}_{GF}}"', Rightarrow, from=1-4, to=2-5]
        \arrow["{G\varepsilon F}", Rightarrow, from=1-5, to=2-5]
    \end{tikzcd}
    \end{equation*}
    Both are results from triangle identities (lemma \ref{lemma:triangle-id-adjoint}). The LHS diagram commutes due to the second triangle identity if we pre-compose with a functor $F$ (i.e its components are of the for $FA$). The RHS diagram commutes due to the first triangle identities if we post-compose with functor $G$. For the multiplication, we have the following LHS diagram:
    \begin{equation*}
    % https://q.uiver.app/#q=WzAsOCxbMCwwLCJHRkdGR0YiXSxbMiwwLCJHRkdGIl0sWzAsMSwiR0ZHRiJdLFsyLDEsIkdGIl0sWzYsMCwiRkdGQSJdLFs2LDEsIkZBIl0sWzQsMSwiRkdGQSJdLFs0LDAsIkZHRkdGQSJdLFswLDEsIkdGR1xcdmFyZXBzaWxvbiBGIiwwLHsibGV2ZWwiOjJ9XSxbMCwyLCJHXFx2YXJlcHNpbG9uIEZHRiIsMix7ImxldmVsIjoyfV0sWzIsMywiR1xcdmFyZXBzaWxvbiBGIiwyLHsibGV2ZWwiOjJ9XSxbMSwzLCJHXFx2YXJlcHNpbG9uIEYiLDAseyJsZXZlbCI6Mn1dLFs0LDUsIlxcdmFyZXBzaWxvbl9BIl0sWzcsNiwiXFx2YXJlcHNpbG9uX3tGR0ZBfSIsMl0sWzcsNCwiRkdcXHZhcmVwc2lsb25fe0ZBfSJdLFs2LDUsIlxcdmFyZXBzaWxvbl97RkF9IiwyXV0=
    \begin{tikzcd}
        GFGFGF && GFGF && FGFGFA && FGFA \\
        GFGF && GF && FGFA && FA
        \arrow["{GFG\varepsilon F}", Rightarrow, from=1-1, to=1-3]
        \arrow["{G\varepsilon FGF}"', Rightarrow, from=1-1, to=2-1]
        \arrow["{G\varepsilon F}", Rightarrow, from=1-3, to=2-3]
        \arrow["{FG\varepsilon_{FA}}", from=1-5, to=1-7]
        \arrow["{\varepsilon_{FGFA}}"', from=1-5, to=2-5]
        \arrow["{\varepsilon_A}", from=1-7, to=2-7]
        \arrow["{G\varepsilon F}"', Rightarrow, from=2-1, to=2-3]
        \arrow["{\varepsilon_{FA}}"', from=2-5, to=2-7]
    \end{tikzcd}
    \end{equation*}
    This follows from the naturality of $\varepsilon$ when acting on $\varepsilon_{FA}$ for any object $A$, and then lift the whole diagram using functor $G$ as show in the RHS diagram.

    \textbf{(Part 2):} For the counit, we have similar proof as the left and right co-unitality law follows from the triangle identities, while the co-multiplication law follows from the naturality condition of $\eta$
\end{proof}

\begin{theorem}
    Given categories $\textbf{C}$ and $\textbf{D}$ and the adjoint functors $F:\textbf{C}\rightarrow\textbf{D}$ and $G:\textbf{D}\rightarrow\textbf{C}$ in which $F\dashv G$, where the unit $\eta$ and counit $\varepsilon$. Let's call the induced monad $G\circ F$ by $T$, then:
    \begin{itemize}
        \item There is canonical ``comparision'' functor $J$ from Kleisli category of $T$ to $\textbf{D}$, unique up to isomorphism, which makes the following diagram commutes (up to isomorphism):
        \begin{equation*}
        % https://q.uiver.app/#q=WzAsNixbMCwwLCJcXHRleHRiZntDfV9UIl0sWzIsMCwiXFx0ZXh0YmZ7RH0iXSxbMSwxLCJcXHRleHRiZntDfSJdLFs0LDAsIlxcdGV4dGJme0N9X1QiXSxbNiwwLCJcXHRleHRiZntEfSJdLFs1LDEsIlxcdGV4dGJme0N9Il0sWzAsMiwiUl9UIiwyXSxbMSwyLCJHIl0sWzAsMSwiSiIsMCx7InN0eWxlIjp7ImJvZHkiOnsibmFtZSI6ImRhc2hlZCJ9fX1dLFszLDQsIkoiLDAseyJzdHlsZSI6eyJib2R5Ijp7Im5hbWUiOiJkYXNoZWQifX19XSxbNSwzLCJMX1QiXSxbNSw0LCJGIiwyXV0=
        \begin{tikzcd}
            {\textbf{C}_T} && {\textbf{D}} && {\textbf{C}_T} && {\textbf{D}} \\
            & {\textbf{C}} &&&& {\textbf{C}}
            \arrow["J", dashed, from=1-1, to=1-3]
            \arrow["{R_T}"', from=1-1, to=2-2]
            \arrow["G", from=1-3, to=2-2]
            \arrow["J", dashed, from=1-5, to=1-7]
            \arrow["{L_T}", from=2-6, to=1-5]
            \arrow["F"', from=2-6, to=1-7]
        \end{tikzcd}
        \end{equation*}
        \item There is canonical ``comparision'' functor $K$ from $\textbf{D}$ to Eilenberg-Moore category of $T$, unique up to isomorphism, which makes the following diagram commutes (up to isomorphism):
        \begin{equation*}
        % https://q.uiver.app/#q=WzAsNixbMCwwLCJcXHRleHRiZntEfSJdLFsyLDAsIlxcdGV4dGJme0N9XlQiXSxbMSwxLCJcXHRleHRiZntDfSJdLFs0LDAsIlxcdGV4dGJme0R9Il0sWzYsMCwiXFx0ZXh0YmZ7Q31eVCJdLFs1LDEsIlxcdGV4dGJme0N9Il0sWzAsMiwiRyIsMl0sWzEsMiwiUl5UIl0sWzAsMSwiSyIsMCx7InN0eWxlIjp7ImJvZHkiOnsibmFtZSI6ImRhc2hlZCJ9fX1dLFszLDQsIksiLDAseyJzdHlsZSI6eyJib2R5Ijp7Im5hbWUiOiJkYXNoZWQifX19XSxbNSwzLCJGIl0sWzUsNCwiTF5UIiwyXV0=
        \begin{tikzcd}
            {\textbf{D}} && {\textbf{C}^T} && {\textbf{D}} && {\textbf{C}^T} \\
            & {\textbf{C}} &&&& {\textbf{C}}
            \arrow["K", dashed, from=1-1, to=1-3]
            \arrow["G"', from=1-1, to=2-2]
            \arrow["{R^T}", from=1-3, to=2-2]
            \arrow["K", dashed, from=1-5, to=1-7]
            \arrow["F", from=2-6, to=1-5]
            \arrow["{L^T}"', from=2-6, to=1-7]
        \end{tikzcd}
        \end{equation*}
    \end{itemize}
\end{theorem}
\begin{proof}
    \textbf{(Find Functor $J$ Definition):} Let's start with the first claim first.  Let's consider the $R_T:\textbf{C}_T\to\textbf{C}$ first. Given object $A$ of $C$, we have $R_TA=TA=GFA$, then we can expect $J$ to sends $JA=FA$. On the other hand, given $f:X\to GFY$ in $\textbf{C}_T$, then we have 
    \begin{equation*}
        R_Tf=\mu_Y\circ GFf=G\varepsilon_{FY}\circ GFf = G\big(\varepsilon_{FY}\circ Ff\big)
    \end{equation*}
    with this, we can see that $J$ lift the map $Jf = \varepsilon_{FY}\circ Ff=f^\sharp$. 
    
    \textbf{($J$ is a Functor):} Let's check that $J$ is ineeded a functor, first, consider the identities of $\textbf{C}^T$ on object $X$, which is $\eta_X$, in which, we have: $J\eta_X = \varepsilon_{FX}\circ F\eta_X = \operatorname{id}_{FX}$ because of the first triangle identities of adjunction. On the other hand, if we have $g:Y\to GFZ$, then we have the following diagram
    \begin{equation*}
    % https://q.uiver.app/#q=WzAsNyxbMCwwLCJGWCJdLFsxLDAsIkZHRlkiXSxbMiwwLCJGR0ZHRloiXSxbMywwLCJGR0ZaIl0sWzMsMSwiRloiXSxbMSwxLCJGWSJdLFsyLDEsIkZHRloiXSxbMCwxLCJGZiIsMCx7ImNvbG91ciI6WzI3MCw2MCw2MF19LFsyNzAsNjAsNjAsMV1dLFsxLDIsIkZHRmciLDAseyJjb2xvdXIiOlswLDYwLDYwXX0sWzAsNjAsNjAsMV1dLFsyLDMsIkZHXFx2YXJlcHNpbG9uX3tGWn0iLDAseyJjb2xvdXIiOlswLDYwLDYwXX0sWzAsNjAsNjAsMV1dLFszLDQsIlxcdmFyZXBzaWxvbl97Rlp9IiwwLHsiY29sb3VyIjpbMCw2MCw2MF19LFswLDYwLDYwLDFdXSxbNSw2LCJGZyIsMix7ImNvbG91ciI6WzI0MCw2MCw2MF19LFsyNDAsNjAsNjAsMV1dLFs2LDQsIlxcdmFyZXBzaWxvbl97Rlp9IiwyLHsiY29sb3VyIjpbMjQwLDYwLDYwXX0sWzI0MCw2MCw2MCwxXV0sWzEsNSwiXFx2YXJlcHNpbG9uX3tGWX0iLDIseyJjb2xvdXIiOlsyNDAsNjAsNjBdfSxbMjQwLDYwLDYwLDFdXSxbMiw2LCJcXHZhcmVwc2lsb25fe0ZHRlp9Il1d
    \begin{tikzcd}
        FX & FGFY & FGFGFZ & FGFZ \\
        & FY & FGFZ & FZ
        \arrow["Ff", color={rgb,255:red,153;green,92;blue,214}, from=1-1, to=1-2]
        \arrow["FGFg", color={rgb,255:red,214;green,92;blue,92}, from=1-2, to=1-3]
        \arrow["{\varepsilon_{FY}}"', color={rgb,255:red,92;green,92;blue,214}, from=1-2, to=2-2]
        \arrow["{FG\varepsilon_{FZ}}", color={rgb,255:red,214;green,92;blue,92}, from=1-3, to=1-4]
        \arrow["{\varepsilon_{FGFZ}}", from=1-3, to=2-3]
        \arrow["{\varepsilon_{FZ}}", color={rgb,255:red,214;green,92;blue,92}, from=1-4, to=2-4]
        \arrow["Fg"', color={rgb,255:red,92;green,92;blue,214}, from=2-2, to=2-3]
        \arrow["{\varepsilon_{FZ}}"', color={rgb,255:red,92;green,92;blue,214}, from=2-3, to=2-4]
    \end{tikzcd}
    \end{equation*}
    where the blue part represents $Jg\circ Jf$ on the red path represents $J(g\circ_{kl}f)$, where both square represents the naturality condition of $\varepsilon$ (the right square resemblance the associativity law of monad). 
    
    \textbf{($J$ gives the Commutativity):} Now we consider the RHS diagram, in which we want to show that $JL_T=F$. Let's start with the object, given $X$ in $\textbf{C}$, we have $JL_TX=JX=X$. On the other hand, given $f:X\to Y$:
    \begin{equation*}
        JL_Tf = J(\eta_Y\circ f) = \varepsilon_{FY}\circ F\eta_Y\circ Ff = Ff
    \end{equation*}
    The last equality follows from the triangle identity of an adjunction (the same as when we do $J\eta_Y$). \textit{Warning}, we can't just $J\eta_Y\circ Jf$ because $J$ is an functor on $\textbf{C}_T$ which distributed across $\circ_{kl}$ not $\circ$. Thus we see that $F=JL_T$ as needed.

    \textbf{(Find Functor $K$ Definition):} 
    Let's consider the object $D$ in $\textbf{D}$, then we have that $R^TKD=GD$, then we can sets $KD=(GD, e_{GD})$ as an object in $\textbf{C}^T$.  

    On the other hand, given $f:D\to D'$, then we have (since $R^T$ is a forgetful functor) $T$-morphism of $Kf:(GD, e_{GD})\to(GD',e_{GD'})$, let's consider the commutativity condition of $T$-morphism to indicate the structure map. We requires the following diagram to commute:
    \begin{equation*}
    % https://q.uiver.app/#q=WzAsNCxbMCwxLCJHRCJdLFsxLDEsIkdEJyJdLFswLDAsIkdGR0QiXSxbMSwwLCJHRkdEIl0sWzIsMCwiZV97R0R9IiwyXSxbMywxLCJlX3tHRCd9Il0sWzAsMSwiR2YiLDJdLFsyLDMsIkdGR2YiXV0=
    \begin{tikzcd}
        GFGD & GFGD \\
        GD & {GD'}
        \arrow["GFGf", from=1-1, to=1-2]
        \arrow["{e_{GD}}"', from=1-1, to=2-1]
        \arrow["{e_{GD'}}", from=1-2, to=2-2]
        \arrow["Gf"', from=2-1, to=2-2]
    \end{tikzcd}
    \end{equation*}
    We note that $e_{GD}:GFGD\to GD$ so the most natural way is to set it to be be $G\varepsilon_D=e_{GD}$. This would gives us the naturality condition for all $f$ in $\textbf{D}$. 
    
    \textbf{($K$ is a Functor):} It is clear that $K$ is a functor (as there is no fancy composition within $\textbf{C}^T$). 

    \textbf{($K$ gives the Commutativity):} We are left to show that this definition of $K$ satisfies the commutativity on the RHS. Let give object $X$ in $C$, then $KFX=(GFX, e_{GFX})$ where $e_{GFX}=G\varepsilon_{FX}=\mu_X$ as needed. On the other hand, given function $f:X\to Y$, we have $KF:(GFX, \mu_X)\to(GFY, \mu_Y)$ as needed (as the structure map is already been proven, we don't need anything).
\end{proof}

\begin{definition}{\textbf{(Monandic Functor)}}
    Given categories $\textbf{C}$ and $\textbf{D}$ and the adjoint functors $F:\textbf{C}\rightarrow\textbf{D}$ and $G:\textbf{D}\rightarrow\textbf{C}$ in which $F\dashv G$, the adjunction is called monadic iff the comparision functor $\textbf{D}\rightarrow\textbf{C}^T$ is an equivalence of categories.

    The right adjunction $G$, then is called monadic functor.
\end{definition}

\begin{remark}
    Now, given the special case of monadic functor (most of ``free-forgetful'' adjunction we have considered, but in most Kleisli adjunctions of most monads, will not be monadic), we can consier the comparision functor $J:\textbf{C}_T\to \textbf{C}^T$ so that the following diagram commutes:
    \begin{equation*}
    % https://q.uiver.app/#q=WzAsOCxbMCwwLCJcXHRleHRiZntDfV9UIl0sWzMsMCwiXFx0ZXh0YmZ7Q31eVCJdLFsxLDEsIlxcdGV4dGJme0N9Il0sWzUsMCwiXFx0ZXh0YmZ7Q31fVCJdLFs4LDAsIlxcdGV4dGJme0N9XlQiXSxbNiwxLCJcXHRleHRiZntDfSJdLFsyLDAsIlxcdGV4dGJme0R9Il0sWzcsMCwiXFx0ZXh0YmZ7RH0iXSxbMCwyLCJSX1QiLDJdLFsxLDIsIlJeVCJdLFs1LDMsIkxfVCJdLFs1LDQsIkxeVCIsMl0sWzAsNiwiSiIsMCx7InN0eWxlIjp7ImJvZHkiOnsibmFtZSI6ImRhc2hlZCJ9fX1dLFs2LDEsIksiXSxbNiwyLCJHIiwyXSxbMyw3LCJKIiwwLHsic3R5bGUiOnsiYm9keSI6eyJuYW1lIjoiZGFzaGVkIn19fV0sWzcsNCwiSyJdLFs1LDcsIkYiXV0=
    \begin{tikzcd}
        {\textbf{C}_T} && {\textbf{D}} & {\textbf{C}^T} && {\textbf{C}_T} && {\textbf{D}} & {\textbf{C}^T} \\
        & {\textbf{C}} &&&&& {\textbf{C}}
        \arrow["J", dashed, from=1-1, to=1-3]
        \arrow["{R_T}"', from=1-1, to=2-2]
        \arrow["K", from=1-3, to=1-4]
        \arrow["G"', from=1-3, to=2-2]
        \arrow["{R^T}", from=1-4, to=2-2]
        \arrow["J", dashed, from=1-6, to=1-8]
        \arrow["K", from=1-8, to=1-9]
        \arrow["{L_T}", from=2-7, to=1-6]
        \arrow["F", from=2-7, to=1-8]
        \arrow["{L^T}"', from=2-7, to=1-9]
    \end{tikzcd}
    \end{equation*}
\end{remark}

This examples/observation leads us to the relationship between the Kleisli category and the (subcategory of) Eilenberg-Moore category.

\begin{proposition}
    Given the above comparision functor $KJ:\textbf{C}_T\rightarrow \textbf{C}^T$, it establishes an equivalence between Kleisli category $\textbf{C}_T$ and full subcategory of $\textbf{C}^T$ whose object are free algebras.
\end{proposition}
\begin{proof}
    To show that $KJ$ is equivalence of categories we can show that it is fully faithful and essetially surjective, as given in theorem \ref{thm:full-faithful-essentially-surj-equiv}.

    \textbf{(Fully):} Given arrow $f:(TA, \mu_A)\to(TB, \mu_B)$, we want to show that there is a morphism $g:A\to TB$ in $\textbf{C}_T$ such that $KJg=f$. By the fact that $K$ is an equivalence of categories, there is a morphism $h$ such that $Kh=f$, that is $h=Jg$ but note that $Jg=g^\sharp$, and by the isomorphism of adjunction, there is such a morphism $g$ that make the equality.

    \textbf{(Faithful):} This also follows from the fact that $K$ is an equivalence of categories and $Jg=g^\sharp$ (as it has isomorphism between 2 hom-sets).

    \textbf{(Essetially Surjective):} For objects $(TA, \mu_A)$, there is an object $B$ in $\textbf{C}_T$ with an isomorphism $KJB\to(TA, \mu_A)$. Note that $KJB=K(FB)=(GFA,e_{GFB})=(TA,\mu_B)$, so that we can just set an object $B$ to $A$ with the isomorphism being identity. 

\end{proof}


\section{Ends and Coends}

\subsection{Profunctor}

\begin{definition}{\textbf{(Profunctor)}}
    Profunctor is a functor of the form $\textbf{C}^\text{op}\times\textbf{D}\to\textbf{E}$
\end{definition}

\begin{remark}
    Consider the case where $\textbf{E}$ is $\textbf{Set}$, it maps a pair of objects $\brackc{A, B}$ to a set $P\brackc{A, B}$ a pair of morphisms $\brackc{f:S\to A, g:B\to T}$ to a function $P\brackc{f,g}:P\brackc{A,B}\to\brackc{S, T}$.  And, we denote profunctor between two categories in the case where $\textbf{E}$ is $\textbf{Set}$ as $\textbf{C} \profunct \textbf{D}$
\end{remark}

\begin{remark}{(Profunctor and Hom-Functor)}
    The good model of profunctor is hom-functor, thus we can think of them as generalizing the hom-functor (hence the lifting of arrows is seen as generalized composition). It provides additional bridges between object.

    When performing a lifting of arrows individually, given a profunctor $P\brackc{A, B}$ with an arrow $f:S\to A$ to get $P\brackc{S, B}$, in which we have $P\brackc{f, \operatorname{id}_B}:P\brackc{A, B}\to\brackc{S, B}$. On the other hand, we can ``post-compose'' with $g:B\to T$ as we got $P\brackc{\operatorname{id}_A,g}:P\brackc{A,B}\to P\brackc{A, T}$.
\end{remark}


\begin{definition}{\textbf{(Collages)}}
    The collage (or co-graph) between two categories $\textbf{C}$ and $\textbf{D}$ is a category whose objects are from both category (disjoin union). The hom-set between two objects are either hom-set in $\textbf{C}$ if both are in $\textbf{C}$ (same for $\textbf{D}$), or $P\brackc{X, Y}$ if $X$ is in $\textbf{C}$ and $Y$ is in $\textbf{D}$. The composition when the morphism is profunctor is via lifting (note that profunctor is of one direction $\textbf{C} \profunct \textbf{D}$). 
\end{definition}

% Suppose that $P\langle A, B\rangle$ when composed with $P\langle B, C \rangle$ where $A$ and $C$ are objects of $\textbf{C}$, while $B$ is the object of $\textbf{D}$ is defined to be

\begin{proposition}
    Given the category of ``walking arrow'' of two objects and one non-trivial arrow (including two more identity arrows), defined as $\bullet_1\to\bullet_2$, then: 
    
    (1) There is a functor from collage of two categories $\textbf{C}$ and $\textbf{D}$ to this category. (2) On the other hand, if there is a functor from $\textbf{E}$ to the ``walking arrow'', then we can split it into a collage of two categories.
\end{proposition}
\begin{proof}
    
    \textbf{(Part 1):} We define the functor as follows: (1) Maps objects from $\textbf{C}$ to $\bullet_1$ and objects from $\textbf{D}$ to $\bullet_2$. (2) Maps morphism \textit{within} $\textbf{C}$ to $\operatorname{id}_{\bullet_1}$ and morphism \textit{within} $\textbf{D}$ to $\operatorname{id}_{\bullet_2}$. That is they behaves almost like a constant functor. For the profunctor $\textbf{C}\profunct\textbf{D}$, we maps them to $\bullet_1\to\bullet_2$. Please note that we can check that this is actually a functor, as it respect the composition and its action on identity. 

    \textbf{(Part 2):} We can reverse the process described in the first part. We only note that the arrow between $C\to D$ are actually a profunctor, which is due to the property of hom-functor, since in the original category $\textbf{E}$, pre-composing and post-composing induces the functor $\textbf{C}^\textbf{op}\times\textbf{D}\to\textbf{Set}$.
\end{proof}

\begin{remark}{(Profunctor as proof)}
    To understand the profunctor more, we can see that it is a a proof-relevant relation between objects. Consider the following interpretation:
    \begin{itemize}
        \item If the profunctor $P\brackc{A, B}$ is empty then $B$ isn't related to $A$. However, $P\brackc{A, A}$ \textbf{can be} empty and it doesn't have to be symmetric i.e we don't assume anything about this relation. 
        \item Furthermore, consider the lifting that is if $A$ is related to $B$ via $P$ i.e $P\brackc{A, B}\ne\emptyset$ and the hom-set $\operatorname{Hom}_\textbf{C}(S, A)$ and $\operatorname{Hom}_\textbf{D}(B, T)$ both aren't empty, then we can automatically related $S$ and $T$ via $P$. 
    \end{itemize}
    The hom-functor as it is a special case of pro-functor can be seen in this way, although $\operatorname{Hom}_\textbf{C}(A, A)$ isn't empty as it must contains at least the identity morphism.
    % Starting with a co-presheaf of $F:\textbf{C}\to\textbf{Set}$, we can consider the objects $A$ of $\textbf{C}$ whee $FA\ne\emptyset$, that is $A$ has a proof, which treated as an element of the set. 

\end{remark}

\subsection{Profunctor Composition + Co-Ends}

\begin{remark}{(Intuition about Profunctor Composition)}
    Consider the following cases: suppose you want to charge your cellphone, to find a charger you need to find your friend who has a charger \textbf{and} the charger need to be compatible with your phone. 
    
    Taking into profunctor vocabulary, we need 2 kinds of proof, proof of friendship and proof of possession of a charger, and both proof has to satisfies the same person. 
\end{remark}

\begin{remark}{(Double Counting in Profunctor Composition)}
    We are interested in profunctor composition, from the intuition above, we want to \textbf{sum} (this is the ``universal property'' that we will use as a template) over all objects $X$ in the product $P\brackc{A, X}\times Q\brackc{X, B}$. However, we can observe that a double counting can occurs:
    \begin{itemize}
        \item Consider the profunctor $Q\brackc{A, X}$ and $P\brackc{Y, B}$ with a morphism connecting between $X$ and $Y$ i.e $f:X\to Y$. (we distinguish between two objects to make sure they are clearly distinct).
        \item There are 2 ways in order to extends the profunctor: extending $Q$ to the right i.e $Q\brackc{\operatorname{id}_A, f}$ and use $y$ as a middle point, or extending $P$ to the left i.e $P\brackc{f,\operatorname{id}_B}$ and take $x$ as the intermediate node.
    \end{itemize}
    Therefore, we will have to force the sum to be on the diagonal and forces $X$ to be equal to $Y$.    
\end{remark}

\begin{definition}{\textbf{(Co-Wedge)}}
    Given a pro-functor of $P:\textbf{C}^\text{op}\times\textbf{C}\to\textbf{D}$, to do the sum of the diagonal, we can relies on the diagram of sum, which is given on the LHS (the dots between $P\brackc{Y, Y}$ and $P\brackc{X, X}$ represents the rest of the objects):
    \begin{equation*}
    % https://q.uiver.app/#q=WzAsNCxbMCwwLCJQXFxicmFja2N7WSwgWX0iXSxbMiwwLCJQXFxicmFja2N7WCwgWH0iXSxbMSwxLCJEIl0sWzEsMCwiXFxjZG90cyJdLFswLDIsImlfWSIsMl0sWzEsMiwiaV9YIl1d
    \begin{tikzcd}
        {P\brackc{Y, Y}} & \cdots & {P\brackc{X, X}} \\
        & D
        \arrow["{i_Y}"', from=1-1, to=2-2]
        \arrow["{i_X}", from=1-3, to=2-2]
    \end{tikzcd}
    \qquad \quad 
    % https://q.uiver.app/#q=WzAsNCxbMCwxLCJQXFxicmFja2N7WSwgWX0iXSxbMiwxLCJQXFxicmFja2N7WCwgWH0iXSxbMSwyLCJEIl0sWzEsMCwiUFxcYnJhY2tje1ksIFh9Il0sWzAsMiwiaV9ZIiwyXSxbMSwyLCJpX1giXSxbMywwLCJQXFxicmFja2N7XFxvcGVyYXRvcm5hbWV7aWR9X1ksZn0iLDJdLFszLDEsIlBcXGJyYWNrY3tmLFxcb3BlcmF0b3JuYW1le2lkfV9YfSJdXQ==
    \begin{tikzcd}
        & {P\brackc{Y, X}} \\
        {P\brackc{Y, Y}} && {P\brackc{X, X}} \\
        & D
        \arrow["{P\brackc{\operatorname{id}_Y,f}}"', from=1-2, to=2-1]
        \arrow["{P\brackc{f,\operatorname{id}_X}}", from=1-2, to=2-3]
        \arrow["{i_Y}"', from=2-1, to=3-2]
        \arrow["{i_X}", from=2-3, to=3-2]
    \end{tikzcd}
    \end{equation*}
    To restrict the results, since we have functor of 2 variables and varying across the objects, we can find a common ancestor between $P\brackc{Y, Y}$ and $P\brackc{X, X}$, assuming that we have the relating morphism $f:X\to Y$ with appropriate commutative condition. This RHS diagram here is called co-wedge and the commuting condition $i_X\circ P\brackc{f, \operatorname{id}_Y}=i_Y\circ\brackc{\operatorname{id}_X, f}$ is called co-wedge condition.
\end{definition}

\begin{definition}{\textbf{(Co-End)}}
    The universal co-wedge is called a co-end. This follows from the diagram below (Note that the integration represents infinite sum):
    \begin{equation*}
    % https://q.uiver.app/#q=WzAsNSxbMCwxLCJQXFxicmFja2N7WSwgWX0iXSxbMiwxLCJQXFxicmFja2N7WCwgWH0iXSxbMSwxLCJcXGludF5YUFxcYnJhY2tje1gsWH0iXSxbMSwwLCJQXFxicmFja2N7WSwgWH0iXSxbMSwyLCJEIl0sWzAsMiwiaV9ZIl0sWzEsMiwiaV9YIiwyXSxbMywwLCJQXFxicmFja2N7ZiwgXFxvcGVyYXRvcm5hbWV7aWR9X1l9IiwyXSxbMywxLCJQXFxicmFja2N7XFxvcGVyYXRvcm5hbWV7aWR9X1gsIGZ9Il0sWzIsNCwiaCIsMCx7InN0eWxlIjp7ImJvZHkiOnsibmFtZSI6ImRhc2hlZCJ9fX1dLFswLDQsImdfWSIsMV0sWzEsNCwiZ19YIiwxXV0=
    \begin{tikzcd}
        & {P\brackc{Y, X}} \\
        {P\brackc{Y, Y}} & {\int^XP\brackc{X,X}} & {P\brackc{X, X}} \\
        & D
        \arrow["{P\brackc{f, \operatorname{id}_Y}}"', from=1-2, to=2-1]
        \arrow["{P\brackc{\operatorname{id}_X, f}}", from=1-2, to=2-3]
        \arrow["{i_Y}", from=2-1, to=2-2]
        \arrow["{g_Y}"{description}, from=2-1, to=3-2]
        \arrow["h", dashed, from=2-2, to=3-2]
        \arrow["{i_X}"', from=2-3, to=2-2]
        \arrow["{g_X}"{description}, from=2-3, to=3-2]
    \end{tikzcd}
    \end{equation*}
    Note the co-ends have to works for all objects $X$ and $Y$ with morphism $f:X\to Y$. With the universal property, we define $\int^XP\brackc{X,X}\to D$ between co-end to some object $D$, by only define a family of functions from diagonal entries of the functor to $D$ i.e $g_X:P\brackc{X,X}\to D$ that satisfying the co-wedge condition.
\end{definition}

The notion of co-wedge, since it looks like a naturality condition, can be extended to the notion of extranatural transformation. The natural transformation follows from the relationship between functor; extranatural transformation is the relationship between 2 profunctors.

\begin{definition}{\textbf{(Extranatural Transformation)}}
    It is a family of arrows $\alpha_{CD}:P\brackc{C,C}\to Q\brackc{D, D}$ between two functors of the form $P:\textbf{C}^\text{op}\times\textbf{C}\to\textbf{E}$ and $Q:\textbf{D}^\text{op}\times\textbf{D}\to\textbf{E}$. The extranaturality in $C$ given $f:C\to C'$ is given the LHS diagram, and the extranaturality in $D$ given $g:D\to D'$ is given in RHS diagram:
    \begin{equation*}
    % https://q.uiver.app/#q=WzAsOCxbMSwwLCJQXFxicmFja2N7QycsIEN9Il0sWzEsMiwiUVxcYnJhY2tje0QsIER9Il0sWzAsMSwiUFxcYnJhY2tje0MnLCBDJ30iXSxbMiwxLCJQXFxicmFja2N7QywgQ30iXSxbNSwyLCJRXFxicmFja2N7RCwgRCd9Il0sWzQsMSwiUVxcYnJhY2tje0QsIER9Il0sWzYsMSwiUVxcYnJhY2tje0QnLCBEJ30iXSxbNSwwLCJQXFxicmFja2N7QywgQ30iXSxbMCwyLCJQXFxicmFja2N7XFxvcGVyYXRvcm5hbWV7aWR9X0MsIGZ9IiwyXSxbMCwzLCJQXFxicmFja2N7ZixcXG9wZXJhdG9ybmFtZXtpZH1fQ30iXSxbMiwxLCJcXGFscGhhX3tDJ0R9IiwyXSxbMywxLCJcXGFscGhhX3tDRH0iXSxbNyw1LCJcXGFscGhhX3tDRH0iLDJdLFs3LDYsIlxcYWxwaGFfe0NEJ30iXSxbNSw0LCJRXFxicmFja2N7XFxvcGVyYXRvcm5hbWV7aWR9X0QsIGd9IiwyXSxbNiw0LCJRXFxicmFja2N7ZywgXFxvcGVyYXRvcm5hbWV7aWR9X0R9Il1d
    \begin{tikzcd}
        & {P\brackc{C', C}} &&&& {P\brackc{C, C}} \\
        {P\brackc{C', C'}} && {P\brackc{C, C}} && {Q\brackc{D, D}} && {Q\brackc{D', D'}} \\
        & {Q\brackc{D, D}} &&&& {Q\brackc{D, D'}}
        \arrow["{P\brackc{\operatorname{id}_C, f}}"', from=1-2, to=2-1]
        \arrow["{P\brackc{f,\operatorname{id}_C}}", from=1-2, to=2-3]
        \arrow["{\alpha_{CD}}"', from=1-6, to=2-5]
        \arrow["{\alpha_{CD'}}", from=1-6, to=2-7]
        \arrow["{\alpha_{C'D}}"', from=2-1, to=3-2]
        \arrow["{\alpha_{CD}}", from=2-3, to=3-2]
        \arrow["{Q\brackc{\operatorname{id}_D, g}}"', from=2-5, to=3-6]
        \arrow["{Q\brackc{g, \operatorname{id}_D}}", from=2-7, to=3-6]
    \end{tikzcd}
    \end{equation*}
    The extranatural condition of $P$ and constant profunctor $\Delta_D$, defines co-wedge. For the co-end, we can have the component of $\alpha_X=h\circ i_X$ and same for other objects.
\end{definition}

\begin{definition}{\textbf{(Profunctor Composition)}}
    Given two pro-functors $P:\textbf{A}^\text{op}\times\textbf{C}\to\textbf{E}$ and $Q:\textbf{C}^\text{op}\times\textbf{A}\to\textbf{E}$, then their composition is given as:
    \begin{equation*}
        (P\diamond Q)\brackc{A, B} = \int^{X:\textbf{C}}P\brackc{A, X}\times Q\brackc{X, B}
    \end{equation*}
    Note that to proof the associativity, we have to relies on the Fubini rule.
\end{definition}

\begin{definition}{\textbf{(Wedge)}}
    By duality, instead of having the sum that represents the co-wedge, we can define the wedge via the product on the diagonal. Given a profunctor $P:\textbf{C}^\text{op}\times\textbf{C}\to\textbf{D}$, we can on the RHS the diagram for ``infinite'' product:
    \begin{equation*}
    % https://q.uiver.app/#q=WzAsNCxbMCwxLCJQXFxicmFja2N7WSwgWX0iXSxbMiwxLCJQXFxicmFja2N7WCwgWH0iXSxbMSwwLCJEIl0sWzEsMSwiXFxjZG90cyJdLFswLDIsIlxccGlfWSIsMCx7InN0eWxlIjp7InRhaWwiOnsibmFtZSI6ImFycm93aGVhZCJ9LCJoZWFkIjp7Im5hbWUiOiJub25lIn19fV0sWzEsMiwiXFxwaV9YIiwyLHsic3R5bGUiOnsidGFpbCI6eyJuYW1lIjoiYXJyb3doZWFkIn0sImhlYWQiOnsibmFtZSI6Im5vbmUifX19XV0=
    \begin{tikzcd}
        & D \\
        {P\brackc{Y, Y}} & \cdots & {P\brackc{X, X}}
        \arrow["{\pi_Y}", tail reversed, no head, from=2-1, to=1-2]
        \arrow["{\pi_X}"', tail reversed, no head, from=2-3, to=1-2]
    \end{tikzcd}
    \qquad \quad
    % https://q.uiver.app/#q=WzAsNCxbMCwxLCJQXFxicmFja2N7WSwgWX0iXSxbMiwxLCJQXFxicmFja2N7WCwgWH0iXSxbMSwwLCJEIl0sWzEsMiwiUFxcYnJhY2tje1gsIFl9Il0sWzAsMiwiXFxwaV9ZIiwwLHsic3R5bGUiOnsidGFpbCI6eyJuYW1lIjoiYXJyb3doZWFkIn0sImhlYWQiOnsibmFtZSI6Im5vbmUifX19XSxbMSwyLCJcXHBpX1giLDIseyJzdHlsZSI6eyJ0YWlsIjp7Im5hbWUiOiJhcnJvd2hlYWQifSwiaGVhZCI6eyJuYW1lIjoibm9uZSJ9fX1dLFswLDMsIlxcYnJhY2tje2YsIFxcb3BlcmF0b3JuYW1le2lkfV9ZfSIsMl0sWzEsMywiXFxicmFja2N7XFxvcGVyYXRvcm5hbWV7aWR9X1gsZn0iXV0=
    \begin{tikzcd}
        & D \\
        {P\brackc{Y, Y}} && {P\brackc{X, X}} \\
        & {P\brackc{X, Y}}
        \arrow["{\pi_Y}", tail reversed, no head, from=2-1, to=1-2]
        \arrow["{\brackc{f, \operatorname{id}_Y}}"', from=2-1, to=3-2]
        \arrow["{\pi_X}"', tail reversed, no head, from=2-3, to=1-2]
        \arrow["{\brackc{\operatorname{id}_X,f}}", from=2-3, to=3-2]
    \end{tikzcd} 
    \end{equation*}
    And with similar concept to the co-wedge, we can requires the diagram on the RHS to commutes, given $f:X\to Y$, we have $\brackc{f,\operatorname{id}_Y}\circ\pi_Y = \brackc{\operatorname{id}_X,f}\circ\pi_X$
\end{definition}

\begin{definition}{\textbf{(Ends)}}
    The universal wedge is called a end. This follows from the diagram below:
    \begin{equation*}
    % https://q.uiver.app/#q=WzAsNSxbMCwxLCJQXFxicmFja2N7WSwgWX0iXSxbMiwxLCJQXFxicmFja2N7WCwgWH0iXSxbMSwwLCJEIl0sWzEsMiwiUFxcYnJhY2tje1gsIFl9Il0sWzEsMSwiXFxpbnRfWCBQXFxicmFja2N7WCwgWH0iXSxbMCwzLCJcXGJyYWNrY3tmLCBcXG9wZXJhdG9ybmFtZXtpZH1fWX0iLDJdLFsxLDMsIlxcYnJhY2tje1xcb3BlcmF0b3JuYW1le2lkfV9YLGZ9Il0sWzAsNCwiXFxwaV9ZIiwwLHsic3R5bGUiOnsidGFpbCI6eyJuYW1lIjoiYXJyb3doZWFkIn0sImhlYWQiOnsibmFtZSI6Im5vbmUifX19XSxbMSw0LCJcXHBpX1giLDIseyJzdHlsZSI6eyJ0YWlsIjp7Im5hbWUiOiJhcnJvd2hlYWQifSwiaGVhZCI6eyJuYW1lIjoibm9uZSJ9fX1dLFsyLDAsImdfWSIsMl0sWzIsMSwiZ19YIl0sWzIsNCwiaCIsMCx7InN0eWxlIjp7ImJvZHkiOnsibmFtZSI6ImRhc2hlZCJ9fX1dXQ==
    \begin{tikzcd}
        & D \\
        {P\brackc{Y, Y}} & {\int_X P\brackc{X, X}} & {P\brackc{X, X}} \\
        & {P\brackc{X, Y}}
        \arrow["{g_Y}"', from=1-2, to=2-1]
        \arrow["h", dashed, from=1-2, to=2-2]
        \arrow["{g_X}", from=1-2, to=2-3]
        \arrow["{\pi_Y}", tail reversed, no head, from=2-1, to=2-2]
        \arrow["{\brackc{f, \operatorname{id}_Y}}"', from=2-1, to=3-2]
        \arrow["{\pi_X}"', tail reversed, no head, from=2-3, to=2-2]
        \arrow["{\brackc{\operatorname{id}_X,f}}", from=2-3, to=3-2]
    \end{tikzcd}
    \end{equation*}
\end{definition}

\begin{proposition}{\textbf{(Natural Transformations as an End)}}
    \label{prop:nat-trans-as-end}
    The set of natural transformation between functor $F, G:\textbf{C}\to\textbf{D}$ being a hom-set in the functor category is given by End as:
    \begin{equation*}
        \operatorname{Hom}_{[\textbf{C}, \textbf{D}]}(F, G) \cong \int_{X:\textbf{C}}\operatorname{Hom}_\textbf{D}(FX, GX)
    \end{equation*}
    Note that the hom-set is seen as profunctor here.
\end{proposition}
\begin{proof}
    First, we consider the following diagram:
    \begin{equation*}
    % https://q.uiver.app/#q=WzAsNSxbMCwxLCJcXG9wZXJhdG9ybmFtZXtIb219X1xcdGV4dGJme0R9KEZZLCBHWSkiXSxbMiwxLCJcXG9wZXJhdG9ybmFtZXtIb219X1xcdGV4dGJme0R9KEZYLCBHWCkiXSxbMSwyLCJcXG9wZXJhdG9ybmFtZXtIb219X1xcdGV4dGJme0R9KEZYLCBHWSkiXSxbMSwxLCJcXGludF97WDpcXHRleHRiZntDfX1cXG9wZXJhdG9ybmFtZXtIb219X1xcdGV4dGJme0R9KEZYLCBHWCkiXSxbMSwwLCIxIl0sWzAsMiwiLVxcY2lyYyBGZiIsMl0sWzEsMiwiR2ZcXGNpcmMgLSJdLFswLDMsIiIsMCx7InN0eWxlIjp7InRhaWwiOnsibmFtZSI6ImFycm93aGVhZCJ9LCJoZWFkIjp7Im5hbWUiOiJub25lIn19fV0sWzEsMywiIiwyLHsic3R5bGUiOnsidGFpbCI6eyJuYW1lIjoiYXJyb3doZWFkIn0sImhlYWQiOnsibmFtZSI6Im5vbmUifX19XSxbNCwzLCIiLDIseyJzdHlsZSI6eyJib2R5Ijp7Im5hbWUiOiJkYXNoZWQifX19XSxbNCwwLCJcXGFscGhhX1kiLDJdLFs0LDEsIlxcYWxwaGFfWCJdXQ==
    \begin{tikzcd}
        & 1 \\
        {\operatorname{Hom}_\textbf{D}(FY, GY)} & {\int_{X:\textbf{C}}\operatorname{Hom}_\textbf{D}(FX, GX)} & {\operatorname{Hom}_\textbf{D}(FX, GX)} \\
        & {\operatorname{Hom}_\textbf{D}(FX, GY)}
        \arrow["{\alpha_Y}"', from=1-2, to=2-1]
        \arrow[dashed, from=1-2, to=2-2]
        \arrow["{\alpha_X}", from=1-2, to=2-3]
        \arrow[tail reversed, no head, from=2-1, to=2-2]
        \arrow["{-\circ Ff}"', from=2-1, to=3-2]
        \arrow[tail reversed, no head, from=2-3, to=2-2]
        \arrow["{Gf\circ -}", from=2-3, to=3-2]
    \end{tikzcd}
    \end{equation*}
    We can see that when we pick an element from the ends, is the same (by universal properties) as picking an element from $\alpha_Y\in\operatorname{Hom}_\textbf{D}(FY, GY)$ and $\alpha_X\in\operatorname{Hom}_\textbf{D}(FX, GX)$ such that $Gf\circ\alpha_X=\alpha_Y\circ Ff$. This happens to all the objects in $\textbf{C}$, and it is clear that $\alpha_C$ for all objects $C$ in $\textbf{C}$ are the component of the $\alpha:F\Rightarrow G$ with the appropriate naturality condition follows from wedge condition.

    % we note that given $\alpha:F\Rightarrow G$, and given $f:X\to Y$, the naturality condition is given by $Gf\circ\alpha_X=\alpha_Y\circ Ff$
\end{proof}

\begin{proposition}{\textbf{((Co)-Limit as (Co)-Ends)}}
    Given a profunctor $F:1\times\textbf{J}\to\textbf{D}$ that ignores the first parameters i.e $F\langle *, X\rangle = FX$ and given $g:X\to Y$, $F\langle *, g\rangle = Fg$. Then
    \begin{equation*}
        \int_XFX=\lim F \qquad \quad \int^XFX=\operatorname{colim} F
    \end{equation*}
\end{proposition}
\begin{proof}
    For any morphism $f:X\to Y$ in $\textbf{J}$ with objects $X$ and $Y$. We have the diagram below. Observe that (co)-wedge is turned into a (co)-cone and by the universal property, the (co)-ends becomes (co)-limit.
    \begin{equation*}
    % https://q.uiver.app/#q=WzAsNSxbMSwwLCJGWCJdLFswLDEsIkZZIl0sWzIsMSwiRlgiXSxbMSwxLCJcXGludF5YRlgiXSxbMSwyLCJEIl0sWzEsM10sWzIsM10sWzAsMSwiRmYiLDJdLFswLDIsIlxcb3BlcmF0b3JuYW1le2lkfV97Rlh9IiwwLHsic3R5bGUiOnsiYm9keSI6eyJuYW1lIjoiZG90dGVkIn19fV0sWzEsNF0sWzIsNF0sWzMsNCwiIiwxLHsic3R5bGUiOnsiYm9keSI6eyJuYW1lIjoiZGFzaGVkIn19fV1d
    \begin{tikzcd}
        & FX \\
        FY & {\int^XFX} & FX \\
        & D
        \arrow["Ff"', from=1-2, to=2-1]
        \arrow["{\operatorname{id}_{FX}}", dotted, from=1-2, to=2-3]
        \arrow[from=2-1, to=2-2]
        \arrow[from=2-1, to=3-2]
        \arrow[dashed, from=2-2, to=3-2]
        \arrow[from=2-3, to=2-2]
        \arrow[from=2-3, to=3-2]
    \end{tikzcd}
    \qquad \quad
    % https://q.uiver.app/#q=WzAsNSxbMSwwLCJEIl0sWzEsMSwiXFxpbnRfWEZYIl0sWzAsMSwiRlgiXSxbMiwxLCJGWSJdLFsxLDIsIkZZIl0sWzEsMl0sWzEsM10sWzAsMSwiIiwxLHsic3R5bGUiOnsiYm9keSI6eyJuYW1lIjoiZGFzaGVkIn19fV0sWzAsMl0sWzAsM10sWzIsNCwiRmYiLDJdLFszLDQsIlxcb3BlcmF0b3JuYW1le2lkfV97Rll9IiwwLHsic3R5bGUiOnsiYm9keSI6eyJuYW1lIjoiZG90dGVkIn19fV1d
    \begin{tikzcd}
        & D \\
        FX & {\int_XFX} & FY \\
        & FY
        \arrow[from=1-2, to=2-1]
        \arrow[dashed, from=1-2, to=2-2]
        \arrow[from=1-2, to=2-3]
        \arrow["Ff"', from=2-1, to=3-2]
        \arrow[from=2-2, to=2-1]
        \arrow[from=2-2, to=2-3]
        \arrow["{\operatorname{id}_{FY}}", dotted, from=2-3, to=3-2]
    \end{tikzcd}
    \end{equation*}
\end{proof}

\begin{remark}{(Hom-Functor and Ends)}
    By the theorem \ref{thm:rep-functor-continuous}, the hom-functor is continuous and since the ends and co-ends can be represented as limit and co-limit, we have the following formulas, given $P:\textbf{C}^\text{op}\times\textbf{C}\to\textbf{D}$:
    \begin{equation*}
        \operatorname{Hom}_\textbf{D}\left( D, \int_AP\brackc{A, A} \right) \cong \int_A\operatorname{Hom}_\textbf{D}\Big( D, P\brackc{A, A} \Big) \quad \operatorname{Hom}_\textbf{D}\left(\int^A P\brackc{A, A}, D \right) \cong \int_A\operatorname{Hom}_\textbf{D}\Big( P\brackc{A, A}, D \Big)
    \end{equation*}
    Note that on the RHS, we replace co-end with end because we are dealing with the opposite category, as the first parameter. 
\end{remark}

\begin{proposition}{\textbf{(Functoriality of Ends)}}
    \label{prop:functoriality-of-ends}
    Given a natural transformation $\eta:P\Rightarrow P'$ between 2 profunctors $P:\textbf{C}^\text{op}\times\textbf{C}\to\textbf{D}$, then we can define an induced arrow: $\int_X\eta:\int_X P\brackc{X, X}\to\int_X P'\brackc{X, X}$, as follows (using universal properties):
    \begin{equation*}
    % https://q.uiver.app/#q=WzAsOCxbMCw1LCJQXFxsYW5nbGUgWSwgWVxccmFuZ2xlIl0sWzIsMywiUFxcbGFuZ2xlIFgsIFhcXHJhbmdsZSJdLFsyLDUsIlBcXGxhbmdsZSBYLCBZXFxyYW5nbGUiXSxbMCwzLCJcXGludF9YIFBcXGxhbmdsZSBYLCBYXFxyYW5nbGUiXSxbMSwwLCJcXGludF9YUCdcXGxhbmdsZSBYLCBYXFxyYW5nbGUiXSxbMSwyLCJQJ1xcbGFuZ2xlIFksIFlcXHJhbmdsZSJdLFszLDAsIlAnXFxsYW5nbGUgWCwgWFxccmFuZ2xlIl0sWzMsMiwiUCdcXGxhbmdsZSBYLCBZXFxyYW5nbGUiXSxbMSwyLCJQXFxsYW5nbGUgXFxvcGVyYXRvcm5hbWV7aWR9X1gsZlxccmFuZ2xlIiwxXSxbMCwzLCJcXHBpX1kiLDEseyJzdHlsZSI6eyJ0YWlsIjp7Im5hbWUiOiJhcnJvd2hlYWQifSwiaGVhZCI6eyJuYW1lIjoibm9uZSJ9fX1dLFsxLDMsIlxccGlfWCIsMSx7InN0eWxlIjp7InRhaWwiOnsibmFtZSI6ImFycm93aGVhZCJ9LCJoZWFkIjp7Im5hbWUiOiJub25lIn19fV0sWzUsNywiUCdcXGxhbmdsZSBmLCBcXG9wZXJhdG9ybmFtZXtpZH1fWVxccmFuZ2xlIiwxXSxbNCw1LCJcXHBpJ19ZIiwxXSxbNCw2LCJcXHBpJ19YIiwxXSxbNiw3LCJQJ1xcbGFuZ2xlIFxcb3BlcmF0b3JuYW1le2lkfV9YLGZcXHJhbmdsZSIsMV0sWzAsMiwiIFBcXGxhbmdsZSBmLCBcXG9wZXJhdG9ybmFtZXtpZH1fWVxccmFuZ2xlIiwxXSxbMCw1LCJcXGV0YV97XFxsYW5nbGUgWSwgWVxccmFuZ2xlfSIsMV0sWzIsNywiXFxldGFfe1xcbGFuZ2xlIFgsIFlcXHJhbmdsZX0iLDFdLFsxLDYsIlxcZXRhX3tcXGxhbmdsZSBYLCBYXFxyYW5nbGV9IiwxXSxbMyw0LCIiLDEseyJzdHlsZSI6eyJib2R5Ijp7Im5hbWUiOiJkYXNoZWQifX19XV0=
    \begin{tikzcd}
        & {\int_XP'\langle X, X\rangle} && {P'\langle X, X\rangle} \\
        \\
        & {P'\langle Y, Y\rangle} && {P'\langle X, Y\rangle} \\
        {\int_X P\langle X, X\rangle} && {P\langle X, X\rangle} \\
        \\
        {P\langle Y, Y\rangle} && {P\langle X, Y\rangle}
        \arrow["{\pi'_X}"{description}, from=1-2, to=1-4]
        \arrow["{\pi'_Y}"{description}, from=1-2, to=3-2]
        \arrow["{P'\langle \operatorname{id}_X,f\rangle}"{description}, from=1-4, to=3-4]
        \arrow["{P'\langle f, \operatorname{id}_Y\rangle}"{description}, from=3-2, to=3-4]
        \arrow[dashed, from=4-1, to=1-2]
        \arrow["{\eta_{\langle X, X\rangle}}"{description}, from=4-3, to=1-4]
        \arrow["{\pi_X}"{description}, tail reversed, no head, from=4-3, to=4-1]
        \arrow["{P\langle \operatorname{id}_X,f\rangle}"{description}, from=4-3, to=6-3]
        \arrow["{\eta_{\langle Y, Y\rangle}}"{description}, from=6-1, to=3-2]
        \arrow["{\pi_Y}"{description}, tail reversed, no head, from=6-1, to=4-1]
        \arrow["{ P\langle f, \operatorname{id}_Y\rangle}"{description}, from=6-1, to=6-3]
        \arrow["{\eta_{\langle X, Y\rangle}}"{description}, from=6-3, to=3-4]
    \end{tikzcd}
    \end{equation*}

    And note that this preserves the composition i.e given $\eta':P'\Rightarrow P''$, we have that $\int_X(\eta'\circ\eta)=\int_X\eta'\circ\int_X\eta$. And $\int_X\operatorname{id}_{P}=\operatorname{id}_{\int_XP\brackc{X,X}}$, meaning that Ends induces a functor.
\end{proposition}
\begin{proof}
    Note that we have used the natural transformation laid out in LHS diagram (to ensure that everything commutes), which construct the ``faces'' of the box above. And, the arrow between two ends is constructed via universal properties shown in LHS:
    \begin{equation*}
    % https://q.uiver.app/#q=WzAsOSxbNSwxLCJcXGludF9YUCdcXGxhbmdsZSBYLCBYXFxyYW5nbGUiXSxbNCwyLCJQJ1xcbGFuZ2xlIFksIFlcXHJhbmdsZSJdLFs2LDIsIlAnXFxsYW5nbGUgWCwgWFxccmFuZ2xlIl0sWzUsMywiUCdcXGxhbmdsZSBYLCBZXFxyYW5nbGUiXSxbMCwxLCJQXFxsYW5nbGUgWSwgWVxccmFuZ2xlIl0sWzIsMSwiUCdcXGxhbmdsZSBZLCBZXFxyYW5nbGUiXSxbMCwyLCJQXFxsYW5nbGUgWCwgWVxccmFuZ2xlIl0sWzIsMiwiUCdcXGxhbmdsZSBYLCBZXFxyYW5nbGUiXSxbNSwwLCJcXGludF9YIFBcXGxhbmdsZSBYLCBYXFxyYW5nbGUiXSxbMSwzLCJQJ1xcbGFuZ2xlIGYsIFxcb3BlcmF0b3JuYW1le2lkfV9ZXFxyYW5nbGUiLDJdLFswLDEsIlxccGknX1kiLDJdLFswLDIsIlxccGknX1giXSxbMiwzLCJQJ1xcbGFuZ2xlIFxcb3BlcmF0b3JuYW1le2lkfV9YLGZcXHJhbmdsZSJdLFs0LDUsIlxcZXRhX3tcXGxhbmdsZSBZLCBZXFxyYW5nbGV9Il0sWzQsNiwiIFBcXGxhbmdsZSBmLCBcXG9wZXJhdG9ybmFtZXtpZH1fWVxccmFuZ2xlIiwyXSxbNSw3LCJQJ1xcbGFuZ2xlIGYsIFxcb3BlcmF0b3JuYW1le2lkfV9ZXFxyYW5nbGUiXSxbNiw3LCJcXGV0YV97XFxsYW5nbGUgWCwgWVxccmFuZ2xlfSIsMl0sWzgsMSwiXFxldGFfe1xcbGFuZ2xlIFksIFlcXHJhbmdsZX1cXGNpcmNcXHBpX1kiLDIseyJsYWJlbF9wb3NpdGlvbiI6NjAsImN1cnZlIjoyfV0sWzgsMiwiXFxldGFfe1xcbGFuZ2xlIFgsIFhcXHJhbmdsZX1cXGNpcmNcXHBpX1giLDAseyJjdXJ2ZSI6LTJ9XSxbOCwwLCIiLDAseyJzdHlsZSI6eyJib2R5Ijp7Im5hbWUiOiJkYXNoZWQifX19XV0=
    \begin{tikzcd}
        &&&&& {\int_X P\langle X, X\rangle} \\
        {P\langle Y, Y\rangle} && {P'\langle Y, Y\rangle} &&& {\int_XP'\langle X, X\rangle} \\
        {P\langle X, Y\rangle} && {P'\langle X, Y\rangle} && {P'\langle Y, Y\rangle} && {P'\langle X, X\rangle} \\
        &&&&& {P'\langle X, Y\rangle}
        \arrow[dashed, from=1-6, to=2-6]
        \arrow["{\eta_{\langle Y, Y\rangle}\circ\pi_Y}"'{pos=0.6}, curve={height=12pt}, from=1-6, to=3-5]
        \arrow["{\eta_{\langle X, X\rangle}\circ\pi_X}", curve={height=-12pt}, from=1-6, to=3-7]
        \arrow["{\eta_{\langle Y, Y\rangle}}", from=2-1, to=2-3]
        \arrow["{ P\langle f, \operatorname{id}_Y\rangle}"', from=2-1, to=3-1]
        \arrow["{P'\langle f, \operatorname{id}_Y\rangle}", from=2-3, to=3-3]
        \arrow["{\pi'_Y}"', from=2-6, to=3-5]
        \arrow["{\pi'_X}", from=2-6, to=3-7]
        \arrow["{\eta_{\langle X, Y\rangle}}"', from=3-1, to=3-3]
        \arrow["{P'\langle f, \operatorname{id}_Y\rangle}"', from=3-5, to=4-6]
        \arrow["{P'\langle \operatorname{id}_X,f\rangle}", from=3-7, to=4-6]
    \end{tikzcd}
    \end{equation*}
    It is clear from the diagram how the horizon composition of the natural transformation is worked out (adding another block and the universal property will makes sure that 2 map $a$ and $a$ are equal). This is the same for the case of identity (the identity case means that we do nothing).
\end{proof}

\begin{theorem}{\textbf{(Fubini Rule)}}
    Given the functor of the form $P:\textbf{C}\times\textbf{C}^\text{op}\times\textbf{D}\times\textbf{D}^\text{op}\to\textbf{E}$, as long as the ends exits, we can show that:
    \begin{equation*}
        \int_{C:\textbf{C}}\int_{D:\textbf{D}}P\brackc{C,C}\brackc{ D, D}\cong \int_{D:\textbf{D}}\int_{C:\textbf{C}} P\brackc{C,C}\brackc{ D, D}\cong\int_{\brackc{C,D}:\textbf{C}\times\textbf{D}} P\brackc{C,C}\brackc{ D, D}
    \end{equation*}
    Note that the last one, the functor $P$ is re-interpreted as $P:(\textbf{C}\times\textbf{D})^\text{op}\times(\textbf{C}\times\textbf{D})\to\textbf{E}$. This is the same for co-end case.
\end{theorem}
\begin{proof}
    \todo   
\end{proof}

\subsection{Ninja Yoneda Lemma}

\begin{proposition}{\textbf{(Ninja Yoneda Lemma)}}
    We can restate Yoneda lemma in terms of Ends and Co-ends as:
    \begin{equation*}
        \int_{X:\textbf{C}}\operatorname{Hom}_\textbf{Set}\Big( \operatorname{Hom}_\textbf{C}(X, A), FX \Big) \cong FA \qquad \quad
        \int^{X:\textbf{C}}\operatorname{Hom}_\textbf{C}(X, A)\times FX\cong FA
    \end{equation*}
\end{proposition}
\begin{proof}
    \textbf{(Part 1):} We will start with the original formulation, and then we simply use the 
    \begin{equation*}
    \begin{aligned}
        FA\cong\operatorname{Hom}_{[\textbf{C}^\text{op}, \textbf{Set}]}\Big( \operatorname{Hom}_\textbf{C}(-, A), F \Big) &\cong \int_{X:\textbf{C}^\text{op}}\operatorname{Hom}_\textbf{Set}\Big( \operatorname{Hom}_\textbf{C}(X, A), FX \Big) \\
        &\cong \int_{X:\textbf{C}}\operatorname{Hom}_\textbf{Set}\Big( \operatorname{Hom}_\textbf{C}(X, A), FX \Big) \\
    \end{aligned}
    \end{equation*}
    To see why the second reduction works, let's trying to consider the commutative diagram of Ends as given here.
    \begin{equation*}
    % https://q.uiver.app/#q=WzAsNSxbMSwxLCJcXGludF97WDpcXHRleHRiZntDfV5cXHRleHR7b3B9fVxcb3BlcmF0b3JuYW1le0hvbX1fXFx0ZXh0YmZ7U2V0fVxcQmlnKCBcXG9wZXJhdG9ybmFtZXtIb219X1xcdGV4dGJme0N9KFgsIEEpLCBGWCBcXEJpZykiXSxbMCwxLCJcXG9wZXJhdG9ybmFtZXtIb219X1xcdGV4dGJme1NldH1cXEJpZyggXFxvcGVyYXRvcm5hbWV7SG9tfV9cXHRleHRiZntDfShYLCBBKSwgRlggXFxCaWcpIl0sWzIsMSwiXFxvcGVyYXRvcm5hbWV7SG9tfV9cXHRleHRiZntTZXR9XFxCaWcoIFxcb3BlcmF0b3JuYW1le0hvbX1fXFx0ZXh0YmZ7Q30oWSwgQSksIEZZIFxcQmlnKSJdLFsxLDIsIlxcb3BlcmF0b3JuYW1le0hvbX1fXFx0ZXh0YmZ7U2V0fVxcQmlnKCBcXG9wZXJhdG9ybmFtZXtIb219X1xcdGV4dGJme0N9KFgsIEEpLCBGWSBcXEJpZykiXSxbMSwwLCIxIl0sWzEsMywiRmZcXGNpcmMtIiwyXSxbMCwxXSxbMCwyXSxbMiwzLCItXFxjaXJjWy1cXGNpcmMgZl0iXSxbNCwxXSxbNCwyXSxbNCwwLCIiLDEseyJzdHlsZSI6eyJib2R5Ijp7Im5hbWUiOiJkYXNoZWQifX19XV0=
    \begin{tikzcd}
        & 1 \\
        {\operatorname{Hom}_\textbf{Set}\Big( \operatorname{Hom}_\textbf{C}(X, A), FX \Big)} & {\int_{X:\textbf{C}^\text{op}}\operatorname{Hom}_\textbf{Set}\Big( \operatorname{Hom}_\textbf{C}(X, A), FX \Big)} & {\operatorname{Hom}_\textbf{Set}\Big( \operatorname{Hom}_\textbf{C}(Y, A), FY \Big)} \\
        & {\operatorname{Hom}_\textbf{Set}\Big( \operatorname{Hom}_\textbf{C}(X, A), FY \Big)}
        \arrow[from=1-2, to=2-1]
        \arrow[dashed, from=1-2, to=2-2]
        \arrow[from=1-2, to=2-3]
        \arrow["{Ff\circ-}"', from=2-1, to=3-2]
        \arrow[from=2-2, to=2-1]
        \arrow[from=2-2, to=2-3]
        \arrow["{-\circ[-\circ f]}", from=2-3, to=3-2]
    \end{tikzcd}
    \end{equation*}
    We note that  $[-\circ f]:\operatorname{Hom}_\textbf{C}(Y, A)\to \operatorname{Hom}_\textbf{C}(X, A)$. That is we required $f:X\to Y$ sits in the opposite category, where $-\circ f$ is the maps between sets after it get lifted from $\operatorname{Hom}_\textbf{Set}(-, A):\textbf{C}^\text{op}\to\textbf{Set}$. Therefore, the opposite category only needs to define the morphism on the RHS. In which we can define the ends over $\textbf{C}$ instead of $\textbf{C}^\text{op}$ and by universal property, they are the same. We can think of it as an integration over the ``objects'' and not the morphism.

    \textbf{(Part 2):} Let's consider the maps out from the co-end to some set $S$ that is:
    \begin{equation*}
    \begin{aligned}
        \operatorname{Hom}_\textbf{Set}\bigg( \int^{X:\textbf{C}}&\operatorname{Hom}_\textbf{C}(X, A)\times FX, S \bigg) \cong \int_{X:\textbf{C}}\operatorname{Hom}_\textbf{Set}\Big(\operatorname{Hom}_\textbf{C}(X, A)\times FX, S \Big) \\
        &\cong \int_{X:\textbf{C}}\operatorname{Hom}_\textbf{Set}\Big(\operatorname{Hom}_\textbf{C}(X, A), S^{FX} \Big) \cong S^{FA}\cong \operatorname{Hom}_\textbf{Set}(FA, S) \\
    \end{aligned}
    \end{equation*}
    Note that one the third isomorphism, we have used the first Ninja Yoneda lemma, as we can view $S^{F-}:\textbf{C}^\text{op}\to\textbf{Set}$ as the functor as the limit/universal properties are functorial. Finally, via the corollary \ref{coro:iso-hom-functor-iso-represent}, we the desired isomorphism.
\end{proof}

\begin{proposition}{\textbf{(Ninja Co-Yoneda Lemma)}}
    We also have consider the co-pre-sheaf, instead of pre-sheaf, seen in Yoneda lemma.
    \begin{equation*}
        \int_{X:\textbf{C}}\operatorname{Hom}_\textbf{Set}\Big( \operatorname{Hom}_\textbf{C}(A, X), GX \Big) \cong GA \qquad \quad
        \int^{X:\textbf{C}}\operatorname{Hom}_\textbf{C}(A, X)\times GX\cong GA
    \end{equation*}
\end{proposition}

\begin{remark}{(Ninja Yoneda Lemma)}
    We can think of hom-profunctor as the identity matrix i.e Kronecker delta, in which the Yoneda lemma will looks like $\sum_j\delta^j_iv_j=v_i$.
\end{remark}

\subsection{Day Convolution}

\begin{definition}{\textbf{(Day Convolution)}}
    Given a monoidal category $(\textbf{C},\otimes, I)$ and a pre-sheaf $F, G:\textbf{C}\to\textbf{Set}$, then the Day convolution:
    \begin{equation*}
        (G\star F)(X)=\int^{A,B:\textbf{C}}\operatorname{Hom}_\textbf{C}\big( A\otimes B, X \big)\times GA\times FB
    \end{equation*}
\end{definition}

\begin{proposition}
    Day convolution endows the category of co-preserve $\textbf{C}\to\textbf{Set}$ with a monoidal structure, where the convolution is associative (shown in LHS) with the unit object is $\operatorname{Hom}_\textbf{C}(I,-)$ (shown in RHS).
    \begin{equation*}
        \big(H\star(G\star F)\big)(X) \cong \big((H\star G)\star F\big)(X) \qquad \quad \Big( \operatorname{Hom}_\textbf{C}(I, -)\star F \Big)(X)\cong FX
    \end{equation*}
\end{proposition}
\begin{proof}
    \textbf{(Associativity):} We have the following isomorphisms:
    \begin{equation*}
    \begin{aligned}
        \big(H\star(G\star F)\big)(X) &= \int^{D, E:\textbf{C}}\operatorname{Hom}_\textbf{C}(D\otimes E, X)\times HD\times\underbrace{\left( \int^{A,B:\textbf{C}}\operatorname{Hom}_\textbf{C}\big( A\otimes B, E \big)\times GA\times FB \right)}_{(G\star F)(E)} \\
        &\cong\int^{A, B, D, E:\textbf{C}}\operatorname{Hom}_\textbf{C}(D\otimes E, X)\times\operatorname{Hom}_\textbf{C}\big( A\otimes B, E \big)\times GA\times HD\times FB \\
        &\cong\int^{A, B, D:\textbf{C}}\operatorname{Hom}_\textbf{C}(D\otimes A\otimes B, X)\times GA\times HD\times FB \\
    \end{aligned}
    \end{equation*}
    On the other hand, we have that:
    \begin{equation*}
    \begin{aligned}
        \big((H\star G)\star F\big)(X) &= \int^{A,B:\textbf{C}}\operatorname{Hom}_\textbf{C}\big( A\otimes B, X \big)\times \underbrace{\left(\int^{D,E:\textbf{C}}\operatorname{Hom}_\textbf{C}\big( D\otimes E, A \big)\times HD\times GE\right)}_{(H\star G)A}\times FB \\
        &\cong \int^{A,B,D,E:\textbf{C}}\operatorname{Hom}_\textbf{C}\big( D\otimes E, A \big)\times \operatorname{Hom}_\textbf{C}\big( A\otimes B, X \big)\times GE\times HD\times FB \\
        &\cong \int^{B,D,E:\textbf{C}}\operatorname{Hom}_\textbf{C}\big( D\otimes E\otimes B, X \big)\times GE\times HD\times FB \\
    \end{aligned}
    \end{equation*}
    Note that on the third isomorphism, we simply use the Ninja Yoneda lemma. And we just rename the object from $E$ to $A$ and we are done.

    \textbf{(Unit):} We simply have:
    \begin{equation*}
    \begin{aligned}
        \Big( \operatorname{Hom}_\textbf{C}(I, -)\star F \Big)(X) &\cong \int^{A,B:\textbf{C}}\operatorname{Hom}_\textbf{C}\big( A\otimes B, X \big)\times \operatorname{Hom}_\textbf{C}(I, A)\times FB \\
        &\cong \int^{B:\textbf{C}}\operatorname{Hom}_\textbf{C}\big(I\otimes B, X \big)\times FB \cong FX
    \end{aligned}
    \end{equation*}
\end{proof}

\todo Skip section \textbf{Applicative functors as monoids} and \textbf{Free Applicatives}

\begin{proposition}
    The profunctor composition is associative and contains the identity functor being the hom-functor. Both of them are true up to isomorphism, in which we have:
    \begin{equation*}
        \big( (P\diamond Q)\diamond R \big) \cong \big( P\diamond (Q\diamond R) \big) \qquad \qquad \operatorname{Hom}_\textbf{C}(-,=)\diamond P \cong P
    \end{equation*}
\end{proposition}
\begin{proof}
    This can be shown using the Fubini theorem, in which we have:
    \begin{equation*}
    \begin{aligned}
        \big( (P\diamond Q)\diamond &R \big) \langle A, B\rangle = \int^{Y:\textbf{C}}(P\diamond Q)\brackc{A, Y}\times R\brackc{Y, B} \\
        &= \int^{Y:\textbf{C}}\left(\int^{X:\textbf{C}}P\brackc{A, X}\times Q\brackc{X, Y}\right)\times R\brackc{Y, B} \\
        &\cong  \int^{X:\textbf{C}}P\brackc{A, X}\times \left(\int^{Y:\textbf{C}}Q\brackc{X, Y}\times R\brackc{Y, B}\right) = \big( P\diamond (Q\diamond R) \big)\langle A, B\rangle
    \end{aligned}
    \end{equation*}
    On the other hand, for the identity we have that (using the ninja Yoneda lemma)
    \begin{equation*}
        \big(\operatorname{Hom}_\textbf{C}(-,=)\diamond P\big)\langle A, B\rangle = \int^{X:\textbf{C}}\operatorname{Hom}_\textbf{C}(A, X)\times P\langle X, B\rangle \cong P\langle A, B\rangle
    \end{equation*}
\end{proof}

\begin{definition}{\textbf{(Bicategory of Profunctor)}}
    \label{def:cat-profunctor}
    The category whose categorical law is satisfies up to isomorphism is called bicategory. In which the objects are categories. The morphism is the profunctor between them i.e $\textbf{C}\profunct \textbf{D}$. 
    
    And it has to be equipped with a 2-cell too i.e morphism between morphism, which will let that be a natural transformation. That is with $P:\textbf{C}\profunct \textbf{D}$ and $Q:\textbf{C}\profunct \textbf{D}$, we have the following naturality condition $f:S\to A$ and $g:B\to T$ in the diagram of:
    \begin{equation*}
    % https://q.uiver.app/#q=WzAsNCxbMCwwLCJQXFxicmFja2N7QSwgQn0iXSxbMiwwLCJQXFxicmFja2N7UywgVH0iXSxbMCwxLCJRXFxicmFja2N7QSwgQn0iXSxbMiwxLCJRXFxicmFja2N7UywgVH0iXSxbMCwxLCJQXFxicmFja2N7ZiwgZ30iXSxbMCwyLCJcXGFscGhhX3tBLCBCfSIsMl0sWzEsMywiXFxhbHBoYV97UywgVH0iXSxbMiwzLCJRXFxicmFja2N7ZiwgZ30iLDJdXQ==
    \begin{tikzcd}
        {P\brackc{A, B}} && {P\brackc{S, T}} \\
        {Q\brackc{A, B}} && {Q\brackc{S, T}}
        \arrow["{P\brackc{f, g}}", from=1-1, to=1-3]
        \arrow["{\alpha_{A, B}}"', from=1-1, to=2-1]
        \arrow["{\alpha_{S, T}}", from=1-3, to=2-3]
        \arrow["{Q\brackc{f, g}}"', from=2-1, to=2-3]
    \end{tikzcd}
    \end{equation*}
\end{definition}

\begin{remark}{(Further Notes on 2-Category)}
    We have observed that categories, functors, and natural transformations form a 2-Category $\textbf{Cat}$. On one object, a category $\textbf{C}$ that is 0-cell:

    \begin{itemize}
        \item (Behavior of 1-Cells): The 1-cells that start and end at this object form a regular category, which is the functor category $[\textbf{C}, \textbf{C}]$. The object in this category are endo-1-cells of the outer 2-category $\textbf{Cat}$, where arrows between them are 2-cells of the outer 2-category.
        \item (Monoidal Structure): This endo-1-cell is automatically equipped with a monoidal structure, where the tensor product is the composition, and the monoidal unit object is the identity 1-cell.
    \end{itemize}

    If we focus our attention on just one endo-1-cell, we can ``square'' it i.e creating $F\circ F$. We say that $F$ is a monad if we can find 2-cells: $\mu:F\circ F\Rightarrow F$ and $\eta:I\Rightarrow F$ that behave like multiplication and unit making associativity and unit diagrams commute In fact a monad can be defined in an arbitrary bicategory.
\end{remark}

\begin{remark}{(Monad Structure for Profunctor)}
    Given the discussion above and the fact that $\textbf{Prof}$ is a bicategory, we can define a monad in it, which is an endo-profunctor $\textbf{C}\profunct \textbf{C}$ or $P:\textbf{C}^\text{op}\times\textbf{C}\to\textbf{Set}$ with 2 natural transformations (2-cells): $\mu:P\diamond P\to P$ and $\eta:\operatorname{Hom}_\textbf{C}(-, =)\to P$ monad can be defined in an arbitrary bicategory. We have the natural transformation as:
    \begin{equation*}
    \begin{aligned}[t]
        \mu&\in\int_{\brackc{A, B}}\operatorname{Hom}_\textbf{Set}\left( \int^XP\brackc{A, X} \times P\brackc{X, B}, P\brackc{A, B} \right) \\
        &\cong \int_{\brackc{A, B}, X}\operatorname{Hom}_\textbf{Set}\Big( P\brackc{A, X} \times P\brackc{X, B},  P\brackc{A, B} \Big)
    \end{aligned}
    \qquad \quad 
    \eta \in \int_{\brackc{A, B}}\operatorname{Hom}_\textbf{Set}\Big( \operatorname{Hom}_\textbf{C}(A, B),  P\brackc{A, B} \Big)
    \end{equation*}
\end{remark}

\subsection{Lens}

\begin{remark}{(Lens)}
    Given a composite objects, which contains parts. We should be able to extract some part of the object, or replace them with a new one. That is we have $\texttt{get} : S\to A$ and $\texttt{set} : S\times A\to S$ with the laws (following from \href{https://arxiv.org/abs/1809.00738}{here}):
    \begin{equation*}
    % https://q.uiver.app/#q=WzAsMTAsWzAsMCwiU1xcdGltZXMgQSJdLFsyLDAsIlMiXSxbMSwxLCJBIl0sWzQsMCwiUyJdLFs2LDAsIlNcXHRpbWVzIEEiXSxbNSwxLCJTIl0sWzgsMCwiU1xcdGltZXMgQVxcdGltZXMgQSJdLFs5LDAsIlNcXHRpbWVzIEEiXSxbOCwxLCJTXFx0aW1lcyBBIl0sWzksMSwiQSJdLFswLDEsIlxcdGV4dHR0e3NldH0iXSxbMSwyLCJcXHRleHR0dHtnZXR9Il0sWzAsMiwiXFxwaV8yIiwyXSxbMyw0LCJcXGJyYWNrY3tcXG9wZXJhdG9ybmFtZXtpZH1fUywgXFx0ZXh0dHR7Z2V0fX0iXSxbNCw1LCJcXHRleHR0dHtzZXR9Il0sWzMsNSwiXFxvcGVyYXRvcm5hbWV7aWR9X1MiLDJdLFs4LDksIlxcdGV4dHR0e3NldH0iLDJdLFs3LDksIlxcdGV4dHR0e3NldH0iXSxbNiw3LCJcXGJyYWNrY3tcXHRleHR0dHtzZXR9LCBcXG9wZXJhdG9ybmFtZXtpZH19Il0sWzYsOCwiXFxwaV97MSwzfSIsMl1d
    \begin{tikzcd}
        {S\times A} && S && S && {S\times A} && {S\times A\times A} & {S\times A} \\
        & A &&&& S &&& {S\times A} & A
        \arrow["{\texttt{set}}", from=1-1, to=1-3]
        \arrow["{\pi_2}"', from=1-1, to=2-2]
        \arrow["{\texttt{get}}", from=1-3, to=2-2]
        \arrow["{\brackc{\operatorname{id}_S, \texttt{get}}}", from=1-5, to=1-7]
        \arrow["{\operatorname{id}_S}"', from=1-5, to=2-6]
        \arrow["{\texttt{set}}", from=1-7, to=2-6]
        \arrow["{\brackc{\texttt{set}, \operatorname{id}}}", from=1-9, to=1-10]
        \arrow["{\pi_{1,3}}"', from=1-9, to=2-9]
        \arrow["{\texttt{set}}", from=1-10, to=2-10]
        \arrow["{\texttt{set}}"', from=2-9, to=2-10]
    \end{tikzcd}
    \end{equation*}
    One can see that the object $S$ that lens perform the zooming can be decomposed as $S\to (C, A)$ where we call $C$ an residue. In similar cases, we can compose $(C, A)\to S$ to an unified object.
\end{remark}

With the motivation above, we can consider the following profunctor below (and show that it is really a profunctor). Note that we aimed keep the residue ($X$ and $Y$) the same while we make a change on other component. Furthermore, we also allow a new composite object type from $S$ to $T$.

\begin{lemma}{\textbf{(Existential Lens Profunctor)}}
    This below is a profunctor $\textbf{C}^\text{op}\times\textbf{C}\to\textbf{Set}$ in $\brackc{X, Y}$:
    \begin{equation*}
        \operatorname{Hom}_\textbf{Set}\big( S, Y\times A \big) \times \operatorname{Hom}_\textbf{Set}\big( X\times B, T \big)
    \end{equation*}
\end{lemma}
\begin{proof}
    Its action on an object $\brackc{C, D}$ is clear. So, we will consider the action of $\brackc{f, \operatorname{id}_D}$ on profunctor at $\brackc{C, D}$ where $f:E\to C$ in $\textbf{C}^\text{op}$, we have:
    \begin{equation*}
    \begin{aligned}
        \operatorname{Hom}_\textbf{Set}\big( S, D\times A \big) \times \operatorname{Hom}_\textbf{Set}&\big( C\times B, T \big) \\
        &\xmapsto{\brackc{\operatorname{id}, \brackc{-\circ\brackc{f, \operatorname{id}}, \operatorname{id}}}} \operatorname{Hom}_\textbf{Set}\big( S, D\times A \big) \times \operatorname{Hom}_\textbf{Set}\big( E\times B, T \big)
    \end{aligned}
    \end{equation*}

    This works the same way as the case of $g:D\to F$. Furthermore, since it is simply a pre/post-composition, it is obvious that it preserves the composition and identity (product is functorial too as seen in generally in proposition \ref{prop:limit-functoriality}) as we have: $\brackc{h, \operatorname{id}}\circ\brackc{f, \operatorname{id}}=\brackc{h\circ f, \operatorname{id}}$
\end{proof}

Finally, having co-ends allows us to ``sum'' all possible but still preserve the type of residual. 

\begin{definition}{\textbf{(Existential Lens)}}
    \label{def:existential-len}
    Given the profunctor above, existential lens is the co-end over its diagonal:
    \begin{equation*}
        \mathcal{L}\brackc{S, T}\brackc{A, B} = \int^{C:\textbf{C}} \operatorname{Hom}_\textbf{Set}\big( S, C\times A \big) \times \operatorname{Hom}_\textbf{Set}\big( C\times B, T \big)
    \end{equation*}
\end{definition}

\begin{remark}{(Simplification of Existential Lens)}
    Let's trying to simplify the lens, as we note that $\operatorname{Hon}_\textbf{Set}(S, C\times A)\cong \operatorname{Hom}_\textbf{Set}(S, C)\times\operatorname{Hom}_\textbf{Set}(S, A)$, then we can use Yoneda lemma:
    \begin{equation*}
    \begin{aligned}
        \mathcal{L}\brackc{S, T}\brackc{A, B} &= \int^{C:\textbf{C}} \operatorname{Hom}_\textbf{Set}\big( S, C\times A \big) \times \operatorname{Hom}_\textbf{Set}\big( C\times B, T \big) \\
        &\cong \int^{C:\textbf{C}} \operatorname{Hom}_\textbf{Set}(S, C)\times\operatorname{Hom}_\textbf{Set}(S, A) \times \operatorname{Hom}_\textbf{Set}\big( C\times B, T \big) \\
        &\cong \operatorname{Hom}_\textbf{Set}(S, A) \times \operatorname{Hom}_\textbf{Set}\big( S\times B, T \big) \\
    \end{aligned}
    \end{equation*}
    Each of the sets are $\texttt{get}:S\to A$ and $\texttt{set}:S\times B\to T$.
\end{remark}

\begin{remark}{(Get/Set Implementation of Lens Composition)}
    The essential notion of lens composition follows from the fact that we can have ``double zooming'' that is we perform a zooming of an zoomed object. This is clearly illustrated in how $\texttt{get}$ and $\texttt{set}$ are defined for composition of lens (following from \href{https://arxiv.org/abs/1809.00738}{here}):
    
    \begin{equation*}
    \begin{aligned}
        S' \xrightarrow{\texttt{get}} & \ S \xrightarrow{\texttt{get}} A \\
        S' \times A' \xrightarrow{\brackc{\operatorname{id}, \texttt{get}}}& \ (S'\times S)\times A'\xrightarrow{\operatorname{assoc}} S'\times (S\times A') \xrightarrow{\brackc{\operatorname{id}, \texttt{set}}} S'\times S_\text{new} \xrightarrow{\texttt{set}} S'_\text{new}
    \end{aligned}
    \end{equation*}
    where we have the decomposition of $S'\to (C', S)$ and $S\to(C, A)$. Observe that we didn't even consider its decomposition here and/or the residual.
\end{remark}

\begin{definition}{\textbf{(Existential Lens Composition)}}
    {\color{violet} Here, given two lens: $\mathcal{L}\brackc{S, T}\brackc{A, B}$ and $\mathcal{L}\brackc{A', B'}\brackc{A, B}$, we can define their composition as follows:}
    \begin{equation*}
    \begin{aligned}
        \mathcal{L}\brackc{S, T}\brackc{A', B'} &= \int^{\brackc{A, B}:\textbf{C}\times\textbf{C}} \mathcal{L}\brackc{S, T}\brackc{A, B}\times \mathcal{L}\brackc{A, B}\brackc{A', B'} \\
        &\cong \int^{C,C'} \operatorname{Hom}_\textbf{Set}\big( S, (C\times C')\times A' \big)\times\operatorname{Hom}_\textbf{Set}\big( (C\times C')\times B', T \big)
    \end{aligned}
    \end{equation*}
\end{definition}

\begin{remark}{(Two Versions of Existential Lens Composition)}
    {\color{violet} Note that the len can be defined as (the lens $\mathcal{L}\brackc{S, T}\brackc{A, B}$ is already defined above): }
    \begin{equation*}
    \begin{aligned}
        \mathcal{L}\brackc{A, B}\brackc{A', B'} &= \int^{C':\textbf{C}} \operatorname{Hom}_\textbf{Set}\big( A, C'\times A' \big) \times \operatorname{Hom}_\textbf{Set}\big( C'\times B', B \big) \\
    \end{aligned}
    \end{equation*}
    Then we have the following isomorphism starting from the LHS of the statement of the definition above. Note that we have used ninja Yoneda lemma of contravariant version (we didn't have to change to co-end) in the blue and cyan section, since everything is nice and functorial with Fubini theorem:
    \begin{equation*}
        \begin{aligned}
        \mathcal{L}\brackc{S, T}\brackc{A', B'} &=  
        \begin{aligned}[t]
            \int^{\brackc{A, B}:\textbf{C}\times\textbf{C}} \Bigg(&\int^{C:\textbf{C}} \operatorname{Hom}_\textbf{Set}\big( S, C\times A \big) \times \operatorname{Hom}_\textbf{Set}\big( C\times B, T \big)\Bigg) \\
            &\times \left( \int^{C':\textbf{C}} \operatorname{Hom}_\textbf{Set}\big( A, C'\times A' \big) \times \operatorname{Hom}_\textbf{Set}\big( C'\times B', B \big) \right) \\
        \end{aligned} \\
        &\cong \begin{aligned}[t]
            \int^{\brackc{C, C'}:\textbf{C}\times\textbf{C}} \Bigg(&{\int^{A:\textbf{C}}\operatorname{Hom}_\textbf{Set}(A, C'\times A')\times\operatorname{Hom}_\textbf{Set}\big( S, C\times A \big)}\Bigg) \\
            &\times \left( {\int^{B:\textbf{C}} \operatorname{Hom}_\textbf{Set}\big( A, C'\times A' \big) \times \operatorname{Hom}_\textbf{Set}\big( C'\times B', B \big)} \right) \\
        \end{aligned} \\
        &\cong \int^{\brackc{C, C'}:\textbf{C}\times\textbf{C}}\operatorname{Hom}_\textbf{Set}\big( S, C\times (C'\times A') \big)\times \operatorname{Hom}_\textbf{Set}\big( C\times ( C'\times B'), T \big) \\
        &\cong \int^{\brackc{C, C'}:\textbf{C}\times\textbf{C}}\operatorname{Hom}_\textbf{Set}\big( S, (C\times C')\times A' \big) \times \operatorname{Hom}_\textbf{Set}\big( (C\times C')\times B', T \big) 
        \end{aligned}
    \end{equation*}
\end{remark}


\begin{definition}{\textbf{(Category of Lens/Identity Lens)}}
    We can, thus, see lens as the morphism between a pair of objects, together with the compositions of lens, we can define the $\textbf{Len}$ category. It is clear that this composition is associative (inherited from the associativity of products)
    {\color{violet} and the identity lens is:}
    \begin{equation*}
    \begin{aligned}
        \mathcal{L}_\text{id}\brackc{A, B}\brackc{A, B} &= \operatorname{Hom}_\textbf{Set}\big( A, \boldsymbol{1}\times A \big) \times \operatorname{Hom}_\textbf{Set}\big( \boldsymbol{1}\times B, B \big) \\
        &\cong \operatorname{Hom}_\textbf{Set}\big( A, A \big) \times \operatorname{Hom}_\textbf{Set}\big(B, B \big) \\
    \end{aligned}
    \end{equation*}
    The intuition is that if we have no core/residual object, then the zooming will just be a full map. It is clear from our formulation of composition that this identity lens act like identity. Note that if we view $\times$ as a tensor product and initial object $\textbf{1}$ as the unit, we are one step toward generalization of the lens.
\end{definition}

\todo Skip \textbf{Lenses and Fibrations}

\section{Tambara Modules}

\begin{remark}{(Representation of Monoid)}
    We will consider a monoid here, in which we can think of it as single-object category $\textbf{M}$, with object $\star$ and the hom-set $\operatorname{Hom}_\textbf{M}(\star ,\star )$. The product is defined as the composition. Thus, representation of a monoid is a functor $F:\textbf{M}\to\textbf{Set}$ (maps $\star$ to a set that a monoid will acts on, and maps arrows $\star\to\star$ as the action on that set).
    \begin{itemize}
        \item If the functor is fully faithful, then we are happy (but this isn't usually the case). In the extreme case, the whole $\operatorname{Hom}_\textbf{M}(\star ,\star )$ is mapped to identity morphism of some set $S$. 
        \item We can also consider the whole set of representation via a functor category $[\textbf{M}, \textbf{Set}]$. The natural transformation is straightforward as it contains on component: $\alpha:F\star \to G\star $. Given $m:\star \to\star $, the naturality condition is $Gm\circ\alpha=\alpha\circ Fm$. The action on different sets should differs but still commutes.
    \end{itemize}
    We can think of the last point as finding how monoid can acts on every sets possible. This provides the motivation of Tannakian reconstruct, where we want to recover the monoid elements/structure based on how it got represented.
\end{remark}

% This goes as follows:

% \begin{remark}{(Exploration of Reconstruction)}
%     % Given the element of monoid (not the object) $m\in\operatorname{Hom}_\textbf{Set}(\star,\star)$, then we can consider natural transformation that relates the action $Fm$ and $Gm$ together. Consider a tuple whose element are from the set $\operatorname{Hom}_\textbf{Set}(F\star,F\star)$ for all functors $F:\textbf{M}\to\textbf{Set}$, where each of its elements are related each other i.e given $g\in \operatorname{Hom}_\textbf{Set}(G\star,G\star)$ and $h\in\operatorname{Hom}_\textbf{Set}(H\star,H\star)$, there is $\alpha$ such that $\alpha\circ g=h\circ\alpha$.
%     % With this, this is exactly the element of the end, 
%     With ends, we have the following wedge condition, as $\alpha$ is the morphism in functor category $[\textbf{M}, \textbf{Set}]$ as $\alpha:G\Rightarrow H$:
%     \begin{equation*}
%     % https://q.uiver.app/#q=WzAsNCxbMSwwLCJcXGludF9GXFxvcGVyYXRvcm5hbWV7SG9tfV9cXHRleHRiZntTZXR9KEZcXHN0YXIsRlxcc3RhcikiXSxbMCwxLCJcXG9wZXJhdG9ybmFtZXtIb219X1xcdGV4dGJme1NldH0oR1xcc3RhcixHXFxzdGFyKSJdLFsyLDEsIlxcb3BlcmF0b3JuYW1le0hvbX1fXFx0ZXh0YmZ7U2V0fShIXFxzdGFyLEhcXHN0YXIpIl0sWzEsMiwiXFxvcGVyYXRvcm5hbWV7SG9tfV9cXHRleHRiZntTZXR9KEdcXHN0YXIsSFxcc3RhcikiXSxbMSwzLCJcXGFscGhhXFxjaXJjLSIsMl0sWzAsMSwiXFxwaV9HIiwyXSxbMCwyLCJcXHBpX0giXSxbMiwzLCItXFxjaXJjXFxhbHBoYSJdXQ==
%     \begin{tikzcd}
%         & {\int_F\operatorname{Hom}_\textbf{Set}(F\star,F\star)} \\
%         {\operatorname{Hom}_\textbf{Set}(G\star,G\star)} && {\operatorname{Hom}_\textbf{Set}(H\star,H\star)} \\
%         & {\operatorname{Hom}_\textbf{Set}(G\star,H\star)}
%         \arrow["{\pi_G}"', from=1-2, to=2-1]
%         \arrow["{\pi_H}", from=1-2, to=2-3]
%         \arrow["{\alpha\circ-}"', from=2-1, to=3-2]
%         \arrow["{-\circ\alpha}", from=2-3, to=3-2]
%     \end{tikzcd}
%     \end{equation*}
%     That is we used the profunctor $P\brackc{G, H}=\operatorname{Hom}_\textbf{Set}(G\star,H\star)$ with signature of $P:[\textbf{M}, \textbf{Set}]^\text{op}\times[\textbf{M}, \textbf{Set}]\to\textbf{Set}$. Then, we see that given a natural transformation on such category $\alpha:G'\to G$ and $\beta:H\to H'$, we can lifted them as $P\brackc{\alpha,\beta}:P\brackc{G, H}\to P\brackc{G',H'}$ i.e $P\brackc{\alpha,\beta}=\beta\circ-\circ\alpha$. 
    
%     
% \end{remark}

\begin{theorem}{\textbf{(Tannakian Reconstruction)}}
    We can consider the hom-set between two objects by consider every of its representations combined via ends:
    \begin{equation*}
        \int_{F:[\textbf{C}, \textbf{Set}]}\operatorname{Hom}_\textbf{Set}\big( FA, FB \big) \cong \operatorname{Hom}_\textbf{C}(A, B)
    \end{equation*}
    We can see clearly that the reconstruction of monoid is the special case of this. Intuitively, it is the way to study arrows of $A\to B$ by looking at all possible representation $F$.
\end{theorem}
\begin{proof}
    Following the Yoneda lemma and with LHS, we have:
    \begin{equation*}
    \begin{aligned}
        \int_{F:[\textbf{C}, \textbf{Set}]}&\operatorname{Hom}_\textbf{Set}\bigg( \underbrace{\operatorname{Hom}_{[\textbf{C}, \textbf{Set}]}\Big( \operatorname{Hom}_\textbf{C}(A, -), F \Big)}_{FA}, \underbrace{\operatorname{Hom}_{[\textbf{C}, \textbf{Set}]}\Big( \operatorname{Hom}_\textbf{C}(B, -), F \Big)}_{FB} \bigg) \\
        &\cong \operatorname{Hom}_{[\textbf{C}, \textbf{Set}]}\Big( \operatorname{Hom}_\textbf{C}(A, -), \operatorname{Hom}_\textbf{C}(B, -) \Big) \cong \operatorname{Hom}_\textbf{C}(A, B)
    \end{aligned}
    \end{equation*}
    The second equality, we have used the Ninja Yoneda lemma (aka, the natural transformation definition of end), and we ended with the Yoneda embedding theorem.
\end{proof}

\begin{remark}{(Some Notes on Reconstruction)}
    Here, we used the profunctor $P\brackc{F, G}=\operatorname{Hom}_\textbf{Set}(FA,GB)$ with signature of $P:[\textbf{C}, \textbf{Set}]^\text{op}\times[\textbf{C}, \textbf{Set}]\to\textbf{Set}$. Then, we see that given a natural transformation $\alpha:F'\Rightarrow F$ and $\beta:G\Rightarrow G'$, we can lifted them (with appropriate component) as $P\brackc{\alpha,\beta}:P\brackc{F, G}\to P\brackc{F',G'}$ i.e $P\brackc{\alpha,\beta}=\beta_B\circ-\circ\alpha_A$. We consider the following wedge condition:
    \begin{equation*}
    % https://q.uiver.app/#q=WzAsNSxbMSwwLCJcXGludF9GXFxvcGVyYXRvcm5hbWV7SG9tfV9cXHRleHRiZntTZXR9KEZBLEZCKSJdLFswLDIsIlxcb3BlcmF0b3JuYW1le0hvbX1fXFx0ZXh0YmZ7U2V0fShHQSxHQikiXSxbMiwyLCJcXG9wZXJhdG9ybmFtZXtIb219X1xcdGV4dGJme1NldH0oSEEsSEIpIl0sWzEsMywiXFxvcGVyYXRvcm5hbWV7SG9tfV9cXHRleHRiZntTZXR9KEdBLEhCKSJdLFsxLDEsIlxcb3BlcmF0b3JuYW1le0hvbX1fXFx0ZXh0YmZ7Q30oQSxCKSJdLFsxLDMsIlxcYWxwaGFfQlxcY2lyYy0iLDJdLFswLDEsIlxccGlfRyIsMl0sWzAsMiwiXFxwaV9IIl0sWzIsMywiLVxcY2lyY1xcYWxwaGFfQSJdLFs0LDEsIkctIiwxXSxbNCwyLCJILSIsMV0sWzAsNCwiXFxjb25nIiwxXV0=
    \begin{tikzcd}
        & {\int_F\operatorname{Hom}_\textbf{Set}(FA,FB)} \\
        & {\operatorname{Hom}_\textbf{C}(A,B)} \\
        {\operatorname{Hom}_\textbf{Set}(GA,GB)} && {\operatorname{Hom}_\textbf{Set}(HA,HB)} \\
        & {\operatorname{Hom}_\textbf{Set}(GA,HB)}
        \arrow["\cong"{description}, from=1-2, to=2-2]
        \arrow["{\pi_G}"', from=1-2, to=3-1]
        \arrow["{\pi_H}", from=1-2, to=3-3]
        \arrow["{G-}"{description}, from=2-2, to=3-1]
        \arrow["{H-}"{description}, from=2-2, to=3-3]
        \arrow["{\alpha_B\circ-}"', from=3-1, to=4-2]
        \arrow["{-\circ\alpha_A}", from=3-3, to=4-2]
    \end{tikzcd}
    \end{equation*}
    Note that we can see $\pi_H$ as the applying functor $H$ to each of the element of $\operatorname{Hon}_\textbf{H}(-,=)\mapsto\operatorname{Hon}_\textbf{Set}(H-,H=)$. And so, the wedge condition encodes the natural transformation.
\end{remark}

\begin{remark}{(Back to Monoid Example)}
    Let's try to reconstruct/recover the monoid from its representation, given the set of all functions of a functor $F\star\to F\star$ i.e a hom-set. The effort will lead to the special case of Tannakian reconstruction:
    \begin{equation*}
        \int_{F:[\textbf{M}, \textbf{Set}]}\operatorname{Hom}_\textbf{Set}\big( F\star, F\star \big) \cong \operatorname{Hom}_\textbf{M}(\star ,\star )
    \end{equation*}
    Note that this has very similarity to Cayley's theorem, in which it is also the special case of Yoneda lemma.
\end{remark}

To generalize the Tannakian reconstruction further, we will consider a special functor category $\textbf{T}$, where we apply the forgetful functor to its functors. 

\begin{proposition}{\textbf{(Tannakian Reconstruction on Adjunction)}}
    \label{prop:adj-tannakan}
    Given a free/forgetful adjunction $F\dashv U$ between two functor category: $\operatorname{Hom}_{\textbf{T}}(FQ, P)\cong\operatorname{Hom}_{[\textbf{C},\textbf{Set}]}(Q,UP)$, where we have, given objects $A$ and $S$:
    \begin{equation*}
        \int_{P:\textbf{T}}\operatorname{Hom}_{\textbf{Set}}\big( (UP)A, (UP)S \big)\cong\big( \Phi\operatorname{Hom}_\textbf{C}(A, -) \big)S
    \end{equation*}
    where $\Phi=U\circ F:[\textbf{C}, \textbf{Set}]\to \textbf{T}\to[\textbf{C}, \textbf{Set}]$, a monad in a functor category (recall theorem \ref{thm:adj-to-monad}) created from the adjoint functor. Note the similarity between this and the Tannakian reconstruction above.
\end{proposition}
\begin{proof}
    Given the LHS, we follows similar procedural to the theorem above, where we apply the Yoneda lemma, use the adjunction, ninja Yoneda lemma, adjunction again, and Yoneda embedding:
    \allowdisplaybreaks
    \begin{align*}
        \int_{P:\textbf{T}}&\operatorname{Hom}_{\textbf{Set}}\big( (UP)A, (UP)S \big) \\
        &\cong \int_{P:\textbf{T}}\operatorname{Hom}_\textbf{Set}\left(\operatorname{Hom}_{[\textbf{C}, \textbf{Set}]}\Big( \operatorname{Hom}_\textbf{C}(A, -), UP \Big), \operatorname{Hom}_{[\textbf{C}, \textbf{Set}]}\Big( \operatorname{Hom}_\textbf{C}(S, -), UP \Big)\right) \\
        &\cong \int_{P:\textbf{T}}\operatorname{Hom}_\textbf{Set}\left(\operatorname{Hom}_{\textbf{T}}\Big( F\operatorname{Hom}_\textbf{C}(A, -), P \Big), \operatorname{Hom}_{\textbf{T}}\Big( F\operatorname{Hom}_\textbf{C}(S, -), P \Big)\right) \\
        &\cong \operatorname{Hom}_{\textbf{T}}\Big( F\operatorname{Hom}_\textbf{C}(S, -), F\operatorname{Hom}_\textbf{C}(A, -) \Big) \cong \operatorname{Hom}_{\textbf{T}}\Big( \operatorname{Hom}_\textbf{C}(S, -), UF\operatorname{Hom}_\textbf{C}(A, -) \Big) \\
        &\cong \big(UF\operatorname{Hom}_\textbf{C}(A, -)\big)S
    \end{align*}
\end{proof}

\begin{remark}{(Fibre Functor)}
    In the proposition above, instead of simply considering $[\textbf{C},\textbf{Set}]$, we have the specialized functor category $\textbf{T}$, and its ``specialty'' is characterized by the adjunction:
    \begin{itemize}
        \item On LHS, we have the fibre functor $P:\textbf{T}\to\textbf{Set}$ parameterize by $A$, defined as $P\mapsto(UP)A$. {\color{violet} This functor describes an ``infinitesimal neighborhood'' of an object, probing the object's environment, and that is due to $T$'s status as having more structure than a mere $[\textbf{C},\textbf{Set}]$, which describe only an singular object}.
        \item Furthermore, this ends can be seen as set of natural transformation (implicitly defined by wedge condition, proposition \ref{prop:nat-trans-as-end}) between two fibre functors (and hence we are back at the reconstructive intuition) that also involved the $\textbf{T}$.
    \end{itemize}
    The result is that when we consider view provided by a special ``filter'' of functor category $\textbf{{T}}$ (which gives rise to fibre functor), the ``image'' of arrow out of object $A$ is ``modified'' by the monad $\Phi$.
\end{remark}

\subsection{Profunctor Lens}

\begin{remark}{(Preparing Lens for Reconstruction)}
    Note that, we can define lens as follow (the first line is the original one and the second one is a more suited for our use here):
    \begin{equation*}
    \begin{aligned}
        \mathcal{L}\brackc{S, T}\brackc{A, B} &= \int^{C:\textbf{C}} \operatorname{Hom}_\textbf{Set}\big( S, C\times A \big) \times \operatorname{Hom}_\textbf{Set}\big( C\times B, T \big) \\
        &= \int^{C:\textbf{C}} \operatorname{Hom}_{\textbf{C}^\text{op}\times\textbf{C}} \Big( C\bullet \brackc{A, B}, \brackc{S, T} \Big)
    \end{aligned}
    \end{equation*}
    where the action on $\textbf{C}^\text{op}\times\textbf{C}$ can be defined generally as $\brackc{C, C'}\bullet\brackc{A, B}=\brackc{C\times A, C'\times B}$. 
\end{remark}

\begin{definition}{\textbf{(Profunctor Representation)}}
    \label{def:pro-repre}
    Given the above definition of lens, we are interested to see the co-presheaves on category $\textbf{C}^\text{op}\times\textbf{C}$ (the integrand $ \operatorname{Hom}_{\textbf{C}^\text{op}\times\textbf{C}} \big( C\bullet -,= \big)$) that is the profunctor representation.
\end{definition}

\begin{definition}{\textbf{(Iso)}}
    Using the proposition \ref{prop:adj-tannakan}, where $\textbf{T}=[\textbf{C}^\text{op}\times\textbf{C}, \textbf{Set}]$ being a category of profunctor (definition \ref{def:cat-profunctor}) without any additional structure, so $\Phi$ or forgetful functor isn't needed, and we have:
    \begin{equation*}
    \begin{aligned}
        \mathcal{O}\brackc{S, T}\brackc{A, B}&=\int_{P:\textbf{T}}\operatorname{Hom}_{\textbf{Set}}\big(P\brackc{A, B}, P\brackc{C, D} \big) \cong \operatorname{Hom}_{\textbf{C}^\text{op}\times\textbf{C}}(\brackc{A, B}, \brackc{S, T}) \\
        &\cong \operatorname{Hom}_{\textbf{C}^\text{op}}(A, S)\times \operatorname{Hom}_{\textbf{C}}(B, T) = \operatorname{Hom}_{\textbf{C}}(S, A)\times \operatorname{Hom}_{\textbf{C}}(B, T)
    \end{aligned}
    \end{equation*}

    {\color{violet} Skip the Haskell explanation for the adaptor.}
\end{definition}

\begin{remark}{(Profunctor and Lens)}
    % We define the forgetful functor $U$ to forgot the structure and doesn't change the set, that is $P\brackc{A, B}$ is the same as $(UP)\brackc{A, B}$. 
    Given the existential lens (definition \ref{def:existential-len}), we have a pair of decomposition and composition morphism i.e $\brackc{f, g}\in \operatorname{Hom}_\textbf{C}(S, C\times A)\times \operatorname{Hom}_\textbf{C}(C\times B, T)$. To build profunctor representation (definition \ref{def:pro-repre}), we need to define $P\brackc{A, B}\to P\brackc{S, T}$, in which we have:
    \begin{equation*}
    % https://q.uiver.app/#q=WzAsMyxbMCwwLCJQXFxicmFja2N7QSwgQn0iXSxbMiwwLCJQXFxicmFja2N7UywgVH0iXSxbMSwwLCJQXFxicmFja2N7Q1xcdGltZXMgQSwgQ1xcdGltZXMgQn0iXSxbMiwxLCJQXFxicmFja2N7ZiwgZ30iXSxbMCwyLCI/Pz8iLDAseyJzdHlsZSI6eyJib2R5Ijp7Im5hbWUiOiJkYXNoZWQifX19XV0=
    \begin{tikzcd}
        {P\brackc{A, B}} & {P\brackc{C\times A, C\times B}} & {P\brackc{S, T}}
        \arrow["{???}", dashed, from=1-1, to=1-2]
        \arrow["{P\brackc{f, g}}", from=1-2, to=1-3]
    \end{tikzcd}
    \end{equation*}
    Than's why we need to add more structure to have the LHS maps, defined as $\alpha_{\brackc{A, B}, C}$
\end{remark}

This leads to the concept of Tambara module. To make sure that the map is well-behaved, we are requires the family of maps to be natural in $A$ and $B$. This conditions can be emphasized on requiring $\alpha$ acts on the diagonal element (when 2 parameter are the same), which gives us:

\begin{definition}{\textbf{(Dinatural Transformation)}}
    A transformation $\alpha$ between diagonal parts of two profunctors $P$ and $Q$ is called dinatural transformation if the following diagram commutes for any $f:C\to C'$:
    \begin{equation*}
    % https://q.uiver.app/#q=WzAsNixbMSwwLCJQXFxicmFja2N7QycsIEN9Il0sWzAsMSwiUFxcYnJhY2tje0MsIEN9Il0sWzIsMSwiUFxcYnJhY2tje0MnLCBDJ30iXSxbMCwyLCJRXFxicmFja2N7QywgQ30iXSxbMiwyLCJRXFxicmFja2N7QycsIEMnfSJdLFsxLDMsIlFcXGJyYWNrY3tDLCBDJ30iXSxbMCwxLCJQXFxicmFja2N7ZixcXG9wZXJhdG9ybmFtZXtpZH1fe0N9fSIsMl0sWzEsMywiXFxhbHBoYV9DIiwyXSxbMiw0LCJcXGFscGhhX3tDJ30iXSxbMCwyLCJQXFxicmFja2N7XFxvcGVyYXRvcm5hbWV7aWR9X3tDJ30sIGZ9Il0sWzMsNSwiUFxcYnJhY2tje1xcb3BlcmF0b3JuYW1le2lkfV97Q30sIGZ9IiwyXSxbNCw1LCJQXFxicmFja2N7ZiwgXFxvcGVyYXRvcm5hbWV7aWR9X3tDfX0iXV0=
    \begin{tikzcd}
        & {P\brackc{C', C}} \\
        {P\brackc{C, C}} && {P\brackc{C', C'}} \\
        {Q\brackc{C, C}} && {Q\brackc{C', C'}} \\
        & {Q\brackc{C, C'}}
        \arrow["{P\brackc{f,\operatorname{id}_{C}}}"', from=1-2, to=2-1]
        \arrow["{P\brackc{\operatorname{id}_{C'}, f}}", from=1-2, to=2-3]
        \arrow["{\alpha_C}"', from=2-1, to=3-1]
        \arrow["{\alpha_{C'}}", from=2-3, to=3-3]
        \arrow["{P\brackc{\operatorname{id}_{C}, f}}"', from=3-1, to=4-2]
        \arrow["{P\brackc{f, \operatorname{id}_{C}}}", from=3-3, to=4-2]
    \end{tikzcd}
    \end{equation*}
\end{definition}

\begin{definition}{\textbf{(Tambara Modules)}}
    It is a profunctor $P$ equipped with the family of dinatural transformations $\alpha_{\brackc{A, B}, C}:P\brackc{A, B}\to P\brackc{C\times A, C\times B}$, that satisfies the following commutative diagram, given $f:C\to C'$
    \begin{equation*}
    % https://q.uiver.app/#q=WzAsNCxbMSwwLCJQXFxicmFja2N7QSwgQn0iXSxbMCwxLCJQXFxicmFja2N7Q1xcdGltZXMgQSwgQ1xcdGltZXMgQn0iXSxbMiwxLCJQXFxicmFja2N7QydcXHRpbWVzIEEsIEMnXFx0aW1lcyBCfSJdLFsxLDIsIlBcXGJyYWNrY3tDXFx0aW1lcyBBLCBDJ1xcdGltZXMgQn0iXSxbMCwxLCJcXGFscGhhX3tcXGJyYWNrY3tBLCBCfSwgQ30iLDFdLFsxLDMsIlBcXGJyYWNrY3tcXG9wZXJhdG9ybmFtZXtpZH1fe0NcXHRpbWVzIEF9LCBcXGJyYWNrY3tmLFxcb3BlcmF0b3JuYW1le2lkfV9CfX0iLDFdLFswLDIsIlxcYWxwaGFfe1xcYnJhY2tje0EsIEJ9LCBDJ30iLDFdLFsyLDMsIlBcXGJyYWNrY3tcXGJyYWNrY3tmLFxcb3BlcmF0b3JuYW1le2lkfV97QX19LCBcXG9wZXJhdG9ybmFtZXtpZH1fe0NcXHRpbWVzIEJ9fSIsMV1d
    \begin{tikzcd}
        & {P\brackc{A, B}} \\
        {P\brackc{C\times A, C\times B}} && {P\brackc{C'\times A, C'\times B}} \\
        & {P\brackc{C\times A, C'\times B}}
        \arrow["{\alpha_{\brackc{A, B}, C}}"{description}, from=1-2, to=2-1]
        \arrow["{\alpha_{\brackc{A, B}, C'}}"{description}, from=1-2, to=2-3]
        \arrow["{P\brackc{\operatorname{id}_{C\times A}, \brackc{f,\operatorname{id}_B}}}"{description}, from=2-1, to=3-2]
        \arrow["{P\brackc{\brackc{f,\operatorname{id}_{A}}, \operatorname{id}_{C\times B}}}"{description}, from=2-3, to=3-2]
    \end{tikzcd}
    \end{equation*}
    To make this more explicit, we consider the profunctor $Q$ being $P\brackc{-\times A,-\times B}$. It also has to respect the monoidal structure of product (see remark \ref{remark:tambara-ext}). To the unit (terminal object), we have: $\operatorname{\alpha}_{\brackc{A, B},\textbf{1}}=\operatorname{id}_{P\brackc{A, B}}$. For multiplication, we have, up to isomorphism:
    \begin{equation*}
    % https://q.uiver.app/#q=WzAsMyxbMCwwLCJQXFxicmFja2N7QSwgQn0iXSxbMiwwLCJQXFxicmFja2N7QydcXHRpbWVzIENcXHRpbWVzIEEsIEMnXFx0aW1lcyBDXFx0aW1lcyBCfSJdLFsxLDEsIlBcXGJyYWNrY3tDXFx0aW1lcyBBLCBDXFx0aW1lcyBCfSJdLFswLDEsIlxcYWxwaGFfe1xcYnJhY2tje0EsIEJ9LCBDJ1xcdGltZXMgQ30iXSxbMCwyLCJcXGFscGhhX3tcXGJyYWNrY3tBLCBCfSwgQ30iLDFdLFsyLDEsIlxcYWxwaGFfe1xcYnJhY2tje0NcXHRpbWVzIEEsIENcXHRpbWVzIEJ9LCBDJ30iLDFdXQ==
    \begin{tikzcd}
        {P\brackc{A, B}} && {P\brackc{C'\times C\times A, C'\times C\times B}} \\
        & {P\brackc{C\times A, C\times B}}
        \arrow["{\alpha_{\brackc{A, B}, C'\times C}}", from=1-1, to=1-3]
        \arrow["{\alpha_{\brackc{A, B}, C}}"{description}, from=1-1, to=2-2]
        \arrow["{\alpha_{\brackc{C\times A, C\times B}, C'}}"{description}, from=2-2, to=1-3]
    \end{tikzcd}
    \end{equation*}
\end{definition}

\begin{definition}{\textbf{(Category of Tambara Module)}}
    \label{def:tambara-module-cat}
    We define a category of Tabara module with morphism being natural transformation that preserves the added structure. Given $\rho:(P,\alpha)\to(Q,\beta)$, the following diagram commutes:
    \begin{equation*}
    % https://q.uiver.app/#q=WzAsNCxbMCwwLCJQXFxicmFja2N7QSwgQn0iXSxbMCwxLCJRXFxicmFja2N7QSwgQn0iXSxbMiwwLCJQXFxicmFja2N7Q1xcdGltZXMgQSwgQ1xcdGltZXMgQn0iXSxbMiwxLCJRXFxicmFja2N7Q1xcdGltZXMgQSwgQ1xcdGltZXMgQn0iXSxbMCwxLCJcXHJob197XFxicmFja2N7QSwgQn19IiwyXSxbMCwyLCJcXGFscGhhX3tcXGJyYWNrY3tBLCBCfSwgQ30iXSxbMSwzLCJcXGJldGFfe1xcYnJhY2tje0EsIEJ9LCBDfSIsMl0sWzIsMywiXFxyaG9fe1xcYnJhY2tje0NcXHRpbWVzIEEsIENcXHRpbWVzIEJ9fSJdXQ==
    \begin{tikzcd}
        {P\brackc{A, B}} && {P\brackc{C\times A, C\times B}} \\
        {Q\brackc{A, B}} && {Q\brackc{C\times A, C\times B}}
        \arrow["{\alpha_{\brackc{A, B}, C}}", from=1-1, to=1-3]
        \arrow["{\rho_{\brackc{A, B}}}"', from=1-1, to=2-1]
        \arrow["{\rho_{\brackc{C\times A, C\times B}}}", from=1-3, to=2-3]
        \arrow["{\beta_{\brackc{A, B}, C}}"', from=2-1, to=2-3]
    \end{tikzcd}
    \end{equation*}
\end{definition}

\begin{remark}{(What we been so far: Reconstruction Lens)}
    We consider the Tannakian construction (and what we expected lens to be), where $\textbf{T}$ is the Tambara modules:
    \begin{equation*}
        \int_{P:\textbf{T}}\operatorname{Hom}_{\textbf{Set}}\Big( (UP)\brackc{A, B}, (UP)\brackc{S, T} \Big)\cong\big( \Phi\operatorname{Hom}_{\textbf{C}^\text{op}\times\textbf{C}}(\brackc{A, B}, -) \big)\brackc{S, T}
    \end{equation*}
    Since we already know $U$ as the forgetful functor of Tambara category. We are left to consider the monad $\Phi$ or the freely generated Tambara module functor $F$.
\end{remark}

\begin{remark}{(Comonad for Lens)}
    It's easier to guess the comonad. Consider the comonad in category of profunctors where, it is defined as:
    \begin{equation*}
        (\Theta P)\brackc{A, B} = \int_{C}P\brackc{C\times A, C\times B}
    \end{equation*}   
    where $\varepsilon_P: \Theta P\to P$ is given to be $\pi_{\textbf{1}}$, while $\nu_P:\Theta P\to\Theta(\Theta P)$ is {\color{violet} defined via universal construction } (see remark below for more details).
\end{remark}

\begin{remark}{(Comonad Equipped Maps)}
    There are 2 map of for the comonad, which are:
    \begin{itemize}
        \item $\boldsymbol{\varepsilon_P: \Theta P\to P}$ is given to be $\pi_{\textbf{1}}$ where $\textbf{1}$ is the terminal object. In which we have:
        \begin{equation*}
            \begin{aligned}
                \pi_{\textbf{1}}\left( \int_{C}P\brackc{C\times A, C\times B} \right) &= P\brackc{\textbf{1}\times A, \textbf{1}\times B} \\
                &\cong P\brackc{A, B}
            \end{aligned}
            \qquad\quad
            % https://q.uiver.app/#q=WzAsNCxbMCwwLCJcXGludF97Q31QXFxicmFja2N7Q1xcdGltZXMgQSwgQ1xcdGltZXMgQn0iXSxbMCwxLCJcXGludF97Q31RXFxicmFja2N7Q1xcdGltZXMgQSwgQ1xcdGltZXMgQn0iXSxbMSwwLCJQXFxicmFja2N7QSwgQn0iXSxbMSwxLCJRXFxicmFja2N7QSwgQn0iXSxbMiwzLCJcXGFscGhhX3tBLCBCfSJdLFswLDEsIiIsMCx7InN0eWxlIjp7ImJvZHkiOnsibmFtZSI6ImRhc2hlZCJ9fX1dLFsxLDMsIlxccGlfMSciLDJdLFswLDIsIlxccGlfMSJdXQ==
            \begin{tikzcd}
                {\int_{C}P\brackc{C\times A, C\times B}} & {P\brackc{A, B}} \\
                {\int_{C}Q\brackc{C\times A, C\times B}} & {Q\brackc{A, B}}
                \arrow["{\pi_1}", from=1-1, to=1-2]
                \arrow[dashed, from=1-1, to=2-1]
                \arrow["{\alpha_{A, B}}", from=1-2, to=2-2]
                \arrow["{\pi_1'}"', from=2-1, to=2-2]
            \end{tikzcd}
        \end{equation*}
        To show that it is a natural transformation, consider the morphism $\alpha:P\Rightarrow Q$, with its component, for any objects $A, B$ be $\alpha_{A, B}:P\brackc{A, B}\to Q\brackc{A, B}$. The map between ends follows from functoriality of ends (proposition \ref{prop:functoriality-of-ends}), and RHS diagram should commutes per construction. 
        \item$\boldsymbol{\nu_P:\Theta P\to\Theta(\Theta P)}$ where its designation is:
        \begin{equation*}
            \Theta(\Theta P)\brackc{A, B} = \Theta\left(\int_{C}P\brackc{C\times A, C\times B}\right) = \int_{C'}\int_C P\brackc{C'\times C\times A, C'\times C\times B}
        \end{equation*}
        Since that the maps between $\Theta P\to\Theta(\Theta P)$ can be defined via universal properties of Ends, and the naturality and commutative are guaranteed by the universal properties, see \texttt{etc.tex} for the diagram.
    \end{itemize}
\end{remark}

\begin{remark}{(Tambara module as Coalgebra)}
    Tambara module is actually the coalgebra of the len's comonad (as it adds a $C\times-$), where the structure map of coalgebra $i_P:P\to\Theta P$ is defined to be (it commutes because of dinatural transformation condition of Tambara module), given any morphism $f:X\to Y$:
    \begin{equation*}
    % https://q.uiver.app/#q=WzAsNSxbMSwxLCJcXGludF97Q31QXFxicmFja2N7Q1xcdGltZXMgQSwgQ1xcdGltZXMgQn0iXSxbMSwwLCJQXFxicmFja2N7QSwgQn0iXSxbMCwxLCJQXFxicmFja2N7WFxcdGltZXMgQSwgWFxcdGltZXMgQn0iXSxbMiwxLCJQXFxicmFja2N7WVxcdGltZXMgQSwgWVxcdGltZXMgQn0iXSxbMSwyLCJQXFxicmFja2N7WFxcdGltZXMgQSwgWVxcdGltZXMgQn0iXSxbMSwwLCJpX1AiLDAseyJzdHlsZSI6eyJib2R5Ijp7Im5hbWUiOiJkYXNoZWQifX19XSxbMSwyLCJcXGFscGhhX3tcXGJyYWNrY3tBLCBCfSwgWH0iLDFdLFsxLDMsIlxcYWxwaGFfe1xcYnJhY2tje0EsIEJ9LCBZfSIsMV0sWzAsMl0sWzIsNCwiUFxcYnJhY2tje1xcb3BlcmF0b3JuYW1le2lkfV97Q1xcdGltZXMgQX0sIFxcYnJhY2tje2YsXFxvcGVyYXRvcm5hbWV7aWR9X0J9fSIsMV0sWzMsNCwiUFxcYnJhY2tje1xcYnJhY2tje2YsXFxvcGVyYXRvcm5hbWV7aWR9X3tBfX0sIFxcb3BlcmF0b3JuYW1le2lkfV97Q1xcdGltZXMgQn19IiwxXSxbMCwzXV0=
    \begin{tikzcd}
        & {P\brackc{A, B}} \\
        {P\brackc{X\times A, X\times B}} & {\int_{C}P\brackc{C\times A, C\times B}} & {P\brackc{Y\times A, Y\times B}} \\
        & {P\brackc{X\times A, Y\times B}}
        \arrow["{\alpha_{\brackc{A, B}, X}}"{description}, from=1-2, to=2-1]
        \arrow["{i_P}", dashed, from=1-2, to=2-2]
        \arrow["{\alpha_{\brackc{A, B}, Y}}"{description}, from=1-2, to=2-3]
        \arrow["{P\brackc{\operatorname{id}_{C\times A}, \brackc{f,\operatorname{id}_B}}}"{description}, from=2-1, to=3-2]
        \arrow[from=2-2, to=2-1]
        \arrow[from=2-2, to=2-3]
        \arrow["{P\brackc{\brackc{f,\operatorname{id}_{A}}, \operatorname{id}_{C\times B}}}"{description}, from=2-3, to=3-2]
    \end{tikzcd}
    \end{equation*}
    Or, we can define $i:P\Rightarrow\Theta P$, in a more compacted manners using (proposition \ref{prop:nat-trans-as-end}), as:
    \begin{equation*}
        \int_{A, B}\operatorname{Hom}_{\textbf{Set}}\big(P\brackc{A, B}, (\Theta P)\brackc{A, B}\big) \cong \int_{A, B}\int_C\operatorname{Hom}_{\textbf{Set}}\big(P\brackc{A, B}, P\brackc{C\times A, C\times B}\big) 
    \end{equation*}
    
    In fact Tambara modules form the Eilenberg-Moore category of coalgebras (definition \ref{def:coalg-cat}) for the comonad $\Theta$. That is because the morphism between Tabara modules respects the structure (see definition \ref{def:tambara-module-cat}) in blue:
    \begin{equation*}
    % https://q.uiver.app/#q=WzAsNixbMCwzLCJcXGludF97Q31QXFxicmFja2N7Q1xcdGltZXMgQSwgQ1xcdGltZXMgQn0iXSxbMCwyLCJQXFxicmFja2N7QSwgQn0iXSxbMSwzLCJQXFxicmFja2N7WVxcdGltZXMgQSwgWVxcdGltZXMgQn0iXSxbMSwwLCJRXFxicmFja2N7QSwgQn0iXSxbMiwxLCJRXFxicmFja2N7WVxcdGltZXMgQSwgWVxcdGltZXMgQn0iXSxbMSwxLCJcXGludF97Q31RXFxicmFja2N7Q1xcdGltZXMgQSwgQ1xcdGltZXMgQn0iXSxbMSwwLCJpX1AiLDAseyJzdHlsZSI6eyJib2R5Ijp7Im5hbWUiOiJkYXNoZWQifX19XSxbMSwyLCJcXGFscGhhX3tcXGJyYWNrY3tBLCBCfSwgWX0iLDEseyJjb2xvdXIiOlsyNDAsNjAsNjBdfSxbMjQwLDYwLDYwLDFdXSxbMCwyLCIiLDAseyJjb2xvdXIiOlswLDYwLDYwXX1dLFsxLDMsIlxccmhvX3tcXGJyYWNrY3tBLCBCfX0iLDAseyJjb2xvdXIiOlsyNDAsNjAsNjBdfSxbMjQwLDYwLDYwLDFdXSxbMyw1LCIiLDAseyJzdHlsZSI6eyJib2R5Ijp7Im5hbWUiOiJkYXNoZWQifX19XSxbMyw0LCJcXGJldGFfe1xcYnJhY2tje0EsIEJ9LCBZfSIsMCx7ImNvbG91ciI6WzI0MCw2MCw2MF19LFsyNDAsNjAsNjAsMV1dLFswLDUsIiIsMSx7InN0eWxlIjp7ImJvZHkiOnsibmFtZSI6ImRhc2hlZCJ9fX1dLFsyLDQsIlxccmhvX3tcXGJyYWNrY3tZXFx0aW1lcyBBLCBZXFx0aW1lcyBCfX0iLDIseyJjb2xvdXIiOlswLDYwLDYwXX0sWzAsNjAsNjAsMV1dLFs1LDRdXQ==
    \begin{tikzcd}
        & {Q\brackc{A, B}} \\
        & {\int_{C}Q\brackc{C\times A, C\times B}} & {Q\brackc{Y\times A, Y\times B}} \\
        {P\brackc{A, B}} \\
        {\int_{C}P\brackc{C\times A, C\times B}} & {P\brackc{Y\times A, Y\times B}}
        \arrow[dashed, from=1-2, to=2-2]
        \arrow["{\beta_{\brackc{A, B}, Y}}", color={rgb,255:red,92;green,92;blue,214}, from=1-2, to=2-3]
        \arrow[from=2-2, to=2-3]
        \arrow["{\rho_{\brackc{A, B}}}", color={rgb,255:red,92;green,92;blue,214}, from=3-1, to=1-2]
        \arrow["{i_P}", dashed, from=3-1, to=4-1]
        \arrow["{\alpha_{\brackc{A, B}, Y}}"{description}, color={rgb,255:red,92;green,92;blue,214}, from=3-1, to=4-2]
        \arrow[dashed, from=4-1, to=2-2]
        \arrow[color={rgb,255:red,214;green,92;blue,92}, from=4-1, to=4-2]
        \arrow["{\rho_{\brackc{Y\times A, Y\times B}}}"', color={rgb,255:red,214;green,92;blue,92}, from=4-2, to=2-3]
    \end{tikzcd}
    \end{equation*}

    Note that the maps $\int_CP\brackc{C\times A,C\times B}\to\int_CQ\brackc{C\times A,C\times B}$ can be derived from the fact that we have the maps from $\int_CP\brackc{C\times A,C\times B}\to Q\brackc{Y\times A, Y\times B}$ colored in red, and so on.

    % This means that we are ready to find the left adjoint to $\Theta$ being the monad $\Phi$}.
\end{remark}

\begin{proposition}{\textbf{(Monad for Lens)}}
    The monad for lens (defined on the second line) is a left adjoint of the $\Theta$ comonad defined above:
    \begin{equation*}
    \begin{aligned}
        \operatorname{Hom}_{[\textbf{C}^\text{op}\times\textbf{C}, \textbf{Set}]}\big( &\Phi P, Q \big) \cong \operatorname{Hom}_{[\textbf{C}^\text{op}\times\textbf{C}, \textbf{Set}]}\left( P, \Theta Q \right) \\
        \text{ where }&(\Phi P)\brackc{S, T} = \int^{U, V, C} \operatorname{Hom}_{\textbf{C}^\text{op}\times\textbf{C}}\big( C\bullet\brackc{U, V}, \brackc{S, T} \big) \times P\brackc{U, V} \\
    \end{aligned}
    \end{equation*}
    where $\operatorname{Hom}_{\textbf{C}^\text{op}\times\textbf{C}}\big( C\bullet\brackc{U, V}, \brackc{S, T} \big)=\operatorname{Hom}_{\textbf{C}}(S, C\times U)\times\operatorname{Hom}_\textbf{C}(C\times V, T)$.
\end{proposition}
\begin{proof}
    The second to last isomorphism is the double application of co-Yoneda lemma and Fubini theorem. The last one is the natural transformation formula in ends with the continuity of hom-functor.
    \begin{equation*}
    \begin{aligned}
        \operatorname{Hom}&{}_{[\textbf{C}^\text{op}\times\textbf{C}, \textbf{Set}]}\Bigg(\int^{U, V, C} \operatorname{Hom}_{\textbf{C}}(-, C\times U)\times\operatorname{Hom}_\textbf{C}(C\times V, =)\times P\brackc{U, V}, Q\brackc{-, =}\Bigg) \\  
        &\cong \int_{U, V, C}\operatorname{Hom}_{[\textbf{C}^\text{op}\times\textbf{C}, \textbf{Set}]}\Bigg( \operatorname{Hom}_{\textbf{C}}(-, C\times U)\times P\brackc{U, V}\times\operatorname{Hom}_\textbf{C}(C\times V, =), Q\brackc{-, =}\Bigg) \\  
        &\cong \int_{U, V, C}\int_{\brackc{A,B}:\textbf{C}^\text{op}\times\textbf{C}}\operatorname{Hom}_\textbf{Set}\Bigg( \operatorname{Hom}_{\textbf{C}}(A, C\times U)\times P\brackc{U, V}\times\operatorname{Hom}_\textbf{C}(C\times V, B), Q\brackc{A, B}\Bigg) \\  
        &\cong \int_{U, V, C}\operatorname{Hom}_\textbf{Set}\Bigg(P\brackc{U, V}, Q\brackc{C\times U, C\times V}\Bigg) \cong \operatorname{Hom}_{[\textbf{C}^\text{op}\times\textbf{C}, \textbf{Set}]}\Bigg(P\brackc{-, =}, \int_{C}Q\brackc{C\times -, C\times =}\Bigg)  \\  
    \end{aligned}
    \end{equation*}
\end{proof}

\begin{remark}{(Equivalence between Algebra and Co-Algebra)}
    Given result above, if we replace $Q$ with $P$, then the set of (monadic) algebras for $\Phi$ is isomorphism to set of coalgebra for $\Theta$. This means that the Eilenberg-Moore category for the monad $\Phi$ is the same as Tambara category.
\end{remark}

\begin{corollary}
    The action of $\Phi$ on the representable functor is lens i.e:
    \begin{equation*}
        \big( \Phi\operatorname{Hom}_{\textbf{C}^\text{op}\times\textbf{C}}(\brackc{A, B}, -) \big)\brackc{S, T} \cong \int^{C:\textbf{C}} \operatorname{Hom}_\textbf{Set}\big( S, C\times A \big) \times \operatorname{Hom}_\textbf{Set}\big( C\times B, T \big) 
    \end{equation*}
\end{corollary}
\begin{proof}
    We have the following isomorphism (using the co-Yoneda lemma):
    \begin{equation*}
    \begin{aligned}
        \big( \Phi\operatorname{Hom}&{}_{\textbf{C}^\text{op}\times\textbf{C}}(\brackc{A, B}, -) \big)\brackc{S, T}  \\
        &= \int^{U, V, C} \operatorname{Hom}_{\textbf{C}}(S, C\times U)\times\operatorname{Hom}_\textbf{C}(C\times V, T) \times \operatorname{Hom}_{\textbf{C}^\text{op}\times\textbf{C}}(\brackc{A, B}, \brackc{U, V}) \\
        &\cong \int^{U, V, C} \operatorname{Hom}_{\textbf{C}}(S, C\times U)\times\operatorname{Hom}_\textbf{C}(C\times V, T) \times \operatorname{Hom}_{\textbf{C}}(U, A)\times\operatorname{Hom}_{\textbf{C}}(B, V) \\
        &\cong \int^{C:\textbf{C}} \operatorname{Hom}_\textbf{Set}\big( S, C\times A \big) \times \operatorname{Hom}_\textbf{Set}\big( C\times B, T \big) 
    \end{aligned}
    \end{equation*}
\end{proof}

\subsection{General Optics}

\begin{remark}{(Monoidal Category Extension of Tambara Module)}
    \label{remark:tambara-ext}
    Given the Tambara module, instead of using a product $\times$, we can extends to use a tensor product $\otimes$ and unit object $I$. That is the structure map is:
    \begin{equation*}
        \alpha_{A, B, C} : P\brackc{A, B} \to P\brackc{C\otimes A, C\otimes B} 
    \end{equation*}
    And, the rest of the construction should be the same.
\end{remark}

\begin{definition}{\textbf{(Prism)}}
    Given the tensor product to be a co-product, we have defined prism, where it can be defined (and simplified) as:
    \begin{equation*}
    \begin{aligned}
        \mathcal{P}\brackc{S, T}\brackc{A, B} &= \int^{C:\textbf{C}} \operatorname{Hom}_\textbf{Set}\big( S, C + A \big) \times \operatorname{Hom}_\textbf{Set}\big( C+ B, T \big) \\
        &\cong \int^{C:\textbf{C}} \operatorname{Hom}_\textbf{Set}\big( S, C + A \big) \times \operatorname{Hom}_\textbf{Set}\big( C, T \big) \times  \operatorname{Hom}_\textbf{Set}\big( B, T \big) \\
        &\cong \operatorname{Hom}_\textbf{Set}\big( S, T + A \big) \times \operatorname{Hom}_\textbf{Set}\big( B, T \big)
    \end{aligned}
    \end{equation*}
    Thus, we have the pair of functions $\texttt{match}:S\to T+A$ and $\texttt{build}:B\to T$. In the form of existential form (Ends), we have $S$ is either the focus $A$ or the residue $C$, and $T$ can be built from new focus $B$ or residual $C$
    
    \todo Adding Haskell logic here to illustrate.
\end{definition}

\begin{remark}{(Traversal)}
    It is a type of optic that focuses on multiple foci at once. Consider the case of traversing the tree, as the traversal can gives you the list of the nodes and the replacement is possible. The issue is that we need to keep track of the length (requires the dependent type).
\end{remark}





\bibliographystyle{plain} % We choose the "plain" reference style
\bibliography{refs} % Entries are in the refs.bib file


\end{document}
