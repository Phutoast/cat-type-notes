\section{Homotopy Type Theory: The Logic of Space}

\begin{remark}{(Synthetic Space)}
    The idea of synthetic spaces is that if all objects in mathematics come naturally with spatial structure, then it is perverse to insist on defining them first in terms of bare sets, and then equipping them with spatial structure. We may replace formal system whose basic objects are spaces.
    \begin{itemize}
        \item \textbf{(Minimal Change)} Since spaces admit most of the same constructions that sets do, we can develop mathematics in such a system with very few changes to its outward appearance, but all the desired spatial structure will automatically be present and preserved.
        \item \textbf{(Generality)} If our formal system is sufficiently general, then its objects will be interpretable as many different kinds of space; thus the same theorems about “groups” will apply to topological groups
    \end{itemize}
    \textbf{Martin-Lof dependent type theory} has these properties, which admit ``spatial'' interpretation. We note that the connection between constructivity/computability and topology/continuity goes back at least to Brouwer. Each ``kind of space'' forms a topos and we can construct ``free topos'' from syntax of type theory.
\end{remark}

\subsection{Type Theory}

\begin{remark}{(Syntax + Semantics)}
    Type theory relies heavy on syntax, a system of formal symbols of some sort, whereas semantics means the interpretation of those symbols as ``things''. In category, we described syntax as free object and semantics as the morphisms out of that object determined by its universal property.

    For example, consider sequence of equation: $(gh)h^{-1}=g(hh^{-1})=ge=g$, the computation can takes place in free group $F\langle g,h\rangle$ generated by $g$ and $h$. 
    \begin{itemize}
        \item (Syntax): Since the element of $F\langle g,h\rangle$ are strings produced by multiplication and inversion $(gh)h^{-1}\in F\langle g,h\rangle$. The sequence of equations holds between these element i.e statement in syntax. 
        \item (Semantics): If we have another group $G$ and 2 elements in it, then there is unique group homomorphism between $F\langle g,h\rangle\to G$. This is the semantics of our syntax.
        \item (Hypothesis) We can also have: If $g^2=e$, then $g^4=(g^2)^2=e^2=e$ by consider the group $F\langle g| g^2=e\rangle$.
    \end{itemize}
    The argument at the start is either a ``semantic'' statement about ``all groups'' or ``syntactic'' statement about a particular free or presented group. The former is consequence of the latter, by the universal property of free groups and presentations.
\end{remark}

\begin{remark}{(Represeting Free Group)}
    We can say more about a free object than is expressed tautologically by its universal property. Usually, This takes the form of an explicit and tractable construction of an object that is then proven to be free. Of course, any construction of a free object must be proven correct can be highly non-trivial. Consider defining $F\langle g,h\rangle$ in following ways:

    \begin{itemize}
        \item \textbf{(Trivial):} The ``tautological'' was is to ``throwing in freely'' the group operations obtaining formal words and then quotienting by an equivalence relation generated by the axioms of a group. The universal property of a free group is then essentially immediate. 
        \item \textbf{(More Useful):} $F\langle g,h\rangle$ contains ``reduced words'' in $g$ and $h$ and their formal inverse (finite sequence in which no cancellation is possible) with multiplication by concatenation and cancellation. This yields a free group that is not entirely trivial, even the definition of the group multiplication is not completely trivial. But once we know it, it can simplify our lives.
    \end{itemize}

    Let's see how differences they are. Consider the proof that conjugation by $g$ is group homomorphism. We denote conjugation of $h$ by $g$ to be $h^g$ and so $h^gk^g=(ghg^{-1})(gkg^{-1}) = ghkg^{-1}(hk)^g$
    \begin{itemize}
        \item But to do it with axioms of a group per trivial approach, we have to choose parenthesizations and use the associativity and unit axioms explicitly, giving us long sequence. 
        \item On the other hand, by using the second description of free groups, we can make formal sense of this as calculation of $F\langle g,h,k\rangle$ where $ghg^{-1}$ and $gkg^{-1}$ are specific elements whose product is $ghkg^{-1}$, and extend this to every other group by freeness.
    \end{itemize}

    There are several ways to present a free object, which make our lives easier. If its elements are ``canonical forms'' (for example, $ghkg^{-1}$ is the canonical form of $((gh)g^{-1})((gk)g^{-1})$), we can simplify the quotient. Often there is a ``reduction'' algorithm to compute canonical forms, making equality in the free object decidable. 
\end{remark}

\begin{remark}{(Further Advantages)}
    Another potential advantage is if we obtain a ``version'' of a free object that is actually simpler. This is particularly common in category theory and higher category theory, where it can be called a coherence theorem.

    A particular construction of a free object might also be psychologically easier to work with, or at least suggest a different viewpoint. The best example is type theory itself, which also offers the advantages of canonical forms and strictness.
\end{remark}

\begin{remark}{(Type Theory and Mathematical Universes)}
    Type theory, roughly speaking, is a convenient construction of free objects for the theory of all of mathematics. Type theory with certain structures presents ``a universe of mathematics'' with a unique ``mathematics homomorphism'' to any other such universe of mathematics. 

    \begin{itemize}
        \item \textbf{(Analogy):} Just like $F\langle g,h|gh=hg\rangle$ admits unique homomorphism to any other group equipped with two commuting elements. 
        \item \textbf{(Universes of Mathematics):} Some formal systems, like Zermelo-Fraenkel set theory, can encode almost all mathematics. And, for a first approximation, universe of mathematics means model of a formal system in which mathematics can be encoded. 
        \item \textbf{(Multiple Universes):} By Godel's incompleteness theorem: any sufficiently powerful formal system contains statements that are neither provable nor disprovable. And, the completeness theorem then implies that there must be some models in which these statements are true and some in which they are false. So, it ensures that any such system has many different models
    \end{itemize}
    
    We can make use of it by construct another universe whose objects are, from the PoV of our original universe, ``spaces''. Thus, when a mathematician in this new world constructs a bare function $A\to B$ between sets, the mathematician in the old world sees as a continuous function between spaces
\end{remark}

\begin{remark}{(Concrete Example):}
    A second approximation of ``universe of mathematics'' is a category or $(\infty,1)$-category with certain structure. Our starting universe is then the category of sets. Type theory gives a way to construct free or presented objects in some category of structured categories. These are called ``syntactic category'' or ``classifying category''. 

    If $G$ is a group object in a category, then the sequence $(gh)h^{-1}=g(hh^{-1})=ge=g$ can be given in the commutativity of the following diagram:
    \begin{equation*}
    % https://q.uiver.app/#q=WzAsNyxbMCwwLCJHXFx0aW1lcyBHIl0sWzIsMCwiR1xcdGltZXMgR1xcdGltZXMgRyJdLFs0LDAsIkdcXHRpbWVzIEdcXHRpbWVzIEciXSxbNiwwLCJHXFx0aW1lcyBHIl0sWzQsMSwiR1xcdGltZXMgRyJdLFswLDEsIkciXSxbNiwxLCJHIl0sWzAsMSwiMVxcdGltZXNcXERlbHRhIl0sWzEsMiwiMVxcdGltZXMxXFx0aW1lc1xcb3BlcmF0b3JuYW1le2ludn0iXSxbMiwzLCJcXG9wZXJhdG9ybmFtZXttdWx0fVxcdGltZXMxIl0sWzIsNCwiMVxcdGltZXNcXG9wZXJhdG9ybmFtZXttdWx0fSIsMV0sWzAsNSwiXFxvcGVyYXRvcm5hbWV7cHJvan1fMSIsMl0sWzUsNCwiMVxcdGltZXNcXG9wZXJhdG9ybmFtZXtpZH0iXSxbNCw2LCJcXG9wZXJhdG9ybmFtZXttdWx0fSJdLFszLDYsIlxcb3BlcmF0b3JuYW1le211bHR9Il0sWzUsNiwiMSIsMix7ImN1cnZlIjozfV1d
    \begin{tikzcd}
        {G\times G} && {G\times G\times G} && {G\times G\times G} && {G\times G} \\
        G &&&& {G\times G} && G
        \arrow["1\times\Delta", from=1-1, to=1-3]
        \arrow["{1\times1\times\operatorname{inv}}", from=1-3, to=1-5]
        \arrow["{\operatorname{mult}\times1}", from=1-5, to=1-7]
        \arrow["{1\times\operatorname{mult}}"{description}, from=1-5, to=2-5]
        \arrow["{\operatorname{proj}_1}"', from=1-1, to=2-1]
        \arrow["{1\times\operatorname{id}}", from=2-1, to=2-5]
        \arrow["{\operatorname{mult}}", from=2-5, to=2-7]
        \arrow["{\operatorname{mult}}", from=1-7, to=2-7]
        \arrow["1"', curve={height=18pt}, from=2-1, to=2-7]
    \end{tikzcd}
    \end{equation*}
    In the elemental form, we have the following maps between objects (up to isomorphism):
    \begin{equation*}
    % https://q.uiver.app/#q=WzAsNyxbMCwwLCIoZyxoKSJdLFsyLDAsIihnLChoLGgpKSJdLFs0LDAsIihnLChoLGheey0xfSkpIl0sWzYsMCwiKChnLGgpaF57LTF9KSJdLFs0LDEsIihnLCBlKSJdLFswLDEsImciXSxbNiwxLCJnIl0sWzAsMSwiMVxcdGltZXNcXERlbHRhIiwwLHsic3R5bGUiOnsidGFpbCI6eyJuYW1lIjoibWFwcyB0byJ9fX1dLFsxLDIsIjFcXHRpbWVzMVxcdGltZXNcXG9wZXJhdG9ybmFtZXtpbnZ9IiwwLHsic3R5bGUiOnsidGFpbCI6eyJuYW1lIjoibWFwcyB0byJ9fX1dLFsyLDMsIlxcb3BlcmF0b3JuYW1le211bHR9XFx0aW1lczEiLDAseyJzdHlsZSI6eyJ0YWlsIjp7Im5hbWUiOiJtYXBzIHRvIn19fV0sWzIsNCwiMVxcdGltZXNcXG9wZXJhdG9ybmFtZXttdWx0fSIsMSx7InN0eWxlIjp7InRhaWwiOnsibmFtZSI6Im1hcHMgdG8ifX19XSxbNSw0LCIxXFx0aW1lc1xcb3BlcmF0b3JuYW1le2lkfSIsMCx7InN0eWxlIjp7InRhaWwiOnsibmFtZSI6Im1hcHMgdG8ifX19XSxbNCw2LCJcXG9wZXJhdG9ybmFtZXttdWx0fSIsMCx7InN0eWxlIjp7InRhaWwiOnsibmFtZSI6Im1hcHMgdG8ifX19XSxbMyw2LCJcXG9wZXJhdG9ybmFtZXttdWx0fSIsMCx7InN0eWxlIjp7InRhaWwiOnsibmFtZSI6Im1hcHMgdG8ifX19XSxbNSw2LCIxIiwyLHsiY3VydmUiOjMsInN0eWxlIjp7InRhaWwiOnsibmFtZSI6Im1hcHMgdG8ifX19XSxbMCw1LCJcXG9wZXJhdG9ybmFtZXtwcm9qfV8xIiwyLHsic3R5bGUiOnsidGFpbCI6eyJuYW1lIjoibWFwcyB0byJ9fX1dXQ==
    \begin{tikzcd}
        {(g,h)} && {(g,(h,h))} && {(g,(h,h^{-1}))} && {((g,h)h^{-1})} \\
        g &&&& {(g, e)} && g
        \arrow["1\times\Delta", maps to, from=1-1, to=1-3]
        \arrow["{1\times1\times\operatorname{inv}}", maps to, from=1-3, to=1-5]
        \arrow["{\operatorname{mult}\times1}", maps to, from=1-5, to=1-7]
        \arrow["{1\times\operatorname{mult}}"{description}, maps to, from=1-5, to=2-5]
        \arrow["{1\times\operatorname{id}}", maps to, from=2-1, to=2-5]
        \arrow["{\operatorname{mult}}", maps to, from=2-5, to=2-7]
        \arrow["{\operatorname{mult}}", maps to, from=1-7, to=2-7]
        \arrow["1"', curve={height=18pt}, maps to, from=2-1, to=2-7]
        \arrow["{\operatorname{proj}_1}"', maps to, from=1-1, to=2-1]
    \end{tikzcd}
    \end{equation*}
    I think it is hard to deny the relative simplicity of of the sequence above, but the benefits are magnified further when we include additional simplifications like those proof of conjugate:

    \begin{itemize}
        \item Type theory allows us to use equations like this to prove things about \textit{all} group objects in all categories. Its syntax involves elements with operations and equations, so we can speak and think as if we were talking about ordinary sets.
        \item But it is nevertheless a description of a free category of a certain sort (there are some limitation), so that its theorems can be uniquely mapped into any other similar category.
    \end{itemize}
    Thus, type theory gives a different perspective on categories that is often more familiar and easier to work with.    
\end{remark}

\begin{remark}{(Type Theory as Semantic and Foundation)}
    The benefit here comes from the interplay between two modes of interacting with type theory:
    \begin{itemize}
        \item \textbf{(Semantic PoV):} On one hand, we can define and study the formal system of type theory inside mathematics, which enables us to talk about its having multiple models, and hence functioning as a syntax for categories, like above.
        \item \textbf{(Foundation PoV):} Because type theory is sufficiently powerful to encode all of mathematics, we are also free to regard it as the “ambient foundation” for any mathematical theory
    \end{itemize}
    The point is that any theorem in ``ordinary'' mathematics can be encoded using the ``foundational'' point of view, obtaining a derivation in the formal system of type theory; but then we can switch to the ``semantic'' point of view and conclude that that theorem is actually true in all categories with appropriate structure. 
    \begin{quote}
        Any mathematical theorem is actually much more general than it appears (this may requires some care on the side of ordinary mathematics).
    \end{quote}
\end{remark}

\begin{remark}{(Type vs Set)}
    The basic objects of type theory, called types, are very set-like with one important difference, that is:
    \begin{itemize}
        \item In ZFC, $x\in A$ is an assertion, which might be true or false. That is, the universe of ZFC is a undifferentiated collection of things called ``sets'', with a relation called ``membership'':
        \begin{itemize}
            \item The type to which an element belongs is ``part of its nature'' rather than something to proof or disprove. For example, in most cases, two distinct types can never share any elements.
            \item Type is like that of categorical or ``structural'' set theory, which axiomatizes the category of sets and functions, contrast to contrasts with membership-based or ``material'' set theory
        \end{itemize}
        The advantage of structural approach is that it generalizes better when thinking of the basic objects as spaces rather than sets, since the spatial relationships between points are specified by an ambient space.
        \item The structural perspective matches the usage of ``set'' in most of mathematics. The notion of subsets is encoded in type theory by a notion of ``subset of $A$'' that like ``element of $A$'' is a basic notion not reducible to something like ``set that happens to be a subset of A''.
    \end{itemize}
    However, to give a mathematical theory of mathematics, the syntactic formulas of the formal system have to to appear quite verbose, and often barely comprehensible to a mathematician accustomed to informal mathematical language. But the syntax of type theory is intrinsically no heavier or unabusable than that of set theory.
\end{remark}

We are ready to describe the syntax of type theory and how it generates a category. Just like element of free group, the syntactic objects of type theory are ``words'' built out of operations.

But since type theory presents a category with both objects and morphisms, it has at least two sorts of ``words'' (free group have only 1 sort of word). Type theorists call a ``sort of word'' a judgment form, and a particular word a judgment.

\begin{definition}{\textbf{(Judgment)}}
    There are 2 kinds of judgment we will consider here:
    \begin{itemize}
        \item \textbf{(Type Judgement):} is written as $B\type$, where $B$ is a syntactic expression like $\mathbb{N}\times(\mathbb{R}\times \mathbb{Q})$ where $\times$ and $+$ are operations on types, analogous to the multiplication of elements represented by concatenation of words in a free group.
        \item \textbf{(Term Judgement):} is written as $b:B$, for eample we can have $(3\cdot2+1,\operatorname{inr}(3/4-17)):\mathbb{N}\times(\mathbb{R}\times \mathbb{Q})$, we can pronounce it as ``$b$ is an element of $B$'' or ``$b$ is a point of $B$'' or ``$b$ is a term of $B$'' emphasizing respectively the set-like, space-like, or syntactic character of $B$.
    \end{itemize}
\end{definition}

\begin{definition}{\textbf{(Context)}}
    Context is a list of variables, each with a specific type that may occur in the term $b$, often denoted as $\Gamma$. For example:
    \begin{equation*}
        x:\mathbb{N}, y: \mathbb{Q} \vdash \big(3x+1,\operatorname{inr}(3/4-y)\big):\mathbb{N}\times(\mathbb{R}\times \mathbb{Q})
    \end{equation*}
\end{definition}

\begin{definition}{\textbf{(Classifiying Category)}}
    For the classifying category (or known as category of contexts $\textbf{Ctx}$) can be defined by specifying the object and the morphisms between them that is:
    \begin{itemize}
        \item \textbf{The object} of the ``classifying category'' is\footnote{More simplified case, it is a generated by type theory are syntactic expressions $B$ for which the judgement $B\type$ can be produced by the rules. We will write $\bracka{B}$ when $B$ is regard an object of this category, and say that $B$ represents the object $\bracka{B}$.} the context rather than types with $\bracka{A}$ being $\bracka{x:A}$
        \item \textbf{Morphisms} is presented by term judgement $\Gamma\vdash a:A$. In the case\footnote{In the simplify case, $\Gamma$ contains only one variable that is $x:A\vdash b:B$, which is a morphisms: $\bracka{b}:\bracka{A}\to\bracka{B}$. } of $\Gamma\vdash a:A$, we have a morphism between $\bracka{\Gamma}\to\bracka{A}$. And we stipulate that $\bracka{\Gamma}$ is a product of type in $\Gamma$ that is $\bracka{x:A,y:B}\cong\bracka{A}\times\bracka{B}$. The empty context is the terminal object $\bracka{\cdot}$. 
    \end{itemize}
\end{definition}

\begin{definition}{\textbf{(Substitution)}}
    The universal property of products implies that for contexts context $\Gamma$ and $\Delta$, a morphisms in $\textbf{Ctx}$ from $\bracka{\Gamma}$ to general context $\bracka{\Delta}$ must consists of a term of judgement $\Gamma\vdash b_i:B_i$ for all variables $y_i:B_i$ occuring in $\Delta$. If we also have $\Delta\vdash c:C$, we get another term judgement of:
    \begin{equation*}
        \Gamma\vdash c[b_1/y_1,\cdots,b_m/y_m]:C
    \end{equation*}
    that is replacing each $y_i$ with $b_i$ in $c$, which is the composite of $\bracka{\Gamma}\to\bracka{\Delta}\to\bracka{C}$.
\end{definition}

\begin{remark}
    Suppose we have a morphism from $\bracka{x:\mathbb{R}}\to\bracka{z:\mathbb{R}, w:\mathbb{R}}$ defined by $x:\mathbb{R}\vdash(x-1):\mathbb{R}$ and $x:\mathbb{R}\vdash(x+1):\mathbb{R}$, then we can substitute to $z:\mathbb{R},w:\mathbb{R}\vdash zw:\mathbb{R}$ giving us $x:\mathbb{R}\vdash(x-1)(x+1):\mathbb{R}$. This is represented by the following diagram ({\color{red} not 100\% sure})
    \begin{equation*}
    % https://q.uiver.app/#q=WzAsNSxbMSwwLCJbW3g6XFxtYXRoYmJ7Un1dXSJdLFsxLDEsIltbejpcXG1hdGhiYntSfSwgdzpcXG1hdGhiYntSfV1dIl0sWzAsMiwiW1t6OlxcbWF0aGJie1J9XV0iXSxbMiwyLCJbW3c6XFxtYXRoYmJ7Un1dXSJdLFsxLDMsIltbenc6XFxtYXRoYmJ7Un1dXSJdLFswLDIsIltbeC0xXV0iLDEseyJjdXJ2ZSI6Mn1dLFswLDMsIltbeCsxXV0iLDEseyJjdXJ2ZSI6LTJ9XSxbMSwyXSxbMSwzXSxbMCwxLCIiLDEseyJzdHlsZSI6eyJib2R5Ijp7Im5hbWUiOiJkYXNoZWQifX19XSxbMSw0XV0=
    \begin{tikzcd}
        & {[[x:\mathbb{R}]]} \\
        & {[[z:\mathbb{R}, w:\mathbb{R}]]} \\
        {[[z:\mathbb{R}]]} && {[[w:\mathbb{R}]]} \\
        & {[[zw:\mathbb{R}]]}
        \arrow["{[[x-1]]}"{description}, curve={height=12pt}, from=1-2, to=3-1]
        \arrow["{[[x+1]]}"{description}, curve={height=-12pt}, from=1-2, to=3-3]
        \arrow[from=2-2, to=3-1]
        \arrow[from=2-2, to=3-3]
        \arrow[dashed, from=1-2, to=2-2]
        \arrow[from=2-2, to=4-2]
    \end{tikzcd}
    \end{equation*}
\end{remark}

\begin{remark}{(Dependent Type)}
    We can consider making $B\type$ as a dependent type or type famility. The dependent type $\Gamma\vdash B\type$ present an object of slice category $\textbf{Ctx}/\bracka{\Gamma}$ i.e object with a morphism to $\bracka{\Gamma}$ i.e fibre over a point of $\bracka{\Gamma}$ as the instance of $B$ corresponding to that point. 

    Consider the example of ``an arbitrary finite cyclic group $C_n$'' for $n:\mathbb{N}_+$. This becomes $n:\mathbb{N}_+\vdash C_n\type$ corresponding categorically to $\coprod_{n:\mathbb{N}}C_n$ with its projection to $\mathbb{N}_+$ i.e a bundle over $\mathbb{N}_+$ with $C_n$ as a fibre. 
    \begin{itemize}
        \item The ability to talk directly about families of types and have them interpreted automatically as bundles is one of the most significant advantages of type theory (as it is hard in traditional way).
    \end{itemize}
\end{remark}

\begin{definition}{\textbf{(Pullback Functor)}}
    Given category $\textbf{C}$ that has the pullbacks with objects $C$ and $D$ and morphism $f:C\to D$, then a pullback functor can be given to be $f^*:\textbf{C}/D\to\textbf{C}/C$ whose action are:
    \begin{itemize}
        \item \textbf{(Object):} Given a morphism $g:A\to D$, we have $f^*g$ is a pullback of $g$ along $f$, see the LHS diagram. 
        \item \textbf{(Morphisms):} Given the morphism $h:A\to B$ between $g_1:A\to D$ and $g_2:B\to D$ such that $g_2\circ h=g_1$, we can define the $f^*h$ to be a pullback of commutative triangle (note that it is a morphism in $\textbf{C}/C$ for sure as it factors $f^*g_1$), as shown in the RHS (from blue to red):
    \end{itemize}
    \begin{equation*}
    % https://q.uiver.app/#q=WzAsNCxbMSwxLCJEIl0sWzEsMCwiQSJdLFswLDEsIkMiXSxbMCwwLCJDXFx0aW1lc19EQSJdLFsyLDAsImYiLDJdLFsxLDAsImciXSxbMywxXSxbMywyLCJmXipnIiwyXSxbMywwLCIiLDEseyJzdHlsZSI6eyJuYW1lIjoiY29ybmVyIn19XV0=
    \begin{tikzcd}
        {C\times_DA} & A \\
        C & D
        \arrow["f"', from=2-1, to=2-2]
        \arrow["g", from=1-2, to=2-2]
        \arrow[from=1-1, to=1-2]
        \arrow["{f^*g}"', from=1-1, to=2-1]
        \arrow["\lrcorner"{anchor=center, pos=0.125}, draw=none, from=1-1, to=2-2]
    \end{tikzcd}
    \qquad \quad
    % https://q.uiver.app/#q=WzAsNixbMiwyLCJEIl0sWzIsMCwiQSJdLFsxLDIsIkMiXSxbMCwwLCJDXFx0aW1lc19EQSJdLFsxLDEsIkNcXHRpbWVzX0RCIl0sWzIsMSwiQiJdLFsyLDAsImYiLDFdLFszLDFdLFszLDIsImZeKmdfMSIsMSx7ImN1cnZlIjoyLCJjb2xvdXIiOlswLDYwLDYwXX0sWzAsNjAsNjAsMV1dLFs0LDVdLFs0LDIsImZeKmdfMiIsMSx7ImNvbG91ciI6WzAsNjAsNjBdfSxbMCw2MCw2MCwxXV0sWzUsMCwiZ18yIiwxLHsiY29sb3VyIjpbMjQwLDYwLDYwXX0sWzI0MCw2MCw2MCwxXV0sWzEsNSwiaCIsMSx7ImNvbG91ciI6WzI0MCw2MCw2MF19LFsyNDAsNjAsNjAsMV1dLFszLDQsImZeKmgiLDEseyJjb2xvdXIiOlswLDYwLDYwXSwic3R5bGUiOnsiYm9keSI6eyJuYW1lIjoiZGFzaGVkIn19fSxbMCw2MCw2MCwxXV0sWzEsMCwiZ18xIiwxLHsiY3VydmUiOi0zLCJjb2xvdXIiOlsyNDAsNjAsNjBdfSxbMjQwLDYwLDYwLDFdXV0=
    \begin{tikzcd}
        {C\times_DA} && A \\
        & {C\times_DB} & B \\
        & C & D
        \arrow["f"{description}, from=3-2, to=3-3]
        \arrow[from=1-1, to=1-3]
        \arrow["{f^*g_1}"{description}, color={rgb,255:red,214;green,92;blue,92}, curve={height=12pt}, from=1-1, to=3-2]
        \arrow[from=2-2, to=2-3]
        \arrow["{f^*g_2}"{description}, color={rgb,255:red,214;green,92;blue,92}, from=2-2, to=3-2]
        \arrow["{g_2}"{description}, color={rgb,255:red,92;green,92;blue,214}, from=2-3, to=3-3]
        \arrow["h"{description}, color={rgb,255:red,92;green,92;blue,214}, from=1-3, to=2-3]
        \arrow["{f^*h}"{description}, color={rgb,255:red,214;green,92;blue,92}, dashed, from=1-1, to=2-2]
        \arrow["{g_1}"{description}, color={rgb,255:red,92;green,92;blue,214}, curve={height=-18pt}, from=1-3, to=3-3]
    \end{tikzcd}
    \end{equation*}
\end{definition}

\begin{proposition}
    A pullback functor is a functor
\end{proposition}
\begin{proof}
    There are 2 things we have to proof. First is that the functor $f^*$ applied to $\operatorname{id}$ will gives us identity. We observe that with $h=\operatorname{id}_B$, we see that the map $g_2=g_1$, and so when perform a pullback (on blue), we get $X=C\times_DB$ as we have a pullback of $g_2$ along $f$. 
    \begin{equation*}
    % https://q.uiver.app/#q=WzAsNixbMiwyLCJEIl0sWzIsMCwiQiJdLFsxLDIsIkMiXSxbMCwwLCJYIl0sWzEsMSwiQ1xcdGltZXNfREIiXSxbMiwxLCJCIl0sWzIsMCwiZiIsMSx7ImNvbG91ciI6WzI0MCw2MCw2MF19LFsyNDAsNjAsNjAsMV1dLFszLDIsImZeKmdfMiIsMSx7ImN1cnZlIjoyfV0sWzQsNV0sWzQsMiwiZl4qZ18yIiwxXSxbNSwwLCJnXzIiLDFdLFsxLDUsIlxcb3BlcmF0b3JuYW1le2lkfV9CIiwxXSxbMyw0LCJmXiooXFxvcGVyYXRvcm5hbWV7aWR9X0IpIiwxLHsic3R5bGUiOnsiYm9keSI6eyJuYW1lIjoiZGFzaGVkIn19fV0sWzEsMCwiZ18yIiwxLHsiY3VydmUiOi0zLCJjb2xvdXIiOlsyNDAsNjAsNjBdfSxbMjQwLDYwLDYwLDFdXSxbMywxXV0=
    \begin{tikzcd}
        X && B \\
        & {C\times_DB} & B \\
        & C & D
        \arrow["f"{description}, color={rgb,255:red,92;green,92;blue,214}, from=3-2, to=3-3]
        \arrow["{f^*g_2}"{description}, curve={height=12pt}, from=1-1, to=3-2]
        \arrow[from=2-2, to=2-3]
        \arrow["{f^*g_2}"{description}, from=2-2, to=3-2]
        \arrow["{g_2}"{description}, from=2-3, to=3-3]
        \arrow["{\operatorname{id}_B}"{description}, from=1-3, to=2-3]
        \arrow["{f^*(\operatorname{id}_B)}"{description}, dashed, from=1-1, to=2-2]
        \arrow["{g_2}"{description}, color={rgb,255:red,92;green,92;blue,214}, curve={height=-18pt}, from=1-3, to=3-3]
        \arrow[from=1-1, to=1-3]
    \end{tikzcd}
    \end{equation*}
    Since the map $X\to A$ and $X\to C$ being the same as $C\times_DB\to A$ and $C\times_DB\to C$ by default, by the universal property $f^*(\operatorname{id}_B)$ is an identity on $C\times_DB$ and thus a idenity morphism. 

    For the composition, we have the following 2 diagrams that are almost the same, except that the LHS computes $f^*(k\circ h)$, while RHS computes $f^*k\circ f^*h$, nonetheless, the arrows are the same. Thus, by universe property both are equal and commutes:
    
    \begin{equation*}
    % https://q.uiver.app/#q=WzAsMTQsWzIsMywiRCJdLFsyLDAsIkEiXSxbMSwzLCJDIl0sWzAsMCwiQ1xcdGltZXNfREEiXSxbMSwyLCJDXFx0aW1lc19ERSJdLFsyLDIsIkUiXSxbNywxLCJCIl0sWzcsMiwiRSJdLFs3LDMsIkQiXSxbNiwyLCJDXFx0aW1lc19ERSJdLFs2LDMsIkMiXSxbNSwxLCJDXFx0aW1lc19EQiJdLFs0LDAsIkNcXHRpbWVzX0RBIl0sWzcsMCwiQSJdLFsyLDAsImYiLDFdLFszLDIsImZeKmdfMSIsMSx7ImN1cnZlIjoyfV0sWzQsNV0sWzQsMiwiZl4qZ18zIiwxXSxbMSw1LCJrXFxjaXJjIGgiLDFdLFszLDQsImZeKihrXFxjaXJjIGgpIiwxLHsic3R5bGUiOnsiYm9keSI6eyJuYW1lIjoiZGFzaGVkIn19fV0sWzMsMV0sWzYsNywiayIsMV0sWzYsOCwiZ18yIiwxLHsiY3VydmUiOi0yfV0sWzcsOCwiZ18zIiwxXSxbMTEsNl0sWzExLDksImZeKmsiLDEseyJzdHlsZSI6eyJib2R5Ijp7Im5hbWUiOiJkYXNoZWQifX19XSxbMTEsMTAsImZeKmdfMiIsMSx7ImN1cnZlIjoyfV0sWzksMTAsImZeKmdfMyIsMV0sWzksN10sWzEwLDgsImYiLDFdLFsxLDAsImdfMSIsMSx7ImN1cnZlIjotM31dLFs1LDAsImdfMyIsMV0sWzEzLDYsImgiLDFdLFsxMyw4LCJnXzEiLDEseyJjdXJ2ZSI6LTR9XSxbMTIsMTNdLFsxMiwxMCwiZl4qZ18xIiwxLHsiY3VydmUiOjN9XSxbMTIsMTEsImZeKmgiLDEseyJzdHlsZSI6eyJib2R5Ijp7Im5hbWUiOiJkYXNoZWQifX19XV0=
    \begin{tikzcd}
        {C\times_DA} && A && {C\times_DA} &&& A \\
        &&&&& {C\times_DB} && B \\
        & {C\times_DE} & E &&&& {C\times_DE} & E \\
        & C & D &&&& C & D
        \arrow["f"{description}, from=4-2, to=4-3]
        \arrow["{f^*g_1}"{description}, curve={height=12pt}, from=1-1, to=4-2]
        \arrow[from=3-2, to=3-3]
        \arrow["{f^*g_3}"{description}, from=3-2, to=4-2]
        \arrow["{k\circ h}"{description}, from=1-3, to=3-3]
        \arrow["{f^*(k\circ h)}"{description}, dashed, from=1-1, to=3-2]
        \arrow[from=1-1, to=1-3]
        \arrow["k"{description}, from=2-8, to=3-8]
        \arrow["{g_2}"{description}, curve={height=-12pt}, from=2-8, to=4-8]
        \arrow["{g_3}"{description}, from=3-8, to=4-8]
        \arrow[from=2-6, to=2-8]
        \arrow["{f^*k}"{description}, dashed, from=2-6, to=3-7]
        \arrow["{f^*g_2}"{description}, curve={height=12pt}, from=2-6, to=4-7]
        \arrow["{f^*g_3}"{description}, from=3-7, to=4-7]
        \arrow[from=3-7, to=3-8]
        \arrow["f"{description}, from=4-7, to=4-8]
        \arrow["{g_1}"{description}, curve={height=-18pt}, from=1-3, to=4-3]
        \arrow["{g_3}"{description}, from=3-3, to=4-3]
        \arrow["h"{description}, from=1-8, to=2-8]
        \arrow["{g_1}"{description}, curve={height=-24pt}, from=1-8, to=4-8]
        \arrow[from=1-5, to=1-8]
        \arrow["{f^*g_1}"{description}, curve={height=18pt}, from=1-5, to=4-7]
        \arrow["{f^*h}"{description}, dashed, from=1-5, to=2-6]
    \end{tikzcd}
    \end{equation*}
    
    Please note that We can see that $g_2\circ h=g_1$ and $g_3\circ k=g_2$ that is $g_3\circ k\circ h=g_1$. 
\end{proof}

\begin{definition}{\textbf{(Dependent Type)}}
    A substitution into a dependent type presents the pullback functor between slice categories (dependent type).
\end{definition}

\begin{remark}{(Example of Dependent Types)}
    Suppose we have judgement $\vdash 3:\mathbb{N}_+$, yielding a context morphisms from a terminal object $\bracka{ \ }\to\bracka{\mathbb{N}_+}$. The substitution into $n:\mathbb{N}_+\vdash C_n\type$ along the inclusion $3:\bracka{ \ }\to\bracka{\mathbb{N}_+}$, see LHS diagram:
    \begin{equation*}
    % https://q.uiver.app/#q=WzAsOCxbMCwwLCJDXzMiXSxbMCwxLCIxIl0sWzEsMCwiXFxjb3Byb2Rfe25cXGluXFxtYXRoYmJ7Tn1fK31DX24iXSxbMSwxLCJcXG1hdGhiYntOfV8rIl0sWzMsMCwiXFxicmFja2F7XFxHYW1tYX1cXHRpbWVzXFxicmFja2F7Qn0iXSxbMywxLCJcXGJyYWNrYXtcXEdhbW1hfSJdLFs0LDAsIlxcYnJhY2the0J9Il0sWzQsMSwiXFxicmFja2F7IFxcIH09MSJdLFsxLDNdLFsyLDNdLFswLDFdLFswLDJdLFswLDMsIiIsMSx7InN0eWxlIjp7Im5hbWUiOiJjb3JuZXIifX1dLFs0LDVdLFs1LDddLFs2LDddLFs0LDZdLFs0LDcsIiIsMSx7InN0eWxlIjp7Im5hbWUiOiJjb3JuZXIifX1dXQ==
    \begin{tikzcd}
        {C_3} & {\coprod_{n\in\mathbb{N}_+}C_n} && {\bracka{\Gamma}\times\bracka{B}} & {\bracka{B}} \\
        1 & {\mathbb{N}_+} && {\bracka{\Gamma}} & {\bracka{ \ }=1}
        \arrow[from=2-1, to=2-2]
        \arrow[from=1-2, to=2-2]
        \arrow[from=1-1, to=2-1]
        \arrow[from=1-1, to=1-2]
        \arrow["\lrcorner"{anchor=center, pos=0.125}, draw=none, from=1-1, to=2-2]
        \arrow[from=1-4, to=2-4]
        \arrow[from=2-4, to=2-5]
        \arrow[from=1-5, to=2-5]
        \arrow[from=1-4, to=1-5]
        \arrow["\lrcorner"{anchor=center, pos=0.125}, draw=none, from=1-4, to=2-5]
    \end{tikzcd}
    \end{equation*}
    Or if $\vdash B\type$ is a non-dependent type, we can substitute it along the unique context morphism for any $\Gamma$ to empty context ($\Gamma\vdash1\vdash B\type$), yielding ``trivially-dependent type'' $\Gamma\vdash B\type$ being the pullback of $\bracka{B}$ to the slice over $\bracka{\Gamma}$ i.e projection of $\bracka{\Gamma}\times\bracka{B}\to\bracka{\Gamma}$ (trivial bundle), see RHS diagram.
\end{remark}

\begin{remark}{(Dependent Type and Term Judgement)}
    Furthermore, we can also have type $B$ in a term judgement $\Gamma\vdash b:B$ to also depends on $\Gamma$. For example:
    \begin{itemize}
        \item The generator of the cyclic group form a term judgement $n:\mathbb{N}_+\vdash g_n:C_n$. Such kind of judgement $\Gamma\vdash b:B$ represents a section of the projection represented by the dependent type $\Gamma\vdash B\type$ i.e ``select one point in each fiber''
        \item This includes the non-dependent case because morphisms $\bracka{\Gamma}\to\bracka{B}$ are equivalent to sections of the projection $\bracka{\Gamma}\times\bracka{B}\to\bracka{\Gamma}$ (To make a section, with every point in base space $\bracka{[\Gamma]}$, we choose a point of $\bracka{B}$, recall that section is ``right inverse'')
    \end{itemize}
\end{remark}

\begin{definition}{\textbf{(Equality Type/Identity Type)}}
    The important example is the diagonal map $\Delta_{\bracka{A}}:\bracka{A}\to\bracka{A}\times\bracka{A}$, which is an object in the slice category $\textbf{Ctx}/(\bracka{A}\times\bracka{A})$ or equivalently $\textbf{Ctx}/\bracka{x:A,y:A}$, which represents a dependent type called equality type or identity type:
    \begin{equation*}
        x:A,y:A\vdash(x=y)\type \qquad\text{ or }\qquad x:A,y:A\vdash\operatorname{Id}(x,y)\type
    \end{equation*}
\end{definition}

\begin{remark}
    We can see that it reduces equalities of terms to existence of terms. For example given $\Gamma\vdash a:A$ an $\Gamma\vdash b:A$ represeting morphism $\bracka{a},\bracka{b}:\bracka{\Gamma}\to\bracka{A}$ substituting them into the equality type we get a dependent type $\Gamma\vdash(a=b)\type$, which is a pullback of $\Delta_{\bracka{A}}$ along $(\bracka{a}, \bracka{b}):\bracka{\Gamma}\to\bracka{A}\times\bracka{A}$ (LHS)
    \begin{equation*}
    % https://q.uiver.app/#q=WzAsOSxbMywyLCJbW0FdXVxcdGltZXNbW0FdXSJdLFsxLDIsIltbXFxHYW1tYV1dIl0sWzMsMSwiW1tBXV0iXSxbMSwxLCJcXGJ1bGxldCJdLFs2LDIsIltbXFxHYW1tYV1dIl0sWzcsMiwiW1tBXV0iXSxbNiwxLCJcXGJ1bGxldCJdLFswLDAsIlgiXSxbNSwyLCJYIl0sWzEsMCwiKFtbYV1dLCBbW2JdXSkiLDJdLFsyLDAsIlxcRGVsdGFfe1tbQV1dfSJdLFszLDFdLFszLDJdLFs0LDUsIltbYV1dIiwwLHsib2Zmc2V0IjotMX1dLFs0LDUsIltbYl1dIiwyLHsib2Zmc2V0IjoxfV0sWzYsNF0sWzcsMSwiIiwyLHsiY3VydmUiOjF9XSxbOCw0XSxbOCw2LCIiLDEseyJzdHlsZSI6eyJib2R5Ijp7Im5hbWUiOiJkYXNoZWQifX19XSxbNywyLCIiLDEseyJjdXJ2ZSI6LTJ9XSxbNywzLCIiLDEseyJzdHlsZSI6eyJib2R5Ijp7Im5hbWUiOiJkYXNoZWQifX19XSxbMyw5LCIiLDEseyJsZXZlbCI6MSwic3R5bGUiOnsibmFtZSI6ImNvcm5lciJ9fV1d
    \begin{tikzcd}
        X \\
        & \bullet && {[[A]]} &&& \bullet \\
        & {[[\Gamma]]} && {[[A]]\times[[A]]} && X & {[[\Gamma]]} & {[[A]]}
        \arrow[""{name=0, anchor=center, inner sep=0}, "{([[a]], [[b]])}"', from=3-2, to=3-4]
        \arrow["{\Delta_{[[A]]}}", from=2-4, to=3-4]
        \arrow[from=2-2, to=3-2]
        \arrow[from=2-2, to=2-4]
        \arrow["{[[a]]}", shift left, from=3-7, to=3-8]
        \arrow["{[[b]]}"', shift right, from=3-7, to=3-8]
        \arrow[from=2-7, to=3-7]
        \arrow[curve={height=6pt}, from=1-1, to=3-2]
        \arrow[from=3-6, to=3-7]
        \arrow[dashed, from=3-6, to=2-7]
        \arrow[curve={height=-12pt}, from=1-1, to=2-4]
        \arrow[dashed, from=1-1, to=2-2]
        \arrow["\lrcorner"{anchor=center, pos=0.125}, draw=none, from=2-2, to=0]
    \end{tikzcd}
    \end{equation*}
    Or the equalizer of $\bracka{a}$ and $\bracka{b}$ (see the RHS), that is because for any $X$, if the pullback diagram commutes, for example: $(\bracka{a}\circ f, \bracka{b}\circ f)=(\bracka{a},\bracka{b})\circ f=\Delta_{\bracka{a}}\circ g = (g, g)$. Then the same map also commutes in the LHS diagram i.e $\bracka{a}\circ f=g=\bracka{b}\circ f$. Thus, the pullback diagram here is actually the equalizer (as needed).

    Thus, a judgement $\Gamma\vdash e:a=b$ says that this equalizer has a section that is $\bracka{a}=\bracka{b}$. So our type and term judgments also suffice to present equality of morphisms.
\end{remark}

\begin{definition}{\textbf{(Variable Dependent Context)}}
    As an extension of classifying category $\textbf{Ctx}$, we can allow types in a context to depend on the variables occurring earlier in the same context. 
\end{definition}

\begin{remark}{(Variable on Context in Category)}
    For example, $n:\mathbb{N}_+,x:C_n\vdash x^2:C_n$ that squares an arbitrary element of an arbitrary cyclic group. Categorically, if $\Gamma\vdash B\type$ presents an object of the slice over $\bracka{\Gamma}$, then the extended context $\bracka{\Gamma, x:B}$ is the domain, and reduces to $\bracka{\Gamma, x:B}=\bracka{\Gamma}\times\bracka{B}$ if $B$ is non-dependent.

    We take the object $\textbf{Ctx}$ to be contexts in this generalized sense, and a morphism from $\bracka{\Gamma}\to\bracka{\Delta}$ to consist of term judgments for all $1\le i\le m$, we have $\Gamma\vdash b_i:B_i[b_1/y_1,\dots,b_{i-1}/y_{i-1}]$ where $\Delta=(y_1:B_1,y_2:B_2,\dots,y_m:B_m)$ with $y_j$ potentially occuring in $B_i$ for $j<i$. 
    
    We have $\Gamma\vdash b_1:B_1$ giving the morphism $\bracka{b_1}:\bracka{\Gamma}\to\bracka{B_1}$. We then substitute $b_1$ for $y_1$ in $B_2$ that is $\Gamma\vdash B_2[b_1/y_1]$, which corresponds to a extended context (top-left) that presents the pullback (LHS):
    \begin{equation*}
    % https://q.uiver.app/#q=WzAsNCxbMCwwLCJcXGJyYWNrYXtcXEdhbW1hLCB5XzI6Ql8yW2JfMS95XzFdfSJdLFsyLDAsIlxcYnJhY2the3lfMTpCXzEsIHlfMjpCXzJ9Il0sWzIsMSwiXFxicmFja2F7Ql8xfSJdLFswLDEsIlxcYnJhY2the1xcR2FtbWF9Il0sWzAsM10sWzAsMV0sWzMsMiwiXFxicmFja2F7Yl8xfSIsMl0sWzEsMl0sWzAsNiwiIiwxLHsibGV2ZWwiOjEsInN0eWxlIjp7Im5hbWUiOiJjb3JuZXIifX1dXQ==
    \begin{tikzcd}
        {\bracka{\Gamma, y_2:B_2[b_1/y_1]}} && {\bracka{y_1:B_1, y_2:B_2}} \\
        {\bracka{\Gamma}} && {\bracka{B_1}}
        \arrow[from=1-1, to=2-1]
        \arrow[from=1-1, to=1-3]
        \arrow[""{name=0, anchor=center, inner sep=0}, "{\bracka{b_1}}"', from=2-1, to=2-3]
        \arrow[from=1-3, to=2-3]
        \arrow["\lrcorner"{anchor=center, pos=0.125}, draw=none, from=1-1, to=0]
    \end{tikzcd}
    \qquad\quad
    % https://q.uiver.app/#q=WzAsMyxbMSwwLCJcXGJyYWNrYXt5XzE6Ql8xLCB5XzI6Ql8yfSJdLFsxLDEsIlxcYnJhY2the3lfMTpCXzF9Il0sWzAsMSwiXFxicmFja2F7XFxHYW1tYX0iXSxbMiwxLCJcXGJyYWNrYXtiXzF9IiwyXSxbMCwxXSxbMiwwLCJcXGJyYWNrYXtiXzJ9Il1d
    \begin{tikzcd}
        & {\bracka{y_1:B_1, y_2:B_2}} \\
        {\bracka{\Gamma}} & {\bracka{y_1:B_1}}
        \arrow["{\bracka{b_1}}"', from=2-1, to=2-2]
        \arrow[from=1-2, to=2-2]
        \arrow["{\bracka{b_2}}", from=2-1, to=1-2]
    \end{tikzcd}
    \end{equation*}
    LHS diagram reprseting a connection between $\Gamma\vdash b_1:B_1$ and $\Gamma, y_1:B_1\vdash B_2$. Then for a term $\Gamma\vdash b_2:B_2[b_1/y_1]$, which presents a section of this pullback i.e morphism $\bracka{\Gamma}\to\bracka{y_1:B_1,y_2:B_2}$, making the RHS diagram commutes (looks like the diagonal from bottom left to top right of the pullback diagram, think of it like changing the base affecting the original section).
    \begin{equation*}
    % https://q.uiver.app/#q=WzAsNSxbMSwyLCJcXGJyYWNrYXt5XzE6Ql8xLCB5XzI6Ql8yfSJdLFsxLDMsIlxcYnJhY2the3lfMTpCXzF9Il0sWzAsMiwiXFxicmFja2F7XFxHYW1tYX0iXSxbMSwxLCJcXHZkb3RzIl0sWzEsMCwiXFxicmFja2F7eV8xOkJfMSx5XzI6Ql8yLFxcZG90cyx5X206Ql9tfT1cXGJyYWNrYXtcXERlbHRhfSJdLFsyLDEsIlxcYnJhY2the2JfMX0iLDJdLFswLDFdLFsyLDAsIlxcYnJhY2the2JfMn0iXSxbMywwXSxbNCwzXSxbMiw0LCJcXGJyYWNrYXtiX219Il1d
    \begin{tikzcd}
        & {\bracka{y_1:B_1,y_2:B_2,\dots,y_m:B_m}=\bracka{\Delta}} \\
        & \vdots \\
        {\bracka{\Gamma}} & {\bracka{y_1:B_1, y_2:B_2}} \\
        & {\bracka{y_1:B_1}}
        \arrow["{\bracka{b_1}}"', from=3-1, to=4-2]
        \arrow[from=3-2, to=4-2]
        \arrow["{\bracka{b_2}}", from=3-1, to=3-2]
        \arrow[from=2-2, to=3-2]
        \arrow[from=1-2, to=2-2]
        \arrow["{\bracka{b_m}}", from=3-1, to=1-2]
    \end{tikzcd}
    \end{equation*}
    where $\bracka{b_m}$ is the overall morphism $\bracka{\Gamma}\to\bracka{\Delta}$. For example, the ``squaring'' injection $i_n:C_n\hookrightarrow C_{2n}$ represents by term judgement of $n:\mathbb{N}_+\vdash 2n:\mathbb{N}_+$ and $n:\mathbb{N}_+,x:C_n\vdash i_n(x):C_{2n}$, which got assembled into the morphism $\bracka{n:\mathbb{N}_+,x:C_n}\to\bracka{m:\mathbb{N}_+,y:C_m}$. Categorically, it is a morphism $\coprod_nC_n\to\coprod_mC_m$ sends the $n$-th summand to the $2n$-th summand.
\end{remark}

\begin{remark}{(Idenity Type)}
    We quotient these morphisms by an equivalence relation arising from the identity type. In the simplest case where each context has only one type, we identify the morphisms presented by $x:A\vdash b_1:B$ and $x:A\vdash b_2:B$ if there is a term $x:A\vdash p:b_1=b_2$. The case of morphisms between arbitary contexts is a generalization of this.
\end{remark}

The complete our definition of classifying category of a type theory. We can now define:

\begin{definition}{\textbf{(Projection Morphism)}}
    $\bracka{\Gamma,z:C}\to\bracka{\Gamma}$ associated with the dependent type $\Gamma\vdash C\type$ exhibiting $\bracka{\Gamma,z:C}$ as object of the slice category over $\bracka{\Gamma}$. This consist of a term in context $\Gamma,z:C$ for each type in $\Gamma$; we take these to be just the variables in $\Gamma$ ignoring $z$.
\end{definition}

\begin{remark}{(Example of Projection Map)}
    The projection map $\bracka{x:A,y:B,z:C}\to\bracka{x:A,y:B}$ is determiend by terms $x:A,y:B,z:C\vdash x:A$ and $x:A,y:B,z:C\vdash y:B$. The section of this projection consists of the terms of:
    \begin{equation*}
        x:A,y:B\vdash a:A \qquad x:A,y:B\vdash b:B[a/x] \qquad x:A, y:B\vdash c:C[a/x,b/y]
    \end{equation*}
    Such that the composite $\bracka{x:A, y:B}\to\bracka{x:A,y:B,z:C}\to\bracka{x:A,y:B}$ is the idenity i.e $a$ and $b$ are the same as $x$ and $y$. Thus, such section is determined by a term $x:A,y:B\vdash c:C$ as needed.
\end{remark}

\begin{remark}
    Not every object of slice category $\textbf{Ctx}/\bracka{\Gamma}$ is of this form (projection map?) but every object of $\textbf{Ctx}/\bracka{\Gamma}$ is isomorphic to one of this form. The trick is to use the identity type to create an isomorphism.
    
    
    Consider the case when $\Gamma$ is single type $B$ and object of $\textbf{Ctx}/\bracka{B}$ whose domain is type $A$ with term $x:A\vdash f(x):B$. Let context $\Psi$ be $(y:B,x:A,p:f(x)=y)$, then $\bracka{\Psi}$ is:
    \begin{equation*}
    % https://q.uiver.app/#q=WzAsNSxbMCwxLCJcXGJyYWNrYXtcXFBzaX0iXSxbMiwxLCJcXGJyYWNrYXtCfSJdLFsyLDIsIlxcYnJhY2the0J9XFx0aW1lc1xcYnJhY2the0J9Il0sWzAsMiwiXFxicmFja2F7Qn1cXHRpbWVzXFxicmFja2F7QX0iXSxbMCwwLCJcXGJyYWNrYXtBfSJdLFszLDIsIjFcXHRpbWVzXFxicmFja2F7Zn0iLDIseyJjb2xvdXIiOlsyNDAsNjAsNjBdfSxbMjQwLDYwLDYwLDFdXSxbMCwzLCIiLDIseyJjb2xvdXIiOlsyNDAsNjAsNjBdfV0sWzAsMV0sWzEsMiwiXFxEZWx0YV97XFxicmFja2F7Qn19Il0sWzQsMCwiXFxjb25nIiwxXSxbNCwxLCJcXGJyYWNrYXtmfSJdLFswLDUsIiIsMCx7ImxldmVsIjoxLCJzdHlsZSI6eyJuYW1lIjoiY29ybmVyIn19XV0=
    \begin{tikzcd}
        {\bracka{A}} \\
        {\bracka{\Psi}} && {\bracka{B}} \\
        {\bracka{B}\times\bracka{A}} && {\bracka{B}\times\bracka{B}}
        \arrow[""{name=0, anchor=center, inner sep=0}, "{1\times\bracka{f}}"', color={rgb,255:red,92;green,92;blue,214}, from=3-1, to=3-3]
        \arrow[color={rgb,255:red,92;green,92;blue,214}, from=2-1, to=3-1]
        \arrow[from=2-1, to=2-3]
        \arrow["{\Delta_{\bracka{B}}}", from=2-3, to=3-3]
        \arrow["\cong"{description}, from=1-1, to=2-1]
        \arrow["{\bracka{f}}", from=1-1, to=2-3]
        \arrow["\lrcorner"{anchor=center, pos=0.125}, draw=none, from=2-1, to=0]
    \end{tikzcd}
    \end{equation*}
    Using identity type $y_1:B,y_2:B\vdash(y_1=y_2)\type$. It is easy to see categorically that such a pullback is isomorphic to $\bracka{A}$. Thus, every object of $\textbf{Ctx}/\bracka{B}$ is at least isomorphic to a composite of two projections from dependent types: $\bracka{y:B,x:A,p:f(x)=y}\to\bracka{y:B,x:A}\to\bracka{y:B}$. (Highlighted in Blue).
\end{remark}

\subsection{Rules and Universal Properties}

\begin{remark}{(Example of Reduced Words Free Group)} As mensioned before, there are 2 ways we can describe the free group:

    \begin{itemize}
        \item \textbf{(Inductive Definition):} We have following rules (1) given elements $X$ and $Y$, we have $(XY)$. (2) given an element $X$, we have $(X^{-1})$. (3) We have an element $e$. We can apply these operations under the set of generator, but we can also view generators as a special case of operator i.e (4) For any generator $g$, we have an element $g$.
        \item \textbf{(Reduced Word):} Similar to the inductive operations case, in which: (1) We have an element $e$. (2) For any generator $g$ and element $X$ not ending with $g^{-1}$, we have element $Xg$. (3) For any generator $g$ and element $X$ not ending with $g$, we have element $Xg^{-1}$.
    \end{itemize}

    The judgement in type theory are generated inductively by operations called rules. Categorically, they build new objects and morphisms from old ones, generally according to them some universal property (for example co-product). And this applies in any context (slice category inherit co-product from base category).
\end{remark}

\begin{remark}{(Free Group Construction in Type Theory)}
    With the specificed recipe to build the element of a free group, let's consider the inductive definition first:
    \begin{equation*}
        \frac{X\elem\qquad Y\elem}{(XY)\elem} \qquad \quad \frac{X\elem}{(X^{-1})\elem} \qquad \quad \frac{}{e\elem} \qquad \quad \frac{g\text{ is a generator}}{g\elem}
    \end{equation*}
    On the other hand, the reduced-words description would be:
    \begin{equation*}
        \frac{}{e\elem} \qquad \frac{X\elem \quad g\text{ is a generator}\quad X\text{ doesn't end with }g^{-1}}{(Xg)\elem}\qquad \frac{X\elem \quad g\text{ is a generator}\quad X\text{ doesn't end with }g}{(Xg^{-1})\elem}
    \end{equation*}
    The variable $X$ and $Y$ are analogous to $\Gamma, A, B$ and $b$ in type theory. The latter is called meta-variable.
\end{remark}

\begin{remark}{(Trivial vs Reduced-Word Description in Type Theory)}
    Conside the introduction rule for co-product type, in which requires to have a certain structure. There shoul be injection from $\bracka{A}$ and $\bracka{B}$ into $\bracka{A+B}$ wihch we may write it as (trivial version):
    \begin{equation*}
        \frac{\Gamma\vdash A\type}{x:A\vdash\operatorname{inl}(x):(A+B)} \qquad \quad
        \frac{\Gamma\vdash B\type}{y:B\vdash\operatorname{inr}(y):(A+B)} 
    \end{equation*}
    On the other hand, with the reduced word description-like rule, which corresponds to describing the morphism $\bracka{\operatorname{inl}}:\bracka{A}\to\bracka{A+B}$, in terms of its image under the Yoneda embedding: for any morphism $\bracka{a}:\bracka{\Gamma}\to\bracka{A}$ gives induced morphism of $\bracka{\operatorname{inl}(a)}:\bracka{\Gamma}\to\bracka{A+B}$:
    \begin{equation*}
        \frac{\Gamma\vdash a:A}{\Gamma\vdash\operatorname{inl}(a):(A+B)}\qquad \quad \frac{\Gamma\vdash v:A}{\Gamma\vdash\operatorname{inl}(v):(A+B)}
    \end{equation*}

    We note the differences by drawing the analogy from the free group, whereby:
    \begin{itemize}
        \item \textbf{In the tautological free group}, we insert the generators as nullary operations, so that we need multiplication as a separate rule, which requires quotienting by an equivalence relation to enforce desired properties of multiplication. 
        \begin{itemize}
            \item We are also required separate ``substitution rule'' to get the composition in the classifying category and quotient to make composition associative and unital.
        \end{itemize}
        \item \textbf{In the reduced words free group}, the generator rules $X\to Xg$ and $X\to Xg^{-1}$ incorporate ``just enough multiplication'' that we can define multiplication of reduced words, and prove that it is associative.
        \begin{itemize}
            \item One can define substitution as on operation on judgments and prove that it is associative, and so on i.e reducing or eliminating the need to pass to a quotient in constructing the classifying category.
        \end{itemize}
    \end{itemize}
    The difference between the need to do the substitution is called the admissibility of substitution.
\end{remark}

\begin{remark}{(Elimination Rule for Co-Product)}
    This which expresses the ``existence'' part of the categorical universal property of a coproduct, where by if we are given the morphism $\bracka{A}\to\bracka{C}$ and $\bracka{B}\to\bracka{C}$, then we have morphism $\bracka{A+B}\to\bracka{C}$:
    \begin{equation*}
        \frac{\Gamma,x:A\vdash c_A:C\qquad \Gamma,y:B\vdash c_B:C\qquad \Gamma\vdash s:A+B}{\Gamma\vdash\operatorname{case}(C, c_A, c_B, s):C}
    \end{equation*}
    The case means that if the form element $s:A+B$ is $\operatorname{inl}(x)$, then we use $c_A$ and else if it is in the form of $\operatorname{inr}(y)$, then we use $c_B$. Given $C$ to be a dependent type, we have that:
    \begin{equation*}
        \frac{\Gamma, z:A+B\vdash C\type\qquad\Gamma\vdash s:A+B\qquad\Gamma,x:A\vdash c_A:C[\operatorname{inl}(x)/z]\qquad\Gamma,y:B\vdash c_B:C[\operatorname{inr}(y)/z]}{\Gamma\vdash\operatorname{case}(C,c_A,c_B,s):C[s/z]}
    \end{equation*}
    Categorically, given a map $\bracka{C}\to\bracka{A+B}$ and {\color{rgb,255:red,92;green,92;blue,214}sections of its pullback to $\bracka{A}$ and $\bracka{B}$}, we define {\color{rgb,255:red,214;green,92;blue,92} section over $\bracka{A+B}$ by universal property of $\bracka{A+B}$}, as shown in the diagram below. On one hand, categorically, uniqueness is what tells us that the induced map $\bracka{A+B}\to\bracka{C}$ is a section:
    \begin{equation*}
    % https://q.uiver.app/#q=WzAsNixbMSwwLCJcXGJyYWNrYXtDfSJdLFsxLDEsIlxcYnJhY2the0ErQn0iXSxbMCwxLCJcXGJyYWNrYXtBfSJdLFsyLDEsIlxcYnJhY2the0J9Il0sWzAsMCwiXFxidWxsZXQiXSxbMiwwLCJcXGJ1bGxldCJdLFsyLDEsIlxcYnJhY2the1xcb3BlcmF0b3JuYW1le2lubH19IiwyXSxbMCwxLCIiLDIseyJvZmZzZXQiOi0yfV0sWzQsMF0sWzQsMl0sWzUsMF0sWzUsM10sWzIsMCwiXFxicmFja2F7Y19BfSIsMSx7ImNvbG91ciI6WzI0MCw2MCw2MF19LFsyNDAsNjAsNjAsMV1dLFszLDAsIlxcYnJhY2the2NfQn0iLDEseyJjb2xvdXIiOlsyNDAsNjAsNjBdfSxbMjQwLDYwLDYwLDFdXSxbMywxLCJcXGJyYWNrYXtcXG9wZXJhdG9ybmFtZXtpbnJ9fSJdLFsxLDAsIiIsMix7Im9mZnNldCI6LTIsImNvbG91ciI6WzAsNjAsNjBdLCJzdHlsZSI6eyJib2R5Ijp7Im5hbWUiOiJkYXNoZWQifX19XSxbNCw2LCIiLDIseyJsZXZlbCI6MSwic3R5bGUiOnsibmFtZSI6ImNvcm5lciJ9fV1d
    \begin{tikzcd}
        \bullet & {\bracka{C}} & \bullet \\
        {\bracka{A}} & {\bracka{A+B}} & {\bracka{B}}
        \arrow[""{name=0, anchor=center, inner sep=0}, "{\bracka{\operatorname{inl}}}"', from=2-1, to=2-2]
        \arrow[shift left=2, from=1-2, to=2-2]
        \arrow[from=1-1, to=1-2]
        \arrow[from=1-1, to=2-1]
        \arrow[from=1-3, to=1-2]
        \arrow[from=1-3, to=2-3]
        \arrow["{\bracka{c_A}}"{description}, color={rgb,255:red,92;green,92;blue,214}, from=2-1, to=1-2]
        \arrow["{\bracka{c_B}}"{description}, color={rgb,255:red,92;green,92;blue,214}, from=2-3, to=1-2]
        \arrow["{\bracka{\operatorname{inr}}}", from=2-3, to=2-2]
        \arrow[shift left=2, color={rgb,255:red,214;green,92;blue,92}, dashed, from=2-2, to=1-2]
        \arrow["\lrcorner"{anchor=center, pos=0.125}, draw=none, from=1-1, to=0]
    \end{tikzcd}
    \end{equation*}
    On the hand, assume rule, if $z:A+B\vdash c:C$ and $z:A+B\vdash d:C$ have equal composites with $\operatorname{inl}$ and $\operatorname{inr}$, then we have use the equality type as: $x:A\vdash e_A:c[\operatorname{inl}(x)/z]=d[\operatorname{inl}(x)/z]$ and $y:B\vdash e_B:c[\operatorname{inr}(y)/z]=d[\operatorname{inr}(y)/z]$ and use the rule to construct $z:A+B\vdash e:(c=d)$
\end{remark}

\begin{remark}{(Computation Rule for Co-product)}
    Universal property requires that ${\color{rgb,255:red,214;green,92;blue,92}\bracka{\operatorname{case}(C,c_A,c_B,s)}}\circ\bracka{\operatorname{inl}}={\color{rgb,255:red,92;green,92;blue,214}\bracka{c_A}}$. This is similar to $\operatorname{inr}$. These are called composition rule. We will postpone discussion the notion of ``equals'' here. Nonetheless, we can write in type theoretic terms as:
    \begin{equation*}
        \frac{\Gamma,z:A+B\vdash C\type\qquad \Gamma,x:A\vdash c_A:C[\operatorname{inl}(x)/z]\qquad\Gamma,y:B\vdash c_B:C[\operatorname{inr}(y)/z]}{\Gamma\vdash\operatorname{case}(C,c_A,c_B,\operatorname{inl}(a))\equiv c_A[a/x]}
    \end{equation*}
\end{remark}

In conclusion, most rules of type theory come in packages like this (formation,introduction, elimination, and computation) come in packages, associated to one ``type constructor'' and expressing some universal property. Given any class of structured categories determined by universal properties, we can obtain a corresponding type theory by choosing all the corresponding packages of rules.